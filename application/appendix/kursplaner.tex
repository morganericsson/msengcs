\chapter{Kursplaner}\label{app:kursplaner}

Preliminära kursplaner för samtliga kurser i det föreslagna programmet följer. Varje kursplan är indelad i följande underrubriker. Nivå och och huvudämne, för de kurser som har ett huvudämne anges i början av kursplanen. Observera att de kurskoder som anges är fiktiva och kommer att ändras innan kursplanerna tas. 

\begin{itemize}
	\tightlist
\item Förkunskaper
\item Lärandemål
\item Kursinnehåll
\item Undervisnings- och arbetsformer
\item Examination
\item Måluppfyllelse
\item Kurslitteratur
\item Övrigt
\end{itemize}

Under Förkunskaper anges samtliga kurser som är förkunskaper, oavsett inbördes förkunskaper mellan dessa. Eventuella krav på poäng inom programmet, t.ex. för att få påbörja det självständiga arbetet anges i utbildningsplanen i bilaga~\ref{app:utbplan}.

Måluppfyllelse kopplar samman kursens lärandemål med dess examination. En markering (\faCheck) skall tolkas som att lärandemålet examineras helt eller delvis genom examinationsmomentet. I ansökan används generella namn på examinationsmomenten, t.ex. \texttt{RAP1}. Dessa kommer att ändras till faktiska provmoment när kursplanerna är tagna. Löpnumret används för att särskilja om kursen har flera examinationsmoment av samma typ, t.ex. individuella och gruppinlämningsuppgifter. Vi använder följande typer av examinationsmoment:

\begin{itemize}
\item \texttt{ASK} -- Auskultation på framläggning av andra studenters självständiga arbeten.
\item \texttt{DAT} -- Datortentamen, en tentamen som fokuserar på tekniska färdigheter, t.ex. förmåga att implementera enklare algoritmer i ett programmeringsspråk.
\item \texttt{HEM} -- Hemtentamen, en tentamen som utföras under en eller flera dagar, med full tillgång till kursmaterial, Internet, osv. 
\item \texttt{LAB} -- Laborationsuppgifter, t.ex. programmeringsuppgifter. Redovisas oftast genom en enkel laborationsrapport eller genom att lösningen (t.ex. källkoden) skickas in.
\item \texttt{MUN} -- Muntlig tentamen, en (individuell) tentamen som genomförs muntligt, ofta med hjälp av skrivtavla eller dator. 
\item \texttt{OPP} -- Opponering på en annan students eller grupps inlämning, t.ex. självständigt arbete, projektrapport, osv.
\item \texttt{PRJ} -- Projekt, där arbetet och leverablerna från projektet bedöms, t.ex. arbetet i grupp, hanteringen av vissa moment, osv.
\item \texttt{PRS} -- Presentation, där förmågan att presentera t.ex. muntligt bedöms. 
\item \texttt{RAP} -- Skriftlig rapport där något, t.ex. ett projektarbete rapporteras kring.
\item \texttt{SEM} -- Seminarie och debatt, där aktivt deltagande krävs och bedöms.
\item \texttt{TEN} -- Skriftlig salstentamen. Kan utföras med hjälp av dator.
\item \texttt{UPG} -- Inlämningsuppgift, t.ex. räkneuppgifter eller enklare modelleringsuppgifter. Redovisas oftast genom en enkel rapport.
\end{itemize}

Vi ser en progression mellan typerna, en \texttt{RAP} förväntas vara mera omfattande än en \texttt{UPG} och större fokus läggs vid t.ex. upplägg och struktur. I vissa fall finns en synergi mellan de olika typerna; t.ex. kan \texttt{PRJ} användas för att omfatta mindre projekt, men när projekten växer och större fokus läggs vid t.ex. genomförande kompletteras den med en eller flera \texttt{RAP} som fokuserar på olika aspekter av projektet. \texttt{PRJ} förväntas fortfarande innehålla samtliga leverabler som är viktiga för projektet, t.ex. tidsrapporter. I de fall där olika moment används för triangulering, dvs för att belysa olika aspekter av något eller några moment har stor vikt lags vid att göra det möjligt för en student att vid ett senare tillfälle komplettera eller genomföra examinationen på nytt utan att behöva göra om ett examinationsmoment den redan är godkänd på. 

De olika typerna av tentamen examinerar förutom kursens innehåll något olika aspekter; \texttt{MUN} kan t.ex. testa studentents förmåga att muntligt förklara och resonera sig fram till en lösning, medan \texttt{HEM} kan användas för ämnen som kräver ett större djup i svaren. \texttt{DAT} används främst som ett komplement till \texttt{LAB} och \texttt{UPG}, och ger tillfälle att observera studenten när den löser programmeringsuppgifter samt begränsa vad den har tillgång till, t.ex. med avseende på böcker eller Internet.

Examinationsmomenten kommer att förtydligas i kursernas ``KursPM''.


{\centering\small
\begin{tabular}[]{@{}lllll@{}}
\toprule
\textbf{\textsf{Kod}} & \textbf{\textsf{Benämning}} & \textbf{\textsf{Poäng}} & \textbf{\textsf{Nivå}} & \textbf{\textsf{Huvudområde}} \tabularnewline
\midrule
\texttt{1MA001} & Diskret matematik                                     & 7,5 & G1N & Matematik \tabularnewline     
\texttt{1DV001} & Programmering och datastrukturer                      & 7,5 & G1N & Datavetenskap \tabularnewline 
\texttt{1MA002} & Linjär algebra                                        & 7,5 & G1F & Matematik \tabularnewline     
\texttt{1DV002} & Introducerande projekt                                & 7,5 & G1F & Datavetenskap \tabularnewline 
\texttt{1MA003} & Envariabelanalys 1                                    & 5   & G1F & Matematik \tabularnewline     
\texttt{1DV003} & Databaser och datamodellering                         & 5   & G1F & Datavetenskap \tabularnewline 
\texttt{1DV004} & Objektorienterad programmering                        & 5   & G1F & Datavetenskap \tabularnewline 
\texttt{1MA004} & Tillämpad sannolikhetslära och statistik              & 7,5 & G1F & Matematik \tabularnewline     
\texttt{1FY001} & Mekanik                                               & 7,5 & G1F & Fysik \tabularnewline         
\midrule
\texttt{1DV005} & Jämnlöpande program                                   & 7,5 & G1F & Datavetenskap \tabularnewline 
\texttt{1FY002} & Ellära och magnetism                                  & 7,5 & G1F & Fysik \tabularnewline         
\texttt{1ZT001} & Teknisk kommunikation                                 & 5   & G1N & \tabularnewline           
\texttt{1DV006} & Algoritmer                                            & 5   & G1F & Datavetenskap \tabularnewline 
\texttt{1DV007} & Mjukvaruutvecklingsprojekt                            & 10  & G1F & Datavetenskap \tabularnewline 
\texttt{1ZT002} & Hållbar utveckling                                    & 5   & G1N & \tabularnewline           
\texttt{1MA005} & Envariabelanalys 2                                    & 5   & G1F & Matematik \tabularnewline     
\texttt{1MA006} & Flervariabelanalys                                    & 7,5 & G1F & Matematik \tabularnewline     
\texttt{2DV001} & Datorns uppbyggnad                                    & 7,5 & G2F & Datavetenskap \tabularnewline 
\midrule
\texttt{1MA007} & Numeriska metoder                                     & 5   & G1F & Matematik \tabularnewline     
\texttt{2DV002} & Mjukvaruarkitektur                                    & 5   & G2F & Datavetenskap \tabularnewline 
\texttt{2DV003} & Inbyggda system                                       & 5   & G2F & Datavetenskap \tabularnewline 
\texttt{1ED001} & Reglerteknik                                          & 5   & G1F & Elektroteknik \tabularnewline 
\texttt{2DV004} & Datorgrafik                                           & 5   & G2F & Datavetenskap \tabularnewline 
\texttt{2DV005} & Datornät                                              & 5   & G2F & Datavetenskap \tabularnewline 
\texttt{1ZT003} & Industriell ekonomi                                   & 5   & G1N & \tabularnewline           
\texttt{2ZT001} & Vetenskapliga metoder                                 & 5   & G2F & \tabularnewline               
\texttt{2DV006} & Datorsäkerhet                                         & 5   & G2F & Datavetenskap \tabularnewline 
\texttt{2DV007} & Självständigt arbete                                  & 15  & G2E & Datavetenskap \tabularnewline 
\midrule
\texttt{4DV001} & Modellering och simulering av system                  & 5   & A1N & Datavetenskap \tabularnewline 
\texttt{4DV002} & Kompilatorkonstruktion                                & 5   & A1N & Datavetenskap \tabularnewline 
\texttt{4DV003} & Formella metoder                                      & 5   & A1F & Datavetenskap \tabularnewline 
\texttt{2MA001} & Optimering                                            & 5   & G2F & Matematik \tabularnewline     
\texttt{4DV004} & Projekt i modellbaserad utveckling                    & 10  & A1N & Datavetenskap \tabularnewline 
\texttt{4DV005} & Maskininlärning                                       & 5   & A1N & Datavetenskap \tabularnewline 
\texttt{4DV006} & Parallelldatorprogrammering                           & 5   & A1N & Datavetenskap \tabularnewline 
\texttt{4DV007} & Djup maskininlärning                                  & 5   & A1F & Datavetenskap \tabularnewline 
\texttt{2ZT002} & Lean startup                                          & 5   & G2F &  \tabularnewline 
\texttt{4DV008} & Projekt i dataintensiva system                        & 10  & A1N & Datavetenskap \tabularnewline 
\midrule
\texttt{4DV009} & Informationsvisualisering                             & 5   & A1N & Datavetenskap \tabularnewline 
\texttt{4DV010} & Datautvinning                                         & 5   & A1F & Datavetenskap \tabularnewline 
\texttt{4DV011} & Avancerad informationsvisualisering och tillämpningar & 5   & A1F & Datavetenskap \tabularnewline 
\texttt{4DV012} & Vetenskapliga metoder inom datavetenskap              & 5   & A1F & Datavetenskap \tabularnewline 
\texttt{4DV013} & Projekt i visualisering och dataanalys                & 10  & A1F & Datavetenskap \tabularnewline 
\texttt{5DV001} & Självständigt arbete                                  & 30  & A2E & Datavetenskap \tabularnewline 
\bottomrule
\end{tabular}
}
\pagebreak 

\section*{1MA001 - Diskret matematik (7,5 hp)}

\begin{tabular}{ll}\emph{Huvudområde}: & Datavetenskap\tabularnewline\emph{Fördjupning}: & G1N\tabularnewline\end{tabular}

\subsection*{Förkunskaper}

Grundläggande behörighet samt Matematik D eller Matematik 4
(områdesbehörighet 9/A9).

\subsection*{Lärandemål}

Efter slutförd kurs skall studenten kunna:

\begin{enumerate}
\def\labelenumi{\Alph{enumi}.}
\tightlist
\item
  \emph{Kunskap och förståelse}

  \begin{enumerate}
  \def\labelenumii{\Alph{enumi}.\arabic{enumii}.}
  \tightlist
  \item
    Förklara begrepp i logik, mängdlära, algebra och diskret matematik,
    samt
  \item
    redogöra för definitioner samt formulera och bevisa teorem som är
    centrala i diskret matematik.
  \end{enumerate}
\item
  \emph{Färdighet och förmåga}

  \begin{enumerate}
  \def\labelenumii{\Alph{enumi}.\arabic{enumii}.}
  \tightlist
  \item
    Använda resultat i logik, mängdlära och diskreta matematiska
    modeller,
  \item
    hantera logisk och algebraisk formalism,
  \item
    lösa problem, utföra beräkningar och föra resonemang i diskret
    matematik,
  \item
    skriftligt presentera beräkningar och resonemang inom diskret
    matematik så att de kan följas av den som inte är insatt i
    problemet, samt
  \item
    tillämpa diskreta matematiska modeller på datavetenskapliga problem.
  \end{enumerate}
\item
  \emph{Värderingsförmåga och förhållningssätt}

  \begin{enumerate}
  \def\labelenumii{\Alph{enumi}.\arabic{enumii}.}
  \tightlist
  \item
    Diskutera relevans, räckvidd och noggrannhet av matematiska modeller
    såsom grafer och differensekvationer.
  \end{enumerate}
\end{enumerate}

\subsection*{Kursinnehåll}

Kursen ger en introduktion till diskret matematik, logik och
kombinatorik. Följande moment behandlas:

\begin{itemize}
\tightlist
\item
  Logik: sanningsvärdestabeller, härledningar, disjunktiv och konjunktiv
  normalform, satslogik, predikatlogisk formalism.
\item
  Mängdlära: dualitetsprincipen, de Morgans lagar, principen för
  inklusion och exklusion.
\item
  Boolesk algebra.
\item
  Relationer och funktioner: funktionslära, egenskaper hos relationer,
  ekvivalensrelationer, ordningsrelationer, matris- och
  grafrepresentation av relationer.
\item
  Induktion: välordningsprincipen, matematisk induktion, rekursion.
\item
  Genererande funktioner.
\item
  Modulär aritmetik.
\item
  Kombinatorik.
\item
  Differensekvationer.
\item
  Grafteori: Eulerkretsar, Hamiltonbanor, plana grafer, färgläggning av
  grafer och kromatiska polynom, träd.
\end{itemize}

\subsection*{Undervisnings- och arbetsformer}

Föreläsningar, lärarledda räkneövningar och lärarledda möten relaterade
till inlämningsuppgifterna.

\subsection*{Examination}

Examinationen av kursen delas in i följande moment:

\begin{longtable}[]{@{}llcc@{}}
\toprule
\textsf{Kod} & \textsf{Benämning} & \textsf{Betyg} & \textsf{Poäng}\tabularnewline
\midrule
\endhead
\texttt{TEN1} & Tentamen: Problemlösning & A-F & 5\tabularnewline
\texttt{TEN2} & Tentamen: Teori & G-U & 1,5\tabularnewline
\texttt{UPG1} & Inlämningsuppgift & G-U & 1,0\tabularnewline
\bottomrule
\end{longtable}

För godkänt betyg på kursen krävs minst betyg E på \texttt{TEN1} samt betyg G på
\texttt{TEN2} och \texttt{UPG1}. Slutbetyget bestäms från \texttt{TEN1}.

\subsection*{Måluppfyllelse}

Examinationsmomenten kopplas till lärandemålen enligt följande:

\begin{longtable}[]{@{}lccc@{}}
\toprule
\textsf{Lärandemål} & \texttt{TEN1} & \texttt{TEN2} & \texttt{UPG1}\tabularnewline
\midrule
\endhead
A.1 & & \faCheck &\tabularnewline
A.2 & & \faCheck &\tabularnewline
B.1 & \faCheck & &\tabularnewline
B.2 & \faCheck & &\tabularnewline
B.3 & \faCheck & \faCheck &\tabularnewline
B.4 & \faCheck & & \faCheck\tabularnewline
B.5 & & & \faCheck\tabularnewline
C.1 & & & \faCheck\tabularnewline
\bottomrule
\end{longtable}

\subsection*{Kurslitteratur}

Obligatorisk litteratur:

\begin{itemize}
\tightlist
\item
  Rosen, K H, \emph{Discrete mathematics and its applications},
  McGraw-Hill, 2012. Antal sidor: 250 av 1024.
\end{itemize}

\subsection*{Övrigt}

Kursen genomförs på ett sätt sådant att både kvinnor och mäns kunskap och erfarenhet utvecklas och görs synlig.
\pagebreak

\section*{1DV001 - Programmering och datastrukturer (7,5 hp)}

\begin{tabular}{ll}\emph{Huvudområde}: & Datavetenskap\tabularnewline\emph{Fördjupning}: & G1N\tabularnewline\end{tabular}

\subsection*{Förkunskaper}

Grundläggande behörighet samt Matematik D eller Matematik 4
(områdesbehörighet 9/A9).

\subsection*{Lärandemål}

Efter slutförd kurs skall studenten kunna:

\begin{enumerate}
\def\labelenumi{\Alph{enumi}.}
\tightlist
\item
  \emph{Kunskap och förståelse}

  \begin{enumerate}
  \def\labelenumii{\Alph{enumi}.\arabic{enumii}.}
  \tightlist
  \item
    Förklara grundläggande programspråkskonstruktioner som t.ex.
    variabler, typer, styrande satser och funktioner, samt
  \item
    förklara grundläggande algoritmer och datastrukturer, samt
    exemplifiera hur och när de bör användas.
  \end{enumerate}
\item
  \emph{Färdighet och förmåga}

  \begin{enumerate}
  \def\labelenumii{\Alph{enumi}.\arabic{enumii}.}
  \tightlist
  \item
    Skapa och implementera en lösning till ett givet problem i
    programmeringsspråket Python,
  \item
    implementera givna algoritmer för att lösa kända typer av problem
    (t.ex. sortering och sökning) och analysera deras tidskomplexitet,
  \item
    installera och använda verktyg och bibliotek som används vid
    programmering,
  \item
    strukturera och genomföra korta muntliga och skriftliga
    presentationer av mindre programmeringsprojekt, samt
  \item
    dokumentera program och följa programkodskonventioner.
  \end{enumerate}
\item
  \emph{Värderingsförmåga och förhållningssätt}

  \begin{enumerate}
  \def\labelenumii{\Alph{enumi}.\arabic{enumii}.}
  \tightlist
  \item
    Resonera kring hur välstrukturerat och lättförstått ett program är,
    samt
  \item
    göra att motiverat val av datastrukturer och algoritmer i olika
    scenarier, t.ex. med avseende på prestanda.
  \end{enumerate}
\end{enumerate}

\subsection*{Kursinnehåll}

Kursen är en inledande programmeringskurs i programspråket Python.
Kursens första del fokuserar på programmeringsfärdigheter och vanliga
programspråkskonstruktioner (t.ex. variabler, typer, styrande satser och
funktioner). Under kursens andra halva introduceras algoritmer och
datastrukturer med exempel från t.ex. sökning och sortering. Egenskap
hos och analys av dessa, t.ex. med avseende på tidskomplexitet
diskuteras.

Följande moment behandlas:

\begin{itemize}
\tightlist
\item
  Introduktion till labbmiljön och andra resurser, t.ex. lärplattformen.
\item
  En dators uppbyggnad och hur program exekveras.
\item
  Introduktion till utvecklingsmiljöer, t.ex. editor, interpretator,
  osv.
\item
  Problemlösning.
\item
  Att formulera lösningar på problem så att datorer kan hantera dem.
\item
  Grundläggande programspråkskonstruktioner.
\item
  Filhantering.
\item
  Att använda externa programvarubibliotek.
\item
  Att använda klasser, objekt och moduler.
\item
  Vanliga datastrukturer (t.ex. listor, mängder, tabeller och träd).
\item
  Vanliga algoritmer för sökning och sortering.
\item
  Enkla uppskattningar av bästa, värsta och genomsnittlig
  tidskomplexitet.
\item
  Dokumentation av kod och kodkonventioner.
\item
  Parvist arbete, problemlösning och kommunikationsfärdigheter.
\end{itemize}

\subsection*{Undervisnings- och arbetsformer}

Undervisningen sker i form av föreläsningar, lärarledda laborationer,
projekt med regelbundna handledning. Kursen avslutas med en muntlig och
skriftlig projektredovisning. Laborationerna är individuella, projekt
och presentationer sker i par.

\subsection*{Examination}

Examinationen av kursen delas in i följande moment:

\begin{longtable}[]{@{}llcc@{}}
\toprule
\textsf{Kod} & \textsf{Benämning} & \textsf{Betyg} & \textsf{Poäng}\tabularnewline
\midrule
\endhead
\texttt{LAB1} & Programmeringsuppgifter & A-F & 3\tabularnewline
\texttt{DAT1} & Datortentamen & G-U & 1\tabularnewline
\texttt{PRJ1} & Projekt i algoritmer och datastrukturer & A-F &
3,5\tabularnewline
\bottomrule
\end{longtable}

För godkänt betyg på kursen krävs betyg G på \texttt{DAT1} och minst betyg E på
övriga moment. Slutbetyget bestäms från: \texttt{LAB1} (50~\%) och \texttt{PRJ1} (50~\%).

\subsection*{Måluppfyllelse}

Examinationsmomenten kopplas till lärandemålen enligt följande:

\begin{longtable}[]{@{}lccc@{}}
\toprule
\textsf{Lärandemål} & \texttt{LAB1} & \texttt{DAT1} & \texttt{PRJ1}\tabularnewline
\midrule
\endhead
A.1 & \faCheck & \faCheck &\tabularnewline
A.2 & \faCheck & \faCheck & \faCheck\tabularnewline
B.1 & \faCheck & \faCheck & \faCheck\tabularnewline
B.2 & \faCheck & \faCheck & \faCheck\tabularnewline
B.3 & \faCheck & & \faCheck\tabularnewline
B.4 & & & \faCheck\tabularnewline
B.5 & \faCheck & & \faCheck\tabularnewline
C.1 & \faCheck & &\tabularnewline
C.2 & & & \faCheck\tabularnewline
\bottomrule
\end{longtable}

\subsection*{Kurslitteratur}

Obligatorisk litteratur:

\begin{itemize}
\tightlist
\item
  Zelle, J. M., \emph{Python Programming: An Introduction to Computer
  Science}, tredje utgåvan, Franklin, Beedle \& Associates, 2016. Antal
  sidor: 450 av 812.
\end{itemize}

\subsection*{Övrigt}

Kursen genomförs på ett sätt sådant att både kvinnor och mäns kunskap och erfarenhet utvecklas och görs synlig.
\pagebreak

\section*{1MA002 - Linjär algebra (7,5 hp)}

\begin{tabular}{ll}\emph{Huvudområde}: & Matematik\tabularnewline\emph{Fördjupning}: & G1F\tabularnewline\end{tabular}

\subsection*{Förkunskaper}

\begin{itemize}
\tightlist
\item
  1MA001 - Diskret matematik
\end{itemize}

\subsection*{Lärandemål}

Efter slutförd kurs skall studenten kunna:

\begin{enumerate}
\def\labelenumi{\Alph{enumi}.}
\tightlist
\item
  \emph{Kunskap och förståelse}

  \begin{enumerate}
  \def\labelenumii{\Alph{enumi}.\arabic{enumii}.}
  \tightlist
  \item
    Redogöra för linjära algebrans grundläggande begrepp och
    operationer, kunna utföra dessa operationer och utnyttja detta i
    problemlösning, samt
  \item
    redogöra för sambanden mellan linjära algebrans grundläggande
    begrepp och utnyttja dessa samband i problemlösning.
  \end{enumerate}
\item
  \emph{Färdighet och förmåga}

  \begin{enumerate}
  \def\labelenumii{\Alph{enumi}.\arabic{enumii}.}
  \tightlist
  \item
    Kombinera kunskaper om olika begrepp, operationer och metoder från
    linjära algebran i problemlösning,
  \item
    redogöra för matematiska resonemang på ett strukturerat och logiskt
    sammanhängande sätt, samt
  \item
    använda programspråket Matlab i problemlösning.
  \end{enumerate}
\item
  \emph{Värderingsförmåga och förhållningssätt}

  \begin{enumerate}
  \def\labelenumii{\Alph{enumi}.\arabic{enumii}.}
  \tightlist
  \item
    Visa förmåga att bedöma rimligheten i resultat av beräkningar.
  \end{enumerate}
\end{enumerate}

\subsection*{Kursinnehåll}

Det övergripande syftet med kursen är att ge en sammanhållen begreppsram
från linjär algebra med tillämpningar inom analys, grafteori,
datorgrafik, mekanik, numerisk analys, matematisk statistik,
reglerteknik, ekonomi och linjär optimering m.fl. ämnen.

\begin{itemize}
\tightlist
\item
  Linjära ekvationssystem.
\item
  Geometriska vektorer, räta linjer och plan.\\
\item
  Linjära rum, euklidiska rum, koordinatsystem, underrum, linjärt
  oberoende, baser, basbyte, skalärprodukt, ortogonalitet,
  vektorprodukt, volymfunktion.
\item
  Matriser: Matrisalgebra, invers matris och linjära ekvationssystem,
  determinant, rang, egenvärde, egenvektor och diagonalisering.
\item
  Linjära avbildningar: Matrisframställnig, nollrum, värderum,
  dimensionssatsen.
\item
  Tillämpningar på projektioner, speglingar och rotationer.
\item
  Introduktion till minsta kvadratmetoden. Gram-Schmidts
  ortogonaliseringsmetod.
\item
  Linjära algebrans tillämpningar inom diskret matematik, analys,
  teknik, fysik och ekonomi.
\item
  Problemlösning med hjälp av programvaran Matlab.
\end{itemize}

\subsection*{Undervisnings- och arbetsformer}

Undervisningen sker i form av föreläsningar, lärarledda räkneövningar
och datorlaborationer. Inlämningsuppgifter sker i par. Obligatorisk
närvaro kan förekomma på vissa moment.

\subsection*{Examination}

Examinationen av kursen delas in i följande moment:

\begin{longtable}[]{@{}llcc@{}}
\toprule
\textsf{Kod} & \textsf{Benämning} & \textsf{Betyg} & \textsf{Poäng}\tabularnewline
\midrule
\endhead
\texttt{LAB1} & Laborationer i Matlab & A-F & 1,5\tabularnewline
\texttt{TEN1} & Skriftlig tentamen & A-F & 6\tabularnewline
\bottomrule
\end{longtable}

För godkänt betyg på kursen krävs minst betyg E på samtliga moment.
Slutbetyget bestäms från: \texttt{LAB1} (20~\%) och \texttt{TEN1} (80~\%).

\subsection*{Måluppfyllelse}

Examinationsmomenten kopplas till lärandemålen enligt följande:

\begin{longtable}[]{@{}lcc@{}}
\toprule
\textsf{Lärandemål} & \texttt{LAB1} & \texttt{TEN1}\tabularnewline
\midrule
\endhead
A.1 & \faCheck & \faCheck\tabularnewline
A.2 & \faCheck & \faCheck\tabularnewline
B.1 & \faCheck & \faCheck\tabularnewline
B.2 & \faCheck & \faCheck\tabularnewline
B.3 & \faCheck &\tabularnewline
C.1 & \faCheck & \faCheck\tabularnewline
\bottomrule
\end{longtable}

\subsection*{Kurslitteratur}

Obligatorisk litteratur:

\begin{itemize}
\tightlist
\item
  Lindstöm, T., \emph{Med fokus på linjär algebra}, tredje upplagan,
  Studentlitteratur, 2017. Antal sidor: 270 av 336.
\end{itemize}

\subsection*{Övrigt}

Kursen genomförs på ett sätt sådant att både kvinnor och mäns kunskap och erfarenhet utvecklas och görs synlig.
\pagebreak

\section*{1DV002 - Introducerande projekt (7,5 hp)}

\begin{tabular}{ll}\emph{Huvudområde}: & Datavetenskap\tabularnewline\emph{Fördjupning}: & G1F\tabularnewline\end{tabular}

\subsection*{Förkunskaper}

\begin{itemize}
\tightlist
\item
  1DV001 - Programmering och datastrukturer
\end{itemize}

\subsection*{Lärandemål}

Efter slutförd kurs skall studenten kunna:

\begin{enumerate}
\def\labelenumi{\Alph{enumi}.}
\tightlist
\item
  \emph{Kunskap och förståelse}

  \begin{enumerate}
  \def\labelenumii{\Alph{enumi}.\arabic{enumii}.}
  \tightlist
  \item
    Namnge och förklara funktionen hos de viktigaste komponenterna i en
    Raspberry Pi,
  \item
    förklara hur man skriver, installerar och exekverar Python-program
    på en Raspberry Pi,
  \item
    förklara hur systemkrav tas fram, specificeras och testas,
  \item
    översiktligt redogöra för vad projektledning och kvalitetsarbete
    innebär i praktiken, samt
  \item
    redogöra för mjukvaruindustrins olika sektorer och olika
    arbetsuppgifter.
  \end{enumerate}
\item
  \emph{Färdighet och förmåga}

  \begin{enumerate}
  \def\labelenumii{\Alph{enumi}.\arabic{enumii}.}
  \tightlist
  \item
    Utveckla Pythonprogram för Raspberry Pi med externa enheter (t.ex.
    sensorer) och nätverkskoppling,
  \item
    analysera ett problem, skapa en kravspecifikation, designa och
    implementera lösningar, och verifiera att lösningen uppfyller alla
    krav,
  \item
    använda vanliga tekniska projektverktyg (tex versionshantering med
    Git),
  \item
    självständigt söka efter och värdera information,
  \item
    strukturera och genomföra en skriftlig och muntlig presentation av
    genomförda laborationer och projekt, samt
  \item
    genomföra ett projekt i grupp under begränsad tid och där tillämpa
    en arbetsform som presenterats i kursen.
  \end{enumerate}
\item
  \emph{Värderingsförmåga och förhållningssätt}

  \begin{enumerate}
  \def\labelenumii{\Alph{enumi}.\arabic{enumii}.}
  \tightlist
  \item
    Reflektera över och värdera en given ansats att lösa ett problem,
  \item
    reflektera över relationen mellan ämneskunskap, ingenjörsfärdigheter
    och yrkesrollen ingenjör, samt
  \item
    reflektera över och värdera sin egen kontra gruppens insats vid
    laborations- och projektarbete.
  \end{enumerate}
\end{enumerate}

\subsection*{Kursinnehåll}

Kursen har två parallella spår: Det första introducerar kort Internet of
Things och det introducerar studenten till yrkesrollen samt fördjupas
dess ämneskunskaper. Inom det första spåret presentation en enkel dator,
Raspberry Pi, och hur man kan skriva program i Python som interagerar
med externa enheter, sensorer och tjänster över Internet. Inom det andra
spåret introduceras hur man arbetar i projekt och grupp, samt
yrkesrollen (mjukvaru)ingenjör.

Följande moment behandlas:

\begin{itemize}
\tightlist
\item
  Introduktion till Internet of Things.
\item
  Introduktion till Raspberry Pi (hårdvara och mjukvara).
\item
  Implementera och exekvera Pythonprogram på Raspberry Pi.
\item
  Interagera med externa enheter (t.ex. sensorer).
\item
  Interagera med nätverkskopplade enheter/Internet.
\item
  Fördjupning av labbmiljön.
\item
  Introduktion till kravhantering, mjukvarudesign och testning.
\item
  Introduktion av verktyg och metoder som används inom ett projekt,
  t.ex. versionshantering, kravhantering, kommunikation, osv..
\item
  Projektmetodik och projektdynamik.
\item
  Hur man arbetar i grupp, vilka roller som finns, vilket ansvar
  individen har, osv.
\item
  Informationssökning.
\item
  Hur man skriver enklare projektdokumentation.
\item
  Muntlig presentation av tekniskt material.
\item
  Skriftlig presentation av tekniskt material.
\item
  Ingenjörens yrkesroll, arbetsuppgifter och förhållningssätt.
\item
  Ingenjörens ansvar och arbetsmiljö.
\end{itemize}

\subsection*{Undervisnings- och arbetsformer}

Undervisningen sker i form av föreläsningar, lärarledda laborationer,
handledning i projektgrupp och en slutpresentation. Laborationerna sker
i par och projekt och presentationer sker i grupper om 4 studenter.
Yrkesrollen ingenjör presenteras via gästföreläsningar och/eller
studiebesök.

\subsection*{Examination}

Examinationen av kursen delas in i följande moment:

\begin{longtable}[]{@{}llcc@{}}
\toprule
\textsf{Kod} & \textsf{Benämning} & \textsf{Betyg} & \textsf{Poäng}\tabularnewline
\midrule
\endhead
\texttt{LAB1} & Programmeringsuppgifter & G-U & 1,5\tabularnewline
\texttt{PRJ1} & Projekt & A-F & 4\tabularnewline
\texttt{PRS1} & Presentation & A-F & 1\tabularnewline
\texttt{UPG1} & Uppgifter om yrkesrollen ingenjör & G-U & 1\tabularnewline
\bottomrule
\end{longtable}

För godkänt betyg på kursen krävs betyg G på \texttt{LAB1} och \texttt{UPG1} samt minst
betyg E på övriga moment. Slutbetyget bestäms från: \texttt{PRJ1} (75~\%) och \texttt{PRS1}
(25~\%).

\subsection*{Måluppfyllelse}

Examinationsmomenten kopplas till lärandemålen enligt följande:

\begin{longtable}[]{@{}lcccc@{}}
\toprule
\textsf{Lärandemål} & \texttt{LAB1} & \texttt{PRJ1} & \texttt{PRS1} & \texttt{UPG1}\tabularnewline
\midrule
\endhead
A.1 & \faCheck & & &\tabularnewline
A.2 & \faCheck & & &\tabularnewline
A.3 & & \faCheck & \faCheck &\tabularnewline
A.4 & & \faCheck & \faCheck & \faCheck\tabularnewline
A.5 & & & & \faCheck\tabularnewline
B.1 & \faCheck & \faCheck & &\tabularnewline
B.2 & & \faCheck & \faCheck &\tabularnewline
B.3 & & \faCheck & &\tabularnewline
B.4 & \faCheck & \faCheck & &\tabularnewline
B.5 & \faCheck & & \faCheck &\tabularnewline
B.6 & & & \faCheck &\tabularnewline
C.1 & & \faCheck & \faCheck &\tabularnewline
C.2 & & & \faCheck & \faCheck\tabularnewline
C.3 & \faCheck & \faCheck & &\tabularnewline
\bottomrule
\end{longtable}

\subsection*{Kurslitteratur}

Kurslitteratur bestäms i samråd med handledare.

\subsection*{Övrigt}

Kursen genomförs på ett sätt sådant att både kvinnor och mäns kunskap och erfarenhet utvecklas och görs synlig.
\pagebreak

\section*{1MA003 - Envariabelanalys 1 (5 hp)}

\begin{tabular}{ll}\emph{Huvudområde}: & Matematik\tabularnewline\emph{Fördjupning}: & G1F\tabularnewline\end{tabular}

\subsection*{Förkunskaper}

\begin{itemize}
\tightlist
\item
  1MA001 - Diskret matematik
\end{itemize}

\subsection*{Lärandemål}

Efter slutförd kurs skall studenten kunna:

\begin{enumerate}
\def\labelenumi{\Alph{enumi}.}
\tightlist
\item
  \emph{Kunskap och förståelse}

  \begin{enumerate}
  \def\labelenumii{\Alph{enumi}.\arabic{enumii}.}
  \tightlist
  \item
    Förklara analytiska begrepp såsom gränsvärden och kontinuitet, samt
  \item
    redogöra för definitioner samt formulera och bevisa teorem som är
    centrala i analys, såsom medelvärdessatsen och analysens huvudsats.
  \end{enumerate}
\item
  \emph{Färdighet och förmåga}

  \begin{enumerate}
  \def\labelenumii{\Alph{enumi}.\arabic{enumii}.}
  \tightlist
  \item
    Hantera elementära funktioner algebraiskt och analytiskt,
  \item
    använda differential- och integralkalkyl i en variabel,
  \item
    lösa problem, utföra beräkningar och föra resonemang i analys,
  \item
    skriftligt presentera beräkningar och resonemang så att de kan
    följas av den som inte är insatt i problemet,
  \item
    tillämpa differential- och integralkalkyl på tekniska, fysikaliska
    och datavetenskapliga problem, samt
  \item
    visualisera resultat såsom grafer till funktioner.
  \end{enumerate}
\item
  \emph{Värderingsförmåga och förhållningssätt}

  \begin{enumerate}
  \def\labelenumii{\Alph{enumi}.\arabic{enumii}.}
  \tightlist
  \item
    Diskutera relevans, räckvidd och noggrannhet av matematiska modeller
    såsom differentialekvationer.
  \end{enumerate}
\end{enumerate}

\subsection*{Kursinnehåll}

Kursen ger en introduktion till envariabelanalys. Följande moment
behandlas:

\begin{itemize}
\tightlist
\item
  Gränsvärden och kontinuitet: gränsvärdesdefinitionen, räkneregler,
  instängningssatsen, standardgränsvärden, talet e.
\item
  Derivata och funktionsstudier: derivatans definition, räkneregler, de
  elementära funktionernas derivator, medelvärdessatsen,
  extremvärdesproblem, kurvritning, asymptoter.
\item
  Integraler: primitiva funktioner, integralens definition, analysens
  huvudsats, integralkalkylens medelvärdessats, partiell integration,
  variabelbyte, integrering av rationella funktioner.
\item
  Differentialekvationer: linjära och separabla ekvationer av första
  ordningen, linjära ekvationer av andra ordningen med konstanta
  koefficienter.
\item
  Matematisk modellering med differentialekvationer.
\end{itemize}

\subsection*{Undervisnings- och
arbetsformer}

Föreläsningar och lärarledda räkneövningar.

\subsection*{Examination}

Examinationen av kursen delas in i följande moment:

\begin{longtable}[]{@{}llcc@{}}
\toprule
\textsf{Kod} & \textsf{Benämning} & \textsf{Betyg} & \textsf{Poäng}\tabularnewline
\midrule
\endhead
\texttt{TEN1} & Tentamen: Problemlösning & A-F & 4\tabularnewline
\texttt{TEN2} & Tentamen: Teori & G-U & 1\tabularnewline
\bottomrule
\end{longtable}

För godkänt betyg på kursen krävs minst betyg E på \texttt{TEN1} samt betyg G på
\texttt{TEN2}. Slutbetyget bestäms från \texttt{TEN1}.

\subsection*{Måluppfyllelse}

Examinationsmomenten kopplas till lärandemålen enligt följande:

\begin{longtable}[]{@{}lcc@{}}
\toprule
\textsf{Lärandemål} & \texttt{TEN1} & \texttt{TEN2}\tabularnewline
\midrule
\endhead
A.1 & & \faCheck\tabularnewline
A.2 & & \faCheck\tabularnewline
B.1 & \faCheck &\tabularnewline
B.2 & \faCheck &\tabularnewline
B.3 & \faCheck & \faCheck\tabularnewline
B.4 & \faCheck &\tabularnewline
B.5 & \faCheck &\tabularnewline
B.6 & \faCheck &\tabularnewline
C.1 & \faCheck &\tabularnewline
\bottomrule
\end{longtable}

\subsection*{Kurslitteratur}

Obligatorisk litteratur:

\begin{itemize}
\tightlist
\item
  Månsson, J. och Nordbeck, P., \emph{Endimensionell analys},
  Studentlitteratur, 2011. Antal sidor: 200 av 400.
\item
  Månsson, J. och Nordbeck, P., \emph{Övningar i endimensionell analys},
  Studentlitteratur, 2011. Antal sidor: 100 av 206.
\end{itemize}

\subsection*{Övrigt}

Kursen genomförs på ett sätt sådant att både kvinnor och mäns kunskap och erfarenhet utvecklas och görs synlig.
\pagebreak
\section*{1DV003 - Databaser och datamodellering (5 hp)}

\begin{tabular}{ll}\emph{Huvudområde}: & Datavetenskap\tabularnewline\emph{Fördjupning}: & G1F\tabularnewline\end{tabular}

\subsection*{Förkunskaper}

\begin{itemize}
\tightlist
\item
  1DV001 - Programmering och datastrukturer
\item
  1DV002 - Introducerande projekt
\item
  1MA001 - Diskret matematik
\end{itemize}

\subsection*{Lärandemål}

Efter slutförd kurs skall studenten kunna:

\begin{enumerate}
\def\labelenumi{\Alph{enumi}.}
\tightlist
\item
  \emph{Kunskap och förståelse}

  \begin{enumerate}
  \def\labelenumii{\Alph{enumi}.\arabic{enumii}.}
  \tightlist
  \item
    Förklara de grundläggande begreppen relaterade till datorbaserade
    informationssystem samt förklara de viktigaste problemen och
    regelverken i samband med datahantering i stora organisationer och
    företag,
  \item
    översiktligt redogöra för olika databastyper, t.ex. relations-,
    dokument- och objekt-baserade,
  \item
    förklara de olika typerna av modeller (konceptuell, logisk och
    fysisk) som används för att ta fram och resonera kring en databas,
  \item
    förklara relationsmodellen, relationsalgebra, kopplingen till
    predikatlogik och normalformer, samt
  \item
    redogöra för hur data, t.ex. tabeller, kan lagras på block-enheter,
    t.ex. hårddiskar, inklusive att kunna beskriva de algoritmer och
    datastrukturer som behövs.
  \end{enumerate}
\item
  \emph{Färdighet och förmåga}

  \begin{enumerate}
  \def\labelenumii{\Alph{enumi}.\arabic{enumii}.}
  \tightlist
  \item
    Utforma datamodeller på olika semantiska nivåer (begreppsmässig,
    logisk, fysisk) med hjälp av lämplig formalism, såsom
    Entity-Relationship och relationsmodeller,
  \item
    optimera en databasdesign genom att använda normalformer (1NF, 2NF,
    3NF, BCNF), med beaktande av egenskaperna hos de fysiska medier som
    används för datalagring, samt
  \item
    implementera relationsdatamodeller i en databashanterare samt skapa,
    fråga och manipulera data med hjälp av SQL via klientprogram och
    program implementerade i Python.
  \end{enumerate}
\item
  \emph{Värderingsförmåga och förhållningssätt}

  \begin{enumerate}
  \def\labelenumii{\Alph{enumi}.\arabic{enumii}.}
  \tightlist
  \item
    Analysera och värdera en domän och dess begränsningar som en
    datamodell samt muntligt och skriftligt diskutera fördelarna och
    nackdelarna med designen,
  \item
    reflektera över egenskaperna hos olika datamodeller och välja de som
    är mest lämpade för det problem som ska lösas, samt
  \item
    resonera om hur olika designalternativ påverkar databasens
    egenskaper, t.ex. prestanda och möjliga frågor.
  \end{enumerate}
\end{enumerate}

\subsection*{Kursinnehåll}

Kursen ger en introduktion till databaser och
informationshanteringssystem. Den utgår från grunderna i hur data kan
lagras, t.ex. via relationsmodellen eller via nätverksmodeller och
diskuterar hur frågespråk kan byggas ovanpå dessa. Bra design diskuteras
på flera olika nivåer, från logiska datamodeller, till t.ex.
relationsmodellen och normalformer och den faktiska fysiska lagringen.

Följande moment behandlas:

\begin{itemize}
\tightlist
\item
  Introduktion till datorbaserade informationshanteringssystem.
\item
  Vikten av databaser och informationshantering i samhället.
\item
  Vilken data kan, får och bör lagras. Vilka regelverk gäller, t.ex.
  GDPR.
\item
  Konceptuella, logiska och fysiska datamodeller.
\item
  Olika typer av databashanterare.
\item
  Diagram för att modellera data, t.ex. E/R.
\item
  Relationsmodeller och relationsalgebra.
\item
  Databasfrågor och databasmanipulation med SQL.
\item
  Funktionella beroenden och normalformer (1NF, 2NF, 3NF, BCNF).
\item
  Installation och användning av vanliga databashanterare, t.ex. MySQL i
  labbmiljön.
\item
  Utveckling av program som använder en databas.
\item
  Predikatlogik och dess relation till databaser.
\item
  Introduktion till samtidighet, låsning och hur transaktioner fungerar.
\item
  Filsystem och hur data lagras på blockenheter (t.ex. hårddiskar).
\end{itemize}

\subsection*{Undervisnings- och
arbetsformer}

Undervisningen sker i form av föreläsningar, lärarledda laborationer och
handledning av inlämningsuppgifter. Laborationer och inlämningsuppgifter
sker i par. I vissa moment av kursen förväntas studenten att på egen
hand söka den information som behövs för att lösa en uppgift.

\subsection*{Examination}

Examinationen av kursen delas in i följande moment:

\begin{longtable}[]{@{}llcc@{}}
\toprule
\textsf{Kod} & \textsf{Benämning} & \textsf{Betyg} & \textsf{Poäng}\tabularnewline
\midrule
\endhead
\texttt{MUN1} & Muntlig tentamen & A-F & 2\tabularnewline
\texttt{LAB1} & Programmeringsuppgifter & A-F & 2\tabularnewline
\texttt{UPG1} & Inlämningsuppgifter & A-F & 1\tabularnewline
\bottomrule
\end{longtable}

För godkänt betyg på kursen krävs minst betyg E på samtliga moment.
Slutbetyget bestäms från: \texttt{MUN1} (40~\%), \texttt{LAB1} (40~\%) och \texttt{UPG1} (20~\%).

\subsection*{Måluppfyllelse}

Examinationsmomenten kopplas till lärandemålen enligt följande:

\begin{longtable}[]{@{}lccc@{}}
\toprule
\textsf{Lärandemål} & \texttt{MUN1} & \texttt{LAB1} & \texttt{UPG1}\tabularnewline
\midrule
\endhead
A.1 & \faCheck & &\tabularnewline
A.2 & \faCheck & &\tabularnewline
A.3 & \faCheck & &\tabularnewline
A.4 & \faCheck & & \faCheck\tabularnewline
A.5 & \faCheck & & \faCheck\tabularnewline
B.1 & \faCheck & \faCheck & \faCheck\tabularnewline
B.2 & \faCheck & \faCheck & \faCheck\tabularnewline
B.3 & & \faCheck &\tabularnewline
C.1 & \faCheck & & \faCheck\tabularnewline
C.2 & \faCheck & \faCheck &\tabularnewline
C.3 & \faCheck & \faCheck &\tabularnewline
\bottomrule
\end{longtable}

\subsection*{Kurslitteratur}

Obligatorisk litteratur:

\begin{itemize}
\tightlist
\item
  Garcia-Molina, H., Ullman, J. D. och Widom, J., \emph{Database
  Systems: The Complete Book}, andra utgåvan, Pearson, 2013. Antal
  sidor: 700 av 1140.
\end{itemize}

\subsection*{Övrigt}

Kursen genomförs på ett sätt sådant att både kvinnor och mäns kunskap och erfarenhet utvecklas och görs synlig.
\pagebreak
\section*{1DV004 - Objektorienterad programmering (5 hp)}

\begin{tabular}{ll}\emph{Huvudområde}: & Datavetenskap\tabularnewline\emph{Fördjupning}: & G1F\tabularnewline\end{tabular}

\subsection*{Förkunskaper}

\begin{itemize}
\tightlist
\item
  1DV001 - Programmering och datastrukturer
\item
  1DV002 - Introducerande projekt
\end{itemize}

\subsection*{Lärandemål}

Efter slutförd kurs skall studenten kunna:

\begin{enumerate}
\def\labelenumi{\Alph{enumi}.}
\tightlist
\item
  \emph{Kunskap och förståelse}

  \begin{enumerate}
  \def\labelenumii{\Alph{enumi}.\arabic{enumii}.}
  \tightlist
  \item
    Förklara grundläggande begrepp inom objekt-orienterad programmering
    såsom klasser, objekt, meddelanden, metoder, arv och polymorfism,
  \item
    förklara begreppen modularisering, abstraktion och inkapsling,
  \item
    förklara och motivera användningen av några vanliga designmönster,
  \item
    förklara de vanligaste konstruktionerna som används i UML:s klass-
    och sekvensdiagram, samt
  \item
    redogör för hur och när modellering med t.ex. UML används inom
    systemutveckling.
  \end{enumerate}
\item
  \emph{Färdighet och förmåga}

  \begin{enumerate}
  \def\labelenumii{\Alph{enumi}.\arabic{enumii}.}
  \tightlist
  \item
    Implementera program med flera klasser i programspråket Java,
  \item
    utföra enhetstester med hjälp av JUnit,
  \item
    skapa klass- och sekvensdiagram enligt UML och kunna implementera
    och testa ett Java-program utifrån UML-modellen,
  \item
    implementera (i Java) några vanligt förekommande designmönster, samt
  \item
    strukturera och genomföra en muntlig och skriftlig presentation av
    ett designprojekt.
  \end{enumerate}
\item
  \emph{Värderingsförmåga och förhållningssätt}

  \begin{enumerate}
  \def\labelenumii{\Alph{enumi}.\arabic{enumii}.}
  \tightlist
  \item
    Resonera om olika designalternativ för ett givet problem, samt
  \item
    göra ett motiverat val av designmönster i olika problemscenarior
  \end{enumerate}
\end{enumerate}

\subsection*{Kursinnehåll}

Det är en inledande kurs i objekt-orienterad analys, design och
programmering. Kursens första del lär ut programmeringsspråket Java och
viktiga begrepp inom objekt-orienterad programmering (t.ex. klasser,
objekt, arv, polymorfism, inkapsling). Denna del förutsätter viss
erfarenhet av programmering, t.ex. från 1DV001 - Programmering och
datastrukturer. I kursens andra del presenteras objekt-orienterad analys
och design, samt UML.

Följande moment behandlas:

\begin{itemize}
\tightlist
\item
  Introduktion till mjukvaruutvecklingsprocessen och hur modellering
  passar in i processen.
\item
  Grundläggande programkonstruktioner i Java så som typer, styrande
  satser, klasser, metoder, fält och exceptions.
\item
  Objektorienterade begrepp såsom abstraktion, modularisering,
  inkapsling, arv, interfaces och polymorfism.
\item
  Enhetstestning med JUnit.
\item
  Objektorienterad modellering med UML klass- och sekvensdiagram.
\item
  Några vanliga designmönster t.ex. Singleton, Iterator, Observer och
  Factory.
\end{itemize}

\subsection*{Undervisnings- och
arbetsformer}

Undervisningen sker i form av föreläsningar, lärarledda laborationer,
handledning i grupp och en avslutande muntlig presentation.
Programmeringsuppgifterna är individuella och projekt och presentationer
sker i par. Obligatorisk närvaro kan förekomma på vissa moment.

\subsection*{Examination}

Examinationen av kursen delas in i följande moment:

\begin{longtable}[]{@{}llcc@{}}
\toprule
\textsf{Kod} & \textsf{Benämning} & \textsf{Betyg} & \textsf{Poäng}\tabularnewline
\midrule
\endhead
\texttt{LAB1} & Programmeringsuppgifter & A-F & 3\tabularnewline
\texttt{PRJ1} & Projekt designmönster & A-F & 2\tabularnewline
\bottomrule
\end{longtable}

För godkänt betyg på kursen krävs minst betyg E på samtliga moment.
Slutbetyget bestäms från: \texttt{LAB1} (60~\%) och \texttt{PRJ1} (40~\%).

\subsection*{Måluppfyllelse}

Examinationsmomenten kopplas till lärandemålen enligt följande:

\begin{longtable}[]{@{}lcc@{}}
\toprule
\textsf{Lärandemål} & \texttt{LAB1} & \texttt{PRJ1}\tabularnewline
\midrule
\endhead
A.1 & \faCheck & \faCheck\tabularnewline
A.2 & \faCheck & \faCheck\tabularnewline
A.3 & \faCheck & \faCheck\tabularnewline
A.4 & & \faCheck\tabularnewline
A.5 & & \faCheck\tabularnewline
B.1 & \faCheck &\tabularnewline
B.2 & \faCheck &\tabularnewline
B.3 & \faCheck & \faCheck\tabularnewline
B.4 & \faCheck & \faCheck\tabularnewline
B.5 & & \faCheck\tabularnewline
C.1 & & \faCheck\tabularnewline
C.2 & & \faCheck\tabularnewline
\bottomrule
\end{longtable}

\subsection*{Kurslitteratur}

Obligatorisk litteratur:

\begin{itemize}
\tightlist
\item
  Jia, X., \emph{Object-oriented Software Development Using Java}, andra
  utgåvan, Addison Wesley, 2003. Antal sidor: 380 av 550.
\end{itemize}

\subsection*{Övrigt}

Kursen genomförs på ett sätt sådant att både kvinnor och mäns kunskap och erfarenhet utvecklas och görs synlig.
\pagebreak
\section*{1MA004 - Tillämpad sannolikhetslära och statistik (7,5 hp)}

\begin{tabular}{ll}\emph{Huvudområde}: & Matematik\tabularnewline\emph{Fördjupning}: & G1F\tabularnewline\end{tabular}

\subsection*{Förkunskaper}

\begin{itemize}
\tightlist
\item
  1MA002 - Linjär algebra
\item
  1MA003 - Envariabelanalys 1
\end{itemize}

\subsection*{Lärandemål}

Efter slutförd kurs skall studenten kunna:

\begin{enumerate}
\def\labelenumi{\Alph{enumi}.}
\tightlist
\item
  \emph{Kunskap och förståelse}

  \begin{enumerate}
  \def\labelenumii{\Alph{enumi}.\arabic{enumii}.}
  \tightlist
  \item
    Redogöra för sannolikhetslärans och statistikens grundläggande
    begrepp, modeller och beräkningsmetoder,
  \item
    redogöra för matematiska resonemang på ett strukturerat och logiskt
    sammanhängande sätt, samt
  \item
    redogöra för problem inom teknik och ekonomi som är lämpliga att
    behandla med grundläggande begrepp och metoder inom sannolikhetslära
    och statistik.
  \end{enumerate}
\item
  \emph{Färdighet och förmåga}

  \begin{enumerate}
  \def\labelenumii{\Alph{enumi}.\arabic{enumii}.}
  \tightlist
  \item
    Kombinera kunskaper om olika begrepp, operationer, modeller och
    metoder från sannolikhetslära och statistik i problemlösning,
  \item
    identifiera en lämplig slumpmodell för att beskriva och analysera
    observerade data och dra slutsatser om intressanta parametrar, samt
  \item
    visa förmåga att utnyttja programspråket Matlab i problemlösning.
  \end{enumerate}
\item
  \emph{Värderingsförmåga och förhållningssätt}

  \begin{enumerate}
  \def\labelenumii{\Alph{enumi}.\arabic{enumii}.}
  \tightlist
  \item
    Visa förmåga att bedöma rimlighet och uppskatta osäkerhet i resultat
    av beräkningar och skattningar.
  \end{enumerate}
\end{enumerate}

\subsection*{Kursinnehåll}

Det övergripande syftet med kursen är att ge en introduktion till
sannolikhetslära och statistisk metodik med tillämpningar inom
dataanalys, teknik och ekonomi m.fl. ämnen. Detta avser bl.a. teoretiskt
arbete med slumpmodeller och utnyttjande av observerade data för att dra
slutsatser.

\begin{itemize}
\tightlist
\item
  Händelser och sannolikheter i utfallsrum.
\item
  Betingade sannolikheter.
\item
  Korrelation.
\item
  Stokastiska variabler, sannolikhetsfördelningar, väntevärden och
  standardavvikelser. Bl.a. behandlas likformig-, exponential-, normal-,
  binomial-, Poission- och Weibullfördelning.
\item
  Tvådimensionella stokastiska variabler.
\item
  Beroende och oberoende variabler.
\item
  Centrala gränsvärdessatsen och stora talens lag.
\item
  Punktskattningar och deras osäkerhet, maximum-likelihood -skattningar,
  momentskattningar och minsta kvadratskattningar.
\item
  Chi-två- och t-fördelning.
\item
  Intervallskattning.
\item
  Hypotestestning.
\item
  Enkel linjär regressionsanalys.
\item
  Problemlösning inriktad på problemställningar från teknik,
  naturvetenskap, kvalitetskontroll, ekonomi m.fl. ämnen.
\item
  Problemlösning med hjälp av programvaran Matlab.
\end{itemize}

\subsection*{Undervisnings- och
arbetsformer}

Undervisningen sker i form av föreläsningar, lärarledda räkneövningar
och datorlaborationer. Inlämningsuppgifter sker i par. Obligatorisk
närvaro kan förekomma på vissa moment.

\subsection*{Examination}

Examinationen av kursen delas in i följande moment:

\begin{longtable}[]{@{}llcc@{}}
\toprule
\textsf{Kod} & \textsf{Benämning} & \textsf{Betyg} & \textsf{Poäng}\tabularnewline
\midrule
\endhead
\texttt{LAB1} & Programmeringsuppgifter i Matlab & A-F & 1,5\tabularnewline
\texttt{TEN1} & Skriftlig tentamen & A-F & 6\tabularnewline
\bottomrule
\end{longtable}

För godkänt betyg på kursen krävs minst betyg E på samtliga moment.
Slutbetyget bestäms från: \texttt{LAB1} (20~\%) och \texttt{TEN1} (80~\%).

\subsection*{Måluppfyllelse}

Examinationsmomenten kopplas till lärandemålen enligt följande:

\begin{longtable}[]{@{}lcc@{}}
\toprule
\textsf{Lärandemål} & \texttt{LAB1} & \texttt{TEN1}\tabularnewline
\midrule
\endhead
A.1 & \faCheck & \faCheck\tabularnewline
A.2 & \faCheck & \faCheck\tabularnewline
A.3 & \faCheck & \faCheck\tabularnewline
B.1 & \faCheck & \faCheck\tabularnewline
B.2 & \faCheck & \faCheck\tabularnewline
B.3 & \faCheck &\tabularnewline
C.1 & \faCheck & \faCheck\tabularnewline
\bottomrule
\end{longtable}

\subsection*{Kurslitteratur}

Obligatorisk litteratur:

\begin{itemize}
\tightlist
\item
  Walpole, R. E., Myers, R. H., Myers, S. L. och Ye, K.,
  \emph{Probability and Statistics for Engineers and Scientists}, nionde
  upplagan, Pearson, 2016. Antal sidor: 443 av 816.
\end{itemize}

\subsection*{Övrigt}

Kursen genomförs på ett sätt sådant att både kvinnor och mäns kunskap och erfarenhet utvecklas och görs synlig.
\pagebreak
\section*{1FY001 - Mekanik (7,5 hp)}

\begin{tabular}{ll}\emph{Huvudområde}: & Fysik\tabularnewline\emph{Fördjupning}: & G1F\tabularnewline\end{tabular}

\subsection*{Förkunskaper}

\begin{itemize}
\tightlist
\item
  1MA002 - Linjär algebra
\item
  1MA003 - Envariabelanalys 1
\end{itemize}

\subsection*{Lärandemål}

Efter slutförd kurs skall studenten kunna:

\begin{enumerate}
\def\labelenumi{\Alph{enumi}.}
\tightlist
\item
  \emph{Kunskap och förståelse}

  \begin{enumerate}
  \def\labelenumii{\Alph{enumi}.\arabic{enumii}.}
  \tightlist
  \item
    Beskriva övergripande den klassiska mekanikens förutsättningar,
  \item
    redogöra för den klassiska mekanikens huvudsakliga resultat och
    sammanhang,
  \item
    förklara hur den klassiska mekaniken kan tillämpas, samt
  \item
    beskriva tillämpning inom styrning av tekniska system.
  \end{enumerate}
\item
  \emph{Färdighet och förmåga}

  \begin{enumerate}
  \def\labelenumii{\Alph{enumi}.\arabic{enumii}.}
  \tightlist
  \item
    Tillämpa teoretiska samband för problemlösning,
  \item
    lösa problem genom tillämpning av matematiska metoder inom linjär
    algebra och differentialekvationer,
  \item
    analysera mekaniska system,
  \item
    planera och utföra mätningar av mekaniska storheter, samt
  \item
    uppskatta fel och precision i mätningar och felberäkna mätdata.
  \end{enumerate}
\item
  \emph{Värderingsförmåga och förhållningssätt}

  \begin{enumerate}
  \def\labelenumii{\Alph{enumi}.\arabic{enumii}.}
  \tightlist
  \item
    På en grundläggande nivå visa insikt om teorins begränsningar,
  \item
    på en grundläggande nivå bedöma tekniska tillämpningar med avseende
    på energiåtgång, stabilitet och kontrollerbarhet, samt
  \item
    på en grundläggande nivå bedöma tillförlitligheten hos vissa
    tekniska lösningar.
  \end{enumerate}
\end{enumerate}

\subsection*{Kursinnehåll}

Kursen behandlar klassisk mekanik enligt Newton baserad på
vektorbeskrivning i två och tre dimensioner. Kursens huvuddel behandlar
statik, dynamik, plan rotation samt enkel svängningsrörelse. Även
relativ rörelse ingår. Avslutningsvis ges en viss inblick i analytisk
mekanik och dess relevans inom automationsstyrning.

\begin{itemize}
\tightlist
\item
  Klassisk mekanik enligt Newton.

  \begin{itemize}
  \tightlist
  \item
    Statik: kraft- och momentjämvikt vid friläggning, masscentrum,
    friktion.
  \item
    Dynamik: kinematik och kinetik, olika koordinatsystem, arbete,
    energi och impuls, konserveringslagar.
  \item
    Svängningsrörelser: dämpad och odämpad svängning, resonans.
  \item
    Roterande objekt: tröghetsmoment och rörelsemängdsmoment.
  \item
    Relativ rörelse: fiktiva krafter, centrifugalkraft och
    Corioliseffekt.
  \end{itemize}
\item
  Analytisk mekanik.

  \begin{itemize}
  \tightlist
  \item
    Om den teoretiska bakgrunden: minsta verkans princip, Lagrange och
    Hamiltons formuleringar.
  \item
    Tillämpning inom automation: styrning av ett dynamiskt system.
  \end{itemize}
\end{itemize}

Kursen innehåller ett antal laborationer som förutom metoder för
behandling av mätdata och bedömning av mätnoggrannhet speciellt
behandlar roterande system, svängningsrörelse respektive styrning av en
dubbelpendel eller inverterad pendel. Dessutom ingår en datorsimulering.

\subsection*{Undervisnings- och
arbetsformer}

Undervisningen sker i form av föreläsningar, räkneövningar och
lärarledda laborationer. Huvuddelen av kursens innehåll presenteras och
förklaras under föreläsningarna. Under räkneövningarna kommer
studenterna sedan att få tillämpa teorin på tekniska problem. Metoder
för problemlösning kommer att demonstreras. Kursen omfattar även ett
antal laborationer, då både grundläggande fenomen och vissa tekniska
tillämpningar demonstreras. Laborationerna omfattar även mättekniska
övningar samt träning i rapportskrivning.

\subsection*{Examination}

Examinationen av kursen delas in i följande moment:

\begin{longtable}[]{@{}llcc@{}}
\toprule
\textsf{Kod} & \textsf{Benämning} & \textsf{Betyg} & \textsf{Poäng}\tabularnewline
\midrule
\endhead
\texttt{TEN1} & Skriftlig tentamen & A-F & 5,5\tabularnewline
\texttt{LAB1} & Laboration och rapport & G-U & 2\tabularnewline
\bottomrule
\end{longtable}

För godkänt betyg på kursen krävs betyg G på \texttt{LAB1} och minst betyg E på
\texttt{TEN1}. Slutbetyget bestäms från \texttt{TEN1}.

Inför varje laboration ska studenten ha gjort ett antal
förberedelseuppgifter. Under laborationen ska labbanteckningar föras.
Dessa och förberedelseuppgifterna ska studenten sedan utnyttja, för att
efter laborationen författa en labbrapport. Rapport (inklusive underlag)
lämnas sedan in för bedömning.

\subsection*{Måluppfyllelse}

Examinationsmoment kopplas till lärandemål enligt följande:

\begin{longtable}[]{@{}lcc@{}}
\toprule
\textsf{Lärandemål} & \texttt{TEN1} & \texttt{LAB1}\tabularnewline
\midrule
\endhead
A.1 & \faCheck &\tabularnewline
A.2 & \faCheck &\tabularnewline
A.3 & \faCheck & \faCheck\tabularnewline
A.4 & & \faCheck\tabularnewline
B.1 & \faCheck &\tabularnewline
B.2 & \faCheck & \faCheck\tabularnewline
B.3 & \faCheck & \faCheck\tabularnewline
B.4 & & \faCheck\tabularnewline
B.5 & & \faCheck\tabularnewline
C.1 & \faCheck &\tabularnewline
C.2 & \faCheck & \faCheck\tabularnewline
C.3 & \faCheck & \faCheck\tabularnewline
\bottomrule
\end{longtable}

\subsection*{Kurslitteratur}

Obligatorisk litteratur:

\begin{itemize}
\tightlist
\item
  McCall, M. W., \emph{Classical Mechanics}, Wiley, 2001. Antal sidor:
  236 av 268 sidor.
\end{itemize}

Referenslitteratur:

\begin{itemize}
\tightlist
\item
  Gamalath, K. A. I. L. W., \emph{Introduction to Analytical Mechanics},
  Alpha Science, 2011.
\end{itemize}

\subsection*{Övrigt}

Kursen genomförs på ett sätt sådant att både kvinnor och mäns kunskap och erfarenhet utvecklas och görs synlig.
\pagebreak
\section*{1DV005 - Jämnlöpande program (7,5 hp)}

\begin{tabular}{ll}\emph{Huvudområde}: & Datavetenskap\tabularnewline\emph{Fördjupning}: & G1F\tabularnewline\end{tabular}

\subsection*{Förkunskaper}

\begin{itemize}
\tightlist
\item
  1DV003 - Databaser och datamodellering
\item
  1DV004 - Objektorienterad programmering
\end{itemize}

\subsection*{Lärandemål}

Efter slutförd kurs skall studenten kunna:

\begin{enumerate}
\def\labelenumi{\Alph{enumi}.}
\tightlist
\item
  \emph{Kunskap och förståelse}

  \begin{enumerate}
  \def\labelenumii{\Alph{enumi}.\arabic{enumii}.}
  \tightlist
  \item
    Förklara problem med delade resurser såsom låsningar, svält och race
    conditions,
  \item
    förklara några vanliga metoder för att hantera låsningar (t.ex.
    semaforer) samt deras egenskaper och begränsningar,
  \item
    förklara olika konsistensmodeller,
  \item
    resonera om olika egenskaper hos jämnlöpande program såsom
    korrekthet och avslutning,
  \item
    förklara skillnaden mellan delat minne och meddelande, samt
  \item
    förklara skillnaden mellan parallellism, samtidighet och asynkron
    exekvering.
  \end{enumerate}
\item
  \emph{Färdighet och förmåga}

  \begin{enumerate}
  \def\labelenumii{\Alph{enumi}.\arabic{enumii}.}
  \tightlist
  \item
    Utveckla jämnlöpande program i Java för en dator med delat minne
    genom att använda trådar och låsning,
  \item
    implementera lås-fria datastrukturer i Java,
  \item
    bevisa att (enkla) jämnlöpande program är korrekta, samt
  \item
    analysera ett problem och implementera en lämplig jämnlöpande
    lösning som är korrekt synkroniserad.
  \end{enumerate}
\item
  \emph{Värderingsförmåga och förhållningssätt}

  \begin{enumerate}
  \def\labelenumii{\Alph{enumi}.\arabic{enumii}.}
  \tightlist
  \item
    Välja och föra ett resonemang om delat minne eller meddelanden är
    mest lämpligt vid ett givet problem.
  \end{enumerate}
\end{enumerate}

\subsection*{Kursinnehåll}

Kursen introducerar hur jämnlöpande programmering och de problem detta
medför, t.ex. låsningar och race-conditions. Olika sätt att hanteras
dessa problem, t.ex. låsningsalgoritmer och meddelande-hantering
diskuteras, samt vilka begränsningar dessa medför. Innehållet och
algoritmerna exemplifieras med hjälp av trådar i Java.

Följande moment behandlas:

\begin{itemize}
\tightlist
\item
  Processer och synkronisering.
\item
  Schemaläggning.
\item
  Delat minne och meddelanden.
\item
  Jämnlöpande programmering med trådar och delade variabler.
\item
  Kritiska sektioner.
\item
  Lås, barriärer, semaforer och monitors.
\item
  Distribuerade/jämnlöpande algoritmer.
\item
  Konsistensmodeller.
\item
  Jämnlöpande och låsfria datastrukturer.
\end{itemize}

\subsection*{Undervisnings- och
arbetsformer}

Undervisningen består av föreläsningar och lärarhandledda laborationer.
Programmeringsuppgifterna sker i par.

\subsection*{Examination}

Examinationen av kursen delas in i följande moment:

\begin{longtable}[]{@{}llcc@{}}
\toprule
\textsf{Kod} & \textsf{Benämning} & \textsf{Betyg} & \textsf{Poäng}\tabularnewline
\midrule
\endhead
\texttt{TEN1} & Skriftlig tentamen & A-F & 4\tabularnewline
\texttt{LAB1} & Programmeringsuppgifter & A-F & 3,5\tabularnewline
\bottomrule
\end{longtable}

För godkänt betyg på kursen krävs minst betyg E på samtliga moment.
Slutbetyget bestäms från: \texttt{TEN1} (60~\%) och \texttt{LAB1} (40~\%).

\subsection*{Måluppfyllelse}

Examinationsmomenten kopplas till lärandemålen enligt följande:

\begin{longtable}[]{@{}lcc@{}}
\toprule
\textsf{Lärandemål} & \texttt{TEN1} & \texttt{LAB1}\tabularnewline
\midrule
\endhead
A.1 & \faCheck &\tabularnewline
A.2 & \faCheck & \faCheck\tabularnewline
A.3 & \faCheck &\tabularnewline
A.4 & \faCheck & \faCheck\tabularnewline
A.5 & \faCheck &\tabularnewline
A.6 & \faCheck &\tabularnewline
B.1 & & \faCheck\tabularnewline
B.2 & & \faCheck\tabularnewline
B.3 & \faCheck &\tabularnewline
B.4 & \faCheck & \faCheck\tabularnewline
C.1 & & \faCheck\tabularnewline
\bottomrule
\end{longtable}

\subsection*{Kurslitteratur}

Obligatorisk litteratur:

\begin{itemize}
\tightlist
\item
  Herlihy, M. och Shavit N., \emph{The Art of Multiprocessor
  Programming}, Morgan Kaufmann, 2012. Antal sidor: 400 av 536.
\end{itemize}

\subsection*{Övrigt}

Kursen genomförs på ett sätt sådant att både kvinnor och mäns kunskap och erfarenhet utvecklas och görs synlig.
\pagebreak
\section*{1FY002 - Ellära och magnetism (7,5 hp)}

\begin{tabular}{ll}\emph{Huvudområde}: & Fysik\tabularnewline\emph{Fördjupning}: & G1F\tabularnewline\end{tabular}

\subsection*{Förkunskaper}

\begin{itemize}
\tightlist
\item
  1MA002 - Linjär algebra
\item
  1MA003 - Envariabelanalys 1
\item
  1FY001 - Mekanik
\end{itemize}

\subsection*{Lärandemål}

Efter slutförd kurs skall studenten kunna:

\begin{enumerate}
\def\labelenumi{\Alph{enumi}.}
\tightlist
\item
  \emph{Kunskap och förståelse}

  \begin{enumerate}
  \def\labelenumii{\Alph{enumi}.\arabic{enumii}.}
  \tightlist
  \item
    Beskriva övergripande elektromagnetismens förutsättningar,
  \item
    redogöra för elektromagnetismens huvudsakliga resultat och
    sammanhang,
  \item
    förklara hur elektromagnetismen kan tillämpas, samt
  \item
    beskriva tillämpning inom elektroniken.
  \end{enumerate}
\item
  \emph{Färdighet och förmåga}

  \begin{enumerate}
  \def\labelenumii{\Alph{enumi}.\arabic{enumii}.}
  \tightlist
  \item
    Tillämpa teoretiska samband för att lösa kretsproblem,
  \item
    analysera enkla kretsar, med t.ex. j-omega-metoden,
  \item
    planera och utföra elektriska mätningar och hantering av
    mätutrustning, samt
  \item
    beräkningar och mätfel och felpropagation.
  \end{enumerate}
\item
  \emph{Värderingsförmåga och förhållningssätt}

  \begin{enumerate}
  \def\labelenumii{\Alph{enumi}.\arabic{enumii}.}
  \tightlist
  \item
    På en grundläggande nivå bedöma tekniska tillämpningar med avseende
    på snabbhet och energiåtgång, samt
  \item
    på en grundläggande nivå bedöma risker och tillförlitlighet hos
    vissa elektrotekniska lösningar.
  \end{enumerate}
\end{enumerate}

\subsection*{Kursinnehåll}

Kursen innehåller två huvuddelar, dels behandlas grundläggande
elektromagnetism och, dels behandlas elektriska tillämpningar, särskilt
inom elektronik. Kursen innehåller ett antal laborationer som förutom
metoder för behandling av mätdata och bedömning av mätnoggrannhet också
omfattar vanliga mätinstrument, såsom multimetrar, oscilloskop och
funktionsgenerator. Speciellt behandlas olika typer av elektriska
kretstillämpningar inkluderande moment som mätning av
tillståndsvariabler och styrning av utrustning.

\begin{itemize}
\tightlist
\item
  Grundläggande elektromagnetism.

  \begin{itemize}
  \tightlist
  \item
    Elektrostatik: laddning och fält, potential och potentiell energi.
  \item
    Dielektriska material: polarisation, kapacitans, kondensatorer.
  \item
    Magnetism och induktion, induktans och spolar.
  \item
    Något om vågutbredning, vågledare och relativistisk dopplereffekt.
  \item
    Något om supraledning.
  \end{itemize}
\item
  Kretsteori.

  \begin{itemize}
  \tightlist
  \item
    Likström, tvåspolssatsen.
  \item
    Växelström: impedans, effekt, resonans.
  \item
    Elektrisk svängningskrets.
  \item
    Linjära kretsar: ingångs-/utgångsimpedans, låg-/högpassfilter.
  \item
    Halvledardioder, aktiva komponenter och sensorer.
  \end{itemize}
\item
  Digitala komponenter: grindar och grundläggande CMOS-kretsar.
\end{itemize}

\subsection*{Undervisnings- och
arbetsformer}

Undervisningen sker i form av föreläsningar, räkneövningar och
lärarledda laborationer. Huvuddelen av kursens innehåll presenteras och
förklaras under föreläsningarna. Under räkneövningarna kommer
studenterna sedan att få tillämpa teorin på tekniska problem. Metoder
för problemlösning kommer att demonstreras. Kursen omfattar även ett
antal laborationer, då både grundläggande fenomen och vissa tekniska
tillämpningar demonstreras. Laborationerna omfattar även mättekniska
övningar samt träning i rapportskrivning.

\subsection*{Examination}

Examinationen av kursen delas in i följande moment:

\begin{longtable}[]{@{}llcc@{}}
\toprule
\textsf{Kod} & \textsf{Benämning} & \textsf{Betyg} & \textsf{Poäng}\tabularnewline
\midrule
\endhead
\texttt{TEN1} & Skriftlig tentamen & A-F & 5,5\tabularnewline
\texttt{LAB1} & Laboration och rapport & G-U & 2\tabularnewline
\bottomrule
\end{longtable}

För godkänt betyg på kursen krävs betyg G på \texttt{LAB1} och minst betyg E på
\texttt{TEN1}. Slutbetyget bestäms från \texttt{TEN1}.

Inför varje laboration ska studenten ha gjort ett antal
förberedelseuppgifter. Under laborationen ska labbanteckningar föras.
Dessa och förberedelseuppgifterna ska studenten sedan utnyttja, för att
efter laborationen författa en labbrapport. Rapport (inklusive underlag)
lämnas sedan in för bedömning.

\subsection*{Måluppfyllelse}

Examinationsmoment kopplas till lärandemål enligt följande:

\begin{longtable}[]{@{}lcc@{}}
\toprule
\textsf{Lärandemål} & \texttt{TEN1} & \texttt{LAB1}\tabularnewline
\midrule
\endhead
A.1 & \faCheck & \faCheck\tabularnewline
A.2 & \faCheck &\tabularnewline
A.3 & \faCheck & \faCheck\tabularnewline
A.4 & & \faCheck\tabularnewline
B.1 & \faCheck & \faCheck\tabularnewline
B.2 & \faCheck &\tabularnewline
B.3 & & \faCheck\tabularnewline
B.4 & & \faCheck\tabularnewline
C.1 & \faCheck & \faCheck\tabularnewline
C.2 & & \faCheck\tabularnewline
\bottomrule
\end{longtable}

\subsection*{Kurslitteratur}

Obligatorisk litteratur:

\begin{itemize}
\tightlist
\item
  Ågren, O., \emph{Elektromagnetism}, Studentlitteratur, 2014. Antal
  sidor: 100 av 174 sidor.
\end{itemize}

Referenslitteratur:

\begin{itemize}
\tightlist
\item
  Hemert, L-H., \emph{Digitala kretsar}, tredje upplagan,
  Studentlitteratur, 2001.
\end{itemize}

\subsection*{Övrigt}

Kursen genomförs på ett sätt sådant att både kvinnor och mäns kunskap och erfarenhet utvecklas och görs synlig.
\pagebreak
\section*{1ZT001 - Teknisk kommunikation (5 hp)}

\begin{tabular}{ll}
\emph{Fördjupning}: & G1N\tabularnewline
\bottomrule
\end{tabular}

\subsection*{Förkunskaper}

Grundläggande behörighet samt Matematik D eller Matematik 4
(områdesbehörighet 9/A9).

\subsection*{Lärandemål}

Efter slutförd kurs skall studenten kunna:

\begin{enumerate}
\def\labelenumi{\Alph{enumi}.}
\tightlist
\item
  \emph{Kunskap och förståelse}

  \begin{enumerate}
  \def\labelenumii{\arabic{enumii}.}
  \tightlist
  \item
    Grundläggande begrepp och perspektiv rörande vetenskapligt
    arbetssätt,
  \item
    grunderna i akademiskt skrivande, samt
  \item
    grunderna i presentationsteknik och hur kroppsspråk samt röstteknik
    påverkar kontakten med publiken vid en muntlig presentation.
  \end{enumerate}
\item
  \emph{Färdighet och förmåga}

  \begin{enumerate}
  \def\labelenumii{\arabic{enumii}.}
  \tightlist
  \item
    Självständigt söka information i olika typer av
    informationsresurser,
  \item
    läsa, tillgodogöra, värdera och använda innehållet i tekniska
    artiklar inom ämnet,
  \item
    skriva mottagar- och genreanpassade texter,
  \item
    planera och genomföra muntliga strukturerade och situationsanpassade
    presentationer av olika slag inför grupp, både informerande och
    argumenterande presentationer,
  \item
    ge konstruktiv respons på andras muntliga och skriftliga
    presentationer,
  \item
    diskussionsteknik med fokus på kommunikationssituationer, samt
  \item
    genomföra uppgifter inom givna ramar och samarbeta med grupper från
    andra ämnen.
  \end{enumerate}
\item
  \emph{Värderingsförmåga och förhållningssätt}

  \begin{enumerate}
  \def\labelenumii{\arabic{enumii}.}
  \tightlist
  \item
    Återge och använda andras material och data på ett korrekt vis.
  \end{enumerate}
\end{enumerate}

\subsection*{Kursinnehåll}

I denna kurs kommer en kunskapsmässig grund till färdigheter inför
skriftliga och muntliga kommunikationssituationer anpassat till dig som
ingenjör ges. Kursen innehåller följande:

\begin{itemize}
\tightlist
\item
  Vetenskapligt arbetssätt.
\item
  Akademiska och populärvetenskapliga texttyper: utformning, analys och
  användande.
\item
  Akademisk skrivande: Grundläggande normer i svenska skrivregler,
  Akademiska texters struktur, Vetenskaplig stil.
\item
  Disposition och formalia rörande hantering av disposition,
  Referenshantering, Utformning och användning av figurer och tabeller.
\item
  Moderna verktyg för skapande av texter och presentationer.
\item
  Presentationsteknik: Retorik, kroppsspråk och röstteknik, Beskrivande
  och argumenterande presentationer, Kommunikationssituationer i
  yrkessammanhang.
\item
  Avancerad informationssökning.
\item
  Etik, upphovsrätt och plagiat.
\item
  Samarbete och gruppdynamik.
\item
  Tidsplanering.
\end{itemize}

\subsection*{Undervisnings- och
arbetsformer}

Undervisningen bedrivs i form av föreläsningar, praktiska övningar,
minikonferenser, möten och seminarier. Studenterna väljer under kursen
ett ämnesexempel utifrån studentens studieinriktning som man arbetar med
under kursens gång, vid flera moment mixas studenter från olika
studieinriktningar. Ämnesexemplet studeras och presenteras utifrån de
olika aspekter som tas upp under kursen.

I de praktiska delarna ingår att förbereda, muntligt framföra och
analysera olika typer av anföranden. I kursen ingår/görs
videoinspelning.

Genomförande av vissa praktiska övningar och seminarier samt
presentationstillfällen är obligatoriska.

\subsection*{Examination}

Examinationen av kursen delas in i följande moment:

\begin{longtable}[]{@{}llcc@{}}
\toprule
\textsf{Kod} & \textsf{Benämning} & \textsf{Betyg} & \textsf{Poäng}\tabularnewline
\midrule
\endhead
UPG1 & Skriftlig rapport & A-F & 2\tabularnewline
PRS1 & Muntlig presentation & G-U & 1\tabularnewline
PRS2 & Poster och opposition & G-U & 2\tabularnewline
\bottomrule
\end{longtable}

För godkänt betyg på kursen krävs betyg G på PRS1 och PRS2 samt minst
betyg E på UPG1. Slutbetyget bestäms från UPG1.

\subsection*{Måluppfyllelse}

Examinationsmomenten kopplas till lärandemålen enligt följande:

\begin{longtable}[]{@{}lccc@{}}
\toprule
Lärandemål & UPG1 & PRS1 & PRS2\tabularnewline
\midrule
\endhead
A.1 & \textbf{X} & \textbf{X} & \textbf{X}\tabularnewline
A.2 & \textbf{X} & &\tabularnewline
A.3 & & \textbf{X} &\tabularnewline
B.1 & \textbf{X} & &\tabularnewline
B.2 & \textbf{X} & &\tabularnewline
B.3 & \textbf{X} & & \textbf{X}\tabularnewline
B.4 & & \textbf{X} &\tabularnewline
B.5 & & & \textbf{X}\tabularnewline
B.6 & & \textbf{X} & \textbf{X}\tabularnewline
B.7 & \textbf{X} & \textbf{X} & \textbf{X}\tabularnewline
C.1 & \textbf{X} & & \textbf{X}\tabularnewline
\bottomrule
\end{longtable}

\subsection*{Kurslitteratur}

Obligatorisk litteratur:

\begin{itemize}
\tightlist
\item
  Eriksson, L-T. och Wiedersheim-Paul, F., \emph{Att utreda, forska och
  rapportera}, Liber, 2014. Antal sidor: 216 av 216 sidor.
\item
  Walla, E., \emph{Presentationsteknik och retorik}, Studentlitteratur,
  2011. Antal sidor: 174 av 174 sidor.
\item
  Schött, K., Hållsten, S., Moberg, B. och Strand, H., \emph{Studentens
  skrivhandbok}, Liber. Antal sidor: 192 av 192 sidor
\item
  Aktuella artiklar ur vetenskapliga tidskrifter, rapporter,
  branschtidningar och böcker tillgängliga via Internet, hänvisning ges
  på kursens webstudieplats.
\end{itemize}

Referenslitteratur:

\begin{itemize}
\tightlist
\item
  Blomström, V. och Persson, C., \emph{Muntlig interaktion},
  Studentlitteratur, 2014.
\item
  Rienecker, L., Stray Jørgense, P. och Gandil, P. \emph{Skriv en
  artikel}, Liber, 2009.
\item
  Blomkvist, P. och Hallin, A., \emph{Metod för teknologer
  --Examensarbete enligt 4-fasmodellen}, Studentlitteratur, 2014.
\item
  Bjork, L. A. och Raisanen, C., \emph{Academic Writing - a university
  writing course}, Studentlitteratur, 2003.
\end{itemize}

\subsection*{Övrigt}

Kursen genomförs på ett sätt sådant att både kvinnor och mäns kunskap och erfarenhet utvecklas och görs synlig.
\pagebreak
\section*{1DV006 - Algoritmer (5 hp)}

\begin{tabular}{ll}\emph{Huvudområde}: & Datavetenskap\tabularnewline\emph{Fördjupning}: & G1F\tabularnewline\end{tabular}

\subsection*{Förkunskaper}

\begin{itemize}
\tightlist
\item
  1DV001 - Programmering och datastrukturer
\item
  1DV003 - Databaser och datamodellering
\item
  1DV004 - Objektorienterad programmering
\item
  1MA001 - Diskret matematik
\item
  1MA003 - Envariabelanalys 1
\end{itemize}

\subsection*{Lärandemål}

Efter slutförd kurs skall studenten kunna:

\begin{enumerate}
\def\labelenumi{\Alph{enumi}.}
\tightlist
\item
  \emph{Kunskap och förståelse}

  \begin{enumerate}
  \def\labelenumii{\Alph{enumi}.\arabic{enumii}.}
  \tightlist
  \item
    Beskriva algoritmer och förklara dess egenskaper (t.ex.
    begränsningar och komplexitet),
  \item
    förklara olika komplexitetsklasser (t.ex. P och NP) och känna igen
    problem som tillhör dem, samt
  \item
    förklara begreppet ohanterbara problem (intractability) och kunna
    identifiera denna typ av problem.
  \end{enumerate}
\item
  \emph{Färdighet och förmåga}

  \begin{enumerate}
  \def\labelenumii{\Alph{enumi}.\arabic{enumii}.}
  \tightlist
  \item
    Bestämma tids- och minneskomplexiteten hos algoritmer och
    datastrukturer,
  \item
    implementera algoritmer och datastrukturer,
  \item
    utföra experiment som validerar en algoritms förväntade egenskaper,
  \item
    utföra reduktion av algoritmiska problem (i vissa enkla fall), samt
  \item
    tillämpa vanliga algoritmstrategier så som söndra-och-härska,
    dynamisk programmering och giriga algoritmer.
  \end{enumerate}
\item
  \emph{Värderingsförmåga och förhållningssätt}

  \begin{enumerate}
  \def\labelenumii{\Alph{enumi}.\arabic{enumii}.}
  \tightlist
  \item
    Resonera om vilken algoritm eller datastruktur som är mest lämpad i
    en given situation, samt
  \item
    resonera om en algoritm är korrekt och kunna identifiera vanligt
    förekommande fallgropar.
  \end{enumerate}
\end{enumerate}

\subsection*{Kursinnehåll}

Kursen ger en fördjupad bild av algoritmer, datastrukturer och
algoritmanalys.

Följande moment behandlas:

\begin{itemize}
\tightlist
\item
  Introduktion till beräkning, beräkningsbarhet och turing-maskiner.
\item
  Asymptotisk analys (big-O, Theta, Omega notation).
\item
  Algoritmanalys (värsta fall, medel, bästa fall).
\item
  Introduktion till skillnad i prestanda och kostnader av att välja
  sämre algoritmer.
\item
  Abstrakta datatyper: associativ lista (dictionary), prioritetskö,
  union-find.
\item
  Sorteringsalgoritmer.
\item
  Grafalgoritmer.
\item
  Strategier för algoritmdesign (söndra-och-härska, dynamisk
  programmering och giriga algoritmer).
\item
  Introduktion till komplexitetsteori: ohanterbara problem
  (intractability), komplexitetsklasser, fullständighet (completeness),
  reduktion.
\item
  Vad medför det att vissa problem inte kan lösas effektivt eller alls?
\end{itemize}

\subsection*{Undervisnings- och
arbetsformer}

Undervisningen sker i form av föreläsningar och lärarledda laborationer.
Laborationer är individuella.

\subsection*{Examination}

Examinationen av kursen delas in i följande moment:

\begin{longtable}[]{@{}llcc@{}}
\toprule
\textsf{Kod} & \textsf{Benämning} & \textsf{Betyg} & \textsf{Poäng}\tabularnewline
\midrule
\endhead
\texttt{MUN1} & Muntlig tentamen & A-F & 3\tabularnewline
\texttt{LAB1} & Programmeringsuppgifter & A-F & 2\tabularnewline
\bottomrule
\end{longtable}

För godkänt betyg på kursen krävs minst betyg E på samtliga moment.
Slutbetyget bestäms från: \texttt{MUN1} (60~\%) och \texttt{LAB1} (40~\%).

\subsection*{Måluppfyllelse}

Examinationsmomenten kopplas till lärandemålen enligt följande:

\begin{longtable}[]{@{}lcc@{}}
\toprule
\textsf{Lärandemål} & \texttt{MUN1} & \texttt{LAB1}\tabularnewline
\midrule
\endhead
A.1 & \faCheck &\tabularnewline
A.2 & \faCheck &\tabularnewline
A.3 & \faCheck &\tabularnewline
B.1 & \faCheck & \faCheck\tabularnewline
B.2 & & \faCheck\tabularnewline
B.3 & & \faCheck\tabularnewline
B.4 & \faCheck &\tabularnewline
B.5 & \faCheck & \faCheck\tabularnewline
C.1 & \faCheck &\tabularnewline
C.2 & \faCheck &\tabularnewline
\bottomrule
\end{longtable}

\subsection*{Kurslitteratur}

Obligatorisk litteratur:

\begin{itemize}
\tightlist
\item
  Weiss, M. A., \emph{Data Structures and Algorithm Analysis in Java},
  tredje utgåvan, Pearson Education, 2012. Antal sidor: 425 av 588.
\end{itemize}

\subsection*{Övrigt}

Kursen genomförs på ett sätt sådant att både kvinnor och mäns kunskap och erfarenhet utvecklas och görs synlig.
\pagebreak
\section*{1DV007 - Mjukvaruutvecklingsprojekt (10 hp)}

\begin{tabular}{ll}\emph{Huvudområde}: & Datavetenskap\tabularnewline\emph{Fördjupning}: & G1F\tabularnewline\end{tabular}

\subsection*{Förkunskaper}

\begin{itemize}
\tightlist
\item
  1DV002 - Introducerande projekt
\item
  1DV004 - Objektorienterad programmering
\end{itemize}

\subsection*{Lärandemål}

Efter slutförd kurs skall studenten kunna:

\begin{enumerate}
\def\labelenumi{\Alph{enumi}.}
\tightlist
\item
  \emph{Kunskap och förståelse}

  \begin{enumerate}
  \def\labelenumii{\Alph{enumi}.\arabic{enumii}.}
  \tightlist
  \item
    Känna till och räkna upp de faktorer som gör mjukvarudesign och
    utveckling komplext,
  \item
    beskriva grundläggande begrepp inom mjukvaruutveckling. Dess olika
    faser, roller, intressenter och metoder,
  \item
    beskriva principer för agila metoder för mjukvaruutveckling, deras
    syfte, och hur de skiljer sig från andra metoder, samt
  \item
    beskriva hur utvecklingsprojekt fungerar i industrin.
  \end{enumerate}
\item
  \emph{Färdighet och förmåga}

  \begin{enumerate}
  \def\labelenumii{\Alph{enumi}.\arabic{enumii}.}
  \tightlist
  \item
    Samla in och prioritera krav samt skapa en lösning till ett
    realistiskt problem,
  \item
    planera och genomföra ett agilt mjukvaruutvecklingsprojekt,
  \item
    självständigt lösa uppgifter med hjälp av för studenten nya verktyg
    och API:er, samt
  \item
    förmåga att arbeta i grupp och hantera de problem som kan uppstå,
    t.ex. konflikter.
  \end{enumerate}
\item
  \emph{Värderingsförmåga och förhållningssätt}

  \begin{enumerate}
  \def\labelenumii{\Alph{enumi}.\arabic{enumii}.}
  \tightlist
  \item
    Reflektera över val av mjukvaruutvecklingsmetoder för projektet,
  \item
    jämföra och resonera kring vilken metodik som är mest lämplig för
    ett givet projekt, samt
  \item
    reflektera över hur effektiva grupper skapas och vad som krävs av
    individ och övriga gruppmedlemmar.
  \end{enumerate}
\end{enumerate}

\subsection*{Kursinnehåll}

Kursen ger en introduktion till software engineering och agila metoder.
Yrkesrollen och vad den innebär diskuteras tillsammans med
mjukvaruutvecklingsindustrin och hur den fungerar.

\begin{itemize}
\tightlist
\item
  Introduktion till mjukvaruutvecklingsprocessen.
\item
  Kravinhämtning, kravbeskrivning och prioritering.
\item
  Metoder och principer för projektplanering.
\item
  Processstrategier och koncept, såsom risk, iterativ och inkrementell.
\item
  Mjukvarutestning, enhets- och intergrationstestning.
\item
  Agil projektledning: användarberättelser, planering, uppskattning.
\item
  Arbete i grupp, kommunikationsstrategier, ansvar och skyldigheter.
\item
  Mjukvaruindustrin och hur utvecklingsprojekt fungerar.
\item
  Organisation och gruppsammansättning i industrin.
\item
  Yrkesrollen mjukvaruutvecklare.
\item
  Verktyg för mjukvaruutveckling och projektledning.
\end{itemize}

\subsection*{Undervisnings- och
arbetsformer}

Undervisningen sker i form av föreläsningar och gästföreläsningar, samt
handledning i projektgrupper. Projektet sker i grupper om 6-7 studenter
enligt en given metodik och med fasta roller. Obligatorisk närvaro kan
förekomma på vissa moment.

\subsection*{Examination}

Examinationen av kursen delas in i följande moment:

\begin{longtable}[]{@{}llcc@{}}
\toprule
\textsf{Kod} & \textsf{Benämning} & \textsf{Betyg} & \textsf{Poäng}\tabularnewline
\midrule
\endhead
\texttt{UPG1} & Reflektionsrapport över projekt & A-F & 1,5\tabularnewline
\texttt{UPG2} & Reflektionsrapport över yrkesrollen & A-F & 1,5\tabularnewline
\texttt{PRJ1} & Projektarbete (inkl. leverabler) & A-F & 7\tabularnewline
\bottomrule
\end{longtable}

För godkänt betyg på kursen krävs minst betyg E på samtliga moment.
Slutbetyget bestäms från: \texttt{UPG1} (25~\%), \texttt{UPG2} (25~\%) och \texttt{PRJ1} (50~\%).

\subsection*{Måluppfyllelse}

Examinationsmomenten kopplas till lärandemålen enligt följande:

\begin{longtable}[]{@{}lccc@{}}
\toprule
\textsf{Lärandemål} & \texttt{UPG1} & \texttt{UPG2} & \texttt{PRJ1}\tabularnewline
\midrule
\endhead
A.1 & \faCheck & \faCheck & \faCheck\tabularnewline
A.2 & \faCheck & \faCheck & \faCheck\tabularnewline
A.3 & \faCheck & & \faCheck\tabularnewline
A.4 & & \faCheck &\tabularnewline
B.1 & & & \faCheck\tabularnewline
B.2 & & & \faCheck\tabularnewline
B.3 & & \faCheck & \faCheck\tabularnewline
B.4 & \faCheck & & \faCheck\tabularnewline
C.1 & \faCheck & & \faCheck\tabularnewline
C.2 & \faCheck & & \faCheck\tabularnewline
C.3 & \faCheck & & \faCheck\tabularnewline
\bottomrule
\end{longtable}

\subsection*{Kurslitteratur}

Obligatorisk litteratur:

\begin{itemize}
\tightlist
\item
  Sommerville, I., \emph{Software Engineering}, tionde utgåvan,
  Addison-Wesley, 2015. Antal sidor: 420 av 757.
\item
  Schwaber, K., \emph{Agile Project Management with Scrum (Developer
  Best Practices)}, Microsoft Press, 2004. Antal sidor: 155 av 155.
\end{itemize}

\subsection*{Övrigt}

Kursen genomförs på ett sätt sådant att både kvinnor och mäns kunskap och erfarenhet utvecklas och görs synlig.
\pagebreak
\section*{1ZT002 - Hållbar utveckling (5 hp)}

\begin{tabular}{ll}
\emph{Fördjupning}: & G1N\tabularnewline
\end{tabular}

\subsection*{Förkunskaper}

Grundläggande behörighet samt Matematik D eller Matematik 4
(områdesbehörighet 9/A9).

\subsection*{Lärandemål}

Efter slutförd kurs skall studenten kunna:

\begin{enumerate}
\def\labelenumi{\Alph{enumi}.}
\tightlist
\item
  \emph{Kunskap och förståelse}

  \begin{enumerate}
  \def\labelenumii{\Alph{enumi}.\arabic{enumii}.}
  \tightlist
  \item
    Redogöra för innebörden i begreppet hållbar utveckling ur såväl
    ekologiska, sociala och ekonomiska aspekter, samt ur ett globalt,
    lokalt respektive individuellt perspektiv,
  \item
    räkna upp några exempel på regler kring arbetsmiljö kopplat till din
    framtida ingenjörsyrkesutövning, samt
  \item
    redogöra för någon metod som används att kvantifiera produkters och
    tjänsters miljöpåverkan.
  \end{enumerate}
\item
  \emph{Färdighet och förmåga}

  \begin{enumerate}
  \def\labelenumii{\Alph{enumi}.\arabic{enumii}.}
  \tightlist
  \item
    Att föreslå och motivera strategier och åtgärder, nationellt och
    internationellt, för olika möjligheter att analysera och reducera
    miljöproblem utifrån ett systemanalytiskt perspektiv,
  \item
    föra ett resonemang runt etiska aspekter, riskhantering och
    ansvarsfrågor, samt
  \item
    analysera och diskutera olika hållbarhetsfrågor och inom givna
    ramar.
  \end{enumerate}
\item
  \emph{Värderingsförmåga och förhållningssätt}

  \begin{enumerate}
  \def\labelenumii{\Alph{enumi}.\arabic{enumii}.}
  \tightlist
  \item
    Föra ett kritiskt resonemang och reflektera över ingenjörens roll i
    ett hållbart samhälle och sitt eget ansvar inom sin egen yrkesroll
    för en hållbar utveckling, med hänsyn till ekologiska, sociala och
    ekonomiska aspekter.
  \end{enumerate}
\end{enumerate}

\subsection*{Kursinnehåll}

I denna kurs kommer en kunskapsmässig grund till begreppet hållbar
utveckling ur ett perspektiv anpassat till dig som ingenjör ges.
Följande moment behandlas:

\begin{itemize}
\tightlist
\item
  Innebörden av hållbar utveckling: ekologiska, sociala och ekonomiska
  förutsättningar, aspekter, definitioner, begrepp,
  kvantifieringsmetoder och praktisk tillämpning.
\item
  Teknikens roll och strategier för att uppnå en hållbar utveckling.
\item
  Globaliseringens påverkan på Hållbar utveckling.
\item
  Globala och nationella miljöhot.
\item
  Hot mot och åtgärder för hållbar utveckling kopplat till ämnesexempel.
\item
  Teknikens roll och påverkan av den egna livsstilen.
\item
  Arbetsmiljölagar.
\item
  Riskbedömning.
\item
  Etik.
\item
  Agenda 2030, roadmap 2050 och de 17 globala målen för hållbar
  utveckling,
\item
  Konsumtionsmönster, ekologiska fotavtryck, resursutnyttjande,
  transporter och avfall.
\item
  Ekonomiska och juridiska styrmedel och verktyg (Sveriges miljömål,
  miljöbalken, utsläppsrätter, ISO 14 000, EMAS mm).
\end{itemize}

\subsection*{Undervisnings- och
arbetsformer}

Undervisningen bedrivs i form av föreläsningar, projektarbete och
seminarier. Studenterna väljer under kursen ett teknikområde utifrån
studentens eget ämne som man arbetar med under kursens gång i olika
form, ämnet skall belysas utifrån de olika aspekter som tas upp under
kursen.

Genomförande av vissa seminarier är obligatoriskt.

\subsection*{Examination}

Examinationen av kursen delas in i följande moment:

\begin{longtable}[]{@{}llcc@{}}
\toprule
\textsf{Kod} & \textsf{Benämning} & \textsf{Betyg} & \textsf{Poäng}\tabularnewline
\midrule
\endhead
\texttt{PRJ1} & Projektarbete, reflektion, opponering & G-U & 3\tabularnewline
\texttt{TEN1} & Skriftlig tentamen & A-F & 2\tabularnewline
\bottomrule
\end{longtable}

För godkänt betyg på kursen krävs betyg G på \texttt{PRJ1} samt minst betyg E på
\texttt{TEN1}. Slutbetyget bestäms från \texttt{TEN1}.

\subsection*{Måluppfyllelse}

Examinationsmomenten kopplas till lärandemålen enligt följande:

\begin{longtable}[]{@{}lcc@{}}
\toprule
\textsf{Lärandemål} & \texttt{PRJ1} & \texttt{TEN1}\tabularnewline
\midrule
\endhead
A.1 & \faCheck & \faCheck\tabularnewline
A.2 & \faCheck & \faCheck\tabularnewline
A.3 & \faCheck & \faCheck\tabularnewline
B.1 & \faCheck &\tabularnewline
B.2 & \faCheck &\tabularnewline
B.3 & \faCheck & \faCheck\tabularnewline
C.1 & \faCheck &\tabularnewline
\bottomrule
\end{longtable}

\subsection*{Kurslitteratur}

Obligatorisk litteratur:

\begin{itemize}
\tightlist
\item
  Gröndahl, F. och Svanström, M., \emph{Hållbar Utveckling - en
  introduktion för ingenjörer och andra problemlösare}, Liber, 2011.
  Antal sidor: 299 av 299.
\item
  Globala målen och Agenda 2030,
  http://www.regeringen.se/regeringens-politik/globala-malen-och-agenda-2030/
\item
  Aktuella artiklar ur vetenskapliga tidskrifter, rapporter,
  branschtidningar och böcker tillgängliga via Internet, hänvisning ges
  på kursens webbplats.
\end{itemize}

\subsection*{Övrigt}

Kursen genomförs på ett sätt sådant att både kvinnor och mäns kunskap och erfarenhet utvecklas och görs synlig.
\pagebreak
\section*{1MA005 - Envariabelanalys 2 (5 hp)}

\begin{tabular}{ll}\emph{Huvudområde}: & Matematik\tabularnewline\emph{Fördjupning}: & G1F\tabularnewline\end{tabular}

\subsection*{Förkunskaper}

\begin{itemize}
\tightlist
\item
  1MA003 - Envariabelanalys 1
\end{itemize}

\subsection*{Lärandemål}

Efter slutförd kurs skall studenten kunna:

\begin{enumerate}
\def\labelenumi{\Alph{enumi}.}
\tightlist
\item
  \emph{Kunskap och förståelse}

  \begin{enumerate}
  \def\labelenumii{\Alph{enumi}.\arabic{enumii}.}
  \tightlist
  \item
    Förklara analytiska begrepp såsom följder, serier och konvergens,
    samt
  \item
    redogöra för definitioner samt formulera och bevisa teorem som är
    centrala i analys, såsom satser om talföljder och serier, samt
    Taylors sats.
  \end{enumerate}
\item
  \emph{Färdighet och förmåga}

  \begin{enumerate}
  \def\labelenumii{\Alph{enumi}.\arabic{enumii}.}
  \tightlist
  \item
    Använda avancerad differential- och integralkalkyl i en variabel,
  \item
    hantera serier och summor samt avgöra konvergens,
  \item
    lösa problem, utföra beräkningar och föra resonemang i avancerad
    analys,
  \item
    skriftligt presentera beräkningar och resonemang så att de kan
    följas av den som inte är insatt i problemet,
  \item
    tillämpa differential- och integralkalkyl, och serier på tekniska,
    fysikaliska och datavetenskapliga problem, samt
  \item
    visualisera resultat såsom tillämpningar av integraler (t ex
    rotationsvolym och -yta)
  \end{enumerate}
\item
  \emph{Värderingsförmåga och förhållningssätt}

  \begin{enumerate}
  \def\labelenumii{\Alph{enumi}.\arabic{enumii}.}
  \tightlist
  \item
    Diskutera relevans, räckvidd och noggrannhet av matematiska modeller
    såsom potensserier
  \end{enumerate}
\end{enumerate}

\subsection*{Kursinnehåll}

Kursen fortsätter introduktionen av analys. Följande moment behandlas:

\begin{itemize}
\tightlist
\item
  Integration av trigonometriska och irrationella funktioner.
\item
  Generaliserade integraler, konvergens, jämförelsekriteriet.
\item
  Tillämpningar av integralkalkyl.
\item
  Talföljder: definition, egenskaper, övre och undre gräns, gränsvärde,
  räkneregler för gränsvärden. Konvergens av monotona talföljder.
\item
  Kontinuerliga funktioner: satsen om mellanliggande värde, Weierstrass
  sats om maximalt och minimalt värde.
\item
  Serier: konvergens, egenskaper. Positiva serier: jämförelsekriterier,
  kvot- och rotkriterier, integralkriteriet. Alternerande serier,
  betingad konvergens och absolutkonvergens.
\item
  Potensserier, konvergensradie, konvergensintervall.
\item
  Maclaurins och Taylors formler med restterm, Taylors formel för de
  elementära funktionerna, beräkningar av gränsvärde och integraler med
  Taylors formel.
\end{itemize}

\subsection*{Undervisnings- och
arbetsformer}

Föreläsningar och lärarledda räkneövningar.

\subsection*{Examination}

Examinationen av kursen delas in i följande moment:

\begin{longtable}[]{@{}llcc@{}}
\toprule
\textsf{Kod} & \textsf{Benämning} & \textsf{Betyg} & \textsf{Poäng}\tabularnewline
\midrule
\endhead
\texttt{TEN1} & Tentamen: Problemlösning & A-F & 4\tabularnewline
\texttt{TEN2} & Tentamen: Teori & G-U & 1\tabularnewline
\bottomrule
\end{longtable}

För godkänt betyg på kursen krävs minst betyg E på \texttt{TEN1} samt betyg G på
\texttt{TEN2}. Slutbetyget bestäms från \texttt{TEN1}.

\subsection*{Måluppfyllelse}

Examinationsmomenten kopplas till lärandemålen enligt följande:

\begin{longtable}[]{@{}lcc@{}}
\toprule
\textsf{Lärandemål} & \texttt{TEN1} & \texttt{TEN2}\tabularnewline
\midrule
\endhead
A.1 & & \faCheck\tabularnewline
A.2 & & \faCheck\tabularnewline
B.1 & \faCheck &\tabularnewline
B.2 & \faCheck &\tabularnewline
B.3 & \faCheck & \faCheck\tabularnewline
B.4 & \faCheck &\tabularnewline
B.5 & \faCheck &\tabularnewline
B.6 & \faCheck &\tabularnewline
C.1 & \faCheck &\tabularnewline
\bottomrule
\end{longtable}

\subsection*{Kurslitteratur}

Obligatorisk litteratur:

\begin{itemize}
\tightlist
\item
  Månsson J, Nordbeck P, \emph{Endimensionell analys},
  Studentlitteratur, 2011. Antal sidor: 200 av 400.
\item
  Månsson J., Nordbeck P. \emph{Övningar i endimensionell analys},
  Studentlitteratur, 2011. Antal sidor: 100 av 206.
\end{itemize}

\subsection*{Övrigt}

Kursen genomförs på ett sätt sådant att både kvinnor och mäns kunskap och erfarenhet utvecklas och görs synlig.
\pagebreak
\section*{1MA006 - Flervariabelanalys (7,5 hp)}

\begin{tabular}{ll}\emph{Huvudområde}: & Matematik\tabularnewline\emph{Fördjupning}: & G1F\tabularnewline\end{tabular}

\subsection*{Förkunskaper}

\begin{itemize}
\tightlist
\item
  1MA002 - Linjär algebra
\item
  1MA003 - Envariabelanalys 1
\item
  1MA005 - Envariabelanalys 2
\end{itemize}

\subsection*{Lärandemål}

Efter slutförd kurs skall studenten kunna:

\begin{enumerate}
\def\labelenumi{\Alph{enumi}.}
\tightlist
\item
  \emph{Kunskap och förståelse}

  \begin{enumerate}
  \def\labelenumii{\Alph{enumi}.\arabic{enumii}.}
  \tightlist
  \item
    Förklara analytiska begrepp i flera variabler såsom kontinuitet och
    differentierbarhet, samt samspelet mellan analys och geometri, samt
  \item
    redogöra för definitioner samt formulera och bevisa teorem som är
    centrala i flervariabelanalys, såsom kedjeregeln och Greens formel.
  \end{enumerate}
\item
  \emph{Färdighet och förmåga}

  \begin{enumerate}
  \def\labelenumii{\Alph{enumi}.\arabic{enumii}.}
  \tightlist
  \item
    Använda differentialkalkyl i flera variabler,
  \item
    beräkna dubbel- och trippelintegraler samt linje- och ytintegraler,
  \item
    lösa problem, utföra beräkningar och föra resonemang i
    flerdimensionell analys,
  \item
    skriftligt presentera beräkningar och resonemang så att de kan
    följas av den som inte är insatt i problemet,
  \item
    tillämpa flerdimensionell differential- och integralkalkyl på
    tekniska, fysikaliska och datavetenskapliga problem, samt
  \item
    utföra beräkningar, approximera och visualisera resultat med hjälp
    av Matlab.
  \end{enumerate}
\item
  \emph{Värderingsförmåga och förhållningssätt}

  \begin{enumerate}
  \def\labelenumii{\Alph{enumi}.\arabic{enumii}.}
  \tightlist
  \item
    Diskutera relevans, räckvidd och noggrannhet av matematiska modeller
    såsom differentialekvationer och vektorfält.
  \end{enumerate}
\end{enumerate}

\subsection*{Kursinnehåll}

Följande moment behandlas:

\begin{itemize}
\tightlist
\item
  Vektorer och koordinatgeometri i R\^{}n, vektorvärda funktioner,
  kurvor och parametrisering.
\item
  Partiella derivator, gradient och riktningsderivata, tangentplan till
  ytor, normalriktning, kedjeregeln, inversa och implicita
  funktionssatsen, funktionalmatris, linjärisering.
\item
  Optimering på kompakta och allmänna områden.
\item
  Dubbel- och trippelintegraler, iterering av multipelintegraler,
  variabelbyte, tillämpning på volym och tyngdpunkt.
\item
  Vektorfält, linjeintegraler och ytintegraler med avseende på
  funktioner och vektorfält.
\item
  Divergens- och rotationsoperatorerna.
\item
  Greens formel i planet, potentialer.
\item
  Gauss sats och Stokes sats.
\item
  Matlab som ett verktyg i flervariabelanalys.
\end{itemize}

\subsection*{Undervisnings- och
arbetsformer}

Föreläsningar, lärarledda räkneövningar och laborationer.

\subsection*{Examination}

Examinationen av kursen delas in i följande moment:

\begin{longtable}[]{@{}llcc@{}}
\toprule
\textsf{Kod} & \textsf{Benämning} & \textsf{Betyg} & \textsf{Poäng}\tabularnewline
\midrule
\endhead
\texttt{TEN1} & Tentamen: Problemlösning & A-F & 5,0\tabularnewline
\texttt{TEN2} & Tentamen: Teori & G-U & 1,5\tabularnewline
\texttt{LAB1} & Laborationer i Matlab & G-U & 1\tabularnewline
\bottomrule
\end{longtable}

För godkänt betyg på kursen krävs minst betyg E på \texttt{TEN1} samt betyg G på
\texttt{TEN2} och \texttt{LAB1}. Slutbetyget bestäms från \texttt{TEN1}.

\subsection*{Måluppfyllelse}

Examinationsmomenten kopplas till lärandemålen enligt följande:

\begin{longtable}[]{@{}lccc@{}}
\toprule
\textsf{Lärandemål} & \texttt{TEN1} & \texttt{TEN2} & \texttt{LAB1}\tabularnewline
\midrule
\endhead
A.1 & & \faCheck &\tabularnewline
A.2 & & \faCheck &\tabularnewline
B.1 & \faCheck & &\tabularnewline
B.2 & \faCheck & &\tabularnewline
B.3 & \faCheck & \faCheck &\tabularnewline
B.4 & \faCheck & &\tabularnewline
B.5 & \faCheck & &\tabularnewline
B.6 & & & \faCheck\tabularnewline
C.1 & & & \faCheck\tabularnewline
\bottomrule
\end{longtable}

\subsection*{Kurslitteratur}

Obligatorisk litteratur:

\begin{itemize}
\tightlist
\item
  Månsson J, Nordbeck P, \emph{Flerdimensionell analys},
  Studentlitteratur, 2013. Antal sidor: 250 av 364.
\item
  Månsson J., Nordbeck P. \emph{Övningar i flerdimensionell analys},
  Studentlitteratur, 2013. Antal sidor: 120 av 173.
\end{itemize}

\subsection*{Övrigt}

Kursen genomförs på ett sätt sådant att både kvinnor och mäns kunskap och erfarenhet utvecklas och görs synlig.
\pagebreak
\section*{2DV001 - Datorns uppbyggnad (7,5 hp)}

\begin{tabular}{ll}\emph{Huvudområde}: & Datavetenskap\tabularnewline\emph{Fördjupning}: & G2F\tabularnewline\end{tabular}

\subsection*{Förkunskaper}

\begin{itemize}
\tightlist
\item
  1DV004 - Objektorienterad programmering
\item
  1DV005 - Jämnlöpande program
\item
  1FY002 - Ellära och magnetism
\item
  1MA001 - Diskret matematik
\end{itemize}

\subsection*{Lärandemål}

Efter slutförd kurs skall studenten kunna:

\begin{enumerate}
\def\labelenumi{\Alph{enumi}.}
\tightlist
\item
  \emph{Kunskap och förståelse}

  \begin{enumerate}
  \def\labelenumii{\Alph{enumi}.\arabic{enumii}.}
  \tightlist
  \item
    Förklara vikten av abstraktion i utformningen av digitala system,
  \item
    förklara de viktigaste mjukvaru- och hårdvaruabstraktionerna i
    dagens datorsystem,
  \item
    förklara funktionen hos måttligt komplexa digitala system, samt
  \item
    beskriva hur virtualisering och virtuellt minne fungerar.
  \end{enumerate}
\item
  \emph{Färdighet och förmåga}

  \begin{enumerate}
  \def\labelenumii{\Alph{enumi}.\arabic{enumii}.}
  \tightlist
  \item
    Analysera prestandan hos digitala system i termer av latens och
    kapacitet,
  \item
    utforma enkla hårdvarusystem baserat på olika digitala abstraktioner
    såsom minnen, logikkretsar, logiska träd, tillståndsmaskiner,
    pipelining och bussar,
  \item
    implementera hårdvarunära program i C, samt
  \item
    överföra enkla program skrivna i något högnivåspråk till maskinkod.
  \end{enumerate}
\item
  \emph{Värderingsförmåga och förhållningssätt}

  \begin{enumerate}
  \def\labelenumii{\Alph{enumi}.\arabic{enumii}.}
  \tightlist
  \item
    Bedöma hur olika hårdvarudesigner, t.ex. med avseende på cache och
    hitrate, påverkar prestanda för applikationsprogram.
  \end{enumerate}
\end{enumerate}

\subsection*{Kursinnehåll}

Kursen startar med olika systemkomponenter som mikrokontroller,
I/O-enheter och givare. För att uppnå förståelse för samspelet mellan
hård och mjukvara sker programmeringen i assembler.

I kursen senare del sätts systemkomponenterna ihop till ett komplett
system, och fokus flyttas till t.ex. virtualisering och virtuellt minne.
Programmering under denna del sker i C.

Följande moment behandlas:

\begin{itemize}
\tightlist
\item
  Digitala kretsar och CMOS.
\item
  Läsa, tolka och förstå datablad.
\item
  Kombinatorisk och sekventiell logik.
\item
  Tillståndsmaskiner.
\item
  Hårdvaruarkitekturer, t.ex. von Neumann.
\item
  Maskinkod och assemblerkod.
\item
  Hårdvarunära programmering i C.
\item
  Gränslandet mellan mjukvara och hårdvara.
\item
  Minneshierarkin.
\item
  Pipelines.
\item
  Virtuellt minne.
\item
  Enheter och avbrott.
\end{itemize}

\subsection*{Undervisnings- och
arbetsformer}

Undervisningen sker i form av föreläsningar och lärarledda laborationer.
Laborationsuppgifterna utförs i par.

Obligatorisk närvaro kan förekomma på vissa moment.

\subsection*{Examination}

Examinationen av kursen delas in i följande moment:

\begin{longtable}[]{@{}llcc@{}}
\toprule
\textsf{Kod} & \textsf{Benämning} & \textsf{Betyg} & \textsf{Poäng}\tabularnewline
\midrule
\endhead
\texttt{TEN1} & Skriftlig tentamen & A-F & 3,5\tabularnewline
\texttt{LAB1} & Programmeringsuppgifter & A-F & 4\tabularnewline
\bottomrule
\end{longtable}

För godkänt betyg på kursen krävs minst betyg E på samtliga moment.
Slutbetyget bestäms från: \texttt{TEN1} (40~\%) och \texttt{LAB1} (60~\%).

\subsection*{Måluppfyllelse}

Examinationsmomenten kopplas till lärandemålen enligt följande:

\begin{longtable}[]{@{}lcc@{}}
\toprule
\textsf{Lärandemål} & \texttt{TEN1} & \texttt{LAB1}\tabularnewline
\midrule
\endhead
A.1 & \faCheck &\tabularnewline
A.2 & \faCheck &\tabularnewline
A.3 & \faCheck & \faCheck\tabularnewline
A.4 & \faCheck &\tabularnewline
B.1 & \faCheck & \faCheck\tabularnewline
B.2 & & \faCheck\tabularnewline
B.3 & \faCheck & \faCheck\tabularnewline
B.4 & & \faCheck\tabularnewline
C.1 & \faCheck &\tabularnewline
\bottomrule
\end{longtable}

\subsection*{Kurslitteratur}

Obligatorisk litteratur:

\begin{itemize}
\tightlist
\item
  Patterson, D. A. och Hennessy, J. L., \emph{Computer Organization and
  Design - The Hardware/Software Interface}, femte utgåvan, Morgan
  Kaufmann, 2013. Antal sidor: 600 av 800.
\end{itemize}

\subsection*{Övrigt}

Kursen genomförs på ett sätt sådant att både kvinnor och mäns kunskap och erfarenhet utvecklas och görs synlig.
\pagebreak
\section*{1MA007 - Numeriska metoder (5 hp)}

\begin{tabular}{ll}\emph{Huvudområde}: & Matematik\tabularnewline\emph{Fördjupning}: & G1F\tabularnewline\end{tabular}

\subsection*{Förkunskaper}

\begin{itemize}
\tightlist
\item
  1MA002 - Linjär algebra
\item
  1MA005 - Envariabelanalys 2
\item
  1MA006 - Flervariabelanalys
\end{itemize}

\subsection*{Lärandemål}

Efter slutförd kurs skall studenten kunna:

\begin{enumerate}
\def\labelenumi{\Alph{enumi}.}
\tightlist
\item
  \emph{Kunskap och förståelse}

  \begin{enumerate}
  \def\labelenumii{\Alph{enumi}.\arabic{enumii}.}
  \tightlist
  \item
    Redogöra för och särskilja grundläggande begrepp och metoder inom
    beräkningsmatematik, samt
  \item
    redogöra för beräkningsmatematiska resonemang på ett strukturerat
    och logiskt sammanhängande sätt.
  \end{enumerate}
\item
  \emph{Färdighet och förmåga}

  \begin{enumerate}
  \def\labelenumii{\Alph{enumi}.\arabic{enumii}.}
  \tightlist
  \item
    Visa förmåga att kombinera kunskaper om olika begrepp, metoder och
    numeriska algoritmer i problemlösning,
  \item
    identifiera och använda lämpliga grundläggande numeriska algoritmer
    för att lösa givna matematiska problem med hjälp av miniräknare,
  \item
    kunna implementera och tillämpa sådana algoritmer med programpaketet
    Matlab, samt
  \item
    identifiera och använda lämpliga grundläggande numeriska metoder för
    att lösa och analysera givna verklighetsanknutna problem inom
    teknikområdet.
  \end{enumerate}
\item
  \emph{Värderingsförmåga och förhållningssätt}

  \begin{enumerate}
  \def\labelenumii{\Alph{enumi}.\arabic{enumii}.}
  \tightlist
  \item
    Bedöma relevans och noggrannhet för numeriska beräkningar, samt
  \item
    uppskatta resursbehov och jämföra och värdera olika numeriska
    algoritmer och metoder för att analysera givna tekniska problem och
    modeller.
  \end{enumerate}
\end{enumerate}

\subsection*{Kursinnehåll}

Det övergripande målet med kursen är att introducera grundläggande
numeriska metoder inom beräkningsmatematik.

Följande moment behandlas:

\begin{itemize}
\tightlist
\item
  Beräkningsmatematik.
\item
  Felanalys.
\item
  Numeriska metoder för linjära och icke-linjära ekvationssystem,
  konditionstal och matrisfaktorisering, minsta kvadratmetoden,
  polynominterpolation.
\item
  Numerisk derivering och integration.
\item
  Egenvärdesberäkning och singulärvärdesuppdelning.
\item
  Diskret Fouriertransform och diskret cosinustransform.\\
\item
  Tillämpningar inom datorgrafik, datakomprimering, sökmotorer,
  signalbehandling och mekanik mm.
\item
  Problemlösning med hjälp av programvaran Matlab.
\end{itemize}

\subsection*{Undervisnings- och
arbetsformer}

Undervisningen sker i form av föreläsningar, lärarledda räkneövningar
och datorlaborationer. Projektuppgifter sker i par. Obligatorisk närvaro
kan förekomma på vissa moment.

\subsection*{Examination}

Examinationen av kursen delas in i följande moment:

\begin{longtable}[]{@{}llcc@{}}
\toprule
\textsf{Kod} & \textsf{Benämning} & \textsf{Betyg} & \textsf{Poäng}\tabularnewline
\midrule
\endhead
\texttt{LAB1} & Laborationer i Matlab & A-F & 1,5\tabularnewline
\texttt{PRJ1} & Projektuppgift & A-F & 1\tabularnewline
\texttt{TEN1} & Skriftlig tentamen & A-F & 2,5\tabularnewline
\bottomrule
\end{longtable}

För godkänt betyg på kursen krävs minst betyg E på samtliga moment.
Slutbetyget bestäms från: \texttt{LAB1} (25~\%), \texttt{PRJ1} (25~\%) och \texttt{TEN1} (50~\%).

\subsection*{Måluppfyllelse}

Examinationsmomenten på kursen kopplas till lärandemålen enligt
följande:

\begin{longtable}[]{@{}lccc@{}}
\toprule
\textsf{Lärandemål} & \texttt{LAB1} & \texttt{PRJ1} & \texttt{TEN1}\tabularnewline
\midrule
\endhead
A.1 & \faCheck & \faCheck & \faCheck\tabularnewline
A.2 & & \faCheck & \faCheck\tabularnewline
B.1 & \faCheck & \faCheck & \faCheck\tabularnewline
B.2 & \faCheck & \faCheck & \faCheck\tabularnewline
B.3 & \faCheck & \faCheck &\tabularnewline
B.4 & \faCheck & \faCheck &\tabularnewline
C.1 & \faCheck & \faCheck & \faCheck\tabularnewline
C.2 & \faCheck & \faCheck &\tabularnewline
\bottomrule
\end{longtable}

\subsection*{Kurslitteratur}

Obligatorisk litteratur:

\begin{itemize}
\tightlist
\item
  Sauer, T., \emph{Numerical analysis}, andra upplagan, Pearson
  Education, 2013. Antal sidor: 350 av 607.
\end{itemize}

\subsection*{Övrigt}

Kursen genomförs på ett sätt sådant att både kvinnor och mäns kunskap och erfarenhet utvecklas och görs synlig.
\pagebreak
\section*{2DV002 - Mjukvaruarkitektur (5 hp)}

\begin{tabular}{ll}\emph{Huvudområde}: & Datavetenskap\tabularnewline\emph{Fördjupning}: & G2F\tabularnewline\end{tabular}

\subsection*{Förkunskaper}

\begin{itemize}
\tightlist
\item
  1DV004 - Objektorienterad programmering
\item
  1DV007 - Mjukvaruutvecklingsprojekt
\end{itemize}

\subsection*{Lärandemål}

Efter slutförd kurs skall studenten kunna:

\begin{enumerate}
\def\labelenumi{\Alph{enumi}.}
\tightlist
\item
  \emph{Kunskap och förståelse}

  \begin{enumerate}
  \def\labelenumii{\Alph{enumi}.\arabic{enumii}.}
  \tightlist
  \item
    Sammanfatta koncept inom mjukvaruarkitekturer,
  \item
    räkna upp och beskriva arkitektoniska mönster/stilar,
  \item
    förtydliga sambandet mellan mjukvaruarkitektur, designmönster,
    produktlinjer, programkvalitet och återanvändning av mjukvara, samt
  \item
    beskriva sambandet mellan mjukvaruarkitektur och dokumentation.
  \end{enumerate}
\item
  \emph{Färdighet och förmåga}

  \begin{enumerate}
  \def\labelenumii{\Alph{enumi}.\arabic{enumii}.}
  \tightlist
  \item
    Känna igen ett arkitektoniskt designproblem,
  \item
    klassificera mjukvaruarkitekturmönster och taktiker samt bestämma
    dess relevans med hänsyn till designproblemet,
  \item
    skapa en mjukvaruarkitektur för ett givet designproblem,
  \item
    skapa en plan för hur en arkitektur kan implementeras, t.ex. med
    avseende på beroenden och prioritet, samt
  \item
    använda verktyg och språk för att definiera en arkitektur och
    översätta den till programkod i t.ex. Java.
  \end{enumerate}
\item
  \emph{Värderingsförmåga och förhållningssätt}

  \begin{enumerate}
  \def\labelenumii{\Alph{enumi}.\arabic{enumii}.}
  \tightlist
  \item
    Välja lämpliga koncept och strategier för att dokumentera en
    mjukvaruarkitektur i en given situation eller för en given målgrupp,
    samt
  \item
    bedöma en mjukvaruarkitektur genom att mäta dess kvalitet med hänsyn
    till ett visst designproblem.
  \end{enumerate}
\end{enumerate}

\subsection*{Kursinnehåll}

Kursen ger en introduktion till mjukvaruarkitektur, arkitekturstilar och
hur arkitektur kan skapas för att stödja återanvändning av
mjukvarukomponents.

Följande moment behandlas:

\begin{itemize}
\tightlist
\item
  Introduktion till mjukvaruarkitektur och centrala begrepp.
\item
  Mjukvaruarkitekturens roll i systemutveckling.
\item
  Mjukvaruarkitektur kontra implementation.
\item
  Beskrivningsteknik av arkitekturer och arkitektoniska synvinklar.
\item
  Arkitektoniska stilar och mönster, samt hur de förhåller sig till
  designmönster.
\item
  Produktlinjer och dess arkitekturer.
\item
  Design och utvärdering av mjukvaruarkitekturer.
\item
  Kvalitet hos mjukvaruarkitekturer.
\item
  Hur arkitektur kan användas för att beskriva ett systems egenskaper.
\item
  Verktyg och språk för att beskriva arkitekturer.
\item
  Enklare transformationer mellan dessa språk och programmeringsspråk,
  t.ex. Java.
\end{itemize}

\subsection*{Undervisnings- och
arbetsformer}

Undervisningen sker i form av föreläsningar, projektarbete och
presentationer. Projekt och presentationer sker i grupper om 4
studenter.

Obligatorisk närvaro kan förekomma på vissa moment.

\subsection*{Examination}

Examinationen av kursen delas in i följande moment:

\begin{longtable}[]{@{}llcc@{}}
\toprule
\textsf{Kod} & \textsf{Benämning} & \textsf{Betyg} & \textsf{Poäng}\tabularnewline
\midrule
\endhead
\texttt{UPG1} & Inlämningsuppgifter & A-F & 1\tabularnewline
\texttt{PRJ1} & Projektuppgift & A-F & 2\tabularnewline
\texttt{HEM1} & Hemtentamen & A-F & 2\tabularnewline
\bottomrule
\end{longtable}

För godkänt betyg på kursen krävs minst betyg E på samtliga moment.
Slutbetyget bestäms från: \texttt{UPG1} (20~\%), \texttt{PRJ1} (40~\%) och \texttt{HEM1} (40~\%).

\subsection*{Måluppfyllelse}

Examinationsmomenten på kursen kopplas till lärandemålen enligt
följande:

\begin{longtable}[]{@{}lccc@{}}
\toprule
\textsf{Lärandemål} & \texttt{UPG1} & \texttt{PRJ1} & \texttt{HEM1}\tabularnewline
\midrule
\endhead
A.1 & & & \faCheck\tabularnewline
A.2 & & \faCheck & \faCheck\tabularnewline
A.3 & \faCheck & & \faCheck\tabularnewline
A.4 & & \faCheck & \faCheck\tabularnewline
B.1 & \faCheck & \faCheck &\tabularnewline
B.2 & \faCheck & \faCheck &\tabularnewline
B.3 & \faCheck & \faCheck & \faCheck\tabularnewline
B.4 & \faCheck & \faCheck &\tabularnewline
B.5 & \faCheck & \faCheck &\tabularnewline
C.1 & & \faCheck &\tabularnewline
C.2 & \faCheck & & \faCheck\tabularnewline
\bottomrule
\end{longtable}

\subsection*{Kurslitteratur}

Obligatorisk litteratur:

\begin{itemize}
\tightlist
\item
  Bass,~L., Clements, P. och Kazman, R.,
  \emph{Software~Architecture~in~Practice}, tredje utgåvan, Addison­
  Wesley,~2012. Antal sidor: 435 av 547.
\end{itemize}

\subsection*{Övrigt}

Kursen genomförs på ett sätt sådant att både kvinnor och mäns kunskap och erfarenhet utvecklas och görs synlig.
\pagebreak
\section*{2DV003 - Inbyggda system (5 hp)}

\begin{tabular}{ll}\emph{Huvudområde}: & Datavetenskap\tabularnewline\emph{Fördjupning}: & G2F\tabularnewline\end{tabular}

\subsection*{Förkunskaper}

\begin{itemize}
\tightlist
\item
  1DV004 - Objektorienterad programmering
\item
  1DV005 - Jämnlöpande program
\item
  2DV001 - Datorns uppbyggnad
\item
  1FY001 - Mekanik
\item
  1FY002 - Ellära och magnetism
\end{itemize}

\subsection*{Lärandemål}

Efter slutförd kurs skall studenten kunna:

\begin{enumerate}
\def\labelenumi{\Alph{enumi}.}
\tightlist
\item
  \emph{Kunskap och förståelse}

  \begin{enumerate}
  \def\labelenumii{\Alph{enumi}.\arabic{enumii}.}
  \tightlist
  \item
    Beskriva de huvudsakliga applikationerna och referensarkitekturerna
    (CPU, buss, gränssnitt etc.) för inbyggda system och
    realtidshantering,
  \item
    definiera gränssnittet mellan hårdvara och programvara och påvisa
    relaterade begränsningar och potentiella risker,
  \item
    sammanfatta hur inbyggda operativsystem är strukturerat och arbetar,
    särskilt när det gäller avbrott, processer, trådar och
    schemaläggare, samt
  \item
    förklara anomalier vid schemaläggning, deras orsaker och hur man
    hanterar dem.
  \end{enumerate}
\item
  \emph{Färdighet och förmåga}

  \begin{enumerate}
  \def\labelenumii{\Alph{enumi}.\arabic{enumii}.}
  \tightlist
  \item
    Använda olika metoder för att bestämma möjligheten att schemalägga
    en uppsättning periodiska uppgifter,
  \item
    använda effektiva språk och designmiljöer för inbyggda system,
  \item
    formge och implementera inbyggda program för att styra
    hårdvaruenheter, sensorer och manöverdon, samt
  \item
    implementera undantags- och avbrottsrutiner.
  \end{enumerate}
\item
  \emph{Värderingsförmåga och förhållningssätt}

  \begin{enumerate}
  \def\labelenumii{\Alph{enumi}.\arabic{enumii}.}
  \tightlist
  \item
    Bedöma för- och nackdelar för olika schemaläggningsmetoder, samt
  \item
    bedöma för- och nackdelar för olika synkroniseringsmetoder.
  \end{enumerate}
\end{enumerate}

\subsection*{Kursinnehåll}

Kursen ger en introduktion till inbyggda system, sensorer och
manöverdon, samt hur dessa nås från mjukvara. Kursen ger sedan en
fördjupning i schemaläggning för realtidssystem och vilka krav som olika
algoritmer kan uppfylla. Konsekvenser av att missa deadline diskuteras
också.

Följande moment behandlas:

\begin{itemize}
\tightlist
\item
  Introduktion till inbyggda system samt fysiska och simulerade miljöer.
\item
  Sensorer och manöverdon i inbyggda system.
\item
  Begreppen tid och tidshantering, monoton och icke-monoton tid samt
  fördröjningar.
\item
  Realtidsoperativsystem och schemaläggning av uppgifter.
\item
  Metoder för avbrottsstyrd schemaläggning.
\item
  Synkronisering av uppgifter, omkastad prioritet, prioritetsarv,
  begränsad prioritet.
\item
  Hantering av undantag och avbrott.
\end{itemize}

\subsection*{Undervisnings- och
arbetsformer}

Undervisningen sker i form av föreläsningar och lärarledda laborationer.
Laborationer är dels individuella, dels i form av grupparbeten.

Obligatorisk närvaro kan förekomma på vissa moment.

\subsection*{Examination}

Examinationen av kursen delas in i följande moment:

\begin{longtable}[]{@{}llcc@{}}
\toprule
\textsf{Kod} & \textsf{Benämning} & \textsf{Betyg} & \textsf{Poäng}\tabularnewline
\midrule
\endhead
\texttt{LAB1} & Programmeringsuppgifter & A-F & 1,5\tabularnewline
\texttt{PRJ1} & Projekt & A-F & 2\tabularnewline
\texttt{HEM1} & Hemtentamen & A-F & 1,5\tabularnewline
\bottomrule
\end{longtable}

För godkänt betyg på kursen krävs minst betyg E på samtliga moment.
Slutbetyget bestäms från: \texttt{LAB1} (25~\%), \texttt{PRJ1} (35~\%) och \texttt{HEM1} (40~\%).

\subsection*{Måluppfyllelse}

Examinationsmomenten på kursen kopplas till lärandemålen enligt
följande:

\begin{longtable}[]{@{}lccc@{}}
\toprule
\textsf{Lärandemål} & \texttt{LAB1} & \texttt{PRJ1} & \texttt{HEM1}\tabularnewline
\midrule
\endhead
A.1 & & & \faCheck\tabularnewline
A.2 & & & \faCheck\tabularnewline
A.3 & & & \faCheck\tabularnewline
A.4 & & & \faCheck\tabularnewline
B.1 & \faCheck & & \faCheck\tabularnewline
B.2 & \faCheck & \faCheck &\tabularnewline
B.3 & \faCheck & \faCheck &\tabularnewline
B.4 & \faCheck & &\tabularnewline
C.1 & & \faCheck & \faCheck\tabularnewline
C.2 & & \faCheck & \faCheck\tabularnewline
\bottomrule
\end{longtable}

\subsection*{Kurslitteratur}

Obligatorisk litteratur:

\begin{itemize}
\tightlist
\item
  Buttazzo, G., \emph{Hard real-time computing systems - predictable
  scheduling algorithms and applications}, Springer, 2011. Antal sidor:
  400 av 485.
\item
  Kopetz, H., \emph{Real-time systems: Design principles for distributed
  embedded applications}. Springer, 2011. Antal sidor: 300 av 339.
\end{itemize}

\subsection*{Övrigt}

Kursen genomförs på ett sätt sådant att både kvinnor och mäns kunskap och erfarenhet utvecklas och görs synlig.
\pagebreak
\section*{1ED001 - Reglerteknik (5 hp)}

\begin{tabular}{ll}\emph{Huvudområde}: & Elektroteknik\tabularnewline\emph{Fördjupning}: & G1F\tabularnewline\end{tabular}

\subsection*{Förkunskaper}

\begin{itemize}
\tightlist
\item
  1MA002 - Linjär algebra
\end{itemize}

\subsection*{Lärandemål}

Efter slutförd kurs skall studenten kunna:

\begin{enumerate}
\def\labelenumi{\Alph{enumi}.}
\tightlist
\item
  \emph{Kunskap och förståelse}

  \begin{enumerate}
  \def\labelenumii{\Alph{enumi}.\arabic{enumii}.}
  \tightlist
  \item
    Förklara syntesmetoderna: polplacering, kompensering, framkoppling
    och kaskadkoppling, samt
  \item
    redogöra för återkopplade systems stabilitetsegenskaper utifrån
    Bode- och Nyquistdiagram.
  \end{enumerate}
\item
  \emph{Färdighet och förmåga}

  \begin{enumerate}
  \def\labelenumii{\Alph{enumi}.\arabic{enumii}.}
  \tightlist
  \item
    Beskriva enkla dynamiska system med hjälp av matematiska modeller.
  \item
    i enkla fall kunna analysera såväl öppna som slutna reglersystem med
    avseende på systemets stabilitet
  \item
    specificera ett reglersystem i såväl tids- som frekvensplanet
  \item
    utifrån en systembeskrivning simulera tidsförlopp. Processparametrar
    ska kunna ändras och olika processvariabler ska kunna studeras,
  \item
    i enkla fall kunna specificera, modellera, konstruera och verifiera
    ett reglersystem för en labbprocess, samt
  \item
    från såväl slutna systemets polplacering som det öppna systemets
    frekvenskarakteristik göra rimliga bedömningar om hur snabbt och hur
    oscillativt det slutna systemet är samt beräkna stationära fel.
  \end{enumerate}
\item
  \emph{Värderingsförmåga och förhållningssätt}

  \begin{enumerate}
  \def\labelenumii{\Alph{enumi}.\arabic{enumii}.}
  \tightlist
  \item
    Utifrån en systembeskrivning kunna reflektera över och motivera
    vilken typ av regulator som är bäst lämpad för att erhålla t.ex.
    önskad robusthet.
  \end{enumerate}
\end{enumerate}

\subsection*{Kursinnehåll}

De dynamiska system som behandlas är samtliga tidskontinuerliga och
tidsinvarianta. Med några undantag är de även linjära.

Följande moment behandlas:

\begin{itemize}
\tightlist
\item
  Introduktion till reglertekniken: historia, exempel på reglersystem
  och reglerteknikens grundbegrepp.
\item
  Beskrivning av dynamiska system med hjälp av tidsinvarianta ordinära
  differentialekvationer.
\item
  Linjarisering, tillståndsbegreppet, viktfunktioner,
  Laplacetransformer, överföringsfunktioner, Nyquist- och Bodediagram.
\item
  Analys av system. Stabilitetsbegrepp.
\item
  Stabilitetsundersökningar med hjälp av rotortmetoden, Routh-Hurwitz
  kriterium, argumentvariationsprincipen och Nyquist-kriteriet. Fas- och
  amplitudmarginal. Syntes av reglersystem. Specifikationer,
  polplacering, kompenseringsfilter, PID-regulatorn, framkoppling,
  kaskadreglering, robusthet, känslighet för störningar och
  parameterändringar.
\item
  Implementering av regulatorer.
\end{itemize}

\subsection*{Undervisnings- och
arbetsformer}

Undervisningen består av föreläsningar, övningar och laborationer.
Övningar utförs i grupp. Deltagande vid vissa övningar kan vara
obligatoriska.

\subsection*{Examination}

Examinationen av kursen delas in följande moment:

\begin{longtable}[]{@{}llcc@{}}
\toprule
\textsf{Kod} & \textsf{Benämning} & \textsf{Betyg} & \textsf{Poäng}\tabularnewline
\midrule
\endhead
\texttt{TEN1} & Skriftlig tentamen & A-F & 3\tabularnewline
\texttt{LAB1} & Laboration och rapport & A-F & 2\tabularnewline
\bottomrule
\end{longtable}

För godkänt betyg på kursen krävs minst betyg E på samtliga moment.
Slutbetyget bestäms från \texttt{TEN1}.

\subsection*{Måluppfyllelse}

Examinationsmomenten kopplas till lärandemålen enligt följande:

\begin{longtable}[]{@{}lcc@{}}
\toprule
\textsf{Lärandemål} & \texttt{TEN1} & \texttt{LAB1}\tabularnewline
\midrule
\endhead
A.1 & \faCheck &\tabularnewline
A.2 & \faCheck &\tabularnewline
B.1 & & \faCheck\tabularnewline
B.2 & \faCheck &\tabularnewline
B.3 & \faCheck &\tabularnewline
B.4 & & \faCheck\tabularnewline
B.5 & &\tabularnewline
B.6 & \faCheck & \faCheck\tabularnewline
C.1 & \faCheck &\tabularnewline
\bottomrule
\end{longtable}

\subsection*{Kurslitteratur}

Obligatorisk litteratur:

\begin{itemize}
\tightlist
\item
  Glad, T. och Ljung, L., \emph{Reglerteknik: grundläggande teori},
  fjärde utgåvan, Studentlitteratur, 2006. Antal sidor: 244 av 244.
\end{itemize}

\subsection*{Övrigt}

Kursen genomförs på ett sätt sådant att både kvinnor och mäns kunskap och erfarenhet utvecklas och görs synlig.
\pagebreak
\section*{2DV004 - Datorgrafik (5 hp)}

\begin{tabular}{ll}\emph{Huvudområde}: & Datavetenskap\tabularnewline\emph{Fördjupning}: & G2F\tabularnewline\end{tabular}

\subsection*{Förkunskaper}

\begin{itemize}
\tightlist
\item
  1DV004 - Objektorienterad programmering
\item
  1DV006 - Algoritmer
\item
  1MA002 - Linjär algebra
\item
  1MA006 - Flervariabelanalys
\end{itemize}

\subsection*{Lärandemål}

Efter slutförd kurs skall studenten kunna:

\begin{enumerate}
\def\labelenumi{\Alph{enumi}.}
\tightlist
\item
  \emph{Kunskap och förståelse}

  \begin{enumerate}
  \def\labelenumii{\Alph{enumi}.\arabic{enumii}.}
  \tightlist
  \item
    Karaktärisera alla aspekter av datorgrafikpipelinjen, dvs de olika
    stegen och algoritmerna som krävs för att gå från en geometrisk
    3D-objektspecifikation till en motsvarande 2D-bild på en datorskärm,
  \item
    definiera och förklara olika typer av objektrepresentationer, samt
  \item
    definiera och förklara de viktigaste modellerna och algoritmerna för
    visning och lokal belysning.
  \end{enumerate}
\item
  \emph{Färdighet och förmåga}

  \begin{enumerate}
  \def\labelenumii{\Alph{enumi}.\arabic{enumii}.}
  \tightlist
  \item
    Utföra och implementera rasteriseringsalgoritmer för grundläggande
    grafiska primitiver,
  \item
    utför och implementera geometriska transformationer,
    kameratransformationer, projektionstransformationer och
    visningstransformationer, samt
  \item
    implementera grundläggande 2D/3D grafiklösningar med hjälp av
    OpenGL.
  \end{enumerate}
\item
  \emph{Värderingsförmåga och förhållningssätt}

  \begin{enumerate}
  \def\labelenumii{\Alph{enumi}.\arabic{enumii}.}
  \tightlist
  \item
    Reflektera över egenskaperna hos olika algoritmer och modeller samt
    välja dem som är lämpliga för det problem som ska lösas, samt
  \item
    reflektera över den inverkan som val av t.ex. specifika
    belysning/skuggning eller färgrepresentationer har på kvaliteten på
    slutresultatet.
  \end{enumerate}
\end{enumerate}

\subsection*{Kursinnehåll}

Kursen går igenom grundläggande tekniker som belysning och färgmodeller
samt diskuterar grundläggande tekniker och algoritmer som används i 2D
och 3D-grafik.

Följande moment behandlas:

\begin{itemize}
\tightlist
\item
  Definition av området datorgrafik och dess omfattning.
\item
  Översikt över visnings- och interaktionsenhetsteknik.
\item
  2D-primitiver och deras rasterisering.
\item
  Fyllningsalgoritmer och antialiasing.
\item
  3D-objektrepresentationer.
\item
  Geometriska transformationer.
\item
  Kamera-, projektions- och visningstransformationer.
\item
  Synlighet och klippningsalgoritmer.
\item
  Färgmodeller.
\item
  Belysning och skuggning, speciellt lokal belysning.
\item
  OpenGL.
\end{itemize}

\subsection*{Undervisnings- och
arbetsformer}

Undervisningen på kursen omfattar föreläsningar och lärarledd
handledning av laborationer och inlämningsuppgifter. Alla uppgifter i
kursen utförs parvis.

\subsection*{Examination}

Examinationen av kursen delas in i följande moment:

\begin{longtable}[]{@{}llcc@{}}
\toprule
\textsf{Kod} & \textsf{Benämning} & \textsf{Betyg} & \textsf{Poäng}\tabularnewline
\midrule
\endhead
\texttt{UPG1} & Inlämningsuppgifter & A-F & 2,5\tabularnewline
\texttt{LAB1} & Programmeringsuppgifter & A-F & 2,5\tabularnewline
\bottomrule
\end{longtable}

För godkänt betyg på kursen krävs minst betyg E på samtliga moment.
Slutbetyget bestäms från: \texttt{UPG1} (50~\%) och \texttt{LAB1} (50~\%).

\subsection*{Måluppfyllelse}

Examinationsmomenten på kursen kopplas till lärandemålen enligt
följande:

\begin{longtable}[]{@{}lcc@{}}
\toprule
\textsf{Lärandemål} & \texttt{UPG1} & \texttt{LAB1}\tabularnewline
\midrule
\endhead
A.1 & \faCheck & \faCheck\tabularnewline
A.2 & \faCheck &\tabularnewline
A.3 & \faCheck &\tabularnewline
B.1 & \faCheck & \faCheck\tabularnewline
B.2 & \faCheck & \faCheck\tabularnewline
B.3 & & \faCheck\tabularnewline
C.1 & & \faCheck\tabularnewline
C.2 & \faCheck &\tabularnewline
\bottomrule
\end{longtable}

\subsection*{Kurslitteratur}

Obligatorisk litteratur:

\begin{itemize}
\tightlist
\item
  Hearn, D. D., Baker, M. P. och Carithers, W., \emph{Computer Graphics
  with OpenGL}, fjärde utgåvan, Pearson, 2010. Antal sidor: 450 av 812.
\end{itemize}

\subsection*{Övrigt}

Kursen genomförs på ett sätt sådant att både kvinnor och mäns kunskap och erfarenhet utvecklas och görs synlig.
\pagebreak
\section*{2DV005 - Datornät (5 hp)}

\begin{tabular}{ll}\emph{Huvudområde}: & Datavetenskap\tabularnewline\emph{Fördjupning}: & G2F\tabularnewline\end{tabular}

\subsection*{Förkunskaper}

\begin{itemize}
\tightlist
\item
  1DV003 - Databaser och datamodellering
\item
  1DV004 - Objektorienterad programmering
\item
  1DV005 - Jämnlöpande program
\item
  1DV006 - Algoritmer
\item
  2DV001 - Datorns uppbyggnad
\item
  1FY002 - Ellära och magnetism
\item
  1MA001 - Diskret matematik
\end{itemize}

\subsection*{Lärandemål}

Efter slutförd kurs skall studenten kunna:

\begin{enumerate}
\def\labelenumi{\Alph{enumi}.}
\tightlist
\item
  \emph{Kunskap och förståelse}

  \begin{enumerate}
  \def\labelenumii{\Alph{enumi}.\arabic{enumii}.}
  \tightlist
  \item
    Beskriva lagren i en nätverksstack, t.ex. TCP/IP och diskutera deras
    syfte,
  \item
    beskriva hur lagren interagerar för att överföra data över ett
    nätverk, och hur varje funktion manipulerar data, t.ex. genom att
    lägga till pakethuvuden eller konvertera signalen,
  \item
    förklara hur routing fungerar i lokala nät och på internet,
  \item
    beskriva de olika typerna av adresser som används, samt
  \item
    beskriva några av de vanligare applikationsprotokollen, t.ex. DNS.
  \end{enumerate}
\item
  \emph{Färdighet och förmåga}

  \begin{enumerate}
  \def\labelenumii{\Alph{enumi}.\arabic{enumii}.}
  \tightlist
  \item
    Använda vanliga felsökningsverktyg för nätverk, t.ex. tcpdump, ping
    och traceroute,
  \item
    skriva program som kommunicerar över TCP/IP, samt
  \item
    konfigurera och administrera routrar enligt en specifikation.
  \item
    tolka standarder för nätverksprotokoll (RFC) och implementera dessa
    i programvara.
  \end{enumerate}
\item
  \emph{Värderingsförmåga och förhållningssätt}

  \begin{enumerate}
  \def\labelenumii{\Alph{enumi}.\arabic{enumii}.}
  \tightlist
  \item
    Givet en applikation och ett förslag på implementation (protokoll),
    resonera kring vilka egenskaper, t.ex. prestanda den kommer att ha
    samt vilka problem som kan uppstå, t.ex. med avseende på
    tillförlitlighet.
  \end{enumerate}
\end{enumerate}

\subsection*{Kursinnehåll}

Kursen ger en introduktion till datornät från ett Internet och
TCP/IP-perspektiv. Huvudsaklig fokus ligger på mjukvaruaspekter, men
grundläggande begrepp inom datakommunikation, t.ex. signaler och
modulering berörs.

Följande moment behandlas:

\begin{itemize}
\tightlist
\item
  Lagerindelade protokollmodeller, OSI och TCP/IP.
\item
  Paketförmedling.
\item
  Datakommunikation på fysisk nivå.
\item
  Datalänkprotokoll.
\item
  Lokala nätverk (t.ex. topologi, åtkomstkontroll, IEEE 802-standarder).
\item
  Transportprotokoll.
\item
  Applikationsprotokoll.
\item
  Standardgränssnitt för nätverksprogrammerings (t.ex. BSD Socket).
\item
  Ruttvalsalgoritmer.
\end{itemize}

\subsection*{Undervisnings- och
arbetsformer}

Undervisningen sker i form av föreläsningar och lärarledda laborationer.
Laborationer är dels individuella, dels i form av grupparbeten.
Laborationer rapporteras med skriftliga rapporter.

Obligatorisk närvaro kan förekomma på vissa moment.

\subsection*{Examination}

Examinationen av kursen delas in i följande moment:

\begin{longtable}[]{@{}llcc@{}}
\toprule
\textsf{Kod} & \textsf{Benämning} & \textsf{Betyg} & \textsf{Poäng}\tabularnewline
\midrule
\endhead
\texttt{TEN1} & Skriftlig tentamen & A-F & 3\tabularnewline
\texttt{LAB1} & Programmeringsuppgifter & A-F & 2\tabularnewline
\bottomrule
\end{longtable}

För godkänt betyg på kursen krävs minst betyg E på samtliga moment.
Slutbetyget bestäms från: \texttt{TEN1} (60~\%) och \texttt{LAB1} (40~\%).

\subsection*{Måluppfyllelse}

Examinationsmomenten kopplas till lärandemålen enligt följande:

\begin{longtable}[]{@{}lcc@{}}
\toprule
\textsf{Lärandemål} & \texttt{TEN1} & \texttt{LAB1}\tabularnewline
\midrule
\endhead
A.1 & \faCheck &\tabularnewline
A.2 & \faCheck &\tabularnewline
A.3 & \faCheck &\tabularnewline
A.4 & \faCheck &\tabularnewline
A.5 & \faCheck &\tabularnewline
B.1 & & \faCheck\tabularnewline
B.2 & \faCheck & \faCheck\tabularnewline
B.3 & & \faCheck\tabularnewline
B.4 & \faCheck & \faCheck\tabularnewline
C.1 & \faCheck & \faCheck\tabularnewline
\bottomrule
\end{longtable}

\subsection*{Kurslitteratur}

Obligatorisk litteratur:

\begin{itemize}
\tightlist
\item
  Comer, D., \emph{Computer Networks and Internets}, sjätte utgåvan,
  Pearson, 2015. Antal sidor: 640 / 667 sidor.
\end{itemize}

\subsection*{Övrigt}

Kursen genomförs på ett sätt sådant att både kvinnor och mäns kunskap och erfarenhet utvecklas och görs synlig.
\pagebreak
\section*{1ZT003 - Industriell ekonomi (5 hp)}

\begin{tabular}{ll}
\emph{Fördjupning}: & G1N\tabularnewline
\end{tabular}

\subsection*{Förkunskaper}

Grundläggande behörighet samt Matematik D eller Matematik 4
(områdesbehörighet 9/A9).

\subsection*{Lärandemål}

Efter slutförd kurs skall studenten kunna:

\begin{enumerate}
\def\labelenumi{\Alph{enumi}.}
\tightlist
\item
  \emph{Kunskap och förståelse}

  \begin{enumerate}
  \def\labelenumii{\Alph{enumi}.\arabic{enumii}.}
  \tightlist
  \item
    Förklara grundläggande ekonomiska konsekvenser av olika tekniska
    beslut och tekniska konsekvenser av olika ekonomiska beslut,
  \item
    redogöra för grundläggande ekonomiska begrepp, samt
  \item
    beskriva metoder för att kunna styra, planera och utveckla
    industriella verksamheter.
  \end{enumerate}
\item
  \emph{Färdighet och förmåga}

  \begin{enumerate}
  \def\labelenumii{\Alph{enumi}.\arabic{enumii}.}
  \tightlist
  \item
    Argumentera för en verksamhet i ekonomiska termer, samt
  \item
    på en grundläggande nivå kunna utföra kalkylering, redovisning och
    finansiering inom givna ramar,
  \end{enumerate}
\item
  \emph{Värderingsförmåga och förhållningssätt}

  \begin{enumerate}
  \def\labelenumii{\Alph{enumi}.\arabic{enumii}.}
  \tightlist
  \item
    Visa insikt i teknikens roll i samhället och människors ansvar för
    hur den används, utifrån ekonomiska aspekter.
  \end{enumerate}
\end{enumerate}

\subsection*{Kursinnehåll}

I denna kurs kommer en grundläggande förståelse av företagsekonomi och
baskunskaper i industriell ekonomi ges. Kursen innehåller följande:

\begin{itemize}
\tightlist
\item
  Ekonomiska begrepp.
\item
  Nyckeltal.
\item
  Ekonomiska modeller.
\item
  Ekonomisk analys.
\item
  Kalkylering på kort och lång sikt.
\item
  Rörelsekapitalbehov.
\item
  Tekniker för att utföra och göra lönsamhetsbedömning i företag.
\item
  Kapitalbindning.
\item
  Budgetering.
\item
  Årsredovisning, redovisning och bokföring.
\item
  Finansiering.
\item
  Ekonomi- och verksamhetsstyrning.
\end{itemize}

\subsection*{Undervisnings- och
arbetsformer}

Undervisningen bedrivs i form av föreläsningar och övningar/seminarier
med räkne- och diskussionskaraktär, där konkreta industri- och
samhällsexempel används för att åskådliggöra teoriernas relevans.

\subsection*{Examination}

Examinationen av kursen delas in i följande moment:

\begin{longtable}[]{@{}llcc@{}}
\toprule
\textsf{Kod} & \textsf{Benämning} & \textsf{Betyg} & \textsf{Poäng}\tabularnewline
\midrule
\endhead
\texttt{SEM1} & Seminarium och debatt & G-U & 1\tabularnewline
\texttt{TEN1} & Skriftlig tentamen & A-F & 4\tabularnewline
\bottomrule
\end{longtable}

För godkänt betyg på kursen krävs betyg G på \texttt{SEM1} och minst betyg E på
\texttt{TEN1}. Slutbetyget bestäms från \texttt{TEN1}.

\subsection*{Måluppfyllelse}

Examinationsmomenten kopplas till lärandemålen enligt följande:

\begin{longtable}[]{@{}lcc@{}}
\toprule
\textsf{Lärandemål} & \texttt{TEN1} & \texttt{SEM1}\tabularnewline
\midrule
\endhead
A.1 & \faCheck & \faCheck\tabularnewline
A.2 & \faCheck & \faCheck\tabularnewline
A.3 & & \faCheck\tabularnewline
B.1 & & \faCheck\tabularnewline
B.2 & \faCheck & \faCheck\tabularnewline
C.1 & & \faCheck\tabularnewline
\bottomrule
\end{longtable}

\subsection*{Kurslitteratur}

Obligatorisk litteratur:

\begin{itemize}
\tightlist
\item
  Lantz, B., Isaksson, A. och Löfsten, H., \emph{Industriell ekonomi --
  grundläggande ekonomisk analys}, Studentlitteratur, 2014. Antal sidor:
  376 av 376.
\item
  Aktuella artiklar ur vetenskapliga tidskrifter tillgängliga via
  Internet, hänvisning ges på kursens webbplats.
\end{itemize}

Referenslitteratur:

\begin{itemize}
\tightlist
\item
  Engwall, M. Jerbrant, A., Karlson, B. och Storm, P., \emph{Modern
  industriell ekonomi, Studentlitteratur}, 2017.
\item
  Skärvad, P-H. och Olsson, J., \emph{Företagsekonomi 100 Fakta}, Liber,
  2017.
\item
  Skärvad, P-H. och Olsson, J., \emph{Företagsekonomi 100 Övningsbok},
  Liber, 2017.
\item
  Skärvad, P-H. och Olsson, J., \emph{Företagsekonomi 100 Lösningar},
  Liber, 2017.
\end{itemize}

\subsection*{Övrigt}

Kursen genomförs på ett sätt sådant att både kvinnor och mäns kunskap och erfarenhet utvecklas och görs synlig.
\pagebreak
\section*{2ZT001 - Vetenskapliga metoder (5 hp)}

\begin{tabular}{ll}\emph{Fördjupning}: & G2F\tabularnewline\end{tabular}

\subsection*{Förkunskaper}

\begin{itemize}
\tightlist
\item
  1ZT001 - Teknisk kommunikation
\item
  1MA001 - Diskret matematik
\item
  1MA004 - Tillämpad sannolikhetslära och statistik
\end{itemize}

\subsection*{Lärandemål}

Efter genomförd kurs förväntas studenten kunna:

\begin{enumerate}
\def\labelenumi{\Alph{enumi}.}
\tightlist
\item
  \emph{Kunskap och förståelse}

  \begin{enumerate}
  \def\labelenumii{\Alph{enumi}.\arabic{enumii}.}
  \tightlist
  \item
    Beskriva olika synsätt på kunskap och vetenskap,
  \item
    beskriva grundläggande epistemologiska begränsningar, t.ex.
    problemen med observation och induktion, samt
  \item
    beskriva relationen mellan ingenjörskonst och vetenskapliga metoder.
  \end{enumerate}
\item
  \emph{Färdighet och förmåga}

  \begin{enumerate}
  \def\labelenumii{\Alph{enumi}.\arabic{enumii}.}
  \tightlist
  \item
    Formulera vetenskapliga frågeställningar,
  \item
    formulera syfte och omfattning hos en vetenskaplig studie,
  \item
    välja en lämplig vetenskaplig metod och genomföra denna på ett
    korrekt sätt, samt
  \item
    planera en vetenskaplig studie och resonera kring vilken relevans
    resultatet kommer att ha för akademi och industri.
  \end{enumerate}
\item
  \emph{Värderingsförmåga och förhållningssätt}

  \begin{enumerate}
  \def\labelenumii{\Alph{enumi}.\arabic{enumii}.}
  \tightlist
  \item
    Bedöma och hantera samhälleliga aspekter av och etiska
    frågeställningar kring vetenskapligt arbete, samt
  \item
    reflektera över likheter och skillnader i rollen som forskare och
    som ingenjör.
  \end{enumerate}
\end{enumerate}

\subsection*{Kursinnehåll}

Kurser ger en kort introduktion till vetenskapsteori och dess historia,
samt olika vetenskapliga metoder, t.ex. systematiska textstudier och
hypotesprövning. Metoderna exemplifieras och fördjupas med
mjukvarutekniska frågeställningar. Kursen syftar till att förbereda
studenterna för det självständiga arbetet samt påvisa samspelet mellan
vetenskaplig metodik och ingenjörskonst, samt hur en civilingenjör i
mjukvaruteknik behöver färdigheter i båda rollerna.

\begin{itemize}
\tightlist
\item
  Introduktion till vetenskapsteori och dess historia.
\item
  Vetenskaplighet och syftet med vetenskapliga metoder.
\item
  Ingenjörskonst och vetenskaplighet.
\item
  Kvantitativ och kvalitativ metod.
\item
  Vetenskapliga frågeställningar och hur sådana kan formuleras.
\item
  Planering och genomförande av ett forskningsprojekt.
\item
  Samspelet mellan forskning och teknisk utveckling, samt deras
  samhälleliga aspekter.
\item
  Forskningsetik.
\item
  Fördjupning i informationssökning.
\item
  Fördjupning i källkritik.
\item
  Fördjupning i vetenskapligt skrivande och referenshantering.
\end{itemize}

\subsection*{Undervisnings- och
arbetsformer}

Undervisningen består av föreläsningar och seminarier där tillämpning av
olika vetenskapliga metoder diskuteras utifrån ett problem. Kursen
innehåller även en serie gästföreläsningar där forskare från akademi och
industri presenterar sitt forskningsämne.

Planeringsrapporten skapas i samråd med och under handledning av en
forskare.

\subsection*{Examination}

Examinationen av kursen delas in i följande moment:

\begin{longtable}[]{@{}llcc@{}}
\toprule
\textsf{Kod} & \textsf{Benämning} & \textsf{Betyg} & \textsf{Poäng}\tabularnewline
\midrule
\endhead
\texttt{UPG1} & Inlämningsuppgifter & A-F & 3\tabularnewline
\texttt{UPG2} & Planeringsrapport & A-F & 2\tabularnewline
\bottomrule
\end{longtable}

För godkänt betyg på kursen krävs minst betyg E på samtliga moment.
Slutbetyget bestäms från: \texttt{UPG1} (50~\%) och \texttt{UPG2} (50~\%).

\subsection*{Måluppfyllelse}

Examinationsmomenten på kursen kopplas till lärandemålen enligt
följande:

\begin{longtable}[]{@{}lcc@{}}
\toprule
\textsf{Lärandemål} & \texttt{UPG1} & \texttt{UPG2}\tabularnewline
\midrule
\endhead
A.1 & \faCheck &\tabularnewline
A.2 & \faCheck &\tabularnewline
A.3 & \faCheck &\tabularnewline
B.1 & \faCheck & \faCheck\tabularnewline
B.2 & \faCheck & \faCheck\tabularnewline
B.3 & \faCheck & \faCheck\tabularnewline
B.4 & & \faCheck\tabularnewline
C.1 & \faCheck & \faCheck\tabularnewline
C.2 & \faCheck &\tabularnewline
\bottomrule
\end{longtable}

\subsection*{Kurslitteratur}

Obligatorisk litteratur:

\begin{itemize}
\tightlist
\item
  Chalmers, A., \emph{What Is This Thing Called Science?}, fjärde
  utgåvan, UQP, 2013. Antal sidor: 250 av 312.
\item
  Höst, M., Regnell, B. och Runeson, P., \emph{Att genomföra
  examensarbete}, Studentlitteratur, 2006. Antal sidor: 130 av 153.
\item
  Kompendium med vetenskapliga artiklar
\end{itemize}

\subsection*{Övrigt}

Kursen genomförs på ett sätt sådant att både kvinnor och mäns kunskap och erfarenhet utvecklas och görs synlig.
\pagebreak
\section*{2DV006 - Datorsäkerhet (5 hp)}

\begin{tabular}{ll}\emph{Huvudområde}: & Datavetenskap\tabularnewline\emph{Fördjupning}: & G2F\tabularnewline\end{tabular}

\subsection*{Förkunskaper}

\begin{itemize}
\tightlist
\item
  1DV003 - Databaser och datamodellering
\item
  1DV004 - Objektorienterad programmering
\item
  2DV001 - Datorns uppbyggnad
\item
  2DV003 - Inbyggda system
\item
  2DV005 - Datornät
\item
  1MA004 - Tillämpad sannolikhetslära och statistik
\end{itemize}

\subsection*{Lärandemål}

Efter genomförd kurs förväntas studenterna att kunna:

\begin{enumerate}
\def\labelenumi{\Alph{enumi}.}
\tightlist
\item
  \emph{Kunskap och förståelse}

  \begin{enumerate}
  \def\labelenumii{\Alph{enumi}.\arabic{enumii}.}
  \tightlist
  \item
    Redogöra för området IT­-säkerhet och dess olika inriktningar,
  \item
    förklara grundläggande säkerhetsmekanismer, samt
  \item
    beskriva de viktigaste hot mot dator- och nätverkssäkerhet och de
    metoder som finns tillgängliga för att motverka dem.
  \end{enumerate}
\item
  \emph{Färdighet och förmåga}

  \begin{enumerate}
  \def\labelenumii{\Alph{enumi}.\arabic{enumii}.}
  \tightlist
  \item
    Utföra en säkerhetsanalys i en organisation, samt
  \item
    upprätta säkerhets-policyer och -planer för en organisation.
  \end{enumerate}
\item
  \emph{Värderingsförmåga och förhållningssätt}

  \begin{enumerate}
  \def\labelenumii{\Alph{enumi}.\arabic{enumii}.}
  \tightlist
  \item
    Utvärdera och relatera till problem av etisk och moralisk natur
    relaterad till datakriminalitet, övervakning och integritet, samt
  \item
    bedöma konsekvenserna av IT-tillämpningar, t.ex. applikationer eller
    system, som inte uppnår en tillräckligt hög nivå av säkerhet.
  \end{enumerate}
\end{enumerate}

\subsection*{Kursinnehåll}

Kursen är en introduktionskurs i IT-säkerhet. Den ger grundläggande
förståelse för olika hot och möjligheter inom området samt kunskap om de
verktyg som kan används för att hantera säkerheten.

Följande moment behandlas:

\begin{itemize}
\tightlist
\item
  IT och samhällets sårbarhet.
\item
  Informationssäkerhet.
\item
  Informationsklassificering.
\item
  Etik och lagstiftning.
\item
  Sårbarhetsanalys på organisationsnivå.
\item
  Grundläggande säkerhetsmekanismer (kryptering, autentisering,
  åtkomstkontroll).
\item
  Programsäkerhet (Buffertöverskridning, säker programmering).
\item
  Säkerhet i operativsystem.
\item
  Säkerhet i databaser.
\item
  Nätverkssäkerhet.
\item
  Illasinnade program (virus / maskar / trojaner).
\end{itemize}

\subsection*{Undervisnings- och
arbetsformer}

Undervisningen sker i form av föreläsningar, lärarledda laborationer och
seminarier. Laborationer är dels individuella, dels i form av
grupparbeten. Laborationer rapporteras med skriftliga rapporter och i
några fall också med muntliga presentationer.

Obligatorisk närvaro kan förekomma på vissa moment.

\subsection*{Examination}

Examinationen av kursen delas in i följande moment:

\begin{longtable}[]{@{}llcc@{}}
\toprule
\textsf{Kod} & \textsf{Benämning} & \textsf{Betyg} & \textsf{Poäng}\tabularnewline
\midrule
\endhead
\texttt{TEN1} & Skriftlig tentamen & A-F & 2,5\tabularnewline
\texttt{UPG1} & Inlämningsuppgifter & A-F & 2,5\tabularnewline
\bottomrule
\end{longtable}

För godkänt betyg på kursen krävs minst betyg E på samtliga moment.
Slutbetyget bestäms från: \texttt{TEN1} (50~\%) och \texttt{UPG1} (50~\%).

\subsection*{Måluppfyllelse}

Examinationsmomenten kopplas till lärandemålen enligt följande:

\begin{longtable}[]{@{}lcc@{}}
\toprule
\textsf{Lärandemål} & \texttt{TEN1} & \texttt{UPG1}\tabularnewline
\midrule
\endhead
A.1 & \faCheck & \faCheck\tabularnewline
A.2 & \faCheck & \faCheck\tabularnewline
A.3 & \faCheck &\tabularnewline
B.1 & & \faCheck\tabularnewline
B.2 & \faCheck & \faCheck\tabularnewline
C.1 & & \faCheck\tabularnewline
C.2 & & \faCheck\tabularnewline
\bottomrule
\end{longtable}

\subsection*{Kurslitteratur}

Obligatorisk litteratur:

\begin{itemize}
\tightlist
\item
  Pfleeger, C., Pfleeger, S., Margulies, J., \emph{Security in
  Computing} femte utgåvan, Prentice Hall, 2015. Antal sidor: 760 av
  944.
\end{itemize}

\subsection*{Övrigt}

Kursen genomförs på ett sätt sådant att både kvinnor och mäns kunskap och erfarenhet utvecklas och görs synlig.
\pagebreak
\section*{2DV007 - Självständigt arbete (15 hp)}

\begin{tabular}{ll}\emph{Huvudområde}: & Datavetenskap\tabularnewline\emph{Fördjupning}: & G2E\tabularnewline\end{tabular}

\subsection*{Förkunskaper}

\begin{itemize}
\tightlist
\item
  1DV007 - Mjukvaruutvecklingsprojekt
\item
  1ZT001 - Teknisk kommunikation
\item
  2ZT001 - Vetenskapliga metoder
\end{itemize}

\subsection*{Lärandemål}

Efter slutförd kurs skall studenten kunna:

\begin{enumerate}
\def\labelenumi{\Alph{enumi}.}
\tightlist
\item
  \emph{Kunskap och förståelse}

  \begin{enumerate}
  \def\labelenumii{\Alph{enumi}.\arabic{enumii}.}
  \tightlist
  \item
    Integrera och fördjupa förvärvade kunskaper.
  \end{enumerate}
\item
  \emph{Färdighet och förmåga}

  \begin{enumerate}
  \def\labelenumii{\Alph{enumi}.\arabic{enumii}.}
  \tightlist
  \item
    Formulera, välja och tillämpa vetenskapliga frågeställningar och
    metoder samt avgränsa dessa,
  \item
    planera, genomföra ett självständigt arbete inom givna tidsramar,
  \item
    presentera det självständiga arbetet muntligt och skriftligt,
  \item
    söka, värdera och referera vetenskaplig litteratur, samt
  \item
    kritiskt granska samt muntligt och skriftligt diskutera ett framlagt
    arbete.
  \end{enumerate}
\item
  \emph{Värderingsförmåga och förhållningssätt}

  \begin{enumerate}
  \def\labelenumii{\Alph{enumi}.\arabic{enumii}.}
  \tightlist
  \item
    Välja och tillämpa en vetenskaplig metod som lämpar sig för
    frågeställningen,
  \item
    bedöma och reflektera över egna resultat jämfört mot tidigare kända
    resultat, samt
  \item
    göra bedömningar med hänsyn till vetenskapliga, samhälleliga och
    etiska hänsynstaganden.
  \end{enumerate}
\end{enumerate}

\subsection*{Kursinnehåll}

Kursen innehåller självständigt arbete (max två studenter) kring ett
problem som valts i samråd med handledare.

Följande moment behandlas:

\begin{itemize}
\tightlist
\item
  Problemformulering.
\item
  Söka i vetenskaplig litteratur.
\item
  Analysera och bearbeta resultat och fakta.
\item
  Vetenskaplig teknisk rapportskrivning.
\item
  Muntlig redovisning.
\item
  Opponering.
\item
  Självständigt ingenjörsmässigt arbete omfattande teoretisk och/eller
  experimentell verksamhet.
\end{itemize}

\subsection*{Undervisnings- och
arbetsformer}

Handledning av lärare med jämna mellanrum. Framläggning av eget arbete
och opponering i slutet av kursen. Auskultation vid framläggning av tre
andra arbeten på samma nivå eller högre måste ske innan framläggning av
eget arbete får ske. Auskultation kan ske från och mer termin fyra.
Närvaro vid egen framläggning och opponering är obligatorisk.

\subsection*{Examination}

Examinationen av kursen delas in i följande moment:

\begin{longtable}[]{@{}llcc@{}}
\toprule
\textsf{Kod} & \textsf{Benämning} & \textsf{Betyg} & \textsf{Poäng}\tabularnewline
\midrule
\endhead
\texttt{RAP1} & Rapport och framläggning & A-F & 12\tabularnewline
\texttt{OPP1} & Opponering & G-U & 1,5\tabularnewline
\texttt{ASK1} & Auskultation & G-U & 1,5\tabularnewline
\bottomrule
\end{longtable}

För godkänt betyg på kursen krävs betyg G på \texttt{OPP1} och \texttt{ASK1} samt minst
betyg E på \texttt{RAP1}. Slutbetyget bestäms från \texttt{RAP1}.

\subsection*{Måluppfyllelse}

Examinationsmomenten på kursen kopplas till lärandemålen enligt
följande:

\begin{longtable}[]{@{}lccc@{}}
\toprule
\textsf{Lärandemål} & \texttt{RAP1} & \texttt{OPP1} & \texttt{ASK1}\tabularnewline
\midrule
\endhead
A.1 & \faCheck & & \faCheck\tabularnewline
B.1 & \faCheck & &\tabularnewline
B.2 & \faCheck & &\tabularnewline
B.3 & & & \faCheck\tabularnewline
B.4 & \faCheck & &\tabularnewline
B.5 & & \faCheck &\tabularnewline
C.1 & \faCheck & & \faCheck\tabularnewline
C.2 & \faCheck & & \faCheck\tabularnewline
C.3 & \faCheck & \faCheck & \faCheck\tabularnewline
\bottomrule
\end{longtable}

\subsection*{Kurslitteratur}

Den studerande väljer i samråd med handledare och examinator ut lämplig
litteratur för aktuellt område.

\subsection*{Övrigt}

Kursen genomförs på ett sätt sådant att både kvinnor och mäns kunskap och erfarenhet utvecklas och görs synlig.
\pagebreak
\section*{4DV001 - Modellering och simulering av system (5 hp)}

\begin{tabular}{ll}\emph{Huvudområde}: & Datavetenskap\tabularnewline\emph{Fördjupning}: & A1N\tabularnewline\end{tabular}

\subsection*{Förkunskaper}

\begin{itemize}
\tightlist
\item
  1DV004 - Objektorienterad programmering
\item
  1MA004 - Tillämpad sannolikhetslära och statistik
\item
  1MA007 - Numeriska metoder
\end{itemize}

\subsection*{Lärandemål}

Efter slutförd kurs skall studenten kunna:

\begin{enumerate}
\def\labelenumi{\Alph{enumi}.}
\tightlist
\item
  \emph{Kunskap och förståelse}

  \begin{enumerate}
  \def\labelenumii{\Alph{enumi}.\arabic{enumii}.}
  \tightlist
  \item
    Redogöra för grundläggande begrepp inom modellering och simulering,
    samt
  \item
    beskriva olika klasser av simulering, t.ex. händelsestyrd, samt när
    de bör tillämpas.
  \end{enumerate}
\item
  \emph{Färdighet och förmåga}

  \begin{enumerate}
  \def\labelenumii{\Alph{enumi}.\arabic{enumii}.}
  \tightlist
  \item
    Implementera enklare simuleringar från utvalda klasser såsom
    händelsestyr, kontinuerlig och agent-baserad,
  \item
    konstruera lämpliga deterministiska och icke-deterministiska
    modeller i ett verktyg för att simulera ett givet system, samt
  \item
    givet ett problem, välja en lämplig simuleringsmetod.
  \end{enumerate}
\item
  \emph{Värderingsförmåga och förhållningssätt}

  \begin{enumerate}
  \def\labelenumii{\Alph{enumi}.\arabic{enumii}.}
  \tightlist
  \item
    Reflektera över olika metoders lämplighet för olika typer av problem
    och vilka konsekvenserna av en olämplig metod blir,
  \item
    värdera resultatet av en simulering t.ex. med avseende på faktorer
    som prestanda eller tillförlitlighet, samt
  \item
    reflektera över värdet av att kunna simulera system innan de
    konstrueras, och vilka samhällsfördelar det ger.
  \end{enumerate}
\end{enumerate}

\subsection*{Kursinnehåll}

Kursen ger en översikt över olika deterministiska och
icke-deterministiska modellerings och simuleringsansatser

Följande moment behandlas:

\begin{itemize}
\tightlist
\item
  Introduktion till modellering och simulering.
\item
  Hur modellering och simulering används.
\item
  Händelsestyrd simulering (Discrete event).
\item
  Kontinuerlig simulering.
\item
  Kömodellering.
\item
  Agent-baserad modellering och simulering.
\item
  Icke-deterministisk och stokastisk modellering och simulering.
\item
  Metoder för sampling.
\item
  Monte Carlo-simulering.
\item
  Validering av simulering, hypotesprövning, ovanliga händelser.
\item
  Verktyg och programbibliotek för modellering och simulering, t.ex.
  Simulink och Modelica.
\end{itemize}

\subsection*{Undervisnings- och
arbetsformer}

Undervisningen består av föreläsningar, seminarier och lärarledda
laborationer. Kursen innehåller även en serie gästföreläsningar där
industrirepresentanter och forskare presenterar hur och varför de
använder simulering.

\subsection*{Examination}

Examinationen av kursen delas in följande moment:

\begin{longtable}[]{@{}llcc@{}}
\toprule
\textsf{Kod} & \textsf{Benämning} & \textsf{Betyg} & \textsf{Poäng}\tabularnewline
\midrule
\endhead
\texttt{LAB1} & Programmeringsuppgifter & A-F & 2,5\tabularnewline
\texttt{HEM1} & Hemtentamen & A-F & 2,5\tabularnewline
\bottomrule
\end{longtable}

För godkänt betyg på kursen krävs minst betyg E på samtliga moment.
Slutbetyget bestäms från: \texttt{HEM1} (50~\%) och \texttt{LAB1} (50~\%).

\subsection*{Måluppfyllelse}

Examinationsmomenten kopplas till lärandemålen enligt följande:

\begin{longtable}[]{@{}lcc@{}}
\toprule
\textsf{Lärandemål} & \texttt{LAB1} & \texttt{HEM1}\tabularnewline
\midrule
\endhead
A.1 & & \faCheck\tabularnewline
A.2 & & \faCheck\tabularnewline
B.1 & \faCheck &\tabularnewline
B.2 & \faCheck &\tabularnewline
B.3 & \faCheck & \faCheck\tabularnewline
C.1 & & \faCheck\tabularnewline
C.2 & \faCheck & \faCheck\tabularnewline
C.3 & & \faCheck\tabularnewline
\bottomrule
\end{longtable}

\subsection*{Kurslitteratur}

\begin{itemize}
\tightlist
\item
  Sokolowski, J. A. och Banks, C. M., \emph{Principles of Modeling and
  Simulation : A Multidisciplinary Approach}, Wiley, 2009. Antal sidor
  153 av 256.
\item
  Birta, L. G. och Arbez, G., \emph{Modelling and Simulation: Exploring
  Dynamic System Behaviour}, Springer, 2013. Antal sidor 341 av 433.
\end{itemize}

\subsection*{Övrigt}

Kursen genomförs på ett sätt sådant att både kvinnor och mäns kunskap och erfarenhet utvecklas och görs synlig.
\pagebreak
\section*{4DV002 - Kompilatorkonstruktion (5 hp)}

\begin{tabular}{ll}\emph{Huvudområde}: & Datavetenskap\tabularnewline\emph{Fördjupning}: & A1N\tabularnewline\end{tabular}

\subsection*{Förkunskaper}

\begin{itemize}
\tightlist
\item
  1DV004 - Objektorienterad programmering
\item
  1DV006 - Algoritmer
\item
  1MA001 - Diskret matematik
\item
  2DV001 - Datorns uppbyggnad
\item
  2DV002 - Mjukvaruarkitektur
\end{itemize}

\subsection*{Lärandemål}

Efter slutförd kurs skall studenten kunna:

\begin{enumerate}
\def\labelenumi{\Alph{enumi}.}
\tightlist
\item
  \emph{Kunskap och förståelse}

  \begin{enumerate}
  \def\labelenumii{\Alph{enumi}.\arabic{enumii}.}
  \tightlist
  \item
    Beskriva en kompilators olika faser,
  \item
    förklara olika parsningstekniker,
  \item
    förklara vad sker i den semantiska analysen,
  \item
    förklara hur typsystem fungerar för några vanliga programspråk, samt
  \item
    förklara hur en stackmaskin fungerar.
  \end{enumerate}
\item
  \emph{Färdighet och förmåga}

  \begin{enumerate}
  \def\labelenumii{\Alph{enumi}.\arabic{enumii}.}
  \tightlist
  \item
    Definiera en finit tillståndsmaskin och en LL(1)-kontextfri
    grammatik för enkla programspråk,
  \item
    designa och implementera en semantisk analys med felhantering,
    typinterferens och som dekorerar syntaxträdet med typinformation,
  \item
    implementera en parser med hjälp av ett givet
    parsergeneratorverktyg, samt
  \item
    designa och implementera en stackmaskinbaserad virtuell maskin.
  \end{enumerate}
\item
  \emph{Värderingsförmåga och förhållningssätt}

  \begin{enumerate}
  \def\labelenumii{\Alph{enumi}.\arabic{enumii}.}
  \tightlist
  \item
    Värdera svårigheten i att hantera olika programkonstruktioner, samt
  \item
    välja och reflektera över lämplig formell notation för att beskriva
    ett givet formellt språk.
  \end{enumerate}
\end{enumerate}

\subsection*{Kursinnehåll}

Kursen presenterar tekniker, teorier och verktyg som används då man
utvecklar en kompilator. Kursen diskuterar också hur dessa idéer kan
användas för att definiera, hantera och interpretera domänspecifika
språk inom modelldriven programvaruutveckling. Ett fokus blir därför
kompilatorns frontend, generering av mellannivå-representationer, och
hur dessa representationer kan exekveras.

Följande moment behandlas:

\begin{itemize}
\tightlist
\item
  Kompilatorns olika faser.
\item
  Objekt­orienterad kompilatordesign.
\item
  Lexikalanalys med hjälp av finita automater och reguljära uttryck.
\item
  Kontextfria grammatiker och språk.
\item
  Olika parsningtekniker för kontextfria språk.
\item
  Typsystem och typinterferens.
\item
  Attribuerade grammatiker.
\item
  Semantisk analys.
\item
  Mellannivå-representationer.
\item
  Kodgenerering.
\item
  Stackmaskiner.
\end{itemize}

\subsection*{Undervisnings- och
arbetsformer}

Undervisningen består av föreläsningar och lärarhandledd hantering av
inlämnings- och programmeringsuppgifter. Inlämningsuppgifterna är
individuella, programmeringsuppgifterna sker i par.

\subsection*{Examination}

Examinationen av kursen delas in i följande moment:

\begin{longtable}[]{@{}llcc@{}}
\toprule
\textsf{Kod} & \textsf{Benämning} & \textsf{Betyg} & \textsf{Poäng}\tabularnewline
\midrule
\endhead
\texttt{UPG1} & Inlämningsuppgifter & A-F & 1\tabularnewline
\texttt{LAB1} & Programmeringsuppgifter & A-F & 2\tabularnewline
\texttt{TEN1} & Skriftlig tentamen & A-F & 2\tabularnewline
\bottomrule
\end{longtable}

För godkänt betyg på kursen krävs minst betyg E på samtliga moment.
Slutbetyget bestäms från: \texttt{UPG1} (20~\%), \texttt{LAB1} (40~\%) och \texttt{TEN1} (40~\%).

\subsection*{Måluppfyllelse}

Examinationsmomenten på kursen kopplas till lärandemålen enligt
följande:

\begin{longtable}[]{@{}lccc@{}}
\toprule
\textsf{Lärandemål} & \texttt{UPG1} & \texttt{LAB1} & \texttt{TEN1}\tabularnewline
\midrule
\endhead
A.1 & & & \faCheck\tabularnewline
A.2 & & & \faCheck\tabularnewline
A.3 & & & \faCheck\tabularnewline
A.4 & & & \faCheck\tabularnewline
A.5 & & & \faCheck\tabularnewline
B.1 & \faCheck & &\tabularnewline
B.2 & \faCheck & \faCheck &\tabularnewline
B.3 & & \faCheck &\tabularnewline
B.4 & & \faCheck &\tabularnewline
C.1 & & & \faCheck\tabularnewline
C.2 & & & \faCheck\tabularnewline
\bottomrule
\end{longtable}

\subsection*{Kurslitteratur}

Obligatorisk litteratur:

\begin{itemize}
\tightlist
\item
  Aho, A. V., Lam, M. S., Sethi, R. och Ullman, J. D., \emph{Compilers:
  Principles, Techniques, and Tools}, Pearson Education, 2006. Antal
  sidor: 510 av 986.
\end{itemize}

\subsection*{Övrigt}

Kursen genomförs på ett sätt sådant att både kvinnor och mäns kunskap och erfarenhet utvecklas och görs synlig.
\pagebreak
\section*{4DV003 - Formella metoder (5 hp)}

\begin{tabular}{ll}\emph{Huvudområde}: & Datavetenskap\tabularnewline\emph{Fördjupning}: & A1F\tabularnewline\end{tabular}

\subsection*{Förkunskaper}

\begin{itemize}
\tightlist
\item
  1DV003 - Databaser och datamodellering
\item
  1DV005 - Jämnlöpande program
\item
  1DV006 - Algoritmer
\item
  2DV006 - Datorsäkerhet
\item
  1MA001 - Diskret matematik
\item
  4DV001 - Modellering och simulering av system
\end{itemize}

\subsection*{Lärandemål}

Efter slutförd kurs skall studenten kunna:

\begin{enumerate}
\def\labelenumi{\Alph{enumi}.}
\tightlist
\item
  \emph{Kunskap och förståelse}

  \begin{enumerate}
  \def\labelenumii{\Alph{enumi}.\arabic{enumii}.}
  \tightlist
  \item
    Resonera kring vad säkerhet (security och safety) betyder för ett
    mjukvaruprogram eller system och vilka egenskaper som krävs.
  \item
    beskriva metoder för och svårigheter med att formellt verifiera
    egenskaper relaterade till säkerhet och korrekthet hos
    mjukvarusystem samt vilka begränsningar olika metoder har,
  \item
    redogöra för de senaste rönen inom formella metoder och
    verifikation, samt
  \item
    redogöra för hur runtime-övervakning kan användas för att genomdriva
    säkerhetskrav.
  \end{enumerate}
\item
  \emph{Färdighet och förmåga}

  \begin{enumerate}
  \def\labelenumii{\Alph{enumi}.\arabic{enumii}.}
  \tightlist
  \item
    Uttrycka säkerhetsegenskaper hos (jämnlöpande) system och
    mjukvaruprogram formellt med hjälp av olika typer av logik,
  \item
    använda olika metoder för att verifiera korrekthet och säkerhet hos
    system under modellering och mjukvaruprogram, samt
  \item
    använda de vanligaste verktygen och programmen för att beskriva och
    verifiera system och mjukvaruprogram.
  \end{enumerate}
\item
  \emph{Värderingsförmåga och förhållningssätt}

  \begin{enumerate}
  \def\labelenumii{\Alph{enumi}.\arabic{enumii}.}
  \tightlist
  \item
    Resonera kring vilka samhällskostnader (och konsekvenser) felaktig
    och osäker mjukvara medför samt hur formella metoder kan spela in
    för att t.ex. reglera mjukvara inom vissa domäner.
  \end{enumerate}
\end{enumerate}

\subsection*{Kursinnehåll}

Kursen ger en introduktion till formell verifikation. Den bygger vidare
på sats och predikatlogik och introducerar t.ex. logiker som tar hänsyn
till tid.

Följande moment behandlas:

\begin{itemize}
\tightlist
\item
  Introduktion till formell verifikation.
\item
  Klassifikation av olika verifikationstekniker.
\item
  Processalgebra (CCS), samt hur dessa kan utökas med tidsfördröjningar.
\item
  Tidstillståndsmaskin (timed automata).
\item
  Fördjupning av sats- och predikatlogik, samt temporallogik (LTL och
  CTL).
\item
  Programspråkssemantik.
\item
  Programverifikation med hjälp av Hoare-logik och separationslogik.
\item
  Modellkontroll med hjälp av LTL och CTL.
\item
  Runtime-övervakning.
\end{itemize}

\subsection*{Undervisnings- och
arbetsformer}

Undervisningen består av föreläsningar och lärarledda laborationer.
Kursen innehåller även en serie gästföreläsningar där
industrirepresentanter och forskare presenterar hur och varför de
använder formell verifikation samt vilka metoder och verktyg de
använder.

\subsection*{Examination}

Examinationen av kursen delas in i följande moment:

\begin{longtable}[]{@{}llcc@{}}
\toprule
\textsf{Kod} & \textsf{Benämning} & \textsf{Betyg} & \textsf{Poäng}\tabularnewline
\midrule
\endhead
\texttt{UPG1} & Formell verifikation av ett mjukvaruprogram & A-F &
1\tabularnewline
\texttt{UPG2} & Formell verifikation av en systemmodell & A-F & 1\tabularnewline
\texttt{TEN1} & Skriftlig tentamen & A-F & 3\tabularnewline
\bottomrule
\end{longtable}

För godkänt betyg på kursen krävs minst betyg E på samtliga moment.
Slutbetyget bestäms från: \texttt{UPG1} (20~\%), \texttt{UPG2} (20~\%) och \texttt{TEN1} (60~\%).

\subsection*{Måluppfyllelse}

Examinationsmomenten kopplas till lärandemålen enligt följande:

\begin{longtable}[]{@{}lccc@{}}
\toprule
\textsf{Lärandemål} & \texttt{UPG1} & \texttt{UPG2} & \texttt{TEN1}\tabularnewline
\midrule
\endhead
A.1 & \faCheck & \faCheck & \faCheck\tabularnewline
A.2 & & & \faCheck\tabularnewline
A.3 & \faCheck & \faCheck & \faCheck\tabularnewline
A.4 & & & \faCheck\tabularnewline
B.1 & \faCheck & \faCheck & \faCheck\tabularnewline
B.2 & \faCheck & \faCheck &\tabularnewline
B.3 & \faCheck & \faCheck &\tabularnewline
C.1 & & & \faCheck\tabularnewline
\bottomrule
\end{longtable}

\subsection*{Kurslitteratur}

Obligatorisk litteratur:

\begin{itemize}
\tightlist
\item
  Aceto, L., Ingólfsdóttir, A., Guldstrand Larsen, K. och Srba, J.,
  \emph{Reactive Systems: Modelling, Specification and Verification},
  Cambridge University Press, 2014. Antal sidor: 150 av 281.
\item
  Huth, M. och Ryan, M., \emph{Logic in Computer Science: Modelling and
  Reasoning about Systems}, andra utgåvan, Cambridge University Press,
  2004. Antal sidor: 300 av 412.
\item
  Kompendium med vetenskapliga artiklar
\end{itemize}

\subsection*{Övrigt}

Kursen genomförs på ett sätt sådant att både kvinnor och mäns kunskap och erfarenhet utvecklas och görs synlig.
\pagebreak
\section*{2MA001 - Optimering (5 hp)}

\begin{tabular}{ll}
\emph{Huvudområde}: & Matematik\tabularnewline
\emph{Fördjupning}: & G2F\tabularnewline
\end{tabular}

\subsection*{Förkunskaper}

\begin{itemize}
\tightlist
\item
  1DV001 - Programmering och datastrukturer
\item
  1DV006 - Algoritmer
\item
  1MA002 - Linjär algebra
\item
  1MA007 - Numeriska metoder
\end{itemize}

\subsection*{Lärandemål}

Efter slutförd kurs skall studenten kunna:

\begin{enumerate}
\def\labelenumi{\Alph{enumi}.}
\tightlist
\item
  \emph{Kunskap och förståelse}

  \begin{enumerate}
  \def\labelenumii{\arabic{enumii}.}
  \tightlist
  \item
    Redogöra för och använda centrala begrepp inom optimering, såsom
    lokal och global optimalitet, konvexitet, svag och stark dualitet,
    samt giltiga olikheter,
  \item
    förklara skillnaden mellan heuristiker och approximativa algoritmer,
  \item
    beskriva och tillämpa grundläggande metodprinciper för att lösa
    några vanligt förekommande typer av optimeringsproblem, som t.ex.
    trädsökning för diskreta problem, samt
  \item
    identifiera frågeställningar av optimeringskaraktär inom teknik och
    ekonomi och klassificera optimeringsproblem utifrån deras
    egenskaper, som till exempel i linjära respektive olinjära problem
    eller i kontinuerliga respektive diskreta problem.
  \end{enumerate}
\item
  \emph{Färdighet och förmåga}

  \begin{enumerate}
  \def\labelenumii{\arabic{enumii}.}
  \tightlist
  \item
    Konstruera matematiska modeller av enkla optimeringsproblem samt
    bedöma deras svårighetsgrad med komplexitetsteori,
  \item
    implementera några grundläggande metoder för optimering och använda
    dem för att lösa optimeringsproblem,
  \item
    använda befintliga program t.ex. Matlab för att lösa
    optimeringsproblem, samt
  \item
    implementera en enkel heuristik för ett strukturerat kombinatoriskt
    optimeringsproblem.
  \end{enumerate}
\item
  \emph{Värderingsförmåga och förhållningssätt}

  \begin{enumerate}
  \def\labelenumii{\arabic{enumii}.}
  \tightlist
  \item
    Reflektera över problem inom mjukvaruutveckling där
    optimeringsmetoder kan användas samt motivera val av metod baserat
    på metodens förmåga att uppnå ett visst resultat, samt
  \item
    diskutera användning av optimeringsmetodik för hushållning med
    personella resurser och begränsning av miljöpåverkan av industriell
    och logistisk verksamhet, samt kunna identifiera sådana
    tillämpningar av optimeringslära.
  \end{enumerate}
\end{enumerate}

\subsection*{Kursinnehåll}

Följande moment behandlas:

\begin{itemize}
\tightlist
\item
  Modellering av optimeringsproblem.
\item
  Linjärprogrammering och simplexmetoden.
\item
  Känslighetsanalys.
\item
  Dualitet.
\item
  Nätverksoptimering (t ex billigaste uppspännande träd, billigaste
  vägproblem, minkostnadsflödesproblem).
\item
  Introduktion till icke­linjär programmering.
\item
  Metoder för obegränsad optimering.
\item
  Heltalsoptimering.
\item
  Trädsökning.
\item
  Heuristiker.
\end{itemize}

\subsection*{Undervisnings- och
arbetsformer}

Undervisningen sker i form av föreläsningar, lärarledda räkneövningar
och datorlaborationer. Projektuppgifter sker i par. Obligatorisk närvaro
kan förekomma på vissa moment.

\subsection*{Examination}

Examinationen av kursen delas in i följande moment:

\begin{longtable}[]{@{}llcc@{}}
\toprule
\textsf{Kod} & \textsf{Benämning} & \textsf{Betyg} & \textsf{Poäng}\tabularnewline
\midrule
\endhead
TEN1 & Skriftlig tentamen & A-F & 2\tabularnewline
UPG1 & Inlämningsuppgifter & A-F & 2\tabularnewline
UPG2 & Optimeringsprojekt & A-F & 1\tabularnewline
\bottomrule
\end{longtable}

För godkänt betyg på kursen krävs minst betyg E på samtliga moment.
Slutbetyget bestäms från: TEN1 (40\%), UPG1 (40\%) och UPG2 (20\%).

\subsection*{Måluppfyllelse}

Examinationsmomenten på kursen kopplas till lärandemålen enligt
följande:

\begin{longtable}[]{@{}lccc@{}}
\toprule
Lärandemål & TEN1 & UPG1 & UPG2\tabularnewline
\midrule
\endhead
A.1 & \textbf{X} & &\tabularnewline
A.2 & \textbf{X} & &\tabularnewline
A.3 & \textbf{X} & &\tabularnewline
A.4 & \textbf{X} & \textbf{X} &\tabularnewline
B.1 & \textbf{X} & \textbf{X} &\tabularnewline
B.2 & \textbf{X} & \textbf{X} &\tabularnewline
B.3 & & \textbf{X} & \textbf{X}\tabularnewline
B.4 & & \textbf{X} &\tabularnewline
C.1 & & & \textbf{X}\tabularnewline
C.1 & & \textbf{X} &\tabularnewline
\bottomrule
\end{longtable}

\subsection*{Kurslitteratur}

\emph{Obligatorisk litteratur}:

\begin{itemize}
\tightlist
\item
  Lundgren, J., Rönnqvist, M. och Värbrand, P., \emph{Optimization},
  Studentlitteratur, 2010. Antal sidor: 421 av 537.
\end{itemize}

\subsection*{Övrigt}

Kursen genomförs på ett sätt sådant att både kvinnor och mäns kunskap och erfarenhet utvecklas och görs synlig.
\pagebreak
\section*{4DV004 - Projekt i modellbaserad utveckling (10 hp)}

\begin{tabular}{ll}\emph{Huvudområde}: & Datavetenskap\tabularnewline\emph{Fördjupning}: & A1N\tabularnewline\end{tabular}

\subsection*{Förkunskaper}

\begin{itemize}
\tightlist
\item
  1DV004 - Objektorienterad programmering
\item
  1DV007 - Mjukvaruutvecklingsprojekt
\item
  2DV002 - Mjukvaruarkitektur
\item
  1MA001 - Diskret matematik
\item
  1ZT001 - Teknisk kommunikation
\item
  1ZT002 - Hållbar utveckling
\item
  1ZT003 - Industriell ekonomi
\end{itemize}

\subsection*{Lärandemål}

Efter slutförd kurs skall studenten kunna:

\begin{enumerate}
\def\labelenumi{\Alph{enumi}.}
\tightlist
\item
  \emph{Kunskap och förståelse}

  \begin{enumerate}
  \def\labelenumii{\Alph{enumi}.\arabic{enumii}.}
  \tightlist
  \item
    Klassificera och förklara centrala principer och koncept inom
    modellbaserad utveckling såsom modeller, meta-modeller,
    begränsningar, transformationer, semantik, abstrakt och konkret
    syntax, samt
  \item
    beskriva arkitekturen hos samtida modelleringsramverk samt hur
    domänspecifika modelleringsspråk kan formuleras med hjälp av dessa.
  \end{enumerate}
\item
  \emph{Färdighet och förmåga}

  \begin{enumerate}
  \def\labelenumii{\Alph{enumi}.\arabic{enumii}.}
  \tightlist
  \item
    Självständigt lära sig att använda olika ramverk och verktyg för
    modellbaserad utveckling,
  \item
    utifrån krav som samlats in från kund modellera ett system som
    uppfyller dessa krav,
  \item
    givet en uppsättning modeller och ett ramverk, skapa modelleditorer,
    modellkontroller och modelltransformationer,
  \item
    använda modellbaserad utveckling för att skapa en exekverbar
    mjukvara och säkerställa egenskaper hos denna, samt
  \item
    planera ett agilt projekt, t.ex. beskriva krav, prioritera dessa och
    uppskatta hur mycket tid som krävs för att implementera dem.
  \end{enumerate}
\item
  \emph{Värderingsförmåga och förhållningssätt}

  \begin{enumerate}
  \def\labelenumii{\Alph{enumi}.\arabic{enumii}.}
  \tightlist
  \item
    Kritiskt reflektera över för- och nackdelar med modellbaserad
    utveckling från ett mjukvaruutvecklingsperspektiv, t.ex. med
    avseende på hur lång tid olika uppgifter tar, hur smidiga verktygen
    är, osv.,
  \item
    resonera kring vilka för- och nackdelar modelldriven utveckling kan
    ha ur ett samhällsperspektiv, t.ex. med avseende på säkerhetskrav,
    ekonomisk vinning osv., samt
  \item
    analysera hur väl ett agilt arbetssätt fungerade inom ett projekt
    från arbetsmiljösynpunkt och föreslå möjliga förbättringar.
  \end{enumerate}
\end{enumerate}

\subsection*{Kursinnehåll}

Kursen är en projektkurs som med hjälp av ett realistiskt problem och
realistiska förutsättningar behandlar hela CDIO-cykeln. Studenterna
rätts i rollen som ett utvecklingsteam inom en industriell agil
organisation med stora krav på sin mjukvara.

Studenterna förväntas jobba agilt i grupper om 5-7 och förväntas besätta
alla roller utom produktägare. Detta är den första av tre kurser under
vilka studenterna förväntas fördjupa sina färdigheter i att jobba agilt.
Fokus under denna kurs är på hur krav samlas in och beskrivs, planering
och estimering, samt dokumentation.

\begin{itemize}
\tightlist
\item
  Fördjupning inom modeller och mjukvaruutveckling.
\item
  Modellbaserad utveckling och arkitektur.
\item
  Diskussion kring problem hos mjukvara, t.ex. säkerhetsproblem,
  kraschar, prestanda.
\item
  Diskussion kring mjukvara för olika domäner, och vilka krav som ställs
  på dessa.
\item
  Fördelar och nackdelar med modellbaserad utveckling.
\item
  Modelleringsspråk, metamodellering, profilering.
\item
  Modelltranformationer och modellbegränsningar.
\item
  Händelsespråk (Action languages).
\item
  Domänspecifika språk.
\item
  Modellbaserad testning.
\item
  Modellvalidering.
\item
  Automatisk kodgenerering.
\item
  Verktyg för modellbaserad utveckling.
\item
  Olika sätt att uppskatta tid i agila projekt.
\item
  Olika sätt att fånga och beskriva krav i agila projekt.
\item
  Olika sätt att dokumentera mjukvara i agila projekt och kopplingen
  till modeller och programkod.
\item
  Hur man skriver reflektionsrapporter och post-mortem analyser av
  projekt.
\end{itemize}

\subsection*{Undervisnings- och
arbetsformer}

Föreläsningar och handledningsmöten. Under föreläsningarna sätter
läraren upp ramar för projektet samt presenterar de verktyg, metoder och
resurser studenterna förväntas använda under projektet. Under projektets
gång kommer studenterna att ha regelbundna handledningsmöten en lärare.

\subsection*{Examination}

Examinationen av kursen delas in följande moment:

\begin{longtable}[]{@{}llcc@{}}
\toprule
\textsf{Kod} & \textsf{Benämning} & \textsf{Betyg} & \textsf{Poäng}\tabularnewline
\midrule
\endhead
\texttt{UPG1} & Vision och planeringsdokument & A-F & 2\tabularnewline
\texttt{PRJ1} & Projektarbete (inkl. leverabler) & A-F & 5\tabularnewline
\texttt{RAP1} & Reflektionsrapport - Modellbaserad utveckling & A-F &
1\tabularnewline
\texttt{RAP2} & Reflektionsrapport - Projektarbete & A-F & 1\tabularnewline
\texttt{PRS1} & Design, implementering och resultat & A-F & 1\tabularnewline
\bottomrule
\end{longtable}

För godkänt betyg på kursen krävs minst betyg E på samtliga moment.
Slutbetyget bestäms från: \texttt{UPG1} (20~\%), \texttt{PRJ1} (50~\%), \texttt{RAP1} (10~\%), \texttt{RAP2}
(10~\%) och \texttt{PRS1} (10~\%).

\subsection*{Måluppfyllelse}

Examinationsmomenten kopplas till lärandemålen enligt följande:

\begin{longtable}[]{@{}lccccc@{}}
\toprule
\textsf{Lärandemål} & \texttt{UPG1} & \texttt{PRJ1} & \texttt{RAP1} & \texttt{RAP2} & \texttt{PRS1}\tabularnewline
\midrule
\endhead
A.1 & \faCheck & & \faCheck & \faCheck & \faCheck\tabularnewline
A.2 & & & \faCheck & &\tabularnewline
B.1 & \faCheck & \faCheck & & &\tabularnewline
B.2 & \faCheck & \faCheck & \faCheck & & \faCheck\tabularnewline
B.3 & & \faCheck & \faCheck & & \faCheck\tabularnewline
B.4 & & \faCheck & \faCheck & & \faCheck\tabularnewline
B.5 & & \faCheck & & \faCheck & \faCheck\tabularnewline
C.1 & & & \faCheck & & \faCheck\tabularnewline
C.2 & & & \faCheck & &\tabularnewline
C.3 & & & \faCheck & \faCheck & \faCheck\tabularnewline
\bottomrule
\end{longtable}

\subsection*{Kurslitteratur}

Studenterna förväntas söka efter lämplig kurslitteratur på egen hand.
Nedanstående referenslitteratur kan användas som en utgångspunkt.

\begin{itemize}
\tightlist
\item
  Brambilla, M., Cabot, J. och Wimmer, M., \emph{Model-Driven Software
  Engineering in Practice}, andra utgåvan, Morgan \& Claypool
  Publishers. 2017.
\item
  Steinberg, D., Budinsky, F., Paternostro, M. och Merks, E., \emph{EMF:
  Eclipse Modeling Framework}, andra utgåvan, Addison-Wesley
  Professional. 2008.
\item
  Kelly, S. och Tolvanen, J-P., \emph{Domain-Specific Modeling},
  Wiley-IEEE Computer Society Press, 2008.
\item
  Mellor, S. J. och Balcer, M. J., \emph{Executable UML: A Foundation
  for Model-Driven Architecture}, Addison-Wesley Professional. 2002.
\item
  Royer, J. och Arboleda, H., \emph{Model-Driven and Software Product
  Line Engineering}, John Wiley \& Sons, Inc.~2013.
\end{itemize}

\subsection*{Övrigt}

Kursen genomförs på ett sätt sådant att både kvinnor och mäns kunskap och erfarenhet utvecklas och görs synlig.
\pagebreak
\section*{4DV005 - Maskininlärning (5 hp)}

\begin{tabular}{ll}\emph{Huvudområde}: & Datavetenskap\tabularnewline\emph{Fördjupning}: & A1N\tabularnewline\end{tabular}

\subsection*{Förkunskaper}

\begin{itemize}
\tightlist
\item
  1MA002 - Linjär algebra
\item
  1MA004 - Tillämpad sannolikhetslära och statistik
\item
  1MA006 - Flervariabelanalys
\item
  1DV001 - Programmering och datastrukturer
\end{itemize}

\subsection*{Lärandemål}

Efter slutförd kurs skall studenten kunna:

\begin{enumerate}
\def\labelenumi{\Alph{enumi}.}
\tightlist
\item
  \emph{Kunskap och förståelse}

  \begin{enumerate}
  \def\labelenumii{\Alph{enumi}.\arabic{enumii}.}
  \tightlist
  \item
    Översiktligt redogöra för olika områden inom artificiell
    intelligens,\\
  \item
    redogöra för grundläggande principer och tillämpningar inom
    maskininlärning,
  \item
    redogöra för svagheter och fördelar med olika
    maskininlärningsalgoritmer, samt
  \item
    redogöra för de olika inlärningsparadigmen i maskininlärning.
  \end{enumerate}
\item
  \emph{Färdighet och förmåga}

  \begin{enumerate}
  \def\labelenumii{\Alph{enumi}.\arabic{enumii}.}
  \tightlist
  \item
    Implementera algoritmer för att lösa typiska
    maskininlärningsproblem,
  \item
    representera data för att underlätta lärandet,
  \item
    utvärdera prestanda hos en modell och kunna välja en lämplig modell
    för ett givet problem
  \item
    känna igen typiska effekter av olämpliga initialiseringsvärden och
    parameterval, och föreslå sätt att förbättra resultaten, samt
  \item
    känna igen fall av över- och underanpassning av modeller och föreslå
    sätt att hantera dem.
  \end{enumerate}
\item
  \emph{Värderingsförmåga och förhållningssätt}

  \begin{enumerate}
  \def\labelenumii{\Alph{enumi}.\arabic{enumii}.}
  \tightlist
  \item
    Resonera kring vilka effekter t.ex. bias från träningsdata får i
    faktiska tillämpningar.
  \end{enumerate}
\end{enumerate}

\subsection*{Kursinnehåll}

Kursen behandlar grundläggande begrepp och metoder inom maskininlärning.

Följande moment behandlas:

\begin{itemize}
\tightlist
\item
  Översikt av området artificiell intelligens.
\item
  Grundläggande principer för maskininlärning.
\item
  Förbehandling av data, särdrags extrahering, dimensionsreducering.
\item
  Modellval, generalisering, över- och underanpassning.
\item
  Optimering för träningsmodellmodeller.
\item
  Regression.
\item
  Närmaste-granne klassificeringar.
\item
  Logistisk regression.
\item
  Naiva Bayes.
\item
  Beslutsträd.
\item
  Neurala nätverk.
\item
  Ensemblemetoder.
\item
  Kärnmetoder och supportvektormaskiner.
\item
  k-medel-klustring och hierarkisk klustring.
\end{itemize}

\subsection*{Undervisnings- och
arbetsformer}

Undervisningen består av föreläsningar och lärarledda laborationer.
Laborationer demonstreras och redovisas för läraren.

\subsection*{Examination}

Examinationen av kursen delas in i följande moment:

\begin{longtable}[]{@{}llcc@{}}
\toprule
\textsf{Kod} & \textsf{Benämning} & \textsf{Betyg} & \textsf{Poäng}\tabularnewline
\midrule
\endhead
\texttt{TEN1} & Skriftlig tentamen & A-F & 2,5\tabularnewline
\texttt{LAB1} & Programmeringsuppgifter & A-F & 2,5\tabularnewline
\bottomrule
\end{longtable}

För godkänt betyg på kursen krävs minst betyg E på samtliga moment.
Slutbetyget bestäms från: \texttt{TEN1} (50~\%) och \texttt{LAB1} (50~\%).

\subsection*{Måluppfyllelse}

Examinationsmomenten på kursen kopplas till lärandemålen enligt
följande:

\begin{longtable}[]{@{}lcc@{}}
\toprule
\textsf{Lärandemål} & \texttt{TEN1} & \texttt{LAB1}\tabularnewline
\midrule
\endhead
A.1 & \faCheck &\tabularnewline
A.2 & \faCheck &\tabularnewline
A.3 & \faCheck & \faCheck\tabularnewline
A.4 & \faCheck &\tabularnewline
B.1 & & \faCheck\tabularnewline
B.2 & \faCheck & \faCheck\tabularnewline
B.3 & \faCheck & \faCheck\tabularnewline
B.4 & \faCheck & \faCheck\tabularnewline
B.5 & \faCheck & \faCheck\tabularnewline
C.1 & \faCheck &\tabularnewline
\bottomrule
\end{longtable}

\subsection*{Kurslitteratur}

Obligatorisk litteratur:

\begin{itemize}
\tightlist
\item
  Bishop, C. M., \emph{Pattern recognition and machine learning}.
  Springer, 2006. Antal sidor: 400 av 700.
\item
  James, G., Witten, D., Hastie, T. och Tibshirani, R., \emph{An
  introduction to statistical learning: with applications in R},
  Springer, 2013. Antal sidor: 350 av 410.
\item
  Kompendium med vetenskapliga artiklar.
\end{itemize}

\subsection*{Övrigt}

Kursen genomförs på ett sätt sådant att både kvinnor och mäns kunskap och erfarenhet utvecklas och görs synlig.
\pagebreak
\section*{4DV006 - Parallelldatorprogrammering (5 hp)}

\begin{tabular}{ll}\emph{Huvudområde}: & Datavetenskap\tabularnewline\emph{Fördjupning}: & A1N\tabularnewline\end{tabular}

\subsection*{Förkunskaper}

\begin{itemize}
\tightlist
\item
  1DV005 - Jämnlöpande program
\item
  1DV006 - Algoritmer
\item
  2DV001 - Datorns uppbyggnad
\item
  2DV004 - Datorgrafik
\end{itemize}

\subsection*{Lärandemål}

Efter slutförd kurs skall studenten kunna:

\begin{enumerate}
\def\labelenumi{\Alph{enumi}.}
\tightlist
\item
  \emph{Kunskap och förståelse}

  \begin{enumerate}
  \def\labelenumii{\Alph{enumi}.\arabic{enumii}.}
  \tightlist
  \item
    Förklara de huvudsakliga likheterna och skillnaderna mellan
    jämnlöpande och parallella program,
  \item
    beskriva olika typer av parallella datorer och acceleratorer samt
    resonera kring vilken typ som bäst lämpar sig för ett givet problem,
    samt
  \item
    redogöra för de senaste rönen inom parallella datorer och
    acceleratorer, samt hur man programmerar dessa på lämpliga sätt.
  \end{enumerate}
\item
  \emph{Färdighet och förmåga}

  \begin{enumerate}
  \def\labelenumii{\Alph{enumi}.\arabic{enumii}.}
  \tightlist
  \item
    Bryta ner problem, formulera parallella algoritmer för att lösa
    dessa, och implementera dessa för olika typer av parallella datorer
    och acceleratorer, t.ex. med hjälp av OpenMP, CUDA, eller MPI,
  \item
    planera och driftsätta ett kluster och lämplig mjukvara (t.ex. MPI)
    för att lösa en viss typ av problem, samt
  \item
    givet ett problem och en implementation, resonera kring förväntad
    prestanda och olika sätt att öka denna.
  \end{enumerate}
\item
  \emph{Värderingsförmåga och förhållningssätt}

  \begin{enumerate}
  \def\labelenumii{\Alph{enumi}.\arabic{enumii}.}
  \tightlist
  \item
    Reflektera kring kostnaden för att lösa vissa typer av problem,
    t.ex. med avseende på faktiska kostnader och energi/miljökostnader,
    samt hur dessa påverkas av val av arkitektur, algoritm, osv.,
  \end{enumerate}
\end{enumerate}

\subsection*{Kursinnehåll}

Kursen ger en fördjupning i hur problem kan lösas med hjälp av
parallelldatorer och acceleratorer, hur problem bryts ner, och hur
program kan optimeras för olika dator- och accelerator-arkitekturer.

Följande moment behandlas:

\begin{itemize}
\tightlist
\item
  Introduktion till homogena och heterogena parallella datorer.
\item
  Introduktion till grafikprocessorer och acceleratorer.
\item
  Fördjupning i hur problem kan brytas ned för parallell exekvering.
\item
  OpenMP.
\item
  Programmering av beräkningsklustrer med hjälp av t.ex. MPI.
\item
  Programmering av grafikprocessorer med hjälp av t.ex. CUDA.
\item
  Parallella mönster såsom prefixsumma, map-reduce, matrisberäkningar,
  merge-sortering och sökning i grafer.
\item
  Exempel på hur parallellprogrammering kan användas inom olika domäner,
  t.ex. maskininlärning, bildbehandling och bildanalys.
\item
  Planering och driftsättning av (virtuella) klustrar i molnet.
\item
  Vanliga benchmarks och hur dessa används för att uppskatta prestanda.
\item
  Enklare verktyg för att testa och felsöka parallella program.
\end{itemize}

\subsection*{Undervisnings- och
arbetsformer}

Undervisningen består av föreläsningar, seminarier och lärarledda
laborationer. Kursen innehåller även en serie gästföreläsningar där
industrirepresentanter och forskare presenterar hur de använder
parallelldatorer och acceleratorer samt vilken typ av problem de löser
med hjälp av dessa.

\subsection*{Examination}

Examinationen av kursen delas in följande moment:

\begin{longtable}[]{@{}llcc@{}}
\toprule
\textsf{Kod} & \textsf{Benämning} & \textsf{Betyg} & \textsf{Poäng}\tabularnewline
\midrule
\endhead
\texttt{TEN1} & Skriftlig tentamen & A-F & 1\tabularnewline
\texttt{LAB1} & Programmeringsuppgifter & A-F & 4\tabularnewline
\bottomrule
\end{longtable}

För godkänt betyg på kursen krävs minst betyg E på samtliga moment.
Slutbetyget bestäms från: \texttt{TEN1} (30~\%) och \texttt{LAB1} (70~\%).

\subsection*{Måluppfyllelse}

Examinationsmomenten kopplas till lärandemålen enligt följande:

\begin{longtable}[]{@{}lcc@{}}
\toprule
\textsf{Lärandemål} & \texttt{TEN1} & \texttt{LAB1}\tabularnewline
\midrule
\endhead
A.1 & \faCheck & \faCheck\tabularnewline
A.2 & \faCheck &\tabularnewline
A.3 & \faCheck & \faCheck\tabularnewline
B.1 & & \faCheck\tabularnewline
B.2 & & \faCheck\tabularnewline
B.3 & & \faCheck\tabularnewline
C.1 & \faCheck &\tabularnewline
\bottomrule
\end{longtable}

\subsection*{Kurslitteratur}

Obligatorisk litteratur:

\begin{itemize}
\tightlist
\item
  Kirk, D. och Hwu, W-M., \emph{Programming Massively Parallel
  Processors - A Hands-on Approach}, tredje utgåvan, Morgan Kaufmann,
  2016. Antal sidor: 500 / 576
\item
  Kompendium med vetenskapliga artiklar.
\end{itemize}

\subsection*{Övrigt}

Kursen genomförs på ett sätt sådant att både kvinnor och mäns kunskap och erfarenhet utvecklas och görs synlig.
\pagebreak
\section*{4DV007 - Djup maskininlärning (5 hp)}

\begin{tabular}{ll}\emph{Huvudområde}: & Datavetenskap\tabularnewline\emph{Fördjupning}: & A1F\tabularnewline\end{tabular}

\subsection*{Förkunskaper}

\begin{itemize}
\tightlist
\item
  1MA002 - Linjär algebra
\item
  1MA004 - Tillämpad sannolikhetslära och statistik
\item
  1MA006 - Flervariabelanalys
\item
  1DV001 - Programmering och datastrukturer
\item
  4DV005 - Maskininlärning
\end{itemize}

\subsection*{Lärandemål}

Efter slutförd kurs skall studenten kunna:

\begin{enumerate}
\def\labelenumi{\Alph{enumi}.}
\tightlist
\item
  \emph{Kunskap och förståelse}

  \begin{enumerate}
  \def\labelenumii{\Alph{enumi}.\arabic{enumii}.}
  \tightlist
  \item
    Redogöra för grunderna och tillämpningar av djup maskininlärning,
  \item
    redogöra för olika metoder för förstärkt inlärning, planering och
    kontroll i sekventiella beslutsprocesser, samt
  \item
    förklara begränsningarna hos en modell i en given situation.
  \end{enumerate}
\item
  \emph{Färdighet och förmåga}

  \begin{enumerate}
  \def\labelenumii{\Alph{enumi}.\arabic{enumii}.}
  \tightlist
  \item
    Implementera algoritmer inom djup maskininlärning med hjälp av
    moderna ramverk,
  \item
    tillämpa relevanta begrepp från djup maskininlärning för att lösa
    praktiska problem såsom bildigenkänning,
  \item
    representera data för att underlätta lärandet, samt
  \item
    känna igen typiska effekter av olämpliga initialiseringsvärden,
    parameterval och hyperparameterval, och föreslå sätt att förbättra
    resultaten.
  \end{enumerate}
\item
  \emph{Värderingsförmåga och förhållningssätt}

  \begin{enumerate}
  \def\labelenumii{\Alph{enumi}.\arabic{enumii}.}
  \tightlist
  \item
    Värdera, sammanfatta, diskutera och muntligt presentera
    vetenskapliga resultat inom området och resonera kring dess påverkan
    på samhället.
  \end{enumerate}
\end{enumerate}

\subsection*{Kursinnehåll}

Kursen omfattar begrepp och metoder från neurala nätverk och djup
maskininlärning.

Följande moment behandlas:

\begin{itemize}
\tightlist
\item
  Neurala nätverk och faltningsnätverk.
\item
  Optimering vid träning av djupa inlärningsmodeller.
\item
  Regularisering för djup maskininlärning.
\item
  Kalibrering av hyperparametrar.
\item
  Återkommande neurala nätverk.
\item
  Långt korttidsminne.
\item
  Förstärkt maskininlärning.
\end{itemize}

\subsection*{Undervisnings- och
arbetsformer}

Föreläsningar, lärarhandledda laborationer och en seminarieserie där
studenter i par presenterar en vetenskaplig artikel inom maskininlärning
och opponerar på ett annat pars presentation.

\subsection*{Examination}

Examinationen av kursen delas in i följande moment:

\begin{longtable}[]{@{}llcc@{}}
\toprule
\textsf{Kod} & \textsf{Benämning} & \textsf{Betyg} & \textsf{Poäng}\tabularnewline
\midrule
\endhead
\texttt{TEN1} & Skriftlig tentamen & A-F & 2\tabularnewline
\texttt{LAB1} & Programmeringsuppgifter & A-F & 2\tabularnewline
\texttt{PRS1} & Presentation av vetenskaplig artikel & G-U & 1\tabularnewline
\bottomrule
\end{longtable}

För godkänt betyg på kursen krävs betyg G på \texttt{PRS1} samt minst betyg E på
övriga moment. Slutbetyget bestäms från: \texttt{TEN1} (50~\%) och \texttt{LAB1} (50~\%).

\subsection*{Måluppfyllelse}

Examinationsmomenten på kursen kopplas till lärandemålen enligt
följande:

\begin{longtable}[]{@{}lccc@{}}
\toprule
\textsf{Lärandemål} & \texttt{TEN1} & \texttt{LAB1} & \texttt{PRS1}\tabularnewline
\midrule
\endhead
A.1 & & \faCheck &\tabularnewline
A.2 & \faCheck & \faCheck &\tabularnewline
A.3 & \faCheck & \faCheck &\tabularnewline
B.1 & \faCheck & &\tabularnewline
B.2 & \faCheck & &\tabularnewline
B.3 & \faCheck & &\tabularnewline
B.4 & \faCheck & \faCheck &\tabularnewline
C.1 & & & \faCheck\tabularnewline
\bottomrule
\end{longtable}

\subsection*{Kurslitteratur}

Obligatorisk litteratur:

\begin{itemize}
\tightlist
\item
  Goodfellow, I., Bengio, Y. och Courville, A., \emph{Deep learning},
  MIT Press, 2016. Antal sidor: 465 av 710.
\item
  Kompendium med vetenskapliga artiklar.
\end{itemize}

\subsection*{Övrigt}

Kursen genomförs på ett sätt sådant att både kvinnor och mäns kunskap och erfarenhet utvecklas och görs synlig.
\pagebreak
\section*{2ZT002 - Lean startup (5 hp)}

\begin{tabular}{ll}
\emph{Huvudområde}: &\tabularnewline
\emph{Fördjupning}: & G2F\tabularnewline
\end{tabular}

\subsection*{Förkunskaper}

\begin{itemize}
\tightlist
\item
  1DV007 - Mjukvaruutvecklingsprojekt
\item
  1ZT001 - Teknisk kommunikation
\item
  1ZT003 - Industriell ekonomi
\end{itemize}

\subsection*{Lärandemål}

Efter slutförd kurs skall studenten kunna:

\begin{enumerate}
\def\labelenumi{\Alph{enumi}.}
\tightlist
\item
  \emph{Kunskap och förståelse}

  \begin{enumerate}
  \def\labelenumii{\Alph{enumi}.\arabic{enumii}.}
  \tightlist
  \item
    Beskriva de lagliga, etiska och strukturella aspekter som gäller för
    att starta och driva en startup,
  \item
    förklara sammanhang, koncept, teorier och processer kring
    entreprenörskap,
  \item
    beskriva affärsutvecklingsprocessen, från idé till etablerad
    verksamhet, samt
  \item
    beskriva koncept såsom ``minimal viable product'', A/B-testning,
    ``product-market fit'' och ``business model canvas''.
  \end{enumerate}
\item
  \emph{Färdighet och förmåga}

  \begin{enumerate}
  \def\labelenumii{\Alph{enumi}.\arabic{enumii}.}
  \tightlist
  \item
    Utveckla nya eller förändra befintliga verksamheter,
  \item
    validera om en affärsmodell är lönsam genom att identifiera och
    närma sig kunder, partners och konkurrenter,
  \item
    aktivt delta i entreprenörsnätverk,
  \item
    skapa och utvärdera idéer för nya företag på ett strukturerat och
    konsekvent sätt, samt
  \item
    tillämpa olika innovationsstrategier.
  \end{enumerate}
\item
  \emph{Värderingsförmåga och förhållningssätt}

  \begin{enumerate}
  \def\labelenumii{\Alph{enumi}.\arabic{enumii}.}
  \tightlist
  \item
    Identifiera entreprenörsmöjligheter och bedöma dem,
  \item
    bedöma av etiska, miljömässiga och hållbarhetshänsyn i
    beslutsfattande i näringslivet,
  \item
    bedöma sociala och kulturella konsekvenser av affärsverksamhet.
  \end{enumerate}
\end{enumerate}

\subsection*{Kursinnehåll}

Kursen ger en introduktion till entreprenörskap och innovation, samt hur
och vad man skall tänka på när man startar eller förändrar verksamheter.
Kursen tar en praktiskt ansats och tidigare erfarenheter från lean och
agil sätts i sammanhang.

\begin{itemize}
\tightlist
\item
  Grundläggande begrepp, teorier och processer för entreprenörskap.
\item
  Affärsmodeller och affärsmodellering.
\item
  Marknads och kundsegmentering.
\item
  Tillväxt- och affärsutvecklingsstrategier.
\item
  Intäkter och prissättningsmodeller.
\item
  Hur en vinnande och agil kultur skapas.
\end{itemize}

\subsection*{Undervisnings- och
arbetsformer}

Undervisningen sker i form av föreläsningar och gästföreläsningar samt
handledning. Obligatorisk närvaro kan förekomma på vissa moment.

\subsection*{Examination}

Examinationen av kursen delas in i följande moment:

\begin{longtable}[]{@{}llcc@{}}
\toprule
\textsf{Kod} & \textsf{Benämning} & \textsf{Betyg} & \textsf{Poäng}\tabularnewline
\midrule
\endhead
\texttt{UPG1} & Uppgifter i grupp & G-U & 2\tabularnewline
\texttt{UPG2} & Individuell uppgift & A-F & 3\tabularnewline
\bottomrule
\end{longtable}

För godkänt betyg på kursen krävs betyg G på \texttt{UPG1} och minst betyg E på
\texttt{UPG2}. Slutbetyget bestäms från \texttt{UPG2}.

\subsection*{Måluppfyllelse}

Examinationsmomenten kopplas till lärandemålen enligt följande:

\begin{longtable}[]{@{}lcc@{}}
\toprule
\textsf{Lärandemål} & \texttt{UPG1} & \texttt{UPG2}\tabularnewline
\midrule
\endhead
A.1 & & \faCheck\tabularnewline
A.2 & & \faCheck\tabularnewline
A.3 & & \faCheck\tabularnewline
A.4 & \faCheck & \faCheck\tabularnewline
B.1 & \faCheck &\tabularnewline
B.2 & \faCheck &\tabularnewline
B.3 & & \faCheck\tabularnewline
B.4 & \faCheck &\tabularnewline
B.5 & \faCheck & \faCheck\tabularnewline
C.1 & \faCheck & \faCheck\tabularnewline
C.2 & \faCheck & \faCheck\tabularnewline
C.3 & \faCheck & \faCheck\tabularnewline
\bottomrule
\end{longtable}

\subsection*{Kurslitteratur}

Obligatorisk litteratur:

\begin{itemize}
\tightlist
\item
  Løwe Nielsen, S., Klyver, K., Rostgaard Evald, M. och Bager, T.,
  \emph{Entrepreneurship in theory and practice -- paradoxes in play},
  andra utgåvan, Edward Elgar Publishing, 2017, Antal sidor: 236 av 368.
\item
  Ries, E., \emph{The Lean Startup}, Crown Publishing Group, 2011. Antal
  sidor: 300 av 336.
\end{itemize}

\subsection*{Övrigt}

Kursen genomförs på ett sätt sådant att både kvinnor och mäns kunskap och erfarenhet utvecklas och görs synlig.
\pagebreak
\section*{4DV008 - Projekt i dataintensiva system (10 hp)}

\begin{tabular}{ll}\emph{Huvudområde}: & Datavetenskap\tabularnewline\emph{Fördjupning}: & A1N\tabularnewline\end{tabular}

\subsection*{Förkunskaper}

\begin{itemize}
\tightlist
\item
  1DV001 - Programmering och datastrukturer
\item
  1DV006 - Algoritmer
\item
  1DV007 - Mjukvaruutvecklingsprojekt
\item
  1MA002 - Linjär algebra
\item
  1MA004 - Tillämpad sannolikhetslära och statistik
\item
  1ZT001 - Teknisk kommunikation
\item
  1ZT002 - Hållbar utveckling
\item
  1ZT003 - Industriell ekonomi
\end{itemize}

\subsection*{Lärandemål}

Efter genomförd kurs förväntas studenten kunna:

\begin{enumerate}
\def\labelenumi{\Alph{enumi}.}
\tightlist
\item
  \emph{Kunskap och förståelse}

  \begin{enumerate}
  \def\labelenumii{\Alph{enumi}.\arabic{enumii}.}
  \tightlist
  \item
    Förklara vilken roll maskininlärning kan ha i ett mjukvarusystem och
    hur det integreras i systems struktur,
  \item
    räkna upp vilka egenskaper ett verktyg eller programvarubibliotek
    för masksininlärning måste ha för att kunna tillämpas på ett visst
    problem, samt
  \item
    namnge och förklara de vanligaste problemen som dyker upp då man
    vill tillämpa maskininlärning på obearbetade data.
  \end{enumerate}
\item
  \emph{Färdighet och förmåga}

  \begin{enumerate}
  \def\labelenumii{\Alph{enumi}.\arabic{enumii}.}
  \tightlist
  \item
    Självständigt lära sig att använda olika verktyg, metoder och
    programvarubibliotek som används inom maskininlärning,
  \item
    samla in krav från en kund och utifrån dessa specificera vilken data
    som behöver samlas och vilken maskininlärningsansats som lämpar sig
    bäst,
  \item
    utifrån en kunds krav skapa metriker som kan användas för att
    utvärdera hur väl en viss ansats uppfyller dessa krav,
  \item
    implementera och utvärdera ett system (hårdvara och mjukvara) med en
    maskininlärningskomponent,
  \item
    prioritera funktionalitet och kontinuerligt släppa ny funktionalitet
    till kund, samt
  \item
    säkerställa systemets drift. \emph{3. Värderingsförmåga och
    förhållningssätt}
  \item
    Kritiskt reflektera över utfallet och hur väl det motsvarade kundens
    krav, men avseende på t.ex. val av teknik, val av arkitektur (mjuk-
    och hårdvara), data, metriker, osv., samt
  \item
    kritiskt reflektera över hur agila metoder och Lean användes under
    projektets genomförande, t.ex. med avseende på arbetsmiljö.
  \end{enumerate}
\end{enumerate}

\subsection*{Kursinnehåll}

Kursen är en projektkurs som med hjälp av ett realistiskt problem och
realistiska förutsättningar behandlar hela CDIO-cykeln. Studenterna
sätts i rollen som ett litet utvecklingsteam i en startup som skall
utveckla en datadriven produkt.

Studenterna förväntas jobba agilt i grupper om 5-7 och förväntas besätta
alla roller utom produktägare. Startup-miljön ställer särskilda krav på
snabba leveranser och effektivt utnyttjande av resurser, så kursen
innehåller förutom tillämpad maskininlärning och databehandling även
s.k. Lean agile.

\begin{itemize}
\tightlist
\item
  Hur maskininlärningsprojekt fungerar i verkligheten.
\item
  Verktyg, tjänster och programvarubibliotek som kan användas för
  dataanalys och maskininlärning, t.ex. Weka och Tensorflow.
\item
  Konfiguration av pipelines för maskininlärningssystem med avseende på
  mjukvara och hårdvara, t.ex. acceleratorer.
\item
  Praxis för att arbeta med verkliga data, t.ex. med avseende på
  insamling, bearbetning och analys.
\item
  Utvärdering av prestanda utifrån kundkrav.
\item
  Experimentdriven utveckling och korta cykler mellan design, träning
  och utvärdering.
\item
  Strategin Lean för tillverkning och Toyota Production System.
\item
  Hur Lean kan tillämpas på mjukvaruutveckling och tillsammans med
  Agile, Lean-Agile.
\item
  Vad slöseri är i mjukvaruutvecklingssammanhang samt tekniker för att
  reducera det.
\item
  Hur ``Just-in-time''-produktion kan tillämpas på mjukvara
\item
  Hur ökad vikt läggs på att lära sig, t.ex. genom reflektion efter
  sprintar, samt hur utvecklingslaget (och deras kompetenser) kan sättas
  i centrum.
\item
  Helheten i mjukvaran.
\item
  Fördjupning i att skriva reflektionsrapporter
\end{itemize}

\subsection*{Undervisnings- och
arbetsformer}

Kursen innehåller en föreläsningsserie samt workshops som presentera och
hjälper studenterna komma igång med de verktyg, metoder och resurser de
förväntas använda under projektet. Under projektets gång kommer
studenterna ha regelbundna möten med produktägare och handledare/lärare.

I slutet av kursen presenteras samtliga projekt vid seminarier.

\subsection*{Examination}

Examinationen av kursen delas in följande moment:

\begin{longtable}[]{@{}llcc@{}}
\toprule
\textsf{Kod} & \textsf{Benämning} & \textsf{Betyg} & \textsf{Poäng}\tabularnewline
\midrule
\endhead
\texttt{UPG1} & Vision och planeringsdokument & A-F & 2\tabularnewline
\texttt{PRJ1} & Projektarbete (inkl. leverabler) & A-F & 5\tabularnewline
\texttt{RAP1} & Reflektionsrapport - Val och uppfyllelse & A-F & 1\tabularnewline
\texttt{RAP2} & Reflektionsrapport - Lean & A-F & 1\tabularnewline
\texttt{PRS1} & Design, implementering och resultat & A-F & 1\tabularnewline
\bottomrule
\end{longtable}

För godkänt betyg på kursen krävs betyg G på \texttt{PRS1} samt minst betyg E på
övriga moment. Slutbetyget bestäms från: \texttt{UPG1} (20~\%), \texttt{PRJ1} (50~\%), \texttt{RAP1}
(10~\%), \texttt{RAP2} (10~\%) och \texttt{PRS1} (10~\%).

\subsection*{Måluppfyllelse}

Examinationsmomenten kopplas till lärandemålen enligt följande:

\begin{longtable}[]{@{}lccccc@{}}
\toprule
\textsf{Lärandemål} & \texttt{UPG1} & \texttt{PRJ1} & \texttt{RAP1} & \texttt{RAP2} & \texttt{PRS1}\tabularnewline
\midrule
\endhead
A.1 & \faCheck & \faCheck & \faCheck & & \faCheck\tabularnewline
A.2 & \faCheck & & & & \faCheck\tabularnewline
A.3 & \faCheck & & \faCheck & & \faCheck\tabularnewline
B.1 & \faCheck & \faCheck & \faCheck & &\tabularnewline
B.2 & \faCheck & & \faCheck & & \faCheck\tabularnewline
B.3 & \faCheck & & \faCheck & & \faCheck\tabularnewline
B.4 & & \faCheck & \faCheck & & \faCheck\tabularnewline
B.5 & & \faCheck & \faCheck & & \faCheck\tabularnewline
B.6 & & \faCheck & \faCheck & & \faCheck\tabularnewline
C.1 & & & \faCheck & &\tabularnewline
C.2 & & & & \faCheck &\tabularnewline
\bottomrule
\end{longtable}

\subsection*{Kurslitteratur}

Studenterna förväntas söka efter lämplig kurslitteratur på egen hand
eller i samråd med handledare. Nedanstående referenslitteratur kan
användas som en utgångspunkt.

\begin{itemize}
\tightlist
\item
  Gollapudi, S., \emph{Practical Machine Learning}. PACKT publishing,
  2016.
\end{itemize}

\subsection*{Övrigt}

Kursen genomförs på ett sätt sådant att både kvinnor och mäns kunskap och erfarenhet utvecklas och görs synlig.
\pagebreak
\section*{4DV009 - Informationsvisualisering (5 hp)}

\begin{tabular}{ll}\emph{Huvudområde}: & Datavetenskap\tabularnewline\emph{Fördjupning}: & A1N\tabularnewline\end{tabular}

\subsection*{Förkunskaper}

\begin{itemize}
\tightlist
\item
  1DV004 - Objektorienterad programmering
\item
  2DV004 - Datorgrafik
\item
  1DV006 - Algoritmer
\item
  1MA004 - Tillämpad sannolikhetslära och statistik
\end{itemize}

\subsection*{Lärandemål}

Efter slutförd kurs skall studenten kunna:

\begin{enumerate}
\def\labelenumi{\Alph{enumi}.}
\tightlist
\item
  \emph{Kunskap och förståelse}

  \begin{enumerate}
  \def\labelenumii{\Alph{enumi}.\arabic{enumii}.}
  \tightlist
  \item
    Klassificera typiska uppgifter för visualiseringar,
  \item
    definiera och förklara visualiseringsteknikerna (angående
    interaktion och visuell representation) och typiska verktyg som
    diskuteras i kursen, samt
  \item
    beskriva och förklara de grundläggande perceptuella principerna som
    påverkar informationsvisualisering.
  \end{enumerate}
\item
  \emph{Färdighet och förmåga}

  \begin{enumerate}
  \def\labelenumii{\Alph{enumi}.\arabic{enumii}.}
  \tightlist
  \item
    Representera data genom expressiva och effektiva visualiseringar med
    hjälp av metoder, programvara och verktyg som är aktuella, samt
  \item
    implementera grundläggande interaktiva visualiseringar, såsom
    sambandsdiagram eller radardiagram.
  \end{enumerate}
\item
  \emph{Värderingsförmåga och förhållningssätt}

  \begin{enumerate}
  \def\labelenumii{\Alph{enumi}.\arabic{enumii}.}
  \tightlist
  \item
    Kritiskt reflektera över visualisering och interaktionsstrategier
    mot bakgrund av aktuella teorier och forskning, samt
  \item
    göra välgrundade designval baserat på olika aspekter och
    databegränsningar
  \end{enumerate}
\end{enumerate}

\subsection*{Kursinnehåll}

Informationsvisualisering fokuserar på abstrakt information som i de
flesta fall inte kan kartläggas i den fysiska världen. Exempel på sådana
abstrakta data är symboliska, tabulära, nätverksbaserade, hierarkiska
eller textuella informationskällor. Kursen ger en översikt över de
viktigaste informationsvisualiseringsteknikerna och applikationerna.

Följande moment behandlas:

\begin{itemize}
\tightlist
\item
  Definition av området informationsvisualisering och hur det relaterar
  till områden som människa-datorinteraktion (HCI) eller vetenskaplig
  visualisering.
\item
  Grunderna i visuell uppfattning (preattentiv behandling,
  gestaltningslagar) och kognition.
\item
  Grunderna i data och visualisering samt hur de kan bearbetas.
\item
  Uppdragsabstraktioner och taxonomier.
\item
  Interaktionskoncept och tekniker (t.ex. dynamiska frågor, zoom och
  panorering eller fokus och sammanhang).
\item
  Visualiseringstekniker för 1D, 2D, 3D och multidimensionella data.
\item
  Översikt över aktuella system och verktyg för
  informationsvisualisering.
\end{itemize}

\subsection*{Undervisnings- och
arbetsformer}

På kursen ges traditionella föreläsningar för att gå igenom huvuddelen
av kursinnehållet. Dessutom bearbetar och fördjupar studenterna
innehållet i begreppsmässiga och praktiska uppgifter. Alla uppgifter
diskuteras också i seminarier där studenterna presenterar sina resultat
och får feedback från sina medstudenter och lärare. Detta ger också en
plats att kritiskt reflektera över informationsvisualiseringar baserat
på teorier och forskningsresultat som presenteras på föreläsningarna.
Alla uppgifter utförs individuellt eller grupper om högst två studenter.

Obligatorisk närvaro kan förekomma på vissa moment.

\subsection*{Examination}

Examinationen av kursen delas in i följande moment:

\begin{longtable}[]{@{}llcc@{}}
\toprule
\textsf{Kod} & \textsf{Benämning} & \textsf{Betyg} & \textsf{Poäng}\tabularnewline
\midrule
\endhead
\texttt{UPG1} & Inlämningsuppgifter & A-F & 3\tabularnewline
\texttt{TEN1} & Muntlig tentamen & A-F & 2\tabularnewline
\bottomrule
\end{longtable}

För godkänt betyg på kursen krävs minst betyg E på samtliga moment.
Slutbetyget bestäms från: \texttt{UPG1} (60~\%) och \texttt{TEN1} (40~\%).

\subsection*{Måluppfyllelse}

Examinationsmomenten kopplas till lärandemålen enligt följande:

\begin{longtable}[]{@{}lll@{}}
\toprule
\textsf{Lärandemål} & \texttt{UPG1} & \texttt{TEN1}\tabularnewline
\midrule
\endhead
A.1 & \faCheck & \faCheck\tabularnewline
A.2 & \faCheck & \faCheck\tabularnewline
A.3 & \faCheck & \faCheck\tabularnewline
B.1 & \faCheck &\tabularnewline
B.2 & \faCheck &\tabularnewline
C.1 & \faCheck & \faCheck\tabularnewline
C.2 & \faCheck & \faCheck\tabularnewline
\bottomrule
\end{longtable}

\subsection*{Kurslitteratur}

Obligatorisk litteratur:

\begin{itemize}
\tightlist
\item
  Spence, R., \emph{Information Visualization -- An Introduction},
  tredje utgåvan, Springer, 2014. Antal sidor: 200 av 292.
\item
  Munzner, T., \emph{Visualization Analysis and Design}, CRC Press,
  2014. Antal sidor: 150 av 404.
\item
  Ware, C., \emph{Information Visualization: Perception for Design},
  tredje utgåvan, Morgan Kaufmann, 2013. Antal sidor: 100 av 512.
\item
  Kompendium med vetenskapliga artiklar
\end{itemize}

\subsection*{Övrigt}

Kursen genomförs på ett sätt sådant att både kvinnor och mäns kunskap och erfarenhet utvecklas och görs synlig.
\pagebreak
\section*{4DV010 - Datautvinning (5 hp)}

\begin{tabular}{ll}\emph{Huvudområde}: & Datavetenskap\tabularnewline\emph{Fördjupning}: & A1F\tabularnewline\end{tabular}

\subsection*{Förkunskaper}

\begin{itemize}
\tightlist
\item
  1DV006 - Algoritmer
\item
  4DV006 - Parallelldatorprogrammering
\item
  4DV005 - Maskininlärning
\item
  1MA002 - Linjär algebra
\item
  1MA003 - Envariabelanalys 1
\item
  1MA007 - Numeriska metoder
\end{itemize}

\subsection*{Lärandemål}

Efter slutförd kurs skall studenten kunna:

\begin{enumerate}
\def\labelenumi{\Alph{enumi}.}
\tightlist
\item
  \emph{Kunskap och förståelse}

  \begin{enumerate}
  \def\labelenumii{\Alph{enumi}.\arabic{enumii}.}
  \tightlist
  \item
    Förklara grundläggande koncept och principer för datautvinning,
    t.ex. avståndsmått och klustering, samt
  \item
    redogöra för de senaste rönen inom datautvinning, t.ex. tekniker och
    tillämpningar.
  \end{enumerate}
\item
  \emph{Färdighet och förmåga}

  \begin{enumerate}
  \def\labelenumii{\Alph{enumi}.\arabic{enumii}.}
  \tightlist
  \item
    Givet ett problem och en datamängd eller dataström, planera och
    strukturera en datautvinningspipeline med avseende på vilka metoder
    som skall användas för att t.ex. dimensionsreducering, avståndsmått
    och klustering, samt vilken kvalitet utdata kommer att ha,
  \item
    implementera några av de enklare algoritmerna, t.ex. Page rank och
    CURE-klustering på ett effektivt sätt, t.ex. med hjälp av Map-reduce
    ramverk, samt
  \item
    använda lämpliga befintliga verktyg och programbibliotek för att
    utvinna data ur en given (ostrukturerad) datamängd eller dataström.
  \end{enumerate}
\item
  \emph{Värderingsförmåga och förhållningssätt}

  \begin{enumerate}
  \def\labelenumii{\Alph{enumi}.\arabic{enumii}.}
  \tightlist
  \item
    Reflektera över en datautvinningstillämpning ur ett
    samhällsperspektiv med avseende på t.ex. etiska frågeställningar
    kontra nytta, samt
  \item
    resonera kring datakvalitet med avseende på hur den påverkar hur
    utvunnen data kan användas samt hur datakvaliteten hos indata kan
    förbättras.
  \end{enumerate}
\end{enumerate}

\subsection*{Kursinnehåll}

Kursen ger en introduktion till datautvinning och vanliga
användningsområden, t.ex. sökmotorer, rekommendationssystem och
webbannonsering. Kursen bygger vidare på kunskaper från numeriska
metoder och maskininlärning.

\begin{itemize}
\tightlist
\item
  Introduktion till datautvinning.
\item
  Relationen mellan datautvinning och maskininlärning.
\item
  Utvinning av information från text.
\item
  Hur hittar man liknande saker, t.ex. dokument?
\item
  Hur utvinns data ur dataströmmar?
\item
  Analys av länkar, t.ex. pagerank och HITS.
\item
  Analys av vanligt förkommande mängder av saker, t.ex. Market-Basket
  och A-Priori algoritmen.
\item
  Fördjupning av klustingsalgoritmer, t.ex CURE och CRGPF.
\item
  Fördjupning av dimensionsreduction, CUR.
\item
  Klustring av strömmande data.
\item
  Tillämpningar inom annonsering på webben, rekommendationssystem och
  analys av sociala nätverk.
\item
  Datautvinning ur ett samhällsperspektiv, t.ex. etiska
  frågeställningar, affärsnytta och nya möjligheter inom t.ex. hälsa.
\item
  Datakvalitet.
\item
  Verktyg och programbibliotek för datautvinning.
\end{itemize}

\subsection*{Undervisnings- och
arbetsformer}

Undervisningen består av traditionella föreläsningar där teori
introduceras, seminarier där tillämpning av olika metoder diskuteras
utifrån ett problem samt lärarledda laborationer där praktiska
färdigheter övas. Kursen innehåller även en serie gästföreläsningar där
industrirepresentanter och forskare presenterar hur och varför de
använder datautvinning samt vilka metoder och verktyg de använder.

\subsection*{Examination}

Examinationen av kursen delas in i följande moment:

\begin{longtable}[]{@{}llcc@{}}
\toprule
\textsf{Kod} & \textsf{Benämning} & \textsf{Betyg} & \textsf{Poäng}\tabularnewline
\midrule
\endhead
\texttt{MUN1} & Muntlig tentamen & A-F & 2\tabularnewline
\texttt{LAB1} & Programmeringsuppgifter & A-F & 2\tabularnewline
\texttt{UPG1} & Reflektionsrapport & A-F & 1\tabularnewline
\bottomrule
\end{longtable}

För godkänt betyg på kursen krävs minst betyg E på samtliga moment.
Slutbetyget bestäms från: \texttt{MUN1} (40~\%), \texttt{LAB1} (40~\%) och \texttt{UPG1} (20~\%).

\subsection*{Måluppfyllelse}

Examinationsmomenten kopplas till lärandemålen enligt följande:

\begin{longtable}[]{@{}lccc@{}}
\toprule
\textsf{Lärandemål} & \texttt{MUN1} & \texttt{LAB1} & \texttt{UPG1}\tabularnewline
\midrule
\endhead
A.1 & \faCheck & &\tabularnewline
A.2 & \faCheck & & \faCheck\tabularnewline
B.1 & \faCheck & \faCheck &\tabularnewline
B.2 & & \faCheck &\tabularnewline
B.3 & & \faCheck &\tabularnewline
C.1 & & & \faCheck\tabularnewline
C.2 & \faCheck & &\tabularnewline
\bottomrule
\end{longtable}

\subsection*{Kurslitteratur}

Obligatorisk litteratur:

\begin{itemize}
\tightlist
\item
  Leskovec, J., Rajaraman, A. och Ullman, J. D., \emph{Mining of Massive
  Datasets}, Cambridge University Press, 2014. Antal sidor: 400 av 511.
\item
  Kompendium med vetenskapliga artiklar
\end{itemize}

\subsection*{Övrigt}

Kursen genomförs på ett sätt sådant att både kvinnor och mäns kunskap och erfarenhet utvecklas och görs synlig.
\pagebreak
\section*{4DV011 - Avancerad informationsvisualisering och tillämpningar (5 hp)}

\begin{tabular}{ll}\emph{Huvudområde}: & Datavetenskap\tabularnewline\emph{Fördjupning}: & A1F\tabularnewline\end{tabular}

\subsection*{Förkunskaper}

\begin{itemize}
\tightlist
\item
  1DV004 - Objektorienterad programmering
\item
  2DV004 - Datorgrafik
\item
  1DV006 - Algoritmer
\item
  4DV009 - Informationsvisualisering
\item
  4DV010 - Datautvinning
\end{itemize}

\subsection*{Lärandemål}

Efter slutförd kurs skall studenten kunna:

\begin{enumerate}
\def\labelenumi{\Alph{enumi}.}
\tightlist
\item
  \emph{Kunskap och förståelse}

  \begin{enumerate}
  \def\labelenumii{\Alph{enumi}.\arabic{enumii}.}
  \tightlist
  \item
    Definiera och förklara visualiseringstekniker (avseende interaktion
    och visuell representation) och känna till verktyg för speciella
    datamängder och applikationsdomäner,
  \item
    beskriva validerings- och utvärderings-metoder för
    visualiserings-verktyg och metoder, samt
  \item
    beskriv de viktigaste utmaningarna inom
    informationsvisualiseringsforskning.
  \end{enumerate}
\item
  \emph{Färdighet och förmåga}

  \begin{enumerate}
  \def\labelenumii{\Alph{enumi}.\arabic{enumii}.}
  \tightlist
  \item
    Representera data genom expressiva och effektiva visualiseringar med
    hjälp av metoder, programvara och verktyg, samt
  \item
    implementera nya interaktiva visualiseringar för komplexa och stora
    datamängder och där det krävs fokus på specifika applikationsdomäner
    eller analysproblem.
  \end{enumerate}
\item
  \emph{Värderingsförmåga och förhållningssätt}

  \begin{enumerate}
  \def\labelenumii{\Alph{enumi}.\arabic{enumii}.}
  \tightlist
  \item
    Kritiskt reflektera över genomgångna visualiserings- och
    interaktionsmetoder mot bakgrund av aktuella teorier och forskning,
    samt
  \item
    skapa välgrundade designval utifrån olika uppgifter och
    databegränsningar.
  \end{enumerate}
\end{enumerate}

\subsection*{Kursinnehåll}

Kursen bygger på och fördjupar innehållet i kursen
informationsvisualisering 1 med interaktiva visualiseringstekniker och
system för speciella datamängder, såsom nätverksdata, tidsberoende data
och textdata. Vidare diskuteras specifika applikationer där
informationsvisualiseringar används för att analysera/utforska
domänspecifika data, t.ex. i bioinformatik, geografi, mjukvaruutveckling
etc. samt att granska exempel på metoder för den interaktiva
visualiseringen av sådana datamängder. Slutligen granskar kursen
möjligheter till hur visualiseringar kan valideras, utvärderas eller
användas i icke-standardiserade sammanhang som samarbetsmiljöer eller
analys av personliga data.

Följande moment behandlas:

\begin{itemize}
\tightlist
\item
  Visualiseringstekniker och ritkonventioner för träd/hierarkier,
  generella nätverksdata (grafer) och multivarianta/dynamiska nätverk.
\item
  Visualiseringstekniker för textdata och dokumentsamlingar (corpus).
\item
  Visualiseringstekniker för generella tidsseriedata.
\item
  Visualiseringar för specifika applikationsdomäner, inklusive en
  översikt över deras vanliga analysuppgifter och dataspecifika
  uppgifter.
\item
  Samarbetande och personliga visualiseringsidéer och tillvägagångssätt.
\item
  Validera och utvärdera visualiseringar.
\item
  Viktigaste olösta utmaningarna inom informationsvisualisering.
\end{itemize}

\subsection*{Undervisnings- och
arbetsformer}

Kursen innefattar traditionella föreläsningar för att gå igenom
huvuddelen av kursinnehållet. Dessutom bearbetar och fördjupar
studenterna innehållet i begreppsmässiga och praktiska uppgifter. Alla
uppgifter diskuteras också i seminarier där studenterna presenterar sina
resultat och får feedback från sina medstudenter och lärare. Detta ger
också en plats att kritiskt reflektera över informationsvisualiseringar
baserat på teorier och forskningsresultat som presenteras på
föreläsningarna. Alla uppgifter utförs individuellt eller grupper om
högst två studenter.

Obligatorisk närvaro kan förekomma på vissa moment.

\subsection*{Examination}

Examinationen av kursen delas in i följande moment:

\begin{longtable}[]{@{}llcc@{}}
\toprule
\textsf{Kod} & \textsf{Benämning} & \textsf{Betyg} & \textsf{Poäng}\tabularnewline
\midrule
\endhead
\texttt{PRJ1} & Programmeringsprojekt & A-F & 2\tabularnewline
\texttt{PRS1} & Presentation & A-F & 1\tabularnewline
\texttt{MUN1} & Muntlig tentamen & A-F & 2\tabularnewline
\bottomrule
\end{longtable}

För godkänt betyg på kursen krävs minst betyg E på samtliga moment.
Slutbetyget bestäms från: \texttt{PRJ1} (40~\%), \texttt{PRS1} (20~\%) och \texttt{TEN1} (40~\%).

\subsection*{Måluppfyllelse}

Examinationsmomenten kopplas till lärandemålen enligt följande:

\begin{longtable}[]{@{}llll@{}}
\toprule
\textsf{Lärandemål} & \texttt{PRJ1} & \texttt{PRS1} & \texttt{TEN1}\tabularnewline
\midrule
\endhead
A.1 & & \faCheck & \faCheck\tabularnewline
A.2 & \faCheck & & \faCheck\tabularnewline
A.3 & \faCheck & & \faCheck\tabularnewline
B.1 & \faCheck & &\tabularnewline
B.2 & \faCheck & &\tabularnewline
C.1 & \faCheck & \faCheck & \faCheck\tabularnewline
C.2 & \faCheck & & \faCheck\tabularnewline
\bottomrule
\end{longtable}

\subsection*{Kurslitteratur}

Obligatorisk litteratur:

\begin{itemize}
\tightlist
\item
  Aigner, W., Miksch, S., Schumann, H. och Tominski, C.,
  \emph{Visualization of Time-Oriented Data}, Springer, 2011. Antal
  sidor: 80 av 286.
\item
  Kerren, A., Ebert, A. och Meyer, J., \emph{Human-Centered
  Visualization Environments}. LNCS Tutorial 4417, Springer, 2007. Antal
  sidor: 150 av 403.
\item
  Ward, M., Grinstein, G. G. och Keim, D., \emph{Interactive Data
  Visualization - Foundations, Techniques, and Applications}, andra
  utgåvan, A. K. Peters Ltd., 2015. Antal sidor: 150 av 558.
\end{itemize}

\subsection*{Övrigt}

Kursen genomförs på ett sätt sådant att både kvinnor och mäns kunskap och erfarenhet utvecklas och görs synlig.
\pagebreak
\section*{4DV012 - Vetenskapliga metoder inom datavetenskap (5 hp)}

\begin{tabular}{ll}\emph{Huvudområde}: & Datavetenskap\tabularnewline\emph{Fördjupning}: & A1F\tabularnewline\end{tabular}

\subsection*{Förkunskaper}

\begin{itemize}
\tightlist
\item
  1ZT001 - Teknisk kommunikation
\item
  2ZT001 - Vetenskapliga metoder
\item
  Minst en av projektkurserna 4DV004 eller 4DV008
\end{itemize}

\subsection*{Lärandemål}

Efter genomförd kurs förväntas studenten kunna:

\begin{enumerate}
\def\labelenumi{\Alph{enumi}.}
\tightlist
\item
  \emph{Kunskap och förståelse}

  \begin{enumerate}
  \def\labelenumii{\Alph{enumi}.\arabic{enumii}.}
  \tightlist
  \item
    Redogöra för de senaste rönen inom ett delområde inom datavetenskap
    och/eller mjukvaruteknik i stor detalj, samt
  \item
    redogöra för forskningsinriktningar, ställningstaganden, samt de
    vanligaste metodvalen inom ett delområde inom datavetenskap
    och/eller mjukvaruteknik.
  \end{enumerate}
\item
  \emph{Färdighet och förmåga}

  \begin{enumerate}
  \def\labelenumii{\Alph{enumi}.\arabic{enumii}.}
  \tightlist
  \item
    Självständigt planera ett vetenskapligt projekt med relevanta
    frågeställningar och lämpliga metoder inom datavetenskap och/eller
    mjukvaruteknik,
  \item
    söka efter, kritiskt utvärdera, och presentera vetenskapliga
    resultat, samt
  \item
    kritiskt analysera metodval och genomförande i en vetenskaplig
    studie.
  \end{enumerate}
\item
  \emph{Värderingsförmåga och förhållningssätt}

  \begin{enumerate}
  \def\labelenumii{\Alph{enumi}.\arabic{enumii}.}
  \tightlist
  \item
    Analysera hur resultaten av en vetenskaplig studie bidrar till att
    öka förståelse och insikt kring ett fenomen e. dyl.,
  \item
    resonera kring vilket bidrag en vetenskaplig studie gör till t.ex.
    samhällsnyttan och teknikutvecklingen, t.ex. hur resultatet kan
    omsättas inom mjukvaruutvecklingsindustrin, samt
  \item
    resonera kring etiska frågeställningar samt analysera hur väl de
    hanteras i en studie.
  \end{enumerate}
\end{enumerate}

\subsection*{Kursinnehåll}

Kursen är en seminariekurs där studenterna presenterar och diskuterar
vetenskapliga arbeten för att fördjupa sin kännedom om aktuell forskning
inom datavetenskap och mjukvaruteknik, samt sin förmåga att kritiskt
diskutera forskningsmetod och resultat.

\begin{itemize}
\tightlist
\item
  Fördjupning av vetenskapliga metoder inom datavetenskap och
  mjukvaruteknik, särskilt inom ett valt delområde.
\item
  Fördjupning i akademiskt skrivande och muntlig presentation.
\item
  Fördjupning av kritisk analys av vetenskapliga publikationer och
  presentationer.
\item
  Fördjupning i planering av forskningsprojekt.
\item
  Fördjupning av etiska frågeställningar och hur dessa hanteras.
\end{itemize}

\subsection*{Undervisnings- och
arbetsformer}

Undervisningen består ett fåtal föreläsningar som introducerar kursens
upplägg och teori samt seminarier där studenterna presenterar och
diskuterar forskningsresultat tillsammans med forskare inom fältet.
Läraren kommer vid behov att styra diskussion så att alla
frågeställningar berörs, t.ex. etiska ställningstaganden.

Planeringsrapporten skapas i samråd med och under handledning av en
forskare.

\subsection*{Examination}

Examinationen av kursen delas in i följande moment:

\begin{longtable}[]{@{}llcc@{}}
\toprule
\textsf{Kod} & \textsf{Benämning} & \textsf{Betyg} & \textsf{Poäng}\tabularnewline
\midrule
\endhead
\texttt{PRS1} & Presentation av vetenskapliga arbeten & A-F & 1\tabularnewline
\texttt{OPP1} & Opponering & A-F & 1\tabularnewline
\texttt{UPG1} & Planeringsdokument & A-F & 3\tabularnewline
\bottomrule
\end{longtable}

För godkänt betyg på kursen krävs minst betyg E på samtliga moment.
Slutbetyget bestäms från: \texttt{PRS1} (20~\%), \texttt{OPP1} (20~\%) och \texttt{UPG1} (60~\%).

\subsection*{Måluppfyllelse}

Examinationsmomenten kopplas till lärandemålen enligt följande:

\begin{longtable}[]{@{}lccc@{}}
\toprule
\textsf{Lärandemål} & \texttt{PRS1} & \texttt{OPP1} & \texttt{UPG1}\tabularnewline
\midrule
\endhead
A.1 & \faCheck & & \faCheck\tabularnewline
A.2 & & & \faCheck\tabularnewline
B.1 & & & \faCheck\tabularnewline
B.2 & \faCheck & \faCheck & \faCheck\tabularnewline
B.3 & & \faCheck & \faCheck\tabularnewline
C.1 & \faCheck & & \faCheck\tabularnewline
C.2 & \faCheck & & \faCheck\tabularnewline
C.3 & & \faCheck & \faCheck\tabularnewline
\bottomrule
\end{longtable}

\subsection*{Kurslitteratur}

Kurslitteraturen bestäms i samråd med handledare och består av
vetenskapliga artiklar och böcker som antingen kan laddas ner eller
lånas.

\subsection*{Övrigt}

Kursen genomförs på ett sätt sådant att både kvinnor och mäns kunskap och erfarenhet utvecklas och görs synlig.
\pagebreak
\section*{4DV013 - Projekt i visualisering och dataanalys (10 hp)}

\begin{tabular}{ll}\emph{Huvudområde}: & Datavetenskap\tabularnewline\emph{Fördjupning}: & A1F\tabularnewline\end{tabular}

\subsection*{Förkunskaper}

\begin{itemize}
\tightlist
\item
  1DV004 - Objektorienterad programmering
\item
  2DV004 - Datorgrafik
\item
  1MA002 - Linjär algebra
\item
  1MA004 - Tillämpad sannolikhetslära och statistik
\item
  1ZT002 - Hållbar utveckling
\item
  4DV005 - Maskininlärning
\item
  Minst en av projektkurserna 4DV004 eller 4DV008
\end{itemize}

\subsection*{Lärandemål}

Efter slutförd kurs skall studenten kunna:

\begin{enumerate}
\def\labelenumi{\Alph{enumi}.}
\tightlist
\item
  \emph{Kunskap och förståelse}

  \begin{enumerate}
  \def\labelenumii{\Alph{enumi}.\arabic{enumii}.}
  \tightlist
  \item
    Förklara och motivera informationsvisualisering och visual analytics
    i ett människa-maskin-perspektiv och hur dessa kan underlätta
    dataanalys.
  \end{enumerate}
\item
  \emph{Färdighet och förmåga}

  \begin{enumerate}
  \def\labelenumii{\Alph{enumi}.\arabic{enumii}.}
  \tightlist
  \item
    Självständigt organisera och genomföra ett agilt projekt,
  \item
    självständigt lära sig använda olika verktyg, metoder och
    programvarubibliotek som används inom informationsvisualisering och
    visual analytics,
  \item
    samla in krav och utifrån dessa bestämma vilka visualiseringar,
    dataanalyser och interaktionsmetoder som lämpar sig bäst,
  \item
    implementera ett visual analytics system med rimliga tekniska
    lösningar och driftsätta detta, samt
  \item
    utifrån en kunds krav specificera och genomföra utvärderingar av ett
    visual analytics-system med hjälp av t.ex. fokusgrupper.
  \end{enumerate}
\item
  \emph{Värderingsförmåga och förhållningssätt}

  \begin{enumerate}
  \def\labelenumii{\Alph{enumi}.\arabic{enumii}.}
  \tightlist
  \item
    Reflektera över vilka typer av frågor och analysprocesser som bäst
    stöds av olika kombinationer av visualiseringstekniker,
    dataanalysalgoritmer och interaktionsmetoder,
  \item
    reflektera över hur val av främst visualiseringar påverkas av de
    grupper av människor som skall använda systemet, t.ex. med avseende
    på metaforer och gemensam förståelse, samt
  \item
    reflektera över vilken bias som finns i systemet, t.ex. beroende på
    insamlade data, databearbetning, analysmetod och
    visualiseringstekniker.
  \end{enumerate}
\end{enumerate}

\subsection*{Kursinnehåll}

Kursen är en projektkurs som med hjälp av ett realistiskt problem och
realistiska förutsättningar behandlar hela CDIO-cykeln. Studenterna
förväntas jobba agilt i grupper om 5-7 och förväntas besätta alla roller
utom produktägare. Det här är den sista projektkursen på programmet där
studenterna jobbar agilt, så de förväntas kunna organisera och genomföra
den agila processen självständigt.

\begin{itemize}
\tightlist
\item
  Vikten av data och visualisering för en organisation.
\item
  Informationsvisualisering i realistiska projekt.
\item
  Dataanalys och databearbetning i realistiska projekt.
\item
  Verktyg, tjänster och programvarubibliotek som kan användas för
  dataanalys och för att utveckla informationsvisualiseringar, till
  exempel D3, yFiles och Bokeh.
\item
  Människa, maskin och visualiseringar.
\item
  Att utvärdera visualiseringar i realistiska projekt.
\item
  Att skapa tillgängliga visualiseringssystem.
\item
  Bias i data, analys och visualisering.
\item
  Metaforer, hur och när skall dessa användas, vilka kulturella
  kopplingar har de, osv?
\end{itemize}

\subsection*{Undervisnings- och
arbetsformer}

Kursen innehåller en föreläsningsserie samt workshops som presentera och
hjälper studenterna komma igång med de verktyg, metoder och resurser de
förväntas använda under projektet. Under projektets gång kommer
studenterna ha regelbundna möten med produktägare och handledare/lärare.

I slutet av kursen presenteras samtliga projekt vid seminarier.

\subsection*{Examination}

Examinationen av kursen delas in följande moment:

\begin{longtable}[]{@{}llcc@{}}
\toprule
\textsf{Kod} & \textsf{Benämning} & \textsf{Betyg} & \textsf{Poäng}\tabularnewline
\midrule
\endhead
\texttt{UPG1} & Vision och planeringsdokument & A-F & 2\tabularnewline
\texttt{PRJ1} & Projektarbete (inkl. leverabler) & A-F & 5\tabularnewline
\texttt{RAP1} & Reflektionsrapport - Att driva ett agilt projekt & A-F &
1\tabularnewline
\texttt{RAP2} & Reflektionsrapport - Människa och maskin & A-F & 1\tabularnewline
\texttt{PRS1} & Design, implementering och resultat & A-F & 1\tabularnewline
\bottomrule
\end{longtable}

För godkänt betyg på kursen krävs minst betyg E på samtliga moment.
Slutbetyget bestäms från: \texttt{UPG1} (20~\%), \texttt{PRJ1} (50~\%), \texttt{RAP1} (10~\%), \texttt{RAP2}
(10~\%) och \texttt{PRS1} (10~\%).

\subsection*{Måluppfyllelse}

Examinationsmomenten kopplas till lärandemålen enligt följande:

\begin{longtable}[]{@{}lccccc@{}}
\toprule
\textsf{Lärandemål} & \texttt{UPG1} & \texttt{PRJ1} & \texttt{RAP1} & \texttt{RAP2} & \texttt{PRS1}\tabularnewline
\midrule
\endhead
A.1 & \faCheck & \faCheck & & \faCheck & \faCheck\tabularnewline
B.1 & \faCheck & \faCheck & \faCheck & & \faCheck\tabularnewline
B.2 & \faCheck & \faCheck & & & \faCheck\tabularnewline
B.3 & \faCheck & \faCheck & & \faCheck & \faCheck\tabularnewline
B.4 & & \faCheck & & &\tabularnewline
B.5 & & \faCheck & & \faCheck & \faCheck\tabularnewline
C.1 & & & & \faCheck & \faCheck\tabularnewline
C.2 & & & & \faCheck &\tabularnewline
C.3 & & \faCheck & & \faCheck & \faCheck\tabularnewline
\bottomrule
\end{longtable}

\subsection*{Kurslitteratur}

Studenterna förväntas söka efter lämplig kurslitteratur på egen hand
eller i samråd med handledare. Nedanstående referenslitteratur kan
användas som en utgångspunkt.

\begin{itemize}
\tightlist
\item
  Keim, D., Kohlhammer, J., Ellis, G. och Mansmann, F., \emph{Mastering
  the Information Age: Solving Problems with Visual Analytics},
  Eurographics, 2010.
\item
  Munzner, T., \emph{Visualization Analysis and Design}, CRC Press,
  2014.
\item
  Purchase, H. C., \emph{Experimental Human-Computer Interaction: A
  Practical Guide with Visual Examples}, Cambridge University Press,
  2012.
\end{itemize}

\subsection*{Övrigt}

Kursen genomförs på ett sätt sådant att både kvinnor och mäns kunskap och erfarenhet utvecklas och görs synlig.
\pagebreak
\section*{5DV001 - Självständigt arbete (30 hp)}

\begin{tabular}{ll}\emph{Huvudområde}: & Datavetenskap\tabularnewline\emph{Fördjupning}: & A2E\tabularnewline\end{tabular}

\subsection*{Förkunskaper}

\begin{itemize}
\tightlist
\item
  2DV007 - Självständigt arbete
\item
  4DV012 - Vetenskapliga metoder inom datavetenskap
\item
  1ZT001 - Teknisk kommunikation
\item
  2ZT001 - Vetenskapliga metoder
\end{itemize}

\subsection*{Lärandemål}

Efter genomförd kurs skall studenten kunna:

\begin{enumerate}
\def\labelenumi{\Alph{enumi}.}
\tightlist
\item
  \emph{Kunskap och förståelse}

  \begin{enumerate}
  \def\labelenumii{\Alph{enumi}.\arabic{enumii}.}
  \tightlist
  \item
    Visa väsentligt fördjupade kunskaper inom datavetenskap och/eller
    mjukvaruteknik, samt fördjupad insikt i aktuellt forsknings- och
    utvecklingsarbete, samt
  \item
    visa fördjupad metodkunskap inom datavetenskap och/eller
    mjukvaruteknik.
  \end{enumerate}
\item
  \emph{Färdighet och förmåga}

  \begin{enumerate}
  \def\labelenumii{\Alph{enumi}.\arabic{enumii}.}
  \tightlist
  \item
    Med helhetssyn kritiskt, självständigt och kreativt identifiera,
    formulera och hantera komplexa frågeställningar,
  \item
    formulera och kritiskt utvärdera, analysera och/eller utvärdera
    vetenskapliga frågeställningar, teorier och metoder,
  \item
    planera och med adekvata metoder genomföra kvalificerade uppgifter
    inom givna ramar samt att utvärdera detta arbete,
  \item
    delta i forsknings‐ och utvecklingsarbete och därigenom bidra till
    kunskapsutvecklingen,
  \item
    kritiskt och systematiskt integrera kunskap och att analysera,
    bedöma och hantera komplexa företeelser, frågeställningar och
    situationer även med begränsad information samt visa förmåga att
    modellera, simulera, förutsäga och utvärdera skeenden även med
    begränsad information, samt
  \item
    muntligt och skriftligt klart redogöra för och diskutera sina
    slutsatser, samt den kunskap och de argument som ligger till grund
    för dessa.
  \end{enumerate}
\item
  \emph{Värderingsförmåga och förhållningssätt}

  \begin{enumerate}
  \def\labelenumii{\Alph{enumi}.\arabic{enumii}.}
  \tightlist
  \item
    Göra bedömningar med hänsyn till relevanta vetenskapliga,
    samhälleliga och etiska aspekter inom ramen för det specifika
    självständiga arbetet samt visa medvetenhet om etiska aspekter på
    forsknings‐ och utvecklingsarbete,
  \item
    identifiera och analysera vetenskapens och ingenjörens roll i
    samhället, samt
  \item
    identifiera sitt behov av ytterligare kunskap och att ta ansvar för
    att fortlöpande utveckla sin kunskap och kompetens.
  \end{enumerate}
\end{enumerate}

\subsection*{Kursinnehåll}

Kursens innehåll bestäms för varje student eller par av studenter i
samråd med handledare och examinator. Det självständiga arbetet skall
utföras inom datavetenskap och mjukvaruteknik.

\subsection*{Undervisnings- och
arbetsformer}

Kursen utgörs av ett självständigt arbete. För varje student eller par
av studenter utses en examinator och en eller flera handledare (t.ex. om
arbetet görs vid ett företag). Det självständiga arbetet utgör det
avslutande momentet på utbildningen.

Varje självständigt arbete skall läggas fram vid ett seminarium. Varje
student skall opponera på ett annat arbete samt auskultera på minst tre
andra framläggningar på samma nivå eller högre. Auskultation kan ske
från och med termin 11 och genomföras innan de egna arbetet för läggas
fram

Närvaro vid egen framläggning och opponering är obligatorisk.

\subsection*{Examination}

Examinationen av kursen delas in i följande moment:

\begin{longtable}[]{@{}llcc@{}}
\toprule
\textsf{Kod} & \textsf{Benämning} & \textsf{Betyg} & \textsf{Poäng}\tabularnewline
\midrule
\endhead
\texttt{RAP1} & Rapport och framläggning & A-F & 27\tabularnewline
\texttt{OPP1} & Opponering & G-U & 1,5\tabularnewline
\texttt{ASK1} & Auskultation & G-U & 1,5\tabularnewline
\bottomrule
\end{longtable}

För godkänt betyg på kursen krävs betyg G på \texttt{OPP1} och \texttt{ASK1} samt minst
betyg E på \texttt{RAP1}. Slutbetyget bestäms från \texttt{RAP1}.

\subsection*{Måluppfyllelse}

Examinationsmomenten till lärandemålen enligt följande:

\begin{longtable}[]{@{}lccc@{}}
\toprule
\textsf{Lärandemål} & \texttt{RAP1} & \texttt{OPP1} & \texttt{ASK1}\tabularnewline
\midrule
\endhead
A.1 & \faCheck & \faCheck & \faCheck\tabularnewline
A.2 & \faCheck & \faCheck & \faCheck\tabularnewline
B.1 & \faCheck & &\tabularnewline
B.2 & \faCheck & \faCheck & \faCheck\tabularnewline
B.3 & \faCheck & &\tabularnewline
B.4 & \faCheck & &\tabularnewline
B.5 & \faCheck & \faCheck & \faCheck\tabularnewline
B.6 & \faCheck & \faCheck &\tabularnewline
C.1 & \faCheck & &\tabularnewline
C.2 & \faCheck & &\tabularnewline
C.3 & \faCheck & \faCheck & \faCheck\tabularnewline
\bottomrule
\end{longtable}

\subsection*{Kurslitteratur}

Studenterna förväntas söka efter lämplig kurslitteratur på egen hand
eller i samråd med handledare.

\subsection*{Övrigt}

Kursen genomförs på ett sätt sådant att både kvinnor och mäns kunskap och erfarenhet utvecklas och görs synlig.
