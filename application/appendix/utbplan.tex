\chapter{Utbildningsplan\label{app:utbplan}}

\section*{Civilingenjör i mjukvaruteknik, 300 hp}

\subsection*{Nivå}

Grundnivå och avancerad nivå.

\subsection*{Förkunskaper}

Grundläggande behörighet samt Fysik 2, Kemi 1 och Matematik 4
(områdesbehörighet A9)

\subsection*{Programbeskrivning}

Mjukvara kan ses som den osynliga infrastrukturen i den digitaliserade
ekonomin, och finns överallt, från självständiga produkter till inbäddad
i och en allt viktigare del av traditionella produkter. Det finns därför
ett behov av välutbildad personal för utveckling av den mjukvara som
styr dagens och morgondagens system.

Utbildningen ger goda kunskaper i datavetenskap samt mjukvaruutveckling och metoder för detta. Studierna skall
förbereda för arbete i verksamheter där mjukvara används och
utvecklas, och en utexaminerad civilingenjör förväntas efter en tid kunna
gå in i samtliga utvecklingsrelaterade roller i ett
mjukvaruutvecklingsprojekt, från teknisk expert till projektledare.
Utbildningen ger en bra grund för att starta och driva egen
verksamhet samt för en akademisk karriär med forskning inom datavetenskap.

Förutom goda kunskaper i datavetenskap ger utbildningen även en bred
matematisk grund och goda kunskaper i ingenjörsmässigt tänkande och
problemlösning. Utbildningen är projektdriven och innehåller stora
inslag av arbeten i grupp, där kommunikation, professionella
färdigheter och förhållningssätt samt systemtänkande tränas. De fyra
större projekten täcker hela utvecklingscykeln, från behov och idé till
operation, i realistiska miljöer och ofta i samarbete med näringslivet.

År 4 och 5 innehåller tre fördjupningsterminer där studenten fördjupar sig inom
modelldriven utveckling, dataintensiva system samt visualisering och
dataanalys.

\subsection*{Mål}

För civilingenjörsexamen skall studenten visa sådan kunskap och förmåga
som krävs för att självständigt arbeta som civilingenjör.

\subsubsection*{A. Kunskap och förståelse}

För civilingenjörsexamen i mjukvaruteknik skall studenten:

\begin{enumerate}
\def\labelenumi{A.\arabic{enumi}.}
\tightlist
\item
  Visa brett kunnande och förståelse inom det valda teknikområdets
  (mjukvaruteknik) vetenskapliga grund och beprövade erfarenhet,
  inbegripet kunskaper i matematik och naturvetenskap, väsentligt
  fördjupade kunskaper inom vissa delar av området, samt fördjupad
  insikt i aktuellt forsknings‐ och utvecklingsarbete, samt
\item
  visa fördjupad metodkunskap inom huvudområdet för utbildningen.
\end{enumerate}

\subsubsection*{B. Färdighet och förmåga}

För civilingenjörsexamen i mjukvaruteknik skall studenten:

\begin{enumerate}
\def\labelenumi{B.\arabic{enumi}.}
\tightlist
\item
  Visa förmåga att med helhetssyn kritiskt, självständigt och kreativt
  identifiera, formulera och hantera komplexa frågeställningar,
\item
  visa förmåga att skapa, analysera och kritiskt utvärdera olika
  tekniska lösningar,
\item
  visa förmåga att planera och med adekvata metoder genomföra
  kvalificerade uppgifter inom givna ramar samt att utvärdera detta
  arbete,
\item
  visa sådan färdighet som fordras för att delta i forsknings‐ och
  utvecklingsarbete eller för att självständigt arbeta i annan
  kvalificerad verksamhet och därigenom bidra till kunskapsutvecklingen,
\item
  visa förmåga att kritiskt och systematiskt integrera kunskap och att
  analysera, bedöma och hantera komplexa företeelser, frågeställningar
  och situationer även med begränsad information samt visa förmåga att
  modellera, simulera, förutsäga och utvärdera skeenden även med
  begränsad information,
\item
  visa förmåga att utveckla och utforma produkter, processer och system
  med hänsyn till människors förutsättningar och behov och samhällets
  mål för ekonomiskt, socialt och ekologiskt hållbar utveckling,
\item
  visa förmåga till lagarbete och samverkan i grupper med olika
  sammansättning, samt
\item
  visa förmåga att i såväl nationella som internationella sammanhang
  muntligt och skriftligt i dialog med olika grupper klart redogöra för
  och diskutera sina slutsatser och den kunskap och de argument som
  ligger till grund för dessa,
\end{enumerate}

\subsubsection*{C. Värderingsförmåga och förhållningssätt}

För civilingenjörsexamen i mjukvaruteknik skall studenten:

\begin{enumerate}
\def\labelenumi{C.\arabic{enumi}.}
\tightlist
\item
  Visa förmåga att göra bedömningar med hänsyn till relevanta
  vetenskapliga, samhälleliga och etiska aspekter samt visa medvetenhet
  om etiska aspekter på forsknings‐ och utvecklingsarbete,
\item
  visa insikt i vetenskapens och teknikens möjligheter och
  begränsningar, dess roll i samhället och människors ansvar för hur den
  används, inbegripet sociala och ekonomiska aspekter samt miljö‐ och
  arbetsmiljöaspekter, samt
\item
  visa förmåga att identifiera sitt behov av ytterligare kunskap och att
  ta ansvar för att fortlöpande utveckla sin kunskap och kompetens.
\end{enumerate}

\subsection*{Innehåll och struktur}

\subsubsection*{Organisation}

Programmet administreras av en programansvarig. Ett programråd finns
inrättat som står för programmets kvalitet, dess utveckling och koppling
till arbetslivet. Programrådet består av programansvarig,
lärarrepresentanter, studeranderepresentanter, samt representanter från
näringslivet.

\subsubsection*{Programöversikt}

Utbildningen omfattar fem år och 300 högskolepoäng (hp). De tre första
årskurserna är på grundnivå och de avslutande två är på avancerad nivå.
Utbildningen innehåller två självständiga arbeten, ett omfattande 15 hp
i slutet av årskurs 3 och ett omfattande 30 hp i slutet av årskurs 5.

Utbildningens 300 hp är fördelade enligt följande: 190
hp datavetenskap, 45 hp matematik, 15 hp fysik, 5 hp elektroteknik.
Utbildningen innehåller även 20 hp kurser inom teknik, människa och samhälle
samt 25 hp valbara kurser. Utbildningen innehåller sammanlagt 32,5 hp
projektarbeten.

Årskurs 1--3 består av obligatoriska kurser. Dessa är schemalagda
på ett sådant sätt att två eller tre kurser läses samtidigt och tenteras
i samma period. Årskurs 4 och 5 består av obligatoriska och
villkorligt valfria kurser och projekt. Varje termin består av fyra
kurser och ett projekt. Kurserna är schemalagda så att projektet går
över hela terminen och två kurser läses samtidigt med projektet.

\subsubsection*{Kurser i programmet}

\paragraph*{Årskurs 1}

\begin{itemize}
\tightlist
\item
  \emph{\textbf{Diskret matematik}, 7,5 hp (G1N)}. Inledande kurs i
  matematik som behandlar diskreta strukturer, såsom träd och grafer,
  logik och kombinatorik.
\item
  \emph{\textbf{Programmering och datastrukturer}$^*$, 7,5 hp (G1N)}.
  Inledande kurs i programmering och datastrukturer i
  programmeringsspråket Python.
\item
  \emph{\textbf{Linjär algebra}, 7,5 hp (G1F)}. Inledande kurs i linjär
  algebra som behandlar vektorer i planet och rummet, lösning av linjära
  ekvationssystem och egenvektorer. Kursen behandlar även programmering
  i Matlab.
\item
  \emph{\textbf{Introducerande projekt}$^*$, 7,5 hp (G1F)}. Kurs som
  introducerar hur man arbetar i mjukvaruutvecklingsprojekt och
  yrkesrollen mjukvaruingenjör. Särskilt fokus läggs på krav, verktyg
  och grupparbete.
\item
  \emph{\textbf{Envariabelanalys 1}, 5 hp (G1F)}. Inledande kurs i
  analys som behandlar gränsvärde och kontinuitet, derivator och
  integraler, samt matematisk modellering med differentialekvationer.
\item
  \emph{\textbf{Databaser och datamodellering}$^*$, 5 hp (G1F)}. Kurs som behandlar hur data
  modelleras och lagras i, och hämtas ur databaser. Behandlar
  frågespråket SQL samt hur program i Python kan kopplas mot databaser.
\item
  \emph{\textbf{Objektorienterad programmering}$^*$, 5 hp (G1F)}.
  Fortsättningskurs i programmering som behandlar objekt-orienterad
  modellering i UML och implementation i Java.
\item
  \emph{\textbf{Tillämpad sannolikhetslära och statistik}, 7,5 hp
  (G1F)}. Inledande kurs i sannolikhetslära och statistik som behandlar
  dataanalys och hur slutsatser kan dras ut observerad data.
\item
  \emph{\textbf{Mekanik}, 7,5 hp (G1F)}. Inledande kurs i fysik som
  behandlar mekanik.
\end{itemize}

\paragraph*{Årskurs 2}

\begin{itemize}
\tightlist
\item
  \emph{\textbf{Jämnlöpande program}$^*$, 7,5 hp (G1F)}. Fortsättningskurs i
  programmering som behandlar jämnlöpande (concurrent) program och
  trådprogrammering i Java, de problem som kan uppstå när resurser
  delas, samt lösningar såsom låsningsalgoritmer.
\item
  \emph{\textbf{Ellära och magnetism}, 7,5 hp (G1F)}. Inledande kurs i
  fysik som behandlar ellära och magnetism, med viss fokus på
  kretselektronik.
\item
  \emph{\textbf{Teknisk kommunikation}, 5 hp (G1N)}. Inledande kurs som
  behandlar muntlig och skriftlig kommunikation exempelvis hur man
  presenterar en lösning till ett tekniskt problem eller skriver en
  teknisk rapport.
\item
  \emph{\textbf{Algoritmer}$^*$, 5 hp (G1F)}. Fortsättningskurs i algoritmer
  som behandlar komplexitetsananlys och komplexitetsklasser, strategier
  för algoritmdesign, samt vanliga algoritmer.
\item
  \emph{\textbf{Mjukvaruutvecklingsprojekt}$^*$, 10 hp (G1F)}. Inledande
  kurs i mjukvaruutveckling och det första i en serie av fyra projekt
  som behandlar hela utvecklingscykeln, från idé och behov till
  operation. Kursen lägger särskilt fokus på agila metoder och arbete i
  grupp.
\item
  \emph{\textbf{Hållbar utveckling}, 5 hp (G1N)}. Inledande kurs i
  hållbar utveckling som behandlar hållbar utveckling ur såväl
  ekologiska, sociala, och ekonomiska aspekter, samt ur ett globalt,
  lokalt respektive industriellt perspektiv.
\item
  \emph{\textbf{Envariabelanalys 2}, 5 hp (G1F)}. Fortsättningskurs i
  analys som behandlar talföljder och serier samt fortsätter
  behandlingen av integraler.
\item
  \emph{\textbf{Flervariabelanalys}, 7,5 hp (G1F)}. Fortsättningskurs i
  analys som behandlar analys i flera variabler och dess tillämpningar
  inom datavetenskap.
\item
  \emph{\textbf{Datorns uppbyggnad}$^*$, 7,5 hp (G2F)}. Inledande kurs i
  datorteknik som behandlar hur en dator är uppbyggd, från kretsar till
  avbrott och minneshantering, samt hårdvarunära programmering i
  assembler och C.
\end{itemize}

\paragraph*{Årskurs 3}

\begin{itemize}
\tightlist
\item
  \emph{\textbf{Numeriska metoder}, 5 hp (G1F)}. Fortsättningskurs i
  matematik som behandlar numeriska metoder och hur de kan användas för
  att lösa problem inom datavetskap, till exempel grafproblem.
\item
  \emph{\textbf{Mjukvaruarkitektur}$^*$, 5 hp (G2F)}. Fortsättningskurs i
  mjukvaruutveckling som behandlar mjukvaruarkitektur och dess roll i
  utvecklingsprocessen exempelvis med avseende på krav.
\item
  \emph{\textbf{Inbyggda system}$^*$, 5 hp (G2F)}. Inledande kurs i inbyggda
  system som behandlar realtidssystemsproblematik såsom schemaläggning
  och garantier, gränsytan mellan hårdvara och mjukvara, exempelvis
  drivrutiner, samt mjukvara i industriella tillämpningar.
\item
  \emph{\textbf{Reglerteknik}, 5 hp (G1F)}. Inledande kurs i
  reglerteknik som behandlar hur system kan styras trots störningar,
  samt tillämpningar av detta inom datavetenskap.
\item
  \emph{\textbf{Datorgrafik}$^*$, 5 hp (G2F)}. Inledande kurs i datorgrafik
  som behandlar bland annat algoritmer för att rita objekt i rastergrafik,
  färgmodeller, samt rendrering av och belysningsmodeller för 3D-objekt.
\item
  \emph{\textbf{Datornät}$^*$, 5 hp (G2F)}. Inledande kurs som behandlar
  datakommunikation och datornät med fokus på TCP/IP-modellen och
  nätverksprogrammering.
\item
  \emph{\textbf{Industriell ekonomi}, 5 hp (G1N)}. Inledande kurs i ekonomi som behandlar industriell och företagsekonomi, exempelvis ekonomiska modeller, ekonomisk analys och verksamhetsstyrning. Ekonomi behandlas även ur ett samhällsperspektiv.  
\item
  \emph{\textbf{Vetenskapliga metoder}, 5 hp (G2F)}. Inledande kurs i vetenskapliga metoder som behandlar vetenskapsteori och dess historia,
samt olika vetenskapliga metoder, t.ex. systematiska textstudier och hypotesprövning. Metoderna exemplifieras och fördjupas med
mjukvarutekniska frågeställningar.
\item
  \emph{\textbf{Datorsäkerhet}$^*$, 5 hp (G1N)}. Inledande kurs i datorsäkerhet som behandlar
  datorsäkerhet utifrån tidigare kurser inom datavetenskap, exempelvis inom
  nätverk och databaser, och belyser deras innehåll ur ett
  säkerhetsperspektiv.
\item
  \emph{\textbf{Självständigt arbete}$^*$, 15 hp (G2E)}. Kurs där studenten självständigt förväntas formulera ett problem
  inom mjukvaruteknik samt presentera och utvärdera en lösning till
  detta.
\end{itemize}

\paragraph*{Årskurs 4}

\begin{itemize}
\tightlist
\item
  \emph{\textbf{Modellering och simulering av system}$^*$, 5 hp (A1N)}.
  Fördjupningskurs i modelldriven utveckling som behandlar hur system
  med stora krav på säkerhet och tillförlitlighet kan modelleras
  och simuleras för att verifiera egenskaper innan de implementeras.
\item
  \emph{\textbf{Kompilatorkonstruktion}$^*$, 5 hp (A1N)}. Fördjupningskurs i
  modelldriven utveckling som behandlar hur datorspråk kan formuleras
  och hur de kan översättas, exempelvis från Java till exekverbar kod.
  Kursen introducerar ett antal viktiga principer såsom grammatiker,
  typinferens, semantisk analys, och en fördjupning i
  tillståndsmaskiner.
\item
  \emph{\textbf{Projekt i modellbaserad utveckling}$^*$, 10 hp (A1N)}. Projektkurs
  i modelldriven utveckling där studenterna tillämpar kunskaper i
  modellering, arkitektur, och simulering för att genomföra ett agilt
  utvecklingsprojekt i en realistisk miljö och med ett öppet problem.
  Särskild fokus läggs på att tillämpa agila metoder.
\item
  \emph{\textbf{Formella metoder}$^*$, 5 hp (A1F)} (kan ersättas av valbar
  kurs). Fördjupningskurs i modelldriven utveckling som behandlar hur
  egenskaper hos modeller och program kan formellt verifieras, exempelvis med
  avseende på säkerhet.
\item
  \emph{\textbf{Optimering}, 5 hp (G2F)} (kan ersättas av valbar kurs).
  Fördjupningskurs i modelldriven utveckling som behandlar hur modeller
  kan optimeras. Kursen ges av matematik med fokus på tillämpningar inom
  datavetenskap såsom approximeringsalgoritmer och heltalsprogrammerings.
\item
  \emph{\textbf{Maskininlärning}$^*$, 5 hp (A1N)}. Fördjupningskurs i
  dataintensiva system som behandlar artificiell intelligens och lärande
  system, med fokus på statistisk maskininlärning och klustering.
\item
  \emph{\textbf{Parallelldatorprogrammering}$^*$, 5 hp (A1N)}. Fördjupningskurs i
  dataintensiva system som behandlar hur parallella datorsystem och
  acceleratorer kan användas för att lösa problem som behandlar stora
  datamängder. Vanliga arkitekturer och metoder för att dela upp problem
  behandlar och särskild fokus läggs på grafikprocessorer och lösningar
  till problem inom linjär algebra.
\item
  \emph{\textbf{Projekt i dataintensiva system}$^*$, 10 hp (A1N)}
  Projektkurs i dataintensiva system där studenterna tillämpar kunskaper
  i maskininlärning och parallellprogrammering för att genomföra
  ett agilt utvecklingsprojekt i en realistisk miljö och med ett öppet
  problem. Särskild fokus läggs på effektiva agila metoder, så kallad
  lean agile.
\item
  \emph{\textbf{Djup maskininlärning}$^*$, 5 hp (A1F) (kan ersättas av valbar kurs)}. Fördjupningskurs i dataintensiva system som behandlar
  så kallad djup maskininlärning och artificiella neuronnät.
\item
  \emph{\textbf{Lean startup}, 5 hp (G2F) (kan ersättas av valbar kurs)}. Fortsättningskurs i ekonomi som behandlar entreprenörsskap och innovation, samt hur man startar, driver och förändrar verksamheter. Kursen tar en praktiskt ansats och tidigare erfarenheter från lean och agil sätts i sammanhang.
\end{itemize}

\paragraph*{Årskurs 5}

\begin{itemize}
\tightlist
\item
  \emph{\textbf{Informationsvisualisering}$^*$, 5 hp (A1N)}.
  Fördjupningskurs i visualisering och dataanalys som behandlar hur
  visualisering kan hjälpa människor att analysera och förstå abstrakt
  data. Interaktion och användarupplevelse behandlas också.
\item
  \emph{\textbf{Datautvinning}$^*$, 5 hp (A1F)}. Fördjupningskurs i
  visualisering och dataanalys som behandlar metoder för att skapa
  mening i ostrukturerad data, exempelvis metoder för analys av (sociala)
  nätverk.
\item
  \emph{\textbf{Projekt i visualisering och dataanalys}$^*$, 10 hp (A1F)}.
  Projektkurs i visualisering och dataanalys där studenterna tillämpar
  kunskaper i visualisering och datautvinning för att genomföra
  ett agilt utvecklingsprojekt i en realistisk miljö och med ett öppet
  problem. Särskild fokus läggs på att självständigt genomföra ett agilt
  projekt.
\item
  \emph{\textbf{Avancerad informationsvisualisering och tillämpningar}$^*$,
  5 hp (A1F) (kan ersättas av valbar kurs).} Fördjupningskurs i
  visualisering och dataanalys som behandlar hur
  informationsvisualisering kan användas för en rad tillämpningar 
  inom exempelvis bioinformatik, geografi och mjukvaruutveckling.
\item
  \emph{\textbf{Vetenskapliga metoder inom datavetenskap}$^*$, 5 hp (A1F)}.
  Fördjupningskurs i vetenskapliga metoder som behandlar aktuell
  forsknings och metoder inom datavetenskap. Kursen är en seminariekurs
  där studenterna presenterar och opponerar på publicerade vetenskapliga
  arbeten.
\item
  \emph{\textbf{Självständigt arbete}$^*$, 30 hp (A2E)}. Kursen är ett
  självständigt arbete där studenten skall utveckla fördjupade
  kunskaper, förståelse, förmågor och förhållningssätt inom
  utbildningens sammanhang. Examensarbetet skall ligga i slutet av
  utbildningen och innebära en fördjupning och syntes av tidigare
  förvärvade kunskaper.
\end{itemize}

$*$ anger kurs inom huvudområdet datavetenskap.

Närmare beskrivning av i programmet ingående kurser ges i separata
kursplaner.

\subsubsection*{Samhällsrelevans}

Programmets studenter får vid flera tillfällen under programmets gång
möta representanter från arbetslivet, till exempel innehåller flera kurser
gästföreläsare eller studiebesök. Programmet innehåller fem projekt och
två självständiga arbeten som kan vara förlagda till eller genomföras
tillsammans med företag eller andra organisationer. Kurserna utformas i
samråd med näringslivet för att säkerställa att realistiska problem och
frågeställningar används.

\subsection*{Internationalisering}

Det finns under termin 7--9 möjlighet att läsa en eller två terminer vid
en utländsk teknisk högskola eller universitet. Lärosäte och studieplan
bestäms i samråd med programansvarig.

\subsubsection*{Perspektiv i utbildningen}

Mjukvara finns i princip över allt och är en allt viktigare del av traditionella produkter. Då mjukvara är
ett tvärsnitt av samhället måste ett antal perspektiv belysas inom
utbildningen.

En stor del av mjukvaruutveckling sker i stora, ofta internationella
team. Detta medför att begrepp som (social) hållbar utveckling, genus,
mångfald osv måste beröras i utbildningen.

Då mjukvara är en viktig del av samhället måste denna roll och
konsekvenserna av den diskuteras. Vilka risker medför datalagring
och vilka konsekvenser kan säkerhets- och tillförlitlighetsproblem få?
Vilken roll har den enskilde mjukvaruingenjören i detta, vilka etiska
frågeställningar finns? Detta perspektiv innehåller även resonemang
kring teknisk hållbarhet, dvs hur utvecklar man mjukvarusystem med lång
livslängd, exempelvis TCP/IP, och vad måste man tänka på.

Att mjukvara används i en allt större omfattning måste ännu större fokus
läggas på användbarhet, användarupplevelse och tillgänglighet, både i
samhället och i utbildningen.

Utbildningen lägger stor vikt vid dessa perspektiv, som belyses i
teoretiska kurser och praktiseras i projektkurser.

\subsection*{Kvalitetsutveckling}

Ett antal studenter utses vid varje kurs till kursrepresentanter som
skall representeras studentgruppen vid utvärderingen. Dessa träffar
lärare/kursansvarig vid några tillfällen under kursens gång. Kurserna
utvärderas genom skriftlig enkät, och efter att denna samställts
sammanträder programansvarig, ansvarig lärare och kursrepresentanterna för
att skapa en en utvärderingsrapport och en åtgärdsplan för nästa gång
kursen ges (om sådan behövs). Utvärderingsrapport och åtgärdsplan från
föregående år skall finnas tillgänglig i kursens kursrum eller via
kursens webbplats.

Programmet utvärderas årligen av programrådet utifrån kursvärderingar,
styrdokument, näringslivet, och alumner. Resultatet av denna utvärdering
presenteras för studenter och lärare vid ett seminarium under
vårterminen. Föregående års utvärderingsrapport skall finns tillgänglig
via programmets programrum eller programmets webbplats.

\subsection*{Examen}

Efter avklarade studier på programmet samt då avklarade studier
motsvarar de fordringar som finns angivna i Högskoleförordningens
examensordning samt i den lokala examensordningen för Linnéuniversitetet
kan studenten ansöka om examen.

Examen benämns Civilingenjörsexamen med inriktning mot mjukvaruteknik.
Examens engelska översättning är Master of Science in Engineering with a
specialization in Software Technology.

Efter den grundläggande delen på 180 hp kan delexamen utfärdas.
Examensbenämningen är Filosofie kandidatexamen, huvudområde
Datavetenskap (Bachelor of Science, major subject Computer Science).

\subsection*{Övrigt}

Det krävs minst 90 hp godkända kurser inom programmet för att få påbörja
det självständiga arbetet i termin 6. Detta krav måste vara uppfyllt vid
starten på termin 5. För tillträde till termin 7 krävs minst 150 hp inom
programmet vid terminsstart. För tillträde till det självständiga
arbetet i termin 10 krävs minst 240 hp inom programmet, samtliga kurser
årskurs 1--3 samt 30 hp på avancerad nivå inom huvudämnet.

Studenter som inte uppfyller dessa krav skall vända sig till programmets
studievägledare och programansvarig för att göra en plan för kommande
studier.
