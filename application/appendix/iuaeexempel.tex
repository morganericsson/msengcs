\chapter{Exempel på ITUE-matris på kursnivå\label{app:itue-exempel}}

\begin{sidewaystable}[H]
\centering
\caption{IUAE-matris för kursen ``Djup maskininlärning''.\label{tab:itue-exempel}}
\resizebox{.8\linewidth}{!}{% 
\begin{tabular}{cccclp{20cm}}
\toprule
\textsf{\textbf{Mål}} & \textsf{\textbf{Introducera}} & \textsf{\textbf{Undervisa}} & \textsf{\textbf{Använda}}&  \textsf{\textbf{Examinera}} & \textsf{\textbf{Kommentar}} \tabularnewline
\midrule
1.1 &   &   & \faCheck &                  & Vektorer och matriser, sannolikhetsfördelningar, vektorvärda funktioner, kedjereglen, partiella derivator \tabularnewline
1.2 &   & \faCheck & \faCheck & \texttt{TEN1}, \texttt{LAB1}, \texttt{PRS1} & \textbf{U}:  neuronnät, faltningsneuronnät, återkommande nätverk, anpassning av hyperparametrar, reglering, långt korttidsminne (LTSM), reinforcement learning.  \textbf{A}: Statistisk maskininläning programmering \tabularnewline
1.3 &   & \faCheck & \faCheck & \texttt{LAB1}, \texttt{PRS1}       & \textbf{U}: Fördjuping i olika arkitkturer för neurala nät och deras tillämpning, verktyg för att skapa neural nät, t.ex. tensorflow. \textbf{A}: Python, programmering för grafikprocessorer (CUDA). \tabularnewline
\midrule
2.1 &   & \faCheck & \faCheck & \texttt{TEN1}, \texttt{LAB1}       & \textbf{U}: Identifiera vilken typ av neuronnät/lärande som bäst lämpar sig för ett problem. Jämför mot  inlärningsmetoder som inte baseras på neuronnät och optimeringsmetoder. Implementation på grafikprocessor, vilka vinster, när lämpar det sig bäst. Uppskattning av hur mycket minne som behövs, etc. \textbf{A}: metoder från problem från den tidigare kursen i maskininlärning. \tabularnewline
2.2 &   & \faCheck & \faCheck & \texttt{LAB1}             & \textbf{U}: Experimentera med olika arkitekturer för neuronnät. \textbf{A}: Dataanvändning (träningsset/test, osv), validering. \tabularnewline
2.3 &   & \faCheck & \faCheck & \texttt{TEN1}             & \textbf{U}: val/motivering av arkitektur. Fördelar och nackdelar med olika val. \textbf{A}: var/hur maskininlärning kan användas i ett system. \tabularnewline
2.4 &   & \faCheck &   & \texttt{PRS1}             & Förmåga att tillgodose sig forskningsresultat inom maskininlärning samt att kritiskt granska dessa.\tabularnewline
2.5 &   &   &   &                  & \tabularnewline
\midrule
3.1 &   &   & \faCheck & \texttt{LAB1}, \texttt{PRS1}       & Arbete sker i grupp \tabularnewline
3.2 &   & \faCheck &   & \texttt{PRS1}             & Presentation av vetenskaplig artikel inom maskininlärning. Opponering på annan artikel/presentation. \tabularnewline
3.3 &   & \faCheck & \faCheck & \texttt{PRS1}             & \textbf{U}: Presentation på engelska \textbf{A}: Kursen ges på engelska \tabularnewline
4.1 & \faCheck &   &   &                  & Maskininlärnings påverkan på samhället, vad blir konsekvenserna av utbredd artificiell intelligens, vilka faror finns osv. Hur regleras det, bör det regleras, osv. \tabularnewline
4.2 &   &   &   &                  & \tabularnewline
4.3 &   &   & \faCheck &                  & Förmåga att planera maskininlärningsstrategier och pipeline för projekt. \tabularnewline
4.4 &   &   & \faCheck &                  & Förmåga att modellera hur maskininlärniningen skall se hur och hur den placeras inom projektet. \tabularnewline
4.5 &   & \faCheck &   & \texttt{LAB1}             & Effektiv implementation av olika algoritmer på hårdvara. Test och verifikation av  prestanda hos systemet. \tabularnewline
4.6 &   & \faCheck &   & \texttt{LAB1}             & Optimering och förbättring av maskininlärningen baserat på systemets prestanda, tillgänglig data, osv. \tabularnewline
\midrule
5.1 & \faCheck &   &   &                  & Maskininlärnings påverkan på samhället, vad blir konsekvenserna av utbredd artificiell intelligens, vilka faror finns osv. Hur regleras det, bör det regleras, osv. \tabularnewline
5.2 &   &   &   &                  & \tabularnewline
5.3 &   &   &   &                  & \tabularnewline
5.4 &   &   &   &                  & \tabularnewline
5.5 & \faCheck &   &   &                  & Presentation av vetenskaplig artikel\tabularnewline
5.6 & \faCheck &   &   &                  & Presentation av vetenskaplig artikel\tabularnewline
\bottomrule
\end{tabular}
}
\end{sidewaystable}