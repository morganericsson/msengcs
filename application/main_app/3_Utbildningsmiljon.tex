\chapter{Utbildningsmiljö\label{ch:miljo}}

\begin{tcbdoublebox}
\emph{Avsnittet täcker aspekten Utbildningsmiljö i den föreslagna ansökningsmallen. Här anges vilken typ av forskning som bedrivs inom de institutioner som är inblandade i genomförandet av den föreslagna utbildningen samt hur utbildningen och dess innehåll formas av forskningen. Avsnittet beskriver även samverkan med det omgivande samhället.}
\end{tcbdoublebox}

\section{Utbildnings- och forskningsmiljön inom datavetenskap och
medieteknik}

Utbildningsmiljön inom datavetenskap och medieteknik är uppbyggd kring de ämnen som institutionen ansvarar för: datavetenskap, medieteknik och datateknik.

Institutionen har en omfattande forskningsverksamhet som spänner över en stor del av det datavetenskapliga fältet med flera tillämpningsområden. Varje forskningsprofil består av en eller flera forskargrupper och bildar en komplett kunskapsmiljö med ansvar för forskning och utbildning inom aktuellt område. Miljöerna består av flera seniora och juniora forskare, doktorander samt i viss utsträckning även adjunkter. Inom varje miljö bedrivs omfattande anslags- och externfinansierad forskning och forskarutbildning inom respektive område.

Institutionen har fyra forskningsområden som lyfts fram för att ytterligare profilera forskningen vid institutionen:

\begin{itemize}
\item
  CPS (Cyber-fysiska System) och AdaptWise -- Mjukvara för självanpassande cyber-fysiska
  system.
\item
  DISTA -- Teknologi för dataintensiva mjukvarusystem.
\item
  ISOVIS -- Informations- och mjukvaruvisualisering.
\item
  CeLeKT -- Teknologi för interaktiva lärmiljöer som stödjer kollaborativ, upptäcktsbaserad inlärning inom
  komplexa ämnen. 
\end{itemize}

Kunskapsmiljön \emph{Cyber-fysiska System} består av två forskargrupper som i samverkan studerar system med inbyggda mjukvarukomponenter (cyber) och fysiska komponenter, exempelvis mekaniska delsystem, energisystem, mänskliga aktiviteter och som en konsekvens den omgivande miljön. Denna typ av system kräver att hänsyn tas till flera faktorer utöver systemets huvudsakliga funktionalitet, såsom realtidsaspekter, energiförbrukning, tillförlitlighet, tillgänglighet och säkerhet. Forskargruppen, AdaptWise, studerar metoder och tekniker för självanpassande mjukvarusystem, vilket har flera tillämpningar inom cyber-fysiska system. Gruppen studerar mjukvarutekniker som skapar förutsättningar för att konstruera system som uppfyller efterfrågade systemfaktorer även om någon förutsättning ändras. Självanpassning är en princip för att hantera osäkerheter i driftsatta system genom kontinuerlig anpassning av mjukvaran, exempelvis vid dynamisk resurstillgång eller förändringar av systemets krav. De personer som ingår i miljön och deras fält anges i tabell~\ref{tab:kmcpsaw}. Observera att samma person kan vara del av flera miljöer.

\begin{table}
\centering
\caption{Kunskapsmiljön CPS\label{tab:kmcpsaw}}
\begin{tabular}{@{}llp{8cm}@{}}
\toprule
\textbf{\textsf{Titel}} & \textbf{\textsf{Namn}} & \textbf{\textsf{Område}}\tabularnewline
\midrule
Professor & Shiyan Hu & Cyber-fysiska system och säkerhet.\tabularnewline
Professor & Danny Weyns & Systemarkitektur och formell
verifiering.\tabularnewline
Professor & Welf Löwe & Kompilatorer och optimering.\tabularnewline
Docent & Mauro Caporuscio & Arkitektur och modellering.\tabularnewline
Lektor & Narges Khakpour & Säkerhet, mjukvarusäkerhet och formella
metoder.\tabularnewline
Lektor & Diego Perez & Modellering, prestandamodellering och modellbaserad
utveckling.\tabularnewline
Lektor & Francesco Flammini & Modellering av cyber-fysiska system,
tillförlitlighet och säkerhet.\tabularnewline
Lektor & Jesper Andersson & Systemarkitektur, modellering och modellbaserad
utveckling.\tabularnewline
Lektor & Jonas Lundberg & Kompilatorer och optimering.\tabularnewline
\bottomrule
\end{tabular}
\end{table}

Den andra forskargruppen studerar hur ingenjörer kan hantera den komplexitet som de möter vid utvecklingen av cyberfysiska system. För detta krävs en samlad kompetens med flera specialiseringar för konkreta systemfaktorer. Några av utmaningarna finns inom det som populärt kallas modellbaserad utveckling i och med modellering av de cyberfysiska systemen för vidare analys, simulering och transformationer. I modellbaserad utveckling tillkommer även olika typer av verifieringar och realtidsanalyser.

Den samlade miljön ansvarar tillsammans med resurser från Institutionen för matematik för kurser och projekt inom Fördjupningsområdet \emph{Modelldriven utveckling} under termin 7 på den föreslagna utbildningen.

Inom kunskapsmiljön \emph{Data Intensive Software Technology and Applications (DISTA)} studeras teknologier för dataintensiva mjukvarusystem. Dataintensiva mjukvarusystem omvandlar olika typer av data till information och vidare konverteras information till kunskap. Artificiell intelligens är ett viktigt område då maskininlärning används för att automatisera transformationer av data och information till kunskap i form av prognoser, resonemang, planering och beslutsstöd baserat på stora datamängder. Ett annat område av central betydelse är skalbara beräkningsteknologier som möjliggör en hantering av stora datavolymer från exempelvis dataströmmar i realtid. Miljön studerar därför effektiva tekniker för parallellbearbetning dels på kraftfulla specialiserade parallelldatorer dels på mer generella molnbaserade infrastrukturer.

Kunskapsmiljön driver forskningscentret \emph{Data Intensive Sciences and Applications (DISA)}, en av universitetets inrättade spetsmiljöer, ett så kallat Linnaeus University Center (LNUC). DISA har ett tvärvetenskapligt perspektiv och samarbete med forskare från universitetets samtliga fakulteter och flera aktörer från det omgivande samhället. DISA arbetar således med ett antal konkreta applikationsområden där samverkan sker tvärvetenskapligt och med det omgivande samhället.

Kunskapsmiljön är ansvarig för fördjupningsområdet \emph{Dataintensiva system} som förläggs till termin 8 i det föreslagna civilingenjörsprogrammet. De personer som ingår i miljön och deras fält anges i tabell~\ref{tab:kmdista}. Fördjupningsområdet genomförs tillsammans med resurser från institutionen för matematik och andra forskare verksamma inom DISA.

\begin{table}
\centering
\caption{Kunskapsmiljön DISTA\label{tab:kmdista}}
\begin{tabular}{@{}llp{8cm}@{}}
\toprule
\textbf{\textsf{Titel}} & \textbf{\textsf{Namn}} & \textbf{\textsf{Område}}\tabularnewline
\midrule
Professor & Welf Löwe & Parallellbearbetning, maskininlärning och dataintensiva
tillämpningar.\tabularnewline
Docent & Morgan Ericsson & Datautvinning och dataintensiva tillämpningar.\tabularnewline
Lektor & Johan Hagelbäck & Maskininlärning och AI. \tabularnewline
Lektor & Jonas Lundberg & Optimering och dataintensiva
tillämpningar.\tabularnewline
Lektor & Sabri Pllana & Parallellbearbetning, optimering och maskininlärning.\tabularnewline
Lektor & Juwel Rana & Maskininlärning och stora datamängder.\tabularnewline
Lektor & Diego Perez & Molninfrastruktur.\tabularnewline
\bottomrule
\end{tabular}
\end{table}

\emph{Information and Software Visualisation (ISOVIS)} omfattar informationsvisualisering för en så kallad explorativ analys av komplexa informationsmängder, exempelvis inom biovetenskap, humaniora eller mjukvaruutveckling. Inom kunskapsmiljön studeras utmaningar kopplade till hur stora datamängder med en kombination av individcentrerad dataanalys och interaktiv visualisering kan skapa förutsättningar för en förbättrad förståelse och därmed ett förbättrat beslutsstöd. En interaktiv visualisering och visuell analys ger effektivare och mer tillförlitliga analyser av de komplexa datamängder som samlas in från anpassningsbara och dataintensiva system. Genom en bättre förståelse av information från system skapas bättre förutsättningar för att förstå och utveckla komplexa system. Kunskapsmiljön ansvarar för det föreslagna civilingenjörsprogrammets tredje fördjupningsområde, \emph{Visualisering och dataanalys}, som förlagts till termin 9. De personer som ingår i miljön och deras fält anges i tabell~\ref{tab:kmisovis}.

\begin{table}
\centering
\caption{Kunskapsmiljön ISOVIS\label{tab:kmisovis}}
\begin{tabular}{@{}llp{8cm}@{}}
\toprule
\textbf{\textsf{Titel}} & \textbf{\textsf{Namn}} & \textbf{\textsf{Område}}\tabularnewline
\midrule
Professor & Andreas Kerren & Informationsvisualisering och mjukvaruvisualisering.\tabularnewline
Lektor & Ilir Jusufi & Informationsvisualisering.\tabularnewline
Lektor & Aris Alissandrakis & Interaktion och virtuell verklighet.\tabularnewline
Lektor & Nuno Ortero & Kognition och interaktion.\tabularnewline
Lektor & Shahrouz Yousefi & Interaktion\tabularnewline
Postdoc. & Rafael Mattias & Informationsvisualisering och mjukvaruvisualisering.\tabularnewline
\bottomrule
\end{tabular}
\end{table}

De tre kunskapsmiljöerna ansvarar för varsitt fördjupningsområde inom det föreslagna programmet. Personer från de olika grupperna ansvarar även för kurser på grundnivå. Forskningsgruppen DISTA ansvarar bland annat för kursen \emph{Jämnlöpande programmering} i årskurs 2 och ISOVIS ansvarar bland annat för kursen \emph{Datorgrafik} i årskurs 3.

\subsection{Utbildnings- och forskningsmiljön inom matematik}

Institutionen för matematik ger kurser i matematik på grund- och avancerad nivå som har nära koppling till olika delar av datavetenskapen. Utbildningsmiljön matematik är till viss del gemensam med datavetenskapen. Kandidatprogrammet i matematik har ett starkt inslag av datavetenskap och därför har flera självständiga arbeten inom matematik på kandidatnivå en inriktning mot diskret matematik och andra till datavetenskapen närliggande områden.

Projekt- och självständiga arbeten i matematik har genomförts i samarbete med Institutionen för datavetenskap.

Forskning i matematik bedrivs i analys, algebra, talteori, matematisk fysik, matematisk statistik och matematisk modellering. Mer specifikt finns följande forskningsområden representerade: matematisk modellering i kvantfysik, vågutbredning, biologi, mikrolokal analys och pseudodifferentialkalkyl, stokastisk analys och finansmatematik, dynamiska system med biologiska tillämpningar, algebraisk dynamik och matematisk kryptering.

Nyligen har optimering och maskininlärning tillkommit som ämnen för såväl forskning som utbildning inom matematik. Dessa områden finns med under årskurs 4 i fördjupningsområdena Modellbaserad utveckling och Dataintensiva system. Institutionen kommer att delta i genomförandet av kurser och projekt. Forskargrupperna i matematik kommer att bidra med relevanta problemställningar och handledning i de avslutande självständiga arbetena inom programmet. Exempel på några problemområden är modellering och simulering samt maskininlärning och optimering.


\subsection{Utbildnings- och forskningsmiljön inom fysik och elektroteknik}

Institutionen för fysik och elektroteknik ansvarar för en kandidatutbildning och en utbildning på avancerad nivå inom fysik. Dessutom erbjuds ett högskoleingenjörsprogram i elektroteknik och två program på avancerad nivå inom signalbehandling och elkraftssystem.

Linnéuniversitetet har två forskargrupper i fysik, en grupp inom experimentell astropartikelfysik och en grupp inom den kondenserade materiens fysik. Astropartikelgruppen medverkar i flera större internationella samarbeten där gruppen bland annat bidrar med betydande instrumentutveckling och disponerar laborationslokaler med särskilda installationer. Gruppen bidrar även med betydande simulerings- och analysarbete till dessa samarbeten. Forskningsarbetet inom astropartikelfysik erbjuder många möjligheter till ingenjörsmässiga uppgifter både inom hårdvara och programmering. Speciellt finns ett samarbete med DISTA inom området dataintensiva system och det finns möjligheter för gruppen att medverka i undervisningen under termin 8. Gruppen inom kondenserade materians fysik bedriver huvudsakligen teoretiska studier av nanomagnetism, spinntronik och molekylär elektronik.

Samtliga forskare i grupperna deltar i betydande utsträckning i undervisningen, både på grundnivå och avancerad nivå. Flera har även en grundutbildning som civilingenjörer i teknisk fysik och har tidigare erfarenhet från utbildning på civilingenjörsutbildningar vid bland annat Uppsala universitet och Chalmers tekniska högskola. Experimentalisterna inom astropartikelgruppen har arbetat fram fysikkurserna på det föreslagna civilingenjörsprogrammet. Den experimentella verksamheten med sin betydande instrumentutveckling är därmed direkt kopplad till utbildningen. Forskningsverksamheten kommer att erbjuda många möjligheter till projekt inom ramen för projektkurser och självständiga arbeten. Dels kommer det att finnas uppgifter direkt kopplade till elektronik och styrning av instrument, dels olika typer av utmaningar med hantering av stora datamängder, visualisering och dataanalys.

\subsection{Utbildnings- och forskningsmiljön inom maskinteknik}

 Forskningsfokus vid institutionen är riktat mot innovationer och hållbar utveckling som tar sig uttryck i simuleringsdriven produktutveckling och produktion med fokus på produktivitet och kvalitet. Genom samverkan med industrin söks lösningar på industrinära problem. Forskning bedrivs för närvarande inom strukturdynamik, industriell ingenjörsvetenskap, materialteknik, övervakning och underhåll av maskinutrustning, tillämpad signalbehandling med fokus på mekaniska och akustiska system samt experiment genomförda på distans. Genom en nyligen genomförd rekrytering av ytterligare en professor i maskinteknik med inriktning mot industriella produktionssystem kommer forskningsfokus att tillföras områden som automation och robotteknik samt industrins digitalisering.

Forskningsverksamheten vid institutionen kommer att erbjuda många möjligheter till projekt inom ramen för projektkurser i termin 7--9 och självständiga arbeten. Industrins digitalisering och automation innehåller flera utmaningar kopplat till inbyggda system men framförallt olika typer av utmaningar med hantering av stora datamängder, visualisering och dataanalys. Detta gäller även institutionens övriga forskningsområden, exempelvis strukturdynamik.

\section{Utbildningens forskningsanknytning}

Som beskrivits ovan är den undervisande personalen på den föreslagna utbildningen till övervägande del aktiva forskare. Forskande lärare har kursansvar och ansvar för examination, samt ansvarar för att leda undervisning och handledning. Viss undervisning på grundnivå och handledning i kurser föreslås ske med resurser som inte är aktiva forskare, exempelvis adjunkter och assistenter.

På avancerad nivå, årskurs 4 och 5, har kurserna koppling till flera starka kunskapsmiljöer inom datavetenskap och matematik. Medarbetare från dessa miljöer leder undervisningen, handledningen och examinationen. På grundnivå finns det dock kurser som inte har en direkt koppling till forskning inom de identifierade starka miljöerna. Detta gäller framförallt generella kurser, exempelvis programmeringskurserna i årskurs 1.

Inom ramen för den kollegiala och individuella medarbetarens kompetensutveckling sker en kontinuerlig uppdatering kring aktuella forskningsfält som dessa kurser berörs av. Detta säkerställer en forskningsanknytning även i de fåtal grundläggande kurser som tar upp områden där institutionen inte i dagsläget bedriver aktiv forskning.

Kopplingen mellan forskning och undervisning genom aktiva forskande lärare skapar förutsättningar men ger inga garantier för en adekvat forskningsanknytning av utbildningen. Institutionen för datavetenskap och medieteknik säkerställer kopplingen genom en kontinuerlig uppföljning av kursplaner, lärandemål och kurslitteratur genom kursvärderingar. I den föreslagna utbildningen kopplas undervisningen specifikt i fördjupningsområdena till aktuell forskning men det finns även andra exempel på hur undervisningen förankras vetenskapligt.

Ett exempel är att tidigt i utbildningen skapa ett vetenskapligt synsätt med frågor, faktainsamling och analys innan beslutsfattande, vilket utgör en viktig del av utbildningens lärandemål under de första åren. I de flesta kurser utökas kurslitteraturen med någon eller några vetenskapliga publikationer under årskurs 3 samt i kurser under de två avslutande åren. Ju senare i utbildningen kursen ligger, desto mer djup och mångfald i publikationerna.

Ett exempel kan illustreras med hjälp av en grundkurs i databasteori. Studenterna läser där publikationer som ligger till grund för viktig teori, exempelvis relationsdatabaser. Liknande moment, med olika typer av examination, kommer att finnas som en del av examinationen på flertalet av kurserna senare i utbildningen.

Ett annat sätt att introducera studenterna till forskning är att aktivt låta dem delta i mindre omfattning i forskningsprojekt. Kurser som innehåller projekt eller större implementationsuppgifter kan kopplas mot behov som finns inom olika forskningsprojekt, exempelvis att implementera algoritmer eller tjänster som forskningsprojekt efterfrågar, exempelvis projekt i maskinteknik eller fysik. Kopplingen mot andra forskningsämnen ger studenterna en bredare förståelse för forskning i andra ämnen, exempelvis inom matematik eller elektroteknik eller ämnen på andra fakulteter på Linnéuniversitetet genom DISA. Ännu ett sätt att introducera forskningsresultat är att använda verktyg eller plattformar utvecklade inom forskningsprojekt som en del i examinationsuppgifter, exempelvis genom att studenterna lägger till funktionalitet eller samlar in eller visualiserar data.

Flera av kurserna, förutom projektkurserna, kommer att innehålla uppgifter som kräver att studenterna på egen hand söker information och litteratur för att kunna förstå, analysera och lösa problemen samt utvärdera lösningar. Redan den inledande projektkursen i årskurs 1 introducerar systematiska metoder för kunskapsinhämtning och problemlösning. Detta utgör en grund som stegvis byggs på i efterföljande kurser där fler, mer konkreta, aspekter av forskning, exempelvis källkritik, introduceras och diskuteras. Detta fördjupar studenternas förståelse för ämnet, även om själva problemen studenterna arbetar med, framför allt på grundnivå, inte alltid har en konkret stark koppling till aktuell forskning.

Som tidigare diskuterats kommer den föreslagna utbildningen att innehålla två kurser i vetenskaplig metod samt två självständiga uppsatsarbeten, ett i årskurs 3 och ett i årskurs 5. Den första av de två kurserna i vetenskapsmetodik introducerar bland annat vetenskapsfilosofi, grundläggande vetenskaplig metodik och etiska frågeställningar från ett generellt perspektiv. Som en del av examinationen bedöms studentens planeringsdokument med frågeställningar och metod för det självständiga arbetet, i slutet av årskurs 3. Studenterna ska sedan genomföra denna plan och dokumentera sina resultat i en rapport.

Den andra kursen i vetenskapsmetodik fokuserar på metoder som är specifika för datavetenskap framför allt med fokus på metoder som är relevanta för forskning inom utbildningens tre fördjupningsområden och de kurser som studenterna läst termin 7--9. Kursen är en fördjupning av den första kursen och vissa delar ges som en seminariekurs som anpassas efter studenternas avslutande självständiga arbete. Samtliga grupper kommer att handledas av en eller flera aktiva forskare som även förväntas vara handledare för respektive students avslutande arbete. Syftet med kursen i vetenskapsmetodik är att ge studenten tillräcklig kännedom kring och förståelse för ett specifikt forskningsproblem, men också att genom seminarier ge studenterna en förståelse för den bredd och variation som finns, både avseende problemställningar och motsvarande tillämpliga metoder inom datavetenskap.

\section{Samverkan med det omgivande samhället}

Samverkan med det omgivande samhället kommer att hanteras på tre nivåer: undervisningsaktiviteter, kvalitetsarbete och kompetenssamverkan.

Undervisningsaktiviteter där det omgivande samhället deltar eller där en samhällelig nytta kan påvisas utgör hörnstenar i all utbildning, men är kanske än viktigare i yrkesexamina som i en civilingenjörsexamen. Vår erfarenhet visar att ett aktivt deltagande från det omgivande samhället ger inspiration till studenter. Det skapar bland annat en bättre förståelse för yrkesrollen efter examen och de utmaningar de kommer att ställas inför.

Aktiviteter som främjar samverkan med det omgivande samhället utgör viktiga återkommande moment i den föreslagna utbildningen. Ett exempel är återkommande gästföreläsningar på flera olika kurser de första åren men även under de avslutande fördjupningsåren. Föreläsningarna kan handla om generella frågeställningar, att beskriva yrkesrollen och speciellt hur det är att arbeta som ingenjör inom mjukvaruindustrin. En annan inriktning är föreläsningar som tar upp och diskuterar konkreta problemställningar, teori och tillämpning i industrin kopplat till kunskaps- och färdighetsmål för en kurs. I utbildningen kommer studenterna även att aktivt arbeta med problemställningar hämtade från det omgivande samhället direkt eller genom olika forskningsmiljöer. Våra erfarenheter av detta arbetssätt visar att det ger relevanta problemställningar och ger kurser en trovärdighet. Det är eftersträvansvärt att skapa situationer i flera kurser där studenter tränas till att fungera som leverantörer till kunder, en roll som företag och andra organisationer agerar. Detta yrkesinriktade arbetssätt tränar studenternas professionalitet i kontakterna med kunder och det ger även erfarenheter som är direkt tillämpbara i studenternas framtida yrkesroll. Företag kan även med fördel delta i seminarier för att ta del av och ge återkoppling på muntliga presentationer, diskussioner och skriftliga rapporter. Under årskurs 3 och 5 erbjuds externa aktörer möjligheter att presentera problemställningar för studenterna som kan utgöra grunden för självständiga arbeten. Företag kan fungera som värd och handleda självständiga arbeten tillsammans med den akademiska handledaren. På så sätt skapas förutsättningar för ett arbete som dels uppfyller kraven på ett självständigt arbete och företagets önskemål. Framförallt kommer det att ge studenterna insikter kring hur de kan arbeta med komplexa problemställningar i industrin och det omgivande samhället och hur de kan tillämpa de teoretiska såväl som de praktiska kunskaper och förmågor de tillägnat sig under utbildningen.

Sammantaget stärker samverkan kring utbildningsaktiviteter utbildningens relevans och ger studenterna erfarenheter och förstärker egenskaper som är viktiga i deras framtida yrkesroller. Flera möjligheter att träffa yrkesverksamma ingenjörer ger studenterna ytterligare möjligheter att bättre förstå deras kommande yrkesroll.

En annan viktig aktivitet som sker i samverkan med det omgivande samhället är utvärderingar av utbildningsprogram vilket beskrivs i avsnitt~\ref{ch:perspektiv}.

Kompetenssamverkan IT är ett nätverk inom aktuellt område för den sökta utbildningen som består av arbetsgivare från både privat och offentlig sektor, utbildningsanordnare (gymnasium, yrkeshögskolor, universitet samt arbetsmarknadsutbildningar), Arbetsförmedlingen och bemanningsföretag. Syftet är att kontinuerligt informera om och analysera nuvarande och framtida kompetensförsörjningsbehov både inom universitetet utifrån utbildningens behov men framför allt det omgivande samhällets behov. Målen för Kompetenssamverkan IT:s arbete är dels att den digitala sektorn får god tillgång till den arbetskraft som efterfrågas, dels att regionen Kalmar/Kronoberg har en effektiv utbildningsverksamhet som motsvarar arbetslivets kompetensbehov. Återkommande aktiviteter som genomförs inom ramen för Kompetenssamverkan IT och som påverkat och även fortsättningsvis kommer att påverka den föreslagna utbildningen är:

\begin{itemize}
\item
  \textbf{Stora IT-kompetensdagen} är en kombinerad karriärdag och ``exjobbsmässa'' där studenter förbereds för steget från utbildning till arbetsliv. Studenter får möjlighet att träffa företag och organisationer för självständiga arbeten, praktik eller annan studierelaterad samverkan.
\item
  \textbf{Stärkt samverkan mellan utbildningsanordnare} för att skapa ett utbildningslandskap med kompletterande utbildningar snarare än konkurrerande. Samarbeten mellan utbildningar på olika nivåer för att exempelvis underlätta för övergången mellan gymnasiet och universitetet.
\item
  \textbf{Utveckling av utbildningar.} Företag ger feedback på utbildningar, hjälper till med inspel för nya utbildningar och diskuterar framtida kompetensbehov. Istället för att skapa separata programråd kan denna gruppering nyttjas för ett flertal utbildningar.
\item
  \textbf{Ökat intresse för IT bland unga.} Grupperingen arbetar med olika initiativ för att öka intresset för IT bland barn och ungdomar.
\end{itemize}