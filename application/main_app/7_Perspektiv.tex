\chapter{Integration av perspektiv i utbildningen\label{ch:perspektiv}}

\begin{tcbdoublebox}
\emph{Avsnittet täcker de perspektiv som anges i den föreslagna ansökningsmallen. Här anges hur utbildningen tar hänsyn till arbetslivets och studenternas perspektiv samt hur jämställdhet integreras i utbildningen och studiemiljön. Utöver de tre perspektiv diskuteras även internationaliserings- och hållbarhetsperspektiv på utbildningen.}
\end{tcbdoublebox}

\section{Arbetslivets perspektiv}

Man kan lägga flera olika perspektiv på en utbildnings användbarhet och hur väl den förbereder studenter för arbetslivet. Civilingenjörsexamen är en yrkesexamen och de generella examensmålen utgör en viktig del för att hantera bredden i en civilingenjörsexamen. För att civilingenjörsstudenter ska kunna fungera i dagens moderna arbetsliv behöver de även, utöver rent teknisk och ingenjörsmässig kompetens, även andra kunskaper och förmågor. Det föreslagna programmet använder CDIO-konceptet för att säkerställa att studenter som tar examen från programmet är ``Ready to engineer''. Under framtagandet av detta koncept lades stor vikt på att skapa en utbildning som motsvarar industrins behov av ingenjörer. Tanken var att utbildningar som följde konceptet skulle vara särskilt attraktiva att rekrytera från\footfullcite{crawley2014rethinking}.

En annan aspekt är det ämnesmässiga djup som utbildningen kan ge. ACM CS2013 används som utgångspunkt och ramar för det generella ämnesinnehållet i utbildningen. I styrgruppen för den senaste versionen fanns företag såsom Microsoft och ABB representerade och bland de personer som granskade och återkopplade kring förslaget fanns exempelvis Intel, NVIDIA, Google och IBM representerade. Examensmålen tillsammans med de mål som hämtas ur CDIO och ACM CS2013 utgör ett samlat teoretiskt perspektiv för den föreslagna utbildningens bredd och djup.

Koncept och rekommendationer är en utgångspunkt, men en avgörande faktor för relevansen av en utbildning är dess förankring lokalt, regionalt och nationellt. Den föreslagna civilingenjörsutbildningen har arbetats fram i samverkan med det omgivande samhället, framför allt genom samverkan med lokal och regional industri. Institutionen för datavetenskap och medieteknik har under en längre tid varit aktiva i ett antal akademi-företagsnätverk såsom Linnaeus Technical Center, Information Engineering Center, Föreningen Tunga Fordon samt den nationella branschföreningen Swedsoft som riktar sig till mjukvaruföretag. Under utbildningens framtagande hölls en serie av arbetsmöten med representanter från näringslivet. Det föreslagna civilingenjörsprogrammet har arbetats fram på GitHub~\footnote{\url{https://github.com/morganericsson/msengcs}}. Inför varje arbetsmöte skickades en hänvisning ut till GitHub tillsammans med en konkret frågeställning via några av dessa nätverk. Några exempel på frågeställningar var ``Hur ser ni på programmets kurser i matematik och fysik?'' och ``Ger grundkurserna i datavetenskap en tillräcklig grund?''. Arbetsmötena resulterade i några förändringar av upplägget, exempelvis valdes den version av programmet som innehöll mer matematik på inrådan av flera representanter från näringslivet. Under arbetsmötena diskuterades även hur näringslivet kunde vara delaktiga i utbildningen under dess genomförande, bland annat via gästföreläsningar eller som externa kunder i projektuppgifter, för att öka kopplingen mot arbetslivet. Flera av företagen och organisationerna som stödjer ansökan (se appendix~\ref{app:stodbrev}) anger på vilket sätt de kan vara delaktiga.

Flera av de organ som granskar styrdokumenten, exempelvis programkommittén för utbildningar inom data- och informationsteknik och fakultetsstyrelsen vid Fakulteten för teknik har externa ledamöter från näringslivet. De har möjlighet att granska vilken påverkan revisioner av exempelvis kursplaner kommer att ha på utbildningens användbarhet. Under det föreslagna programmets uppstart kommer ett programråd att inrättas där representanter från näringslivet, särskilt de som medverkade på arbetsmötena bjuds in. Rådet kommer att fungera som en rådgivande grupp till programansvarig och säkerställa bland annat användbarhet och koppling till arbetslivet. När utbildningen mognat kommer programrådets funktion att tas över av programkommittén. Linnéuniversitetet samverkar med Karlstad universitet och Mittuniversitetet kring ett system för regelbunden granskning av utbildningar, under namnet Treklövern\footnote{\url{https://medarbetare.lnu.se/medarbetare/styrning-och-regelverk/kvalitetsarbete/kvalitetsutvarderingar/kvalitetsutvarderingar/}}. Utvärderingarna ska ta fasta på resultat, förutsättningar och processer, liksom sådant utbildningsnära kvalitetssäkrings- och kvalitetsutvecklingsarbete, som anges i Standard and Guidelines for Quality Assurance in the European Higher Education Area (ESG)\footnote{\url{http://www.enqa.eu/wp-content/uploads/2015/11/ESG_2015.pdf}} samt Högskolelagen och Högskoleförordningen. Självvärderingar som görs inom Treklövern ska enligt ESG Standard 1.9 innehålla en värdering av utbildningens användbarhet för studenter och samhället. Det sker dessutom årliga utvärderingar av utbildningsprogrammet Linnébarometern, där studenter som går sista terminen på sitt program har möjlighet att lämna synpunkter på sin utbildning. En del av denna utvärdering fokuserar på hur användbar studenterna upplever sin utbildning, samt hur stort inslag arbetslivet har inom utbildningen, till exempel med avseende på gästföreläsningar och kontaktskapande.

Slutligen används alumner och alumnnätverk för att mäta utbildningens användbarhet. Linnéuniversitetet har ett alumnnätverk som omfattar alumner från Högskolan i Växjö, Växjö universitet, Högskolan i Kalmar och Linnéuniversitet. Utöver detta nätverk finns en rad specialiserade grupper på till exempel Facebook och LinkedIn, samt resurser kopplade till studerandeföreningarna, exempelvis KodKollektivet\footnote{\url{https://kodkollektivet.se/en/}}. 

\section{Studenters perspektiv}

\subsection{Studentinflytande i Linnéuniversitetets beslutsprocesser}

Studentkåren Linnéstudenterna representerar samtliga studenter vid Linnéuniversitetet och arbetar för att deras studietid ska vara så givande som möjligt. Linnéstudenterna representerar studenterna i kontakterna med universitetsledningen, fakulteter och institutioner och gentemot samhället. Förutom studiesociala verksamheter såsom bostadsförmedling och olika evenemang lägger Linnéstudenterna fokus på studieövervakning och kvalitetsarbete. De har flera anställda studieövervakare och studentombudsmän som erbjuder studenter råd och stöd när de anser att de behandlats orättvist av lärosätet.

Som nämnts i bland annat avsnitt~\ref{ch:styrdokument} har de studerande rätt att vara representerade i alla beslutande och beredande organ vid universitetet. Studenterna är representerade i universitetsstyrelsen, fakultetsstyrelsen, utbildningsråden, kursplaneutskottet, anställningsnämnd, programkommittén samt på dekans beslutsmöte på fakultetsnivån. Detta är en viktig komponent i studentinflytande och ett sätt för studenter att vara delaktiga i och påverka beslut som berör deras studier och studiesociala situation. Studeranderepresentanterna utses av medlemmarna i Linnéstudenterna.

Studenterna har varit representerade i framtagandet av det föreslagna civilingenjörsprogrammet. En student fanns representerad i gruppen som arbetat med bland annat blockschema och kursplaner. De arbetsmöten som hölls för näringslivet var också öppna för studenter. Under programmets uppstart kommer minst en studeranderepresentant att finnas representerad i programrådet. Utöver detta finns som tidigare nämnts studeranderepresentanter i programkommittén.

\subsection{Kurs- och programvärderingar}

En viktig del i arbetet med studentinflytandet är kurs- och programvärderingar. Programmet utvärderas av studenterna genom den årliga Linnébarometern. Resultat presenteras per program i det fall statistiskt signifikant underlag erhållits. Annars redovisas resultatet efter Fakulteten för teknik fyra olika utbildningsområden. Resultatet delges programansvariga och fakultetens utbildningsråd diskuterar den övergripande bilden av resultatet och inhämtar kommentarer samt åtgärdsplaner för program där behov eller brister identifierats. Linnébarometern ger en övergripande bild av studenternas syn på programmet som helhet och kan användas för att fokusera de programvärderingar som initieras på fakultet och institution.

Kursvärderingar regleras i högskoleförordningen och vidare internt i Linnéuniversitetets lokala regler för kurser\footnote{\url{https://lnu.se/contentassets/f292dcdf15c94c2ab0ef8245cd4a69ba/lokala-regler-for-kurs-pa-grundniva-och-avancerad-niva_galler-fr-150301.pdf}}. Enligt rektorsbeslut ska kursvärderingar genomföras via ett digitalt enkätsystem. Inom fakulteteten för teknik finns en fastställd handläggningsrutin: kursvärderingsenkäten öppnas upp för registrerade studenter i slutet av kursen, studenterna ges därefter två veckor på sig att svara (påminnelser skickas automatiskt) och därefter stängs enkäten. Enkätsvaren sammanställs och skickas till ansvarig lärare för analys och återkoppling. Fakulteteten för teknik har utarbetat ett processtöd för medarbetarna\footnote{\url{https://medarbetare.lnu.se/medarbetare/organisation/ftk/verksamhetsstod-ftk/ikt-stod/kursvardering-och-enkater/}}.

Förutom de förutbestämda enkätfrågorna utgår majoriteten av fakultetens kursvärderingar från en gemensam enkätmall. Frågorna behandlar kursplanens övergripande tydlighet avseende mål och innehåll i förhållande till genomförandet, betygskriterier, lärandesituation, studentinflytande, kursmomentens relevans samt studentens syn på kursens relevans i sin helhet. Studenterna kan även lämna mer utförliga synpunkter och ge direkta förslag på förbättringar i så kallade fritextfält. Enligt universitetets principer för det systematiska kvalitetsarbetet ska den ansvarige lärarens analys utgöra grunden för förnyelse och förbättringsförslag som återkopplas till studenterna.

På kurser på det föreslagna civilingenjörsprogrammet kommer ett antal kursrepresentanter att utses per kurs. Dessa kommer att fungera som ett komplement till det digitala enkätsystemet och har till uppgift att stämma av kursen med övriga studenter och sätta kursvärderingen i ett sammanhang, exempelvis genom att ge mer information om vissa svar. Efter avslutad kurs träffas programansvarig, ansvariga lärare och studeranderepresentanter för att diskutera utfallet av enkäten och tillsammans sammanfatta utfallet i en utvärderingsrapport och formulera en åtgärdsplan om sådan behövs. Om en åtgärdsplan formuleras skall programansvarig följa upp denna med ansvariga lärare, före kursen ges nästa gång och vid efterföljande utvärderingsmöte. Utvärderingsrapporter och åtgärdsrapporter från föregående år skall publiceras på kursens kursrum i Moodle och rapporter från tidigare år skall finnas arkiverade och tillgängliga för studenter som efterfrågar dessa.

Det systematiska kvalitetsarbetet kommer att vara intensivt under utbildningens uppbyggnad och de första åren för att säkerställa att utbildningen motsvarar de ställda kvalitetsmålen. Särskilda programvärderingar kommer att ske årligen dels baserat på kursvärderingar, utfall på kurser och andra uppgifter som samlas in. Särskilda programutvärderingar som distribueras till studenterna kommer också att finns med. Målsättningen är att snabbt etablera en kultur där ett systematiskt kvalitetsarbete ingår och som säkerställer studenters inflytande och att utbildningen möter de uppställda kvalitetsmålen.

\section{Jämställdhetsperspektiv}

Linnéuniversitetet tog under våren 2017 fram en plan för jämställdhetsintegrering vid lärosätet\footnote{\url{https://lnu.se/globalassets/ny-katalog/plan-for-jamstalldhetsintegrering1.pdf}}. Planen är en följd av det nationella uppdrag som beslutats av regeringen där lärosäten aktivt ska arbeta med att minska obalansen mellan män och kvinnor på alla nivåer i verksamheten. Centrala processer har initierats för att fullfölja planens tio förbättringsområden. De innehåller bland annat personalpolitik, rekrytering, kunskapslyft och jämställda styrdokument. Fakulteterna och institutionerna håller för närvarande på att analysera effekterna av planen på olika styrdokument, processer och i verksamheten. Universitetet har även en plan för lika rättigheter och möjligheter, som riktar sig mot att uppfylla de krav som diskrimineringslagen ställer. Denna plan innehåller exempelvis åtgärder för att minska risken för diskriminering och trakasserier. Universitetet har även en pågående kampanj mot sexuella trakasserier där studenter och medarbetare kan lämna in berättelser om upplevda sexuella trakasserier anonymt. Dessa berättelser kommer ligga till grund för framtida proaktiva åtgärder för att minska risker för sexuella trakasserier på universitetet.

Den föreslagna utbildningens miljö har idag en kraftig snedfördelning om man ser till kön, både avseende undervisande personal och studenter. Tekniska utbildningar har historiskt haft svårt att hitta en balans i representation mellan könen, så detta är på intet sätt ett problem som är specifikt för Linnéuniversitetet, utan samma tendenser återfinns på flera av de lärosäten som erbjuder ingenjörsutbildningar inom data- och informationsteknik.

Fakulteten för teknik och Institutionen för datavetenskap och medieteknik arbetar aktivt på två nivåer för att förändra situationen. Den första nivån handlar om rekrytering av medarbetare och studenter, den andra handlar om utbildningens organisation och genomförande. Kompetensförsörjningsplanen (se appendix~\ref{app:kompetensplan}) identifierar den sneda könsrepresentationen som ett problem och detta återspeglas i rekryteringsprocessen och anställningsprofiler. När det gäller rekrytering av studenter så är inspiration och intresse två grundpelare. Inspiration handlar om att synliggöra kvinnor och den kompetens som finns i branschen och intresse syftar till att öka intresset för branschen generellt hos flickor i skolan.

Avsaknaden av förebilder anses vara en viktig orsak till snedrekryteringen av studenter. Institutionen för datavetenskap och medieteknik har som en del i arbetet med att rekrytera och behålla en större andel kvinnliga studenter varit med och grundat WiTech\footnote{\url{www.witech.nu}}, ett nätverk för kvinnor inom IT, där studenter, forskare och yrkesverksamma från privat och offentlig sektor möts. Nätverket grundades i november 2017. Sedan dess har fler än 140 kvinnor från mer än 30 organisationer anslutit sig.

WiTech gör det möjligt för kvinnliga studenter att redan från början av sin utbildning komma i kontakt med kvinnor som är verksamma inom mjukvaruutveckling eller angränsade områden och tidigt bygga ett kontaktnätverk. Kontaktnätverket kan sedan användas för att exempelvis hitta handledare för självständiga arbeten eller mentorer. Nätverket skapar även möjligheter att nå ut till framtida studenter genom till exempel aktiviteter på skolor för att synliggöra kvinnor som arbetar inom data- och informationsteknik och göra flickor och unga kvinnor medvetna om möjligheter till en karriär inom området.

Utöver nätverket WiTech så driver och deltar institutionen i flera andra aktiviteter som riktas mot olika grupper för att väcka intresse för tekniska utbildningar, i synnerhet data- och informationsteknik.

Jämställdhetsarbetet inom det föreslagna civilingenjörsprogrammet har främst fokuserat på att arbeta fram lärsituationer som passar en större grupp studenter. Programmet innehåller till exempel flera projekt istället för traditionell katederundervisning, även om det senare fortfarande används. Tanken är att dessa ska skapa ett lärande som handlar mer om individens personliga utveckling i situationer som i så stor grad som möjligt liknar de som individen kommer att befinna sig i sitt yrkesliv; det finns en direkt koppling mellan kursens innehåll och dess framtida nytta.

Institutionen för datavetenskap och medieteknik kommer att sträva efter en jämn könsfördelning inom det föreslagna civilingenjörsprogrammet vad det gäller exempelvis externa representanter i programkommittén, gästföreläsare och externa kunder i projekt. Nätverket WiTech kommer att spela en viktig roll i att säkerställa detta. Minst en representant till programrådet kommer till exempel att hämtas från det.
 Minst en representant till programrådet kommer till exempel att hämtas från det. 

Inom ramen för jämställdhetsintegrering och lika villkor tas specifika moment fram som syftar till att utbilda medarbetare i hur man aktivt kan arbeta för mer jämställda kurser avseende innehåll, material och kommunikation. Denna process är viktig men har precis påbörjats och måste ses i ett längre tidsperspektiv. Den förväntas dock ge direkta avtryck i uppbyggnaden och genomförandet av de nya kurser som arbetats fram inom ramen för arbetet med det föreslagna civilingenjörsprogrammet. Ett konkret exempel på detta arbete är att kursplanerna för varje föreslagen kurs innehåller ett avsnitt under rubriken övrigt som betonar att kursen ska vara anpassad för både män och kvinnor. Denna text finns där för att påminna både lärare och studenter om jämställdhetsperspektivet och kommer konkret att medföra att kursansvarig eller examinator måste kunna resonera kring på vilket sätt kursen är anpassad i hänseende till detta. Anpassningen kommer att följas upp i kursvärderingen och i de fall det finns brister måste ansvarig lärare åtgärda dessa inför nästa kurstillfälle.

Jämställdhetsintegreringen kommer utgöra ett nyckelområde i det intensifierade systematiska kvalitetsarbetet som planerats under utbildningens uppbyggnad och genomförande. Nätverket WiTech kommer att fungera som en viktig part i utbildningens uppbyggnad men även vid utformning av annonser och annan marknadsföring.

\section{Internationaliseringsperspektiv}

Yrkeslivet är idag internationellt och framtidens ingenjörer kommer att arbeta i internationella miljöer, med projekt i flera länder. Behovet av förståelse för olika kulturer ökar och därför är internationalisering av utbildningen viktigt för att förbereda studenterna för deras yrkesroller. Linnéuniversitetet ser internationalisering som en strategisk del i all verksamhet. Historiskt har universitetet haft en stor andel utresande och inresande utbytesstudenter. Inresande studenter i kombination med internationell rekrytering av undervisande personal och internationella gästlärare skapar en miljö med internationalisering på hemmaplan. Exempelvis kommer samtliga kurser i årskurs 4 och 5 på det föreslagna civilingenjörer att ges på engelska. Detta skapar möjligheter att erbjuda kurserna för internationella studenter och på så sätt skapa en internationell miljö där studenter genom samarbeten lär känna andra kulturer, men lika viktigt är att de får erfarenheter från att samarbeta och skriva rapporter samt presentera dem på engelska.

En annan del av internationaliseringen av en utbildning är att skapa förutsättningar för studenter att studera utomlands inom ramen för utbildningen. Linnéniversitetet och Fakulteten för teknik rekommenderar att varje utbildning ger studenterna möjlighet att tillbringa en eller två terminer vid ett utländskt lärosäte. Det kommer att vara möjligt att studera vid en utländsk teknisk högskola eller universitet under termin 7--9. Ett antal strategiska samarbetsavtal kommer att etableras för att kvalitetsäkra att studenter som väljer att studera en eller två terminer vid ett utländskt lärosäte fortfarande uppfyller examensmålen.

\section{Hållbarhetsperspektiv}

En ingenjör kommer att i sin yrkesroll ställas inför utmaningar kopplat till olika aspekter av hållbar utveckling. Hållbarhet och hållbarutveckling utgör därför ett viktigt perspektiv och har integrerats i det föreslagna civilingenjörsprogrammet.

Linnéuniversitetets policy för hållbar utveckling slår fast att dagens och framtidens samhällsutmaningar efterfrågar nya perspektiv och nya sätt att tänka. Som kunskapsorganisation möter universitetet dessa utmaningar genom att erbjuda utbildning av nutida och kommande generationer och genom forskning generera kunskap för en förbättrad samhällsutveckling.

Med utgångspunkt i universitets vision ``Linnéuniversitetet --- en kreativ och internationell kunskapsmiljö som odlar nyfikenhet, nytänkande, nytta och närhet'' samt i FN:s hållbarhetsmål deltar universitetet i den samtida globala debatten för framtidsutveckling.

I Linnéuniversitetets vision och strategi 2015–2020, en resa in i framtiden\footnote{\url{https://lnu.se/globalassets/dokument---gemensamma/universitetsledningens-kansli/en_resa_in_i_framtiden_2015-2020.pdf}}, fastslås att alla studenter och medarbetare är bärare av tankar om hållbar utveckling. Genom att systematiskt integrera hållbar utveckling i all verksamhet kan universitetet direkt och indirekt bidra till en hållbar samhällsutveckling. Utmanande utbildningar och framstående forskning är tillsammans med samhällelig drivkraft och globala värden hörnstenar i en kreativ kunskapsmiljö. I ett hållbarhets perspektiv betyder det:

\begin{itemize}
\item
  Utmanande utbildningar --- Studenter vid Linnéuniversitetet skall ha en god bildning vad det gäller samhällsutmaningar samt den egna utbildningens betydelse i att möta dessa. Sådana kunskaper ger dem verktyg att anta samhällsutmaningar med ett hållbarhetsperspektiv.
\item
  Framstående forskning --- Linnéuniversitetet ska ha en framstående forskning som integrerar ekologiska, ekonomiska och sociala perspektiv för en hållbar utveckling.
\item
  Samhällelig drivkraft --- Linnéuniversitetet ska vara drivande i samverkan med det omgivande samhället för en hållbar utveckling från lokal till global nivå.
\item
  Globala värden --- I Linnéuniversitetets arbete med hållbar utveckling integreras värden inom internationalisering och lika villkor till en helhet i undervisning, forskning, samverkan och stödverksamhet.
\end{itemize}

I den föreslagna civilingenjörsutbildningen sker en integrering av framför allt kunskap men även förmågor och förhållningssätt kopplat till hållbarhet och hållbar utveckling. Detta sker dels genom kurser kopplade till deras framtida yrkesroll men även i de ämnesspecifika kurserna, projekt och de självständiga arbetena. Under årskurs 2 ges en kurs i hållbarhet och hållbar utveckling som ger en förståelse för begrepp men även färdigheter och värderingsförmåga kopplat till exempelvis resursanvändning och arbetsmiljö. I de ämnesspecifika kurserna finns ett flertal exempel kopplat till framför allt resursanvändning vid stora beräkningar men även kopplat till effektiv hantering och planering av personalresurser i projekt.