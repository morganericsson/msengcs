\chapter{Introduktion}

I mitten av 2000-talet tog utbildningsdepartementet ett nationellt
initiativ som uppmuntrade mindre lärosäten att inleda diskussioner kring
samgåenden. I sydöstra delen av Sverige fördes dessa diskussioner inom
ramen för ett strategiskt samarbete, Akademi Sydost, mellan Blekinge
Tekniska Högskola, Högskolan i Kalmar samt Växjö universitet. Samarbetet
ledde till att Högskolan i Kalmar och Växjö universitet beslutade att gå
samman och Linnéuniversitetet bildades den 1 januari 2010.

Linnéuniversitetet är idag lokaliserat till Växjö och Kalmar, men det finns även
verksamhet vid andra orter, exempelvis designutbildningen i Pukeberg och
verksamheten vid Centrum för informationslogistik (CIL) i Ljungby.
Universitetet hade 2017 ca 2~000 anställda, varav drygt 170 professorer,
och utbildar årligen drygt 32~000 studenter motsvarande ca 15~300
helårsstudenter. Baserat på antalet studenter är Linnéuniversitetet Sveriges sjätte
största lärosäte. Antalet forskarstuderande är drygt 300. Universitetet
har ca 730 inresande internationella studenter (utbytesstudenter)
och strategiska utbildnings- och forskningssamarbeten med universitet
och forskningsinstitut i fler än 60 länder. De totala intäkterna under
år 2017 uppgick till 1,83 miljarder kr.

Verksamheten är organisatoriskt uppdelad på fem fakulteter (hälsa och
livsvetenskap, konst och humaniora, samhällsvetenskap, teknik, och
ekonomihögskolan) och en lärarutbildningsnämnd. Inom varje fakultet
finns en fakultetsstyrelse med strategiskt och operativt ansvar. Dekan
för respektive fakultet beslutar bl.a. om anställningar av akademisk
personal förutom professorer samt ansvarar för arbetsmiljön vid
fakulteten. Fakulteterna är uppdelade i institutioner som var och en
leds av en prefekt med strategiskt och operativt ansvar för alla
aktiviteter inom institutionens område. Organisationen för de enskilda
fakulteterna kan variera i mindre utsträckning.

Under de första åren utarbetades visioner, mål och strategier för det
nya universitetets verksamhet. Linnéuniversitetet har som ambition att vara en kreativ
och internationell kunskapsmiljö som odlar nyfikenhet, nytänkande, nytta
och närhet. Denna miljö skapas genom en strategi som vilar på fyra
hörnstenar: utmanande utbildningar, framstående forskning, samhällelig
drivkraft och globala värden. Centralt är att utbildning och forskning
integreras för att uppnå högsta möjliga akademiska kvalitet och bidra
till utvecklingen av hållbara miljöer.

Det enskilt viktigaste projektet inom Linnéuniversitetet för närvarande är etableringen
av en civilingenjörsutbildning inom området data- och
informationsvetenskap. Genomförandet av en sådan utbildning bidrar till
uppfyllelsen av Linnéuniversitetets vision, samtidigt som utbildningen kommer att
medverka till att tillgodose ett samhälleligt behov, eftersom det från
såväl svensk industri som offentliga aktörer rapporteras att det råder
brist på ingenjörer. Detta gäller inte minst inom området data- och
informationsvetenskap, där behoven är betydande både nationellt och i
Linnéuniversitetet närområde.

Linnéuniversitetet har vid två tidigare tillfällen, 2009 och 2011,
ansökt om rättigheter att utbilda och examinera civilingenjörer. 
Våren 2014 avsatte fakultetsstyrelsen vid fakulteten för teknik strategiska medel och anlitande externa granskare för en genomlysning av förutsättningarna för
en förnyad ansökan. Baserat på rapporten från de externa granskarna genomfördes flera åtgärder: en projektgrupp för att ta fram ett utbildningsförslag skapades, en projektgrupp för att utbilda inom och införa Concieve-Design-Implement-Operate-konceptet (CDIO) på samtliga befintliga högskoleingenjörsutbildningar skapades och medel för att rekrytera de kompetenser som de externa granskarna ansåg saknas avsattes. Dessa medel används för att rekrytera t.ex. en professor och två lektorer inom inbyggda system och en lektor i beräkningsmatematik. 

Då näringslivets perspektiv är väldigt viktig har under det under utvecklingen
av programmet anordnats ett antal workshoppar med företag för att i detalj diskutera behov
och även inriktning på en utbildning. Träffarna har fokuserats enligt
den tänkta utbildningens struktur och därmed har både bredd och
specialiseringar diskuterats inom det tänkta huvudområdet mjukvaruteknik
men även matematikbehov för en mjukvaruingenjör i industrin och behov
och inriktning på projektkurser har diskuterats. Institutionens
samverkansenhet har kontinuerligt lyft och diskuterat frågan i samband
med återkommande kompetenssamverkansträffar. På så sätt har regionens
näringsliv och offentliga verksamheter fått insyn och kunnat delta i
processen och på så sätt påverka programstrukturen, kursinriktningar och
i viss mån även kursinnehåll. Detta speglas av de stödbrev som bifogas ansökan.