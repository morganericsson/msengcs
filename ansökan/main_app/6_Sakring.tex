\chapter{Säkring av examensmålen\label{ch:sakring}}

\begin{tcbdoublebox}
\emph{Avsnittet täcker aspekten Säkring av examensmålen i den föreslagna ansökningsmallen. Här anges utbildningens examensmål, samt de verktyg som används för att följa upp och säkra att utbildningen uppfyller dessa mål. Avsnittet innehåller även en diskussion om hur progression säkras genom utbildningen och hur den anpassats för att främja studenters lärande och ta hänsyn till deras förutsättningar.}
\end{tcbdoublebox}

Den föreslagna utbildningen är en sammanhållen femårig utbildning med tydlig uppdelning mellan grund- och avancerad nivå. För att förtydliga denna uppdelning anges ett etappmål (se figur~\ref{fig:etmal}) som en student ska uppfylla efter årskurs 3 utöver de mål som Högskoleförordningen anger för civilingenjörsexamen. Detta etappmål säkerställer att årskurs 1-3 har det innehåll och den progression en grundutbildning i datavetenskap ska ha samt att en tillräcklig nivå av ingenjörsmässighet uppnås. Etappmålen kontrollerar även att studenterna är redo att påbörja studier på avancerad nivå.

Examensmålen för civilingenjör i mjukvaruteknik~\ref{fig:exmal}) är formulerade så att de täcker examensmålen för civilingenjörs- och masterexamen för att utöver yrkesmässighet säkerställa det vetenskapliga djupet i utbildningen.

Etapp- och examensmålen är inspirerade av den utredning av examensmål för civilingenjörsutbildningar som Kungliga tekniska högskolan genomförde inför införandet av Bolognamodellen\footfullcite{rosen2011}.

\begin{figure}[tp]
\begin{tcbframe}
\begin{enumerate}
\def\labelenumi{\emph{\Alph{enumi}.}}
\item \emph{Kunskap och förståelse}
  \begin{enumerate}
  \def\labelenumii{\Alph{enumi}\arabic{enumii}.}
  \tightlist
  \item
    Visa brett kunnande och förståelse inom det valda teknikområdets
    (mjukvaruteknik) vetenskapliga grund och beprövade erfarenhet,
    inbegripet kunskaper i matematik och naturvetenskap, väsentligt
    fördjupade kunskaper inom vissa delar av området, samt fördjupad
    insikt i aktuellt forsknings‐ och utvecklingsarbete, samt
  \item
    visa fördjupad metodkunskap inom huvudområdet för utbildningen
    (datavetenskap).
  \end{enumerate}
\item \emph{Färdighet och förmåga}
  \begin{enumerate}
  \def\labelenumii{\Alph{enumi}\arabic{enumii}.}
  \tightlist
  \item
    Visa förmåga att med helhetssyn kritiskt, självständigt och kreativt
    identifiera, formulera och hantera komplexa frågeställningar, 
  \item
    visa förmåga att skapa, analysera och kritiskt utvärdera olika
    tekniska lösningar, 
  \item
    visa förmåga att planera och med adekvata metoder genomföra
    kvalificerade uppgifter inom givna ramar samt att utvärdera detta
    arbete, 
  \item
    visa sådan färdighet som fordras för att delta i forsknings‐ och
    utvecklingsarbete eller för att självständigt arbeta i annan
    kvalificerad verksamhet och därigenom bidra till
    kunskapsutvecklingen, 
  \item
    visa förmåga att kritiskt och systematiskt integrera kunskap och att
    analysera, bedöma och hantera komplexa företeelser, frågeställningar
    och situationer även med begränsad information samt visa förmåga att
    modellera, simulera, förutsäga och utvärdera skeenden även med
    begränsad information, 
  \item
    visa förmåga att utveckla och utforma produkter, processer och
    system med hänsyn till människors förutsättningar och behov och
    samhällets mål för ekonomiskt, socialt och ekologiskt hållbar
    utveckling, 
  \item
    visa förmåga till lagarbete och samverkan i grupper med olika
    sammansättning, samt
  \item
    visa förmåga att i såväl nationella som internationella sammanhang
    muntligt och skriftligt i dialog med olika grupper klart redogöra
    för och diskutera sina slutsatser och den kunskap och de argument
    som ligger till grund för dessa.
  \end{enumerate}
\item \emph{Värderingsförmåga och förhållningssätt}
  \begin{enumerate}
  \def\labelenumii{\Alph{enumi}\arabic{enumii}.}
  \tightlist
  \item
    Visa förmåga att göra bedömningar med hänsyn till relevanta
    vetenskapliga, samhälleliga och etiska aspekter samt visa
    medvetenhet om etiska aspekter på forsknings‐ och utvecklingsarbete, 
  \item
    visa insikt i vetenskapens och teknikens möjligheter och
    begränsningar, dess roll i samhället och människors ansvar för hur
    den används, inbegripet sociala och ekonomiska aspekter samt miljö‐
    och arbetsmiljöaspekter, samt
  \item
    visa förmåga att identifiera sitt behov av ytterligare kunskap och
    att ta ansvar för att fortlöpande utveckla sin kunskap och kompetens.
  \end{enumerate}
\end{enumerate}
\end{tcbframe}
\caption{Examensmål för civilingenjör i mjukvaruteknik\label{fig:exmal}}
\end{figure}

\begin{figure}[tp]
\begin{tcbframe}
\begin{enumerate}
\def\labelenumi{\emph{\Alph{enumi}.}}
\item \emph{Kunskap och förståelse}
  \begin{enumerate}
  \def\labelenumii{\Alph{enumi}\arabic{enumii}.}
  \tightlist
  \item
    Visa kunskap och förståelse inom huvudområdet, inbegripet kunskap om
    områdets vetenskapliga grund, kunskap om tillämpliga metoder inom
    området, fördjupning inom någon del av området samt orientering om
    aktuellt forsknings‐ och utvecklingsarbete, samt
  \item
    visa kunskap i matematik och naturvetenskap.
  \end{enumerate}
\item \emph{Färdighet och förmåga}
  \begin{enumerate}
  \def\labelenumii{\Alph{enumi}\arabic{enumii}.}
  \tightlist
  \item
    Visa förmåga att söka, samla, värdera och kritiskt tolka relevant
    information i en problemställning samt att kritiskt diskutera
    företeelser, frågeställningar och situationer,
  \item
    visa förmåga att självständigt identifiera, formulera och lösa
    problem samt att analysera och utvärdera olika tekniska lösningar,
  \item
    visa förmåga att självständigt planera, genomföra, och redovisa
    uppgifter inom givna ramar,
  \item
    visa sådan färdighet som fordras för att självständigt arbeta inom
    det område som utbildningen avser,
  \item
    visa förmåga att integrera och använda kunskap,
  \item
    visa förmåga att kritiskt diskutera företeelser, frågeställningar
    och situationer, samt modellera skeenden med utgångspunkt i relevant
    information,
  \item
    visa förmåga att beskriva och utveckla förslag till enklare
    produkter, processer och system med hänsyn till människors
    förutsättningar och behov och samhällets mål för ekonomiskt, socialt
    och ekologiskt hållbar utveckling,
  \item
    visa förmåga att i samarbete planera, genomföra och redovisa givna
    uppgifter,
  \item
    visa förmåga att på svenska, och i viss mån även på engelska,
    muntligt och skriftligt redogöra för och diskutera information,
    problem och lösningar i dialog med olika grupper.
  \end{enumerate}
\item \emph{Värderingsförmåga och förhållningssätt}
  \begin{enumerate}
  \def\labelenumii{\Alph{enumi}\arabic{enumii}.}
  \tightlist
  \item
    Visa förmåga att inom huvudområdet för utbildningen göra bedömningar
    med hänsyn till relevanta vetenskapliga, samhälleliga och etiska
    aspekter,
  \item
    visa insikt om kunskapens och teknikens roll i samhället, och om
    människors ansvar för hur de används, samt
  \item
    visa förmåga att identifiera sitt behov av ytterligare kunskap och
    att utveckla sin kompetens.
  \end{enumerate}
\end{enumerate}
\end{tcbframe}
\caption{Etappmål som en student skall uppfylla efter årskurs 3.\label{fig:etmal}}
\end{figure}

\section{Säkerställande av examensmål inom utbildningen}

Examensmålen ger en övergripande bild av vad en student ska kunna efter genomförd utbildning. Två verktyg kommer att användas för att säkerställa att den föreslagna utbildningens innehåll lever upp till dessa mål: CDIO Syllabus\footfullcite{crawley2011cdio} och ACM:s Computer Science Curricula (CS2013)\footfullcite{cs2013}.

CDIO Syllabus är ett strukturerat sätt att beskriva vilka kunskaper och färdigheter en ingenjör ska ha:

\begin{itemize}
\tightlist
\item
  Ämnes-, matematiska, tekniska- och naturvetenskapliga kunskaper,
\item
  individuella och yrkesmässiga färdigheter och förhållningssätt,
\item
  förmåga att arbeta i grupp och kommunicera,
\item
  förmåga att planera, designa, implementera, samt
\item
  driftsätta system med hänsyn till affärsmässiga och samhälleliga behov
  och krav.
\end{itemize}

Då CDIO Syllabus beskriver vad en civilingenjörsutbildning ska innehålla används denna både för att säkerställa målen för hela programmet och för att säkerställa utbildningens ingenjörsmässighet.

De ämnesmässiga kunskaperna skiljer sig åt för olika civilingenjörsutbildningar. Av denna anledning ger CDIO Syllabus inte några riktlinjer kring ämnesinnehållet. Då ACM CS2013 ger råd och riktlinjer för vad en grundutbildning i datavetenskap bör innehålla samt hur mycket resurser och tid som bör ägnas åt varje moment kommer denna att användas för att säkerställa det datavetenskapliga innehållet i den föreslagna utbildningen.

\subsection{Koppling mellan examensmål, lärandemål, lärandeaktiviteter och examination}

Processen för att säkerställa examensmålen inom utbildningen utgår från en koppling av etapp- och examensmålen till CDIO Syllabus 2.0. Denna extra koppling görs främst då målen i CDIO förtydligar ingenjörsrollen och ämneskunskaperna och således ger ett mer konkret verktyg att arbeta med. Det är enklare att göra en koppling mot CDIO på kursnivå och via sambandet mellan CDIO och examensmålen kan kopplingarna på kursnivå översättas och kopplas till etapp- och examensmålen.

Målen i CDIO Syllabus räknas upp under punkt 1--4 i figur~\ref{fig:cdiosyll}. Då CDIO Syllabus främst fokuserar på ingenjörsrollen har denna utökats för att även täcka de examensmål som rör vetenskap och forskning. De extra mål som lagts till är de som anges under punkt 5 i figur~\ref{fig:cdiosyll}, ``Planering, genomförande och presentation av forskningsprojekt med hänsyn till vetenskapliga och samhälleliga behov och krav''. Denna utökning är inspirerad av Linköpings tekniska högskolas version av CDIO Syllabus från 2007.

\begin{figure}[tp]
\begin{tcbframe}
\begin{enumerate}
\def\labelenumi{\arabic{enumi}.}
\item \emph{Ämneskunskaper.}
\tightlist
  \begin{enumerate}
  \def\labelenumii{\arabic{enumi}.\arabic{enumii}.}
  \tightlist
   \item
    Kunskaper i grundläggande matematiska och naturvetenskapliga ämnen.
  \item
    Kunskaper i grundläggande teknikvetenskapliga ämnen.
  \item
    Fördjupade kunskaper, metoder och verktyg inom något/några
    teknikvetenskapliga ämnen.
  \end{enumerate}
\item
  \emph{Individuella och yrkesmässiga färdigheter och förhållningssätt.}

  \begin{enumerate}
  \def\labelenumii{\arabic{enumi}.\arabic{enumii}.}
  \tightlist
   \item
    Analytiskt tänkande och problemlösning.
  \item 
    Experimenterande och undersökande arbetssätt samt kunskapsbildning.  
  \item
    Systemtänkande.
  \item
    Förhållningssätt, tänkande och lärande.
  \item
    Etik, likabehandling och ansvarstagande.
  \end{enumerate}
\item
  \emph{Förmåga att arbeta i grupp och att kommunicera.}

  \begin{enumerate}
  \def\labelenumii{\arabic{enumi}.\arabic{enumii}.}
  \tightlist
   \item
    Arbete i grupp.
  \item
    Kommunikation.
  \item
    Kommunikation på främmande språk.
  \end{enumerate}
\item
  \emph{Planering, utveckling, realisering och drift av tekniska produkter och
  system med hänsyn till affärsmässiga och samhälleliga behov och krav -- innovationsprocessen.}

  \begin{enumerate}
   \def\labelenumii{\arabic{enumi}.\arabic{enumii}.}
  \tightlist
  \item
    Samhälleliga och miljömässiga villkor.
  \item
    Företags- och affärsmässiga villkor.
  \item
    Att identifiera behov samt strukturera och planera utveckling av
    produkter och system.
  \item
    Att konstruera produkter och system.
  \item
    Att realisera produkter och system.
  \item
    Att ta i drift och använda produkter och system.
  \end{enumerate}
\item
  \emph{Planering, genomförande och presentation av forskningsprojekt med
  hänsyn till vetenskapliga och samhälleliga behov och krav.}
  \begin{enumerate}
  \def\labelenumii{\arabic{enumi}.\arabic{enumii}.}
  \tightlist
   \item
    Samhälleliga villkor, inklusive ekonomiskt, socialt och ekologiskt
    hållbar utveckling.
  \item
    Ekonomiska villkor för forskning och utveckling.
  \item
    Att planera forsknings- och utvecklingsprojekt.
  \item
    Att genomföra forsknings- och utvecklingsprojekt.
  \item
    Att rapportera och redovisa forsknings- och utvecklingsprojekt.
  \item
    Fördjupad insikt i aktuellt forsknings- och utvecklingsarbete.
  \end{enumerate}
\end{enumerate}
\end{tcbframe}
\caption{CDIO Syllabus 2.0 utökat med mål som rör forskning och vetenskap.\label{fig:cdiosyll}}
\end{figure}

Appendix~\ref{app:maltillcdio}  visar kopplingen mellan examensmålen för programmet och den utökade CDIO Syllabus. Motsvarande koppling finns även för det sammanhållna etappmålet. Varje examensmål är i kopplingen en konjunktion av CDIO-mål, där samtliga måste vara uppfyllda på programnivå för att examensmålet ska vara uppfyllt. Målen kan uppfyllas av flera olika kurser.

\begin{table}[t]
\caption{IUA-matris (Introducera, Undervisa, Använda) för årskurs 1 på programmet. Observera att målen under punkt 5 i den utökade CDIO Syllabus utelämnats då kurserna i årskurs 1 inte berör dessa.\label{tab:iua-ar1}}
\resizebox{\columnwidth}{!}{%
\begin{tabular}{lccccccccccccccccc}
\toprule
 \textsf{\textbf{Kurs}}    & \textsf{\textbf{1.1}} & \textsf{\textbf{1.2}} & \textsf{\textbf{1.3}} & \textsf{\textbf{2.1}} & \textsf{\textbf{2.2}} & \textsf{\textbf{2.3}} & \textsf{\textbf{2.4}} & \textsf{\textbf{2.5}} & \textsf{\textbf{3.1}} & \textsf{\textbf{3.2}} & \textsf{\textbf{3.3}} & \textsf{\textbf{4.1}} & \textsf{\textbf{4.2}} & \textsf{\textbf{4.3}} & \textsf{\textbf{4.4}} & \textsf{\textbf{4.5}} & \textsf{\textbf{4.6}} \tabularnewline
\midrule
Diskret matematik                        & \texttt{\texttt{U}}   &     &     & \texttt{UA}  & \texttt{A}   &     & \texttt{UA}  &     &     & \texttt{A}   &     &     &     &     &     &     &     \tabularnewline
Programmering och datastrukturer         & \texttt{U}   & \texttt{IU}  &     & \texttt{IU}  & \texttt{I}   &     & \texttt{I}   &     & \texttt{U}   & \texttt{IU}  & \texttt{A}   &     &     & \texttt{I}   &     & \texttt{I}   &     \tabularnewline
Linjär algebra                           & \texttt{U}   &     & \texttt{U}   & \texttt{UA}  & \texttt{A}   &     & \texttt{UA}  &     &     & \texttt{A}   &     &     &     &     &     &     &     \tabularnewline
Introducerande projekt                   & \texttt{A}   & \texttt{IUA} &     & \texttt{U}   &     & \texttt{U}   & \texttt{UA}  & \texttt{U}   & \texttt{UA}  & \texttt{UA}  &     & \texttt{U}   & \texttt{U}   & \texttt{IU}  & \texttt{I}   & \texttt{U}   & \texttt{I}   \tabularnewline
Envariabelanalys 1                       & \texttt{UA}  &     & \texttt{A}   & \texttt{UA}  & \texttt{A}   &     & \texttt{UA}  &     &     & \texttt{A}   &     &     &     &     &     &     &     \tabularnewline
Databaser och datamodellering            & \texttt{A}   & \texttt{IUA} &     & \texttt{U}   & \texttt{UA}  & \texttt{U}   & \texttt{U}   & \texttt{I}   & \texttt{A}   & \texttt{UA}  & \texttt{A}   & \texttt{I}   & \texttt{I}   & \texttt{UA}  & \texttt{I}   & \texttt{U}   &     \tabularnewline
Objekt-orienterad programmering          & \texttt{A}   & \texttt{UA}  &     & \texttt{U}   &     & \texttt{U}   & \texttt{U}   &     & \texttt{A}   & \texttt{UA}  &     &     &     & \texttt{U}   & \texttt{U}   & \texttt{IU}  &     \tabularnewline
Tillämpad sannolikhetslära och statistik & \texttt{UA}  &     & \texttt{A}   & \texttt{UA}  & \texttt{UA}  &     & \texttt{UA}  &     &     & \texttt{A}   &     &     &     &     &     &     &     \tabularnewline
Mekanik                                  & \texttt{UA}  &     &     & \texttt{UA}  & \texttt{UA}  &     & \texttt{UA}  &     &     &     &     &     &     &     &     &     &     \tabularnewline
\bottomrule
\end{tabular}}
\end{table}

För varje kurs finns utöver kursplanen en så kallad IUAE-matris (Introducera, Undervisa, Använda, Examinera), där kursens lärandemål kopplas till målen i CDIO Syllabus. Mål som \emph{undervisas} kopplas till examination, mål som \emph{används} kopplas till tidigare kurser och så vidare. Från matriserna på kursnivå kan en IUA-matris (Introducera, Undervisa, Använda) för programmet upprättas, som via kopplingen mellan CDIO Syllabus och examensmålen ger en bild över hur väl programmet uppfyller sina mål. Tabell~\ref{tab:iua-ar1} visar IUA-matrisen för årskurs 1 på programmet. Den fullständiga IUA-matrisen för programmet redovisas i appendixå~\ref{app:iuaprogram}.

Tabell~\ref{tab:exkoppling}visar kopplingen mellan examensmålen och kurserna i årskurs 1, för att illustrera hur kopplingen ser ut. En fullständig koppling mellan examensmål och kurser inom programmet bifogas som appendix~\ref{app:kursmal}. Beakta till exempel examensmål B1 (``Visa förmåga att skapa, analysera och kritiskt utvärdera olika tekniska lösningar''), som kopplas mot mål 2.1--2.4 i CDIO enligt tabell~\ref{tab:maltillcdio} i appendix~\ref{app:maltillcdio}. Av kurserna i årskurs 1 är det endast kursen \emph{Databaser och datamodellering} som berör samtliga examensmål, medan övriga kurser bidrar till måluppfyllelsen till olika grad. Observera att endast kurser som undervisar (och såldes examinerar) ett CDIO-mål kan kopplas.

Det är sällan så att en kurs helt ska uppfylla ett examensmål, däremot är det viktigt att ha ett antal kurser som tillsammans bidrar till att uppfylla examensmålen. Kopplingarna mellan examensmål och IUA-matriser på programnivå visar att de tre första åren väl uppfyller etappmålen och att hela programmet väl uppfyller examensmålen för civilingenjörs- och mastersexamen. Den fullständiga IUA-matrisen som bifogas kan tillsammans med tabell~\ref{tab:maltillcdio} användas för att skapa en bättre förståelse av exakt vilka kurser som uppfyller vilka mål och hur.

IUAE-matriser på kursnivå och IUA-matrisen på programnivå har använts för att utveckla programmet och kommer att användas löpande för att följa upp och utvärdera både kurser och programmet i sin helhet. Sammankopplingen ger en struktur för att utvärdera hur väl de olika examinationsmomenten på en kurs uppfyller mål på både kurs- och programnivå och kan på så sätt används som underlag till frågor på kurs- och programutvärderingar.


\begin{table}
\caption{Koppling mellan examensmål och kurser för årskurs 1. \faCircleO\ indikerar att en kurs delvis uppfyller ett mål och \faCircle\ indikerar att den helt uppfyller målet.\label{tab:exkoppling}}
\resizebox{\columnwidth}{!}{%
\begin{tabular}{lccccccccccccc}
\toprule
\textbf{\textsf{Kurs}}                   & \textbf{\textsf{A.1}}          & \textbf{\textsf{A.2}}         & \textbf{\textsf{B.1}}                          & \textbf{\textsf{B.2}} & \textbf{\textsf{B.3}} & \textbf{\textsf{B.4}} & \textbf{\textsf{B.5}} & \textbf{\textsf{B.6}} & \textbf{\textsf{B.7}} & \textbf{\textsf{B.8}} & \textbf{\textsf{C.1}} & \textbf{\textsf{C.2}} & \textbf{\textsf{C.3}} \tabularnewline
\midrule
Diskret matematik & \faCircleO & \faCircleO & \faCircleO & \faCircleO & \faCircleO & \faCircleO & \faCircleO &&&&&& \faCircleO \tabularnewline 
Programmering och datastrukturer & \faCircleO & \faCircleO & \faCircleO & \faCircleO & \faCircleO & \faCircleO & \faCircleO && \faCircle & \faCircleO && \faCircleO & \tabularnewline            
Linjär algebra & \faCircleO & \faCircleO & \faCircleO & \faCircleO & \faCircleO & \faCircleO & \faCircleO &&&&&& \faCircleO \tabularnewline 
Introducerande projekt & \faCircleO & \faCircleO & \faCircleO & \faCircleO & \faCircleO & \faCircleO & \faCircleO & \faCircleO & \faCircle & \faCircleO & \faCircleO & \faCircleO & \faCircleO \tabularnewline 
Envariabelanalys 1 & \faCircleO & \faCircleO & \faCircleO & \faCircleO & \faCircleO & \faCircleO & \faCircleO &&&&&& \faCircleO \tabularnewline 
Databaser och datamodellering & \faCircleO & \faCircleO & \faCircle  & \faCircleO & \faCircleO & \faCircleO & \faCircleO & \faCircleO && \faCircleO &            & \faCircleO & \faCircleO \tabularnewline 
Objekt-orienterade programmering & \faCircleO & \faCircleO & \faCircleO & \faCircleO & \faCircleO & \faCircleO & \faCircleO & \faCircleO && \faCircleO && \faCircleO & \faCircleO \tabularnewline 
Tillämpad sannolikhetslära och statistik & \faCircleO & \faCircleO & \faCircleO & \faCircleO & \faCircleO & \faCircleO & \faCircleO &&&&&& \faCircleO \tabularnewline 
Mekanik & \faCircleO & \faCircleO & \faCircleO & \faCircleO & \faCircleO & \faCircleO & \faCircleO &&&&&& \faCircleO \tabularnewline 
\bottomrule
\end{tabular}%
}
\end{table}

\subsection{Säkerställande av utbildningens datavetenskapliga innehåll}

 ACM:s rekommendationer i CS2013 används för att säkerställa det datavetenskapliga innehållet i den föreslagna utbildningen. ACM CS2013 består av 18 kunskapsområden (Knowledge Areas) som i sin tur delas upp i kunskapsenheter (Knowledge Units). Ett kunskapsområde kan och bör spänna över flera kurser. Varje kunskapsenhet anger innehåll genom en lista av ämnen som ska beröras samt ett antal lärandemål och till vilken nivå studenten ska uppfylla dessa. Då nivåerna anges som Familiarity, Usage och Assessment har dessa kopplats till de tre nivåer som anges i exempelvis examensmålen: Kunskap och förståelse, färdighet och förmåga samt värderingsförmåga och förhållningssätt.

Varje ämne inom ett kunskapsområde klassificeras som Tier-1, Tier-2 eller valbart. Tier-1-kunskaper är något alla ska ha medan vissa Tier-2-kunskaper kan utlämnas beroende på utbildningens övergripande inriktning. För varje ämne anges hur många undervisningstimmar som bör läggas under en utbildning, fördelat över Tier-1, Tier-2 och valbart. ACM rekommenderar att en grundutbildning i datavetenskap bör ha 100~\% av Tier-1 och minst 80~\% av Tier-2, men helst 90–-100~\%.

När kurserna i datavetenskap och diskret matematik inom den föreslagna utbildningen togs fram gjordes först en koppling mot kunskapsområden och kunskapsenheterna för varje kurs. Sedan fördelades ämnen och lärandemål över kurser tillsammans med vilken nivå det berörs på. Till sist gjordes en uppskattning av i vilken utsträckning varje kurs täcker ett ämne som sedan omvandlades till timmar för att kunna jämföras med ACM:s riktvärden. Kopplingen mellan varje kurs och ACM CS2013 anges i ett dokument som är kopplat till kursplanen. Ett utdrag ur detta dokument redovisas i appendix~\ref{app:acm-exempel}. Dessa kopplingar uppdateras när kurser ändras och kan användas som verktyg för att till exempel följa upp innehåll och examination i kursen eller som underlag till kurs- och programutvärderingar.

Det föreslagna programmet i sin helhet täcker 100~\% av Tier-1 och cirka 95~\% av Tier-2 och uppfyller således ACM:s riktlinjer. Det finns ett stort överlapp mellan målen i CDIO Syllabus och de i kunskapsområdet \emph{Social Issues and Professional Practice} exempelvis avseende kommunikation och professionell etik. Kunskaper som berörs av detta område spänner ofta över flera kurser, så för att förenkla kan man se att Tier-1 och Tier-2 uppfylls av att CDIO Syllabus uppfylls.

De delar av Tier-2 som inte täcks rör främst kunskapsområdet \emph{Intelligent Systems}. Det föreslagna programmet innehåller inte någon generell kurs i artificiell intelligens, utan fokuserar istället på kunskapsenheten maskininlärning. Detta medför att några algoritmer såsom vissa typer av sökalgoritmer inte tas upp i den föreslagna utbildningen.

Den föreslagna utbildningen täcker även över 40~\% av de valfria kunskapsenheter som beskrivs i ACM CS2013.


\section{Utbildningens progression}

Progression i ämneskunskaper följer av att kurser bygger på varandra och använder kunskaper och färdigheter från tidigare kurser. Kursen \emph{Databaser och datamodellering} kräver programmeringsfärdigheter och ger progression, dels genom tillämpning av dessa färdigheter men även fördjupning i hur studenten använder de tillägg som krävs för att koppla upp sig mot databasen. På liknande sätt tillämpar kursen funktions- och relationsbegreppen från matematik och fördjupar förståelsen för dessa genom att relatera dem till datamodellering och de problem som detta medför. IUAE-matriserna på kursnivå har använts för att detaljerat beskriva vilka kunskaper och färdigheter som används inom varje kurs. Dessa har sedan via undervisning och examination (i matriserna) kopplats till kurser. För att förtydliga detta anges samtliga av dessa kurser som förkunskaper, oavsett eventuella förkunskapskrav. Exempelvis anges både \emph{Diskret matematik} och \emph{Linjär algebra} om en kurs använder färdigheter från båda, trots att \emph{Diskret matematik} är ett förkunskapskrav för \emph{Linjär algebra}.

Vad det gäller övriga färdigheter såsom kommunikation eller arbete i grupp, så fördelas dessa efter vad en kurs bäst kan bidra med. I ovan nämnda kurs i \emph{Databaser och datamodellering} är datamodellering en viktig del, vilken kräver att studenterna gör antaganden och tar beslut som behöver motiveras och förklaras. Detta moment kan examineras genom diskussion, vilket övar studenterna i muntlig kommunikation och att förklara krav, antaganden och resonemang. Dessa färdigheter kan sedan fångas upp i efterföljande kurser, till exempel i \emph{Mjukvaruutvecklingsprojekt}, där studenter behöver resonera kring krav och designbeslut gentemot en kund.

I vissa fall finns det inte någon kurs där kombinationer av övriga färdigheter passar in, exempelvis kommunikation och arbete i grupp. I dessa fall har särskilda kurser skapats. Dessa kurser är främst projektkurserna i det föreslagna programmet, där hela CDIO-konceptet belyses. Dessa projektkurser bidrar till en progression av färdigheter genom att varje kurs antingen belyser aspekter av färdighetskombinationerna eller genom att kraven på uppgiften som ska lösas ökar. Vad det gäller färdigheter i arbete i grupp så introduceras detta först i ett mindre projekt i den inledande kursen \emph{Programmering och datastrukturer}, där studenterna lär sig att samarbeta kring utveckling av en mjukvara. Fokus i denna kurs läggs på hur uppgifter kan fördelas, hur mjukvaruartefakter såsom källkod kan delas och så vidare. I den följande projektkursen \emph{Introducerande projekt} är problemet som ska lösas mer öppet och studenterna måste själva definiera krav. Storleken på grupperna ökar till cirka fyra personer per grupp och mer fokus läggs på hur studenterna samarbetar i en grupp, vilka skyldigheter en individ har, hur de kan hantera konflikter och så vidare. I kursen \emph{Mjukvaruutvecklingsprojekt} i årskurs 2 kommer gruppstorleken istället vara cirka sex personer per grupp och undervisningen fokuserar på hur mjukvaruvecklingsprojekt leds och utförs. Industriella (agila) processmodeller införs och studenterna kommer att arbeta i enlighet med dessa. Problemställningen begränsas, men en kund som studenterna måste interagera med införs, vilket kan försvåra hantering av krav under projekttiden, vilket studenterna då får träning i.

De två avslutande åren fokuserar vidare på progression inom CDIO-cykeln i verklighetsliknande scenarion. Termin 7-9 innehåller projekt där studenterna under realistiska förhållande avseende både problem och verktyg genomgår hela cykeln. Kurserna fokuserar på att ge teoretiska grunder för delar av cykeln; de två i terminens första läsperiod fokuserar främst på Conceive och Design, det vill säga grundläggande teori inom området som belyses i projekten. För projektet \emph{Modelldriven utveckling} i termin 7 innefattar detta hur mjukvarusystem kan modelleras och simuleras innan de implementeras, samt hur modeller representeras av språk och hur dessa kan översättas till programkod. Kurserna i terminens andra läsperiod fokuserar främst på teori och metoder för Implement och Operate, för att fördjupa studenternas förståelse. I projektet som diskuteras ovan handlar dessa om hur modeller kan verifieras och egenskaper hos dem bevisas samt hur de kan optimeras.

De tre projekten i termin 7-9 förutsätter att studenterna har tillgodogjort sig grunderna i att arbeta i ett mjukvaruprojekt. Detta ger progression i projektarbetet och ämneskunskaperna genom att introducera nya domäner, problemställningar och verktyg. Projektet i termin 7 fokuserar på att fördjupa studenternas förståelse för olika aktiviteter inom agila metoder såsom sätt att fånga krav och ger dem större möjlighet att pröva och skräddarsy dem jämfört med projektet i kursen \emph{Mjukvaruutvecklingsprojekt} som främst fokuserar på att lära ut dessa. Projektet i termin 8 fokuserar på hur man gör ett agilt projekt effektivt genom idéer från lean agile. Upplägget på projektet simulerar en startup med begränsade resurser, som studenterna ska göra det mesta av. I det sista projektet i termin 9 testas studenternas förmåga att självständigt genomföra ett agilt projekt och de ska där på egen hand driva projektet.

Under utbildningens utveckling användes erfarenheter kring i vilken ordning kurser/ kunskapsenheter normalt presenteras som riktlinjer när kurserna placerades ut i ett blockschema. Sedan skapades IUAE-matriser på kursnivå där mål och behov tydligt specificerades (exempelvis ``Studenten ska kunna utföra en kortast-väg-beräkning på en graf''), både kring ämneskunskaper och övriga färdigheter. Utifrån dessa matriser skapades en partiell ordning av kurser, med cykler (cirkulära beroenden mellan kurser), som fångar förkunskaper oberoende av blockschemat. Utifrån denna ordning genomfördes en revision av kurser och mål, genom förtydligande, införande eller flyttande, tills alla cykler var borttagna. Ett exempel på mål som infördes under denna process är kunskapsenheten kring filsystem (\emph{OS/File systems} i ACM CS2013), som krävdes i kursen \emph{Inbyggda system}, men inte undervisades av någon kurs i programmet. Efter diskussion ansågs kursen \emph{Databaser och datamodellering} mest lämplig att beröra denna kunskapsenhet, då den redan tar upp hur tabeller lagras i minne och på hårddiskar. Särskilda ansträngningar gjordes för de ingenjörsmässiga färdigheterna där det för varje kurs diskuterades vilka färdigheter som bör användas.

Efter att samtliga kurser kunde placeras in i ordningen, det vill säga att samtliga kursmål uppfylldes samt att alla cykler var borttagna skapades det blockschema för utbildningen som visas av figur~\ref{fig:bs13} på sidan~\pageref{fig:bs13} och figur~\ref{fig:bs45} på sidan~\pageref{fig:bs45}.

\section{Hänsyn och främjande av studenternas lärande}

CDIO-Standards beskriver 12 olika särdrag som karakteriserar en CDIO-utbildning. Dessa standarder beskriver inte i första hand mål som studenten bör uppnå, något som finns i CDIO Syllabus, eller utbildningens huvudsakliga innehåll utan CDIO-Standards beskriver istället utbildningens grundfilosofi i termer av helhet, integration, undervisningssätt, kunskapssyn, kvalitetssäkring etcetera. De 12 standarderna är:

\begin{enumerate}
\tightlist
\item Utbildningens syfte.
\item Lärandemål för ingenjörsfärdigheter och ämneskunskaper.
\item Integrerad utbildning.
\item Introduktion till ingenjörsarbete.
\item Utvecklingsprojekt.
\item Lärmiljöer för praktiskt lärande.
\item Integrerat lärande.
\item Aktivt lärande.
\item Utveckling av lärarkollegiets ingenjörskompetens.
\item Utveckling av lärarkollegiets pedagogiska kompetens.
\item Bedömning och examination.
\item Programutvärdering.
\end{enumerate}

Fem av CDIO:s Standards avser kursutveckling eller studenternas förutsättningar och lärande. Den första av dessa, Standard 4, rör det inledande ingenjörsprojektet. Det föreslagna programmet innehåller ett sådant projekt, kursen \emph{Introducerande projekt}, i årskurs 1. I detta projekt kommer studenterna att introduceras till ingenjörsrollen och med vad och hur en mjukvaruingenjör arbetar. Projektet ska skapa intresse för ämnet och rollen, samt visa på den mångfald som finns i mjukvaruingenjörsrollen och på så sätt motivera de olika kurser som ingår i programmet. Dessa kunskaper och färdigheter utvecklas vidare under kommande projekt.

Standard 6 berör laboration och arbetsytor. En del av laborationsmiljön för den föreslagna utbildningen är virtuell och således alltid tillgänglig för studenterna, oavsett var de befinner sig. Den kommer också att vara flexibel och studenterna kommer att uppmuntras att experimentera med olika sätt att lösa problem på, oavsett om det är en kurs i matematik där Matlab används eller en projektkurs senare i programmet. Då laborationsmiljön är virtuell kan studenterna komma åt den så länge de har en dator med nätverksuppkoppling. Detta gör att studenterna kan arbeta från olika platser, till exempel de många grupprum som erbjuds via biblioteket.

Standard 7 rör integrerat lärande, där bland annat ämneskunskap, yrkesmässiga färdigheter och systembyggande blandas inom kurser. Detta förekommer i samtliga projektkurser, men många av de renodlade ämneskurserna kommer också att innehålla denna typ av moment för att ge studenterna olika perspektiv på ämnesmässig kunskap och på så sätt uppmuntra och underlätta lärande. Ett exempel på detta är kursen \emph{Diskret matematik} som kommer att innehålla frivilliga programmeringsuppgifter för att hjälpa studenterna som har lättare för programmering än matematik. Skillnaderna och likheterna kan sedan belysas under föreläsningarna för att stödja studentens progression och lärande.

Ett annat sätt att främja lärande är Standard 8, aktivt lärande, enligt vilken lärsituationen förflyttas från ett passivt överförande av kunskap till en där studenterna aktiveras och engageras genom problemlösning. Även detta är något som återfinns i projektkurserna, där projektbaserat lärande kommer att tillämpas. Aktivt lärande kommer även att vara del av traditionella ämneskurser. I kurser som behandlar teknik och algoritmer kan så kallat ``flipped classroom'' användas, där läraren förbereder läsanvisningar och material, exempelvis inspelade föreläsningar och sedan ägnas lektionen till diskussion och gemensam problemlösning. Denna teknik används idag vid en rad kurser i datavetenskap och kommer även att tillämpas i den föreslagna utbildningen. Den virtuella laboratoriemiljön gör det enkelt för en lärare att skapa experiment under en lektion och genomföra dessa tillsammans med studenterna och diskutera olika strategier och utfall. Ett sådant upplägg används idag på en kurs i nätverksteknik, där ett mindre nätverk skapas och sedan kan applikationer startas och fel simuleras, baserat på frågeställningar och förslag från studenter.

Examinationstypen för varje kurs har valts så att den så väl som möjligt motsvarar lärandemålen. Vikten av olika examinationsformer diskuteras i Standard 11, bedömning och examination. Processen med att välja examinationsform började med en kartläggning av fördelar och nackdelar hos olika former och vad de bäst mäter. Sedan diskuterades syftet med de olika kurserna och målen grupperades och kopplades mot examinationsformer som bäst kontrollerar måluppfyllelse på ett sådant sätt att kursens syfte uppnås. Då många kurser har liknande syfte återkommer vissa examinationsformer; kurser som lär ut någon teknisk del av datavetenskap, såsom programmering, innehåller en examination där studenterna utvecklar datorprogram. På liknande sätt examineras ofta teoretiskt innehåll genom skriftlig tentamen. I fall där det är viktigt att som student kunna resonera kring och motivera används muntlig tentamen, exempelvis för kurserna \emph{Databaser och datamodellering} och \emph{Algoritmer}. Totalt används cirka tio olika examinationsformer inom programmet som dessutom anpassas efter kurs och mål. Däremot kan en presentation och exakt hur den bedöms variera stort beroende på om det är en mjukvara från ett projektarbete eller en vetenskaplig artikel som presenteras. På liknande sätt finns en progression i examinations formen, så tidiga rapporter kan vara enklare och bedöms inte enligt lika många kriterier. Examinationsformerna för varje kurs beskrivs i de bifogande kursplanerna.