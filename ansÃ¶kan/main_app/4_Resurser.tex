\chapter{Studiemiljö\label{ch:resurser}}

\begin{tcbdoublebox}
\emph{Avsnittet täcker aspekten Resurser i den föreslagna ansökningsmallen. Här anges den infrastruktur som finns för att stödja studenterna under deras studier, såsom bibliotek och laborationssalar. Avsnittet täcker även andra funktioner som till exempel det studiestöd biblioteket och studenthälsan erbjuder.}
\end{tcbdoublebox}

\section{Lokaler för undervisning}

Utbildningen kommer att förläggas till Linnéuniversitetets campus i Växjö. Där finns moderna undervisningssalar av varierande storlek som vanligtvis är utrustade med whiteboardtavlor och AV-utrustning (projektor, dator och ljudsystem). Vissa lokaler har även utrustning för inspelning och livesändning, så att en föreläsning eller ett föredrag genom en knapptryckning kan sändas direkt via exempelvis YouTube eller Zoom. Trådlöst nätverk via Eduroam finns tillgängligt för studenter över hela campusområdet.

Universitetet arbetar aktivt och systematiskt, enligt diskrimineringsombudsmannens modell, för att universitetets lokaler ska vara tillgängliga för alla och att ingen diskriminering ska förekomma. Det erbjuds centralt stöd för tillgänglighetsinsatser som riktar sig till både studenter och undervisande personal. Undervisningslokalerna är anpassade för att vara tillgängliga för alla och flera lokaler har förutom den tekniska utrustning som beskrivs ovan även hörslingor.

\section{Universitetsbiblioteket}

Universitetsbiblioteket tillhandahåller litteratur och textdatabaser. I lokalerna finns även studieplatser och grupprum samt ett antal rum anpassade för distansutbildning. Biblioteket har öppet 8-22 måndag till torsdag, 8-18 på fredagar samt 8-17 på helger.

Biblioteket ansvarar för den universitetsgemensamma lärplattformen Moodle samt IT-system för exempelvis kursvärderingar. Inom ramen för bibliotekets verksamhet erbjuds utbildning i informationssökning, informationshantering, akademiskt skrivande och textgranskning.
 
\subsection{Litteratur och databaser}

Universitetsbibliotekets samling uppgår till cirka 320 000 tryckta böcker och cirka 160 000 e-böcker. Årligen köps cirka 10 000 nya tryckta böcker in. Biblioteket prenumererar på fler än 100 olika databaser\footnote{\url{https://lnu.se/ub/soka-och-vardera/artiklar--databaser/}}(genom avtal med Bibsamkonsortiet) inom olika ämnesområden. Tillgången på litteratur och prenumerationer på databaser bedöms i nuläget vara fullgott och täcka det behov av böcker och vetenskapliga publikationer som den föreslagna utbildningen har. Behovet analyseras kontinuerligt och eventuella behov kommer att tillgodoses inom ramen för bibliotekets ordinarie processer.

\subsection{Studieplatser}

Det finns 840 arbets- och studieplatser på universitetsbiblioteket. Dessa är fördelade över tre zoner (ljudlösa, lugna och livfulla) som styr vilka aktiviteter som är lämpliga vid platserna\footnote{\url{https://lnu.se/globalassets/dokument---gemensamma/bibliotek/pa-biblioteket/aktivitetszoner-ubvaxjo.pdf}}. Dessutom finns en pauszon med pentryn och ett café.

Det finns 50 grupprum på biblioteket, varav 15 är bokningsbara. Varje grupprum har plats för 4-8 studenter och flera av dem har skrivtavlor och bildskärm eller projektor. Det finns även soffgrupper med avskärmningar som kan användas för grupparbeten. Utöver grupprummen på biblioteket finns även cirka 30 bokningsbara grupprum i andra lokaler på campus i Växjö. Det finns utöver studieplatserna och grupprummen cirka 80 datorarbetsplatser på biblioteket som samtliga studenter har tillgång till.

\subsection{Studiestöd}

Universitetsbiblioteket erbjuder flera mindre moment i exempelvis informationssökning, akademiskt skrivande och studieteknik, som kan ingå i ordinarie kurser. Personal från biblioteket och kursansvariga för en dialog kring kursens innehåll och studenternas behov. På så sätt anpassas kursmomenten så att de blir ändamålsenliga. Vid särskilda behov kan biblioteket producera anpassade digitala läromedel för dessa moment. Universitetsbiblioteket erbjuder även öppna föreläsningar som studenter kan gå på samt digitala läromedel via websidorna kring dessa ämnen.

Studieverkstaden\footnote{\url{https://lnu.se/ub/skriva-och-referera/studieverkstaden/}} vid universitetsbiblioteket erbjuder handledning i akademiskt skrivande på både svenska och engelska. Studenter kan kontakta verkstaden när som helst i skrivprocessen för att boka handledning. De kan bland annat hjälpa till med hur en text skall struktureras och hur referenshantering går till. Utöver handledning i akademiskt skrivande erbjuder verkstaden även vägledning och stöd i studieteknik, inom till exempel studieplanering och hur studenterna kan effektivisera sitt antecknande från föreläsningar.

\section{Dator- och laborationssalar}

Institutionen för datavetenskap och medieteknik har under de senaste åren noterat att användningen av allmänna arbetsplatser i datorlaborationssalar minskat drastiskt. Detta beror främst på att de flesta studenterna har tillgång till en egen bärbar dator och att de föredrar att kunna arbeta i sin egen miljö, när och var de vill. En stor del av den mjukvara som behövs för studierna, såsom utvecklingsmiljöer, är antingen fritt tillgängligt eller kan installeras via campuslicenser.

Den minskade användningen av datorlaborationssalarna har medfört att institutionen:

\begin{itemize}
\item Tillhandahåller studieplatser med utrustning såsom bildskärmar, tangentbord och pekdon som studenterna kan använda tillsammans med sina egna datorer för att skapa en bra studiemiljö.
\item Skapat en virtuell laborationsmiljö, en molninfrastruktur, som kan användas när laborationsbehoven överstiger vad studenter kan göra på sina egna datorer, exempelvis i fall då särskild programvara eller särskilda tjänster krävs för att genomföra studieuppgifter.
\item Skapat specialiserade laborationssalar med utrustning för specifika kurser och/eller moment, såsom robotar och 3D-skrivare.
\end{itemize}

I de fall en student inte har tillgång till egen dator finns 30 kompletta datorarbetsplatser att tillgå. Satsningen på den virtuella laborationsmiljön gör det också möjligt att i stor utsträckning använda allmänna datorarbetsplatser för laborationsmoment exempelvis de cirka 80 datorarbetsplatser som finns på biblioteket.

Arbetet med att erbjuda de bästa förutsättningarna avseende lokaler och resurser kopplade till den föreslagna civilingenjörsutbildningen kommer att fortgå kontinuerligt under de närmsta åren och under utbildningens uppbyggnadsår.

\subsection{Molninfrastruktur}

Flera kurser, främst inom datavetenskap, har laborationsmoment som kan vara svåra att genomföra på en enskild persondator eller arbetsstation. Det finns situationer där en speciell miljö, som kan vara svår att installera krävs, eller då flera datorer krävs för att genomföra laborationer. Institutionen för datavetenskap och medieteknik har under flera år arbetat med att bygga upp en miljö där studenterna kan arbeta med virtuella maskiner och nätverk för att utföra laborationer och under 2017 investerade Institutionen för datavetenskap och medieteknik 3 MSEK på utrustning för detta. Den nuvarande molninfrastrukturen kan hantera hundratals samtidiga användare och har visst stöd för särskilda typer av laborationer och projekt, bland annat grafikprocessorer och acceleratorer. Systemet kommer att byggas ut och uppdateras efter behov.

Molninfrastrukturen ger lärare och studenter möjligheter att arbeta med virtuella maskiner och nätverk som kan användas på flera olika sätt. En lärare kan skapa en miljö med specifik mjukvara för att genomföra en laboration och dela denna med studenterna som sedan startar sina egna kopior och genomför laborationerna på detta sätt. Alternativt kan studenterna få mer fria händer och skapa sina egna miljöer med exempelvis databashanterare och webbservrar.

Molninfrastrukturen bygger på öppna standarder och ett flertal av studenterna kommer med stor sannolikhet att stöta på motsvarande lösningar i arbetslivet, exempelvis Amazon Web Services eller Google Cloud Platform. Så, förutom den flexibilitet infrastrukturen erbjuder inom utbildningen, får studenterna även med sig värdefulla erfarenheter och färdigheter kring exempelvis hur man driftsätter och underhåller mjukvarutjänster.

Molninfrastrukturen används även av forskare som studieobjekt och för att genomföra experiment. Detta bidrar till en tydlig koppling mellan undervisning och forskning, studenterna kan på detta sätt använda i princip samma miljöer och verktyg som används i forskningen. I kurser som behandlar olika aspekter av datorsystem såsom \emph{Datorns uppbyggnad} eller \emph{Datornät} kan infrastrukturen studeras för att exemplifiera koncept, exempelvis virtualisering och ``Software-defined networking''. Studenterna kommer också tidigt i studieförloppet i kontakt med den molninfrastruktur som de kan dra nytta av i sina självständiga arbeten.

\subsection{Övriga resurser för laborativa moment}

Vissa laborationsmoment kräver särskilda, ofta fysiska resurser såsom robotar. Det finns ett antal mindre specialiserade laborationssalar vid Linnéuniversitetet med exempelvis automationsutrustning med avancerade robotar och AGV:er (Automated Guided Vehicle), 3D-skrivare, utrustning för virtual reality, stora skärmar för visualiseringar samt verktyg för ögonföljning.

Kurserna \emph{Mekanik}, \emph{Ellära och magnetism} och \emph{Reglerteknik} ställer särskilda krav på laborationsutrustning. Det finns laborationssalar med olika uppställningar med datorstödd mätutrustning som i första hand används för laborationer i mekanik och laseroptik. Det finns även utrustning för laborationer i atom- och kärnfysik med spektrometrar för olika energiområden och olika datorstödda mätningar, samt en neutronkälla för experiment med neutronaktivering. Den senare laborationsutrustningen kan exempelvis användas i projektkurserna under årskurs 4 och 5.

En sal är specialdesignad för laborationer i elektronik och har uppställningar med olika standardutrustningar såsom oscilloskop (4-kanaler, 1 GSa/s), signalgenerator, multimeter, symmetriska spänningsaggregat, samt dator med LabView och Matlab. Det finns också en sal för laborationer i elkraft med Terco SD1500 Scan Drive system. Institutionen för fysik och elektroteknik har även utrustning för experiment med ``Software-defined radio''.

\subsection{Laborationsresurser i vår närhet}

EPIC-laboratoriet (Entrepreneurship Production Innovation Communication) är ett samverkansprojekt mellan Linnéuniversitetet, Växjö kommun och näringslivet i regionen för att stärka industrinära utbildning och forskning. EPIC invigs under 2018 och har utrustning för bland annat automation och produktionssystem, tillverkningsprocesser, strukturdynamik (inkluderande ljud och vibrationer) och icke-förstörande provning, mät- och styrsystem, IT-relaterad utrustning som sensorer, IT-infrastruktur och styrning.

EPIC kommer långsiktigt att vara en viktig resurs för den föreslagna utbildningen. Även om vissa av ovan beskrivna resurser inte direkt kopplas till utbildningens grundläggande behov, kommer studenter att komma i kontakt med dessa i samband med både fördjupningsprojekt och självständiga arbeten.


Nätverket Videum VR\footnote{\url{https://www.videum.se/videum-vr}}  har en demonstrationsmiljö och en laborationsyta med avancerad utrustning för virtual reality, som förutom datorer och headsets även har ett ``rullande golv'' som gör det möjligt att fritt röra sig i den virtuella rymden.

\section{Studenthälsan}

Studenthälsan är en viktig resurs för att hjälpa studenterna under studietiden med att till exempel hantera stress. De erbjuder kontinuerligt ett antal föreläsningar och workshops i stresshantering, mindfulness och prokrastination, samt lunchmeditation en gång i veckan.

De håller även ett flertal kurser såsom \emph{Första hjälpen till psykisk hälsa} och \emph{Att våga tala}. Den första kursen utbildar frivilliga studenter i att känna igen tecken på psykisk ohälsa hos sina studiekamrater (och sig själva), samt hur de kan bemöta och hjälpa varandra. Den senare är en kurs som hjälper studenter möta rädslan och obehaget kring att tala offentligt. För studenter som inte är redo att delta i en kurs tillsammans med andra erbjuds även individuella stödsamtal.