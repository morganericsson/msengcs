\chapter{Introduktion}

Utbildningsdepartementet tog i mitten av 2000-talet ett nationellt initiativ som uppmuntrade mindre lärosäten att inleda diskussioner kring samgåenden. I sydöstra delen av Sverige fördes dessa diskussioner inom ramen för ett strategiskt samarbete, Akademi Sydost, mellan Blekinge Tekniska Högskola, Högskolan i Kalmar samt Växjö universitet. Samarbetet ledde till att Högskolan i Kalmar och Växjö universitet beslutade att gå samman och Linnéuniversitetet bildades den 1 januari 2010.

Linnéuniversitetet är idag lokaliserat till Växjö och Kalmar, men det finns även verksamhet vid andra orter, exempelvis designutbildningen i Pukeberg och verksamheten vid Centrum för informationslogistik (CIL) i Ljungby. Universitetet hade 2017 cirka 2 000 anställda, varav drygt 170 professorer. Varje år utbildas drygt 32 000 studenter motsvarande cirka 15 300 helårsstudenter. Baserat på antalet studenter är Linnéuniversitetet Sveriges sjätte största lärosäte. Antalet forskarstuderande är drygt 300. Universitetet har cirka 730 inresande internationella studenter (utbytesstudenter) och strategiska utbildnings- och forskningssamarbeten med universitet och forskningsinstitut i fler än 60 länder. De totala intäkterna under år 2017 uppgick till 1,83 miljarder kronor.

Linnéuniversitetet är uppdelat i fem fakulteter (hälsa och livsvetenskap, konst och humaniora, samhällsvetenskap, teknik och ekonomihögskolan) och en lärarutbildningsnämnd. Inom varje fakultet finns en fakultetsstyrelse med strategiskt och operativt ansvar. Dekan för respektive fakultet beslutar bland annat om anställningar av akademisk personal förutom professorer samt ansvarar för arbetsmiljön vid fakulteten. Fakulteterna är uppdelade i institutioner som var och en leds av en prefekt med strategiskt och operativt ansvar för alla aktiviteter inom institutionens område. Organisationen för de enskilda fakulteterna kan variera i mindre utsträckning.

Under Linnéuniversitetets första år utarbetades visioner, mål och strategier för det dess verksamhet. Universitetet har som ambition att vara en kreativ och internationell kunskapsmiljö som odlar nyfikenhet, nytänkande, nytta och närhet. Denna miljö skapas genom en strategi som vilar på fyra hörnstenar: utmanande utbildningar, framstående forskning, samhällelig drivkraft och globala värden. Centralt är att utbildning och forskning integreras för att uppnå hög akademisk kvalitet och bidra till utvecklingen av hållbara miljöer.

Det enskilt viktigaste projektet inom Linnéuniversitetet för närvarande är etableringen av en civilingenjörsutbildning inom området data- och informationsteknik. Genomförandet av en sådan utbildning bidrar till uppfyllelsen av Linnéuniversitetets vision, samtidigt som utbildningen kommer att medverka till att tillgodose ett samhälleligt behov, eftersom det från såväl svensk industri som offentliga aktörer rapporteras att det råder brist på ingenjörer. Detta gäller inte minst inom området data- och informationsvetenskap, där behoven är betydande både nationellt och i universitetets närområde.

Linnéuniversitetet har vid två tidigare tillfällen, 2009 och 2011, ansökt om rättigheter att utbilda och examinera civilingenjörer. Våren 2014 avsatte fakultetsstyrelsen vid Fakulteten för teknik strategiska medel och anlitande externa granskare för en genomlysning av förutsättningarna för en förnyad ansökan. Baserat på rapporten från de externa granskarna genomfördes flera åtgärder: en projektgrupp för att ta fram ett utbildningsförslag skapades, en projektgrupp för att utbilda inom och införa Conceive-Design-Implement-Operate-konceptet (CDIO) på samtliga befintliga högskoleingenjörsutbildningar inrättades och medel för att rekrytera de kompetenser som de externa granskarna ansåg saknas avsattes. Dessa medel används för att rekrytera en professor, två lektorer inom inbyggda system och en lektor i beräkningsmatematik.

För att fånga näringslivets perspektiv har det under utvecklingen av programmet anordnats arbetsmöten med företag och organisationer för att i detalj diskutera deras behov och vilken inriktning de vill se på utbildningen. Varje möte fokuserade på en särskild aspekt av utbildningen såsom hur projekten bäst används för att förbereda för arbetslivet och hur mycket matematik och fysik som bör ingå, samt vilken inriktning dessa bör ha. Institutionens samverkansenhet har kontinuerligt lyft och diskuterat frågan i samband med återkommande kompetenssamverkansträffar. På så sätt har regionens näringsliv och offentliga verksamheter fått insyn och kunnat delta i processen; de har varit med och påverkat programstrukturen, kursinriktningar och i viss mån även kursinnehåll. Detta speglas av de stödbrev och avsiktsförklaringar som bifogas i ansökan.