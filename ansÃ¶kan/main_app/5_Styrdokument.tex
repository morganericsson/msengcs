\chapter{Säkring av styrdokument\label{ch:styrdokument}}

\begin{tcbdoublebox}
\emph{Avsnittet täcker aspekten Styrdokument i den föreslagna ansökningsmallen. Här anges hur styrdokument tas fram, följs upp, utvecklas och kvalitetssäkras av olika instanser vid Linnéuniversitetet.}
\end{tcbdoublebox}

Utbildningsplan för den föreslagna utbildningen bifogas i appendix~\ref{app:utbplan} och samtliga kursplaner bifogas i appendix ~\ref{app:kursplaner}. Appendix~\ref{app:progstrukt} visar strukturen hos programmet samt hur förkunskaper kopplar samman kurserna.
 
Styrdokument för utbildning hanteras inom Linnéuniversitetet på tre olika nivåer: universitet, fakultet och institution. Något förenklat hanteras regler kring och processer för styrdokumenten på universitetsnivå, programplaner samt nya kursplaner på fakultetsnivå och reviderade kursplaner på institutionsnivå. Den huvudsakliga granskningen av program- och kursplaner sker alltså på fakultetsnivå. Utöver dessa tillkommer en fjärde nivå, program, då samtliga kursplaner, nya och reviderade, inom det föreslagna civilingenjörsprogrammet måste godkännas av programansvarig innan de behandlas på övriga nivåer.

\section{Universitetsnivå}

Rådet för utbildning och lärande inrättades av rektor 1 april 2018 och har som övergripande uppgift att samordna det fakultetsövergripande arbetet och stödja fakulteternas arbete med utbildningsutbud och utbildningskvalitet. Som diskuterades i avsnitt~\ref{ch:personal} ska rådet även samordna och utveckla universitetets högskolepedagogiska arbete. Rådet leds av vicerektor med särskilt ansvar för frågor om utbildning och lärande och består av fakulteternas dekaner eller prodekaner samt två studeranderepresentanter.

Som stöd för rådets arbete har rektor inrättat ett kvalitetsutskott som har i uppdrag att samordna universitetets arbete med utbildningskvalitet, exempelvis följa och utveckla universitetets systematiska arbete med utbildningskvalitet och koordinera samarbetet inom Treklöverprojektet och UKÄ:s utvärderingar. Utskottet består av en representant från varje fakultet samt två studeranderepresentanter.

Rådet för utbildning och lärande bereder styr- och kvalitetsdokument inom utbildningsområdet inför rektorsbeslut. Detta innefattar Linnéuniversitetets flera övergripande styrdokument för utbildning\footnote{\url{https://medarbetare.lnu.se/medarbetare/styrning-och-regelverk/styrdokument/utbildning}} som bland annat innehåller principer vid prövning av nya utbildningsprogram för generell examen och för yrkesexamen, lokala regler för vad som gäller för utbildningsprogram och för kurser inom universitetet.

\section{Fakultetsnivå vid Fakulteten för teknik}

Utifrån Linnéuniversitetets styrdokument har Fakulteten för teknik fastställt specifika handläggningsrutiner för fastställande och revidering av både kurs- och utbildningsplaner\footnote{\url{https://medarbetare.lnu.se/medarbetare/organisation/ftk/beslut/kurs--och-utbildningsplaner-pa-ftk/}}. Rektors besluts- och delegationsordning\footnote{\url{https://lnu.se/contentassets/3dec6ef953444afd995da189b4ab8359/rektors-besluts--och-delegationsordning.pdf}} delegerar till dekan att besluta om kursplaner för nya kurser samt till fakultetsstyrelsen att besluta om utbildningsplaner för utbildning på grund- och avancerad nivå.

Arbetet med att säkerställa kurs- och utbildningsplaner fördelas operativt till Fakulteten för tekniks utbildningsråd och dess kursplaneutskott. Utbildningsrådet består av prodekan, två studeranderepresentanter och fyra studierektorer som representerar fakultetens huvudsakliga utbildningsområden, varav data och informationsteknik är ett. Rådets huvudsakliga uppgift är att samordna utvecklings- och kvalitetssäkringsarbetet och bereda nya och reviderade utbildningsplaner. Under beredningen granskas utbildningsplanen gentemot universitetets lokala regler för utbildningsprogram. Utbildningsrådet tillser att programmets kurser sammantaget svarar för att studenterna uppnår de nationella examensmålen och lokala gemensamma och utbildningsspecifika programmål som specificerats i utbildningsplanen. Detta sker i samråd med berörd studierektor och programansvarig. I och med införandet av CDIO vid fakulteten kommer utbildningsrådet att begära att matriser som kopplar kurser mot examensmål finns bifogade till samtliga ingenjörsutbildningars utbildningsplaner.

Kursplaneutskottet bereder och kvalitetssäkrar nya kursplaner utifrån lokala regler för kurser och kursplaner samt kontrollerar att det finns en konstruktiv länkning mellan angivna kursmål, undervisningsformer och examination i de fall detta efterfrågas. Utskottet består av fyra lärarrepresentanter, en från varje utbildningsområde samt en studeranderepresentant.

Varje studierektor och utbildningsområde har en programkommitté, som förutom studierektor består av samtliga programansvariga inom utbildningsområdet, minst två studeranderepresentanter, samt två till tre externa ledamöter. Programkommitténs huvudsakliga uppdrag är att kvalitetssäkra utbildningsprogrammen samt övervaka vilka behov som finns i samråd med näringslivet. Detta innebär att nya utbildningar och förändringar av befintliga ofta bereds av programkommittén innan de skickas till fakultetsstyrelsen och utbildningsrådet.

\section{Institutions och programnivå}

Reviderade kursplaner fastställs av prefekt. Dessa bereds ofta av programkommittén och/eller programansvarig innan de skickas till prefekten för fastställande. Vad det gäller kursplaner inom det föreslagna civilingenjörsprogrammet så kommer en ny process att införas där alla förändringar först måste godkännas av programansvarig. Det måste för varje kurs finnas en så kallad Introducera-Använda-Undervisa-Examinera-matris som kopplar kursens innehåll och examination mot mål i CDIO. Detta dokument beskrivs i mera detalj i avsnitt~\ref{ch:sakring} och ett exempel bifogas i appendix~\ref{app:itue-exempel}. I samband med i princip varje förändring av en kurs måste detta dokument uppdateras och innan ändring kan godkännas måste dess konsekvenser för utbildningen analyseras (med hjälp av Introducera-Använda-Undervisa-matrisen för programmet). För kurser i karaktärsämnet datavetenskap måste även dokumentet som kopplar innehållet i ACM:s curriculum för datavetenskap mot kurser i programmet (se exempel i appendix~\ref{app:acm-exempel}) uppdateras och konsekvenserna för programmet analyseras. Om ändringarna leder till att programmet inte längre uppfyller examensmålen, CDIO-målen eller målen i ACM:s curriculum kan förändringen inte godkännas och en process för att hitta lämpliga lösningar, såsom att även förändra andra kurser, måste inledas.

\section{Arbetsprocessen med förändring av styrdokument}

Behovet att ändra en kursplan kommer främst från lärare, studenter och/eller programkommittén. Det kan till exempel röra sig om att ett behov uppkommit i kursvärderingen eller att de externa representanterna i programkommittén tycker att något bör förändras för att förbättra kopplingen mot arbetslivets behov. Om det rör sig om en revision av en befintlig kursplan börjar lärarna på kursen arbetet och genomför ändringen i samråd med programansvarig. När programansvarig godkänner revisionen diskuteras den i programkommittén och när denna är nöjd med revisionen fastställs den nya kursplanen av prefekten för berörd institution. Om det rör sig om en ny kursplan är de första stegen på processen detsamma, men den går sedan till kursplaneutskottet istället för till prefekten och fastställs slutligen av dekan. Om ändringen kräver att utbildningsplanen ändras är de första stegen samma, men istället för kursplaneutskottet går utbildningsplanen vidare till utbildningsrådet vid Fakulteten för teknik och sedan fastställs den av fakultetsstyrelsen. I varje led granskas kursplanen enligt det regelverk som gäller, om det exempelvis rör sig om en reviderad kursplan säkerställer programkommittén och prefekten att kursen uppfyller målen och att examinationen är lämplig.