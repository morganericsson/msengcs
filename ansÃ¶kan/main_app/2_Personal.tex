\chapter{Lärarkompetens och lärarkapacitet\label{ch:personal}}

\begin{tcbdoublebox}
\emph{Avsnittet täcker aspekten Personal i den föreslagna ansökningsmallen. Här anges vilken kompetens som finns inom de institutioner som är inblandade i genomförandet av den föreslagna utbildningen samt en analys om varför denna anses vara tillräcklig. Avsnittet innehåller även en diskussion kring lärarnas utrymme för och tillgång till kompetensutveckling.}
\end{tcbdoublebox}

\section{Lärarresurser för undervisning, handledning och examination}

Institutionen för datavetenskap och medieteknik kommer att få det direkta ansvaret för att genomföra det föreslagna civilingenjörsprogrammet. Institutionerna för matematik, fysik och elektroteknik samt maskinteknik är delaktiga i det föreslagna civilingenjörsprogrammet och ansvarar för kurser eller moment inom kurser. Kurserna i teknik, människa och samhälle hämtas från de institutioner som är bäst lämpade att tillsammans med lärare från Institutionen från datavetenskap och medieteknik undervisa inom ämnet, exempelvis Institutionen för byggd miljö och energiteknik när det gäller miljöaspekterna inom hållbar utveckling. En preliminär lista över kursansvariga och examinatorer anges i appendix~\ref{app:examinatorer} och en lärartabell bifogas ansökan som separat dokument.

Den formella kompetensen har både den bredd och det djup som behövs för att säkerställa att kompetenta lärarresurser finns att tillgå i alla utbildningens kurser för undervisning, handledning och examination. Nedan följer en översiktlig beskrivning över de institutioner som är involverade i utbildningen:

\begin{itemize}
	\item Institutionen för datavetenskap och medieteknik har 5 professorer, 5 docenter, 16 lektorer, 17 adjunkter varav 3 i forskarutbildning, 2 forskarassistenter, 2 postdoktorer och 17 doktorander.
	\item  Institutionen för matematik har 6 professorer samt 1 gästprofessor, 6 docenter, 5 lektorer, 14 adjunkter varav 3 i forskarutbildning, 1 postdoktor och 6 doktorander.
	\item  Institutionen för maskinteknik har 4 professorer, 2 docenter, 3 lektorer, 4 universitetsadjunkter och 4 doktorander.
	\item Institutionen för maskinteknik har 4 professorer, 2 docenter, 3 lektorer, 4 universitetsadjunkter och 4 doktorander.
\end{itemize}

Enligt de lokala reglerna för examinationer vid Linnéuniversitetet\footfullcite{lr_examination} skall examinator för kurser på grundnivå minst vara disputerad inom relevant ämne (motsvarande lektor i våra tabeller) och för kurser på avancerad nivå minst vara docent inom relevant ämne. Vad det gäller kompetens att undervisa på eller examinera en kurs används även följande principer: den som undervisar på kurs på grundnivå skall ha tillräcklig ämneskunskap och pedagogisk färdighet och den som undervisar kurs på avancerad nivå skall dessutom vara aktiv forskare inom ämnet för kursen eller ett närliggande ämne. För att skapa forskningskoppling och ett forskande arbetssätt på kurser på grundnivå väljs lärare som antingen är forskningsaktiva i ämnet eller ett närliggande ämne i så stor utsträckning som möjligt även för dessa kurser.

Enligt Högskoleförordningen måste man visa pedagogisk skicklighet för att vara behörig att anställas som lektor eller professor, så samtliga lektorer och professorer har bedömts ha tillräcklig pedagogisk färdighet. Appendix~\ref{app:kompetens} anger vilken pedagogisk utbildning samtliga lärare som anges i appendix~\ref{app:examinatorer} har, samt om de har minst fem års erfarenhet av att undervisa inom högre utbildning. I de flesta fall är den utbildning som anges den så kallade behörighetsgivande högskolepedagogiska utbildningen som erbjuds svenska lärosäten; i de fall där utbildningen skett utomlands anges omfattningen på utbildningen i dagar eller veckor. Observera att kortare utbildningar och seminarieserier, exempelvis utbildning inom CDIO eller i kursplansskrivande, inte ger högskolepoäng och därför inte syns i sammanställningen. Många av lärarna har en bred erfarenhet från undervisning inom professionsutbildning på olika nivåer vid lärosäten i flera länder och därför presenteras även undervisningserfarenhet (i år). En lärare som har minst fem års erfarenhet av undervisning till cirka 50~\% av heltid anses vara tillräckligt, så därför anges bara ``mer än'' om en lärare har mer än fem års erfarenhet. Kompetensen hos medarbetarna har både den nödvändiga bredden och djupet för att möta kraven från en utbildning på civilingenjörsnivå.

För att uppskatta de resurser som krävs används en modell som bygger på genomsnittlig lärarresurs per högskolepoäng. Varje högskolepoäng uppskattas kräva 40 lärartimmar. Det exakta behovet varierar naturligtvis från kurs till kurs, men baserat på erfarenhet från befintliga ingenjörsprogram är detta en rimlig genomsnittlig uppskattning. Enligt denna uppskattning kräver programmet i sin helhet 12 000 lärartimmar. Observera att denna beräkning innehåller de valbara kurserna i termin 7–9 och inte tar hänsyn till eventuell samläsning. Dessa 12 000 lärartimmar fördelar sig över de olika kurserna och de ansvariga institutionerna och grupperingarna (Teknik, människa och samhälle) enligt tabell~\ref{tab:larh}. 

\begin{table}[tbh]
\caption{Fördelning av lärartimmar mellan de olika ingående institutionerna.\label{tab:larh}}
\centering
\resizebox{\columnwidth}{!}{%
\begin{tabular}{lrrr}
\toprule
\textsf{\textbf{Institution/område}} & \textsf{\textbf{Högskolepoäng}} & \textsf{\textbf{Timmar}} & \textsf{\textbf{Andel}} \tabularnewline
\midrule
Datavetenskap och medieteknik & 205~hp & 8~200~h & 68~\%\tabularnewline
Matematik & 50~hp & 2~000~h & 17~\%\tabularnewline
Fysik och elektroteknik tillsammans med maskinteknik & 20~hp & 800~h & 7~\% \tabularnewline
Teknik, människa och samhälle & 25~hp & 1~000~h & 8~\%\tabularnewline
\bottomrule
\end{tabular}}
\end{table}

En tillsvidareanställd har normalt en årsarbetstid om 1 700 timmar. Detta medför att det behövs en lärarresurs som motsvarar fem heltidsanställda vid Institutionen för datavetenskap och medieteknik för att täcka kurserna i datavetenskap. Motsvarande siffror för övriga institutioner är en heltid för Institutionen för matematik och en halvtid för Institutionen för fysik och elektroteknik samt Institutionen för maskinteknik och avslutningsvis en halvtid för kurserna inom teknik, människa och samhälle.

Professorer, docenter och lektorer kommer bära huvudansvaret för kurserna i den föreslagna civilingenjörsutbildningen och även ansvara för merparten av undervisning och examination. I utbildningen kommer dessutom adjunkter och assistenter att vara verksamma, företrädelsevis med handledning vid exempelvis laborationer och övningar, men adjunkter kommer även ta ett visst ansvar för andra undervisningsaktiviteter såsom föreläsningar på vissa grundkurser. En rimlig uppskattning, något i överkant, av fördelningen är 70/30, det vill säga att professorer och lektorer kommer ansvara för 70~\% av aktiviteter kopplat till undervisning. Med det som utgångspunkt motsvarar den seniora lärarresursen på Institutionen för datavetenskap och medieteknik 3,5 heltider för professorer och lektorer. Resterande 1,5 heltider utförs av adjunkter och assistenter.

\section{Kompetensförsörjning}

Kompetensförsörjning avser säkerställa att rätt kompetens finns för att nå de berörda verksamheternas mål både på kort och på lång sikt. Universitetet har utarbetat ett universitetsövergripande instrument, kompetensförsörjningsplanen, vilken syftar till att ge en översikt av befintlig kompetens och tydliggöra kommande kompetensbehov. Befintligt och kommande behov beskriver ett möjligt kompetensgap som används för att strategiskt planera kompetensutvecklingsbehov av befintlig personal och rekryteringar. Planerna används även för att identifiera områden där viss kompetens inte längre behövs och därmed kan avvecklas. Arbetet med kompetensförsörjning har sin grund i Linnéuniversitetets styrdokument, främst dess strategidokument, men även i beslutade policys, planer och program. Även omvärlden påverkar, exempelvis demografi, politiska beslut och konjunkturen på arbetsmarknaden. Institutionernas personalkonsulter deltar som resurs i arbetsprocesser som rör kompetensförsörjning.

Institutionen för datavetenskap och medieteknik arbetar systematiskt och rutinmässigt med kompetensförsörjningsplaner för att identifiera kortsiktiga och mer långsiktiga behov kopplade till både utbildning och forskning, både på kollegial och individuell nivå. Institutionens prefekt, ämnesansvariga och studierektor analyserar kontinuerligt behov på kort och lång sikt och dokumenterar detta i kompetensförsörjningsplanerna. Kompetensförsörjningsplanen för Institutionen för datavetenskap och medieteknik bifogas i appendix~\ref{app:kompetensplan}.

\section{Kompetensutveckling}

Kompetensförsörjningsplaner arbetas fram på institutionsnivå och kopplas således direkt till behov inom grundutbildning, forskarutbildning och forskning. Det kollegiala behovet analyseras och hanteras strategiskt via de ovan nämnda kompetensförsörjningsplanerna. I samband med årliga medarbetarsamtal och uppföljningssamtal planeras kompetensutvecklingsaktiviteter i samverkan med medarbetare. Kompetensutvecklingstiden planeras inom ramen för årsarbetstiden. Vid Institutionen för datavetenskap och medieteknik avsätts i regel 20~\% av årsarbetstiden för planerad kompetensutveckling för lektorer. Genom att aktiviteterna planeras i samverkan och dessutom följs upp kontinuerligt säkerställs såväl det kollegiala som det individuella kompetensförsörjningsbehovet vid institutionen. Kompetensutvecklingstiden kan överstiga 20~\% för vissa medarbetare under kortare eller längre tid om det är strategiskt motiverat eller om man i planen har identifierat ett akut behov som snabbt måste åtgärdas. Kompetensutveckling sker inom flera områden, exempelvis högskolepedagogik och språk, men merparten av aktiviteterna sker inom det egna ämnesområdet. Exempel på aktiviteter inom det egna området är breddning eller fördjupning för att hantera teknikutveckling som påverkar en kurs. För lektorer utgör forskning en naturlig del av kompetensutvecklingsarbetet. Flera adjunkter erbjuds forskarutbildning och de tre som för tillfället är aktiva beräknas att disputera under 2018–2019. För adjunkter i forskarutbildning överstiger kompetensutvecklingstiden ofta 50~\% av tjänst. Lektorer och adjunkter bereds sedan möjlighet till kompetensutveckling inom ordinarie tjänst i enlighet med planen.

Ett problem som identifierats inom arbetet med kompetensförsörjning är den relativt låga andel av docenter inom Institutionen för datavetenskap och medieteknik. Institutionen jobbar där strategiskt med att öka andelen docenter genom individuell karriärplanering där yngre lektorer tillsammans med ämnesansvarig och prefekt diskuterar och planerar karriärval som leder till docentkompetens.

Linnéuniversitetet har under senare år omorganiserat det centrala arbetet med kompetensutveckling avseende högskolepedagogik och handledning och en operativ enhet för högskolepedagogik har inrättats vid universitetsbiblioteket\footnote{\url{https://medarbetare.lnu.se/social/groups/linneuniversitetets-nyhetsbrev/posts/65293}}. En verksamhetsledare kommer att ansvara för arbetet vid den nyinrättade enheten som samordnas av Rådet för utbildning och lärande~\footnote{\url{https://medarbetare.lnu.se/medarbetare/organisation/rad-for-utbildning-och-larande/}}. Enheten kommer även att ansvara för universitetets strategi för digitalt lärande och hur denna integreras i det högskolepedagogiska arbetet. Institutionen för datavetenskap och medieteknik avser att samarbeta med den nya enheten för att ta fram kurser och seminarieserier som berör vad som identifierat som viktiga pedagogiska färdigheter, exempelvis hur man arbetar med jämställdhet inom kurser och projektbaserat lärande.

Fakulteten för teknik har sedan beslut togs om att införa CDIO-konceptet i juni 2017 en grupp som regelbundet genomför utbildningar, seminarier och arbetsmöten kring CDIO och hur man inför det i en utbildning. Att utbilda lärare i detta koncept är en viktig kompetensutveckling för lärare som förväntas undervisa på det föreslagna civilingenjörsprogrammet. De exakta kraven på utbildning i CDIO varierar beroende på vilka kurser en lärare är inblandad i, den som ger exempelvis tidiga projektkurser skall ha mera utbildning. Ungefär hälften av de lärarna som anges i appendix~\ref{app:examinatorer} har deltagit i något av de utbildningstillfällen som anordnats och samtliga planeras göra det innan den föreslagna civilingenjörsutbildningen inleds.