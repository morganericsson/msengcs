\chapter{Koppling av kurser och examensmål\label{app:kursmal}}

Kurserna på det föreslagna civilingenjörprogrammet kopplas i första hand mot mål i CDIO Syllabus. För att säkerställa att examensmålen uppfylls används en koppling mellan målen i CDIO Syllabus och examensmålen (se tabell~\ref{tab:maltillcdio}). Denna koppling används för att översätta kopplingen mellan kurser och mål i CDIO Syllabus till en koppling mellan kurser och examensmål. 

Tabell~\ref{tab:appkmd1} anger hur varje kurs i årskurs 1--3 uppfyller examensmålen. En kurs kan helt (\faCircle) eller delvis (\faCircleO) uppfylla ett mål. Det senare är en konsekvens av översättningen, då en flera mål i CDIO Syllabus kopplas till varje examensmål, och en kurs uppfyller nödvändigtvis inte samtliga av dessa mål i CDIO Syllabus. Kursen ``Diskret matematik'' uppfyller exempelvis delvis mål A.1, då mål A.1 är kopplat till mål 1.1--1.3 och 5.6 i CDIO Syllabus och kursen uppfyller mål 1.1. Tabell~\ref{tab:appkmr1} anger till vilken grad (i procent) varje kurs i årskurs 1--3 uppfyller examensmålen. Då kursen ``Diskret matematik'' uppfyller ett av de fyra mål i CDIO Syllabus som kopplas till A.1 uppfyller den målet till 25~\%. Tabellerna~\ref{tab:appkmd2} och~\ref{tab:appkmr2} innehåller kopplingar mellan examensmålen och kurserna i årskurs 4 och 5.

Samtliga examensmål uppfylls av kurserna på det föreslagna programmet. I många fall uppfylls ett mål delvis av flera kurser, men tillsammans uppfyller de målet helt. Observera att kopplingen som redovisas här innehåller de valbara kurserna i årskurs 4 och 5. I de fall en student vill läsa en annan kurs istället för dessa måste valet diskuteras med programansvarig, som då kontrollerar att examensmålen fortfarande uppfylls.

\begin{sidewaystable}[H]
\centering
\caption{Koppling mellan examensmål och kurser i årskurs 1--3. För varje kurs och examensmål anges om målet uppfylls helt (\faCircle) eller delvis (\faCircleO).\label{tab:appkmd1}}
\resizebox{.9\linewidth}{!}{% 
\begin{tabular}{p{10cm}ccccccccccccc}
\toprule
\textsf{\textbf{Kurs}} & \textsf{\textbf{A.1}} & \textsf{\textbf{A.2}} & \textsf{\textbf{B.1}} & \textsf{\textbf{B.2}} & \textsf{\textbf{B.3}} & \textsf{\textbf{B.4}} & \textsf{\textbf{B.5}} & \textsf{\textbf{B.6}} & \textsf{\textbf{B.7}} & \textsf{\textbf{B.8}} & \textsf{\textbf{C.1}} & \textsf{\textbf{C.2}} & \textsf{\textbf{C.3}} \tabularnewline
\midrule
Diskret matematik & \faCircleO & \faCircleO & \faCircleO & \faCircleO & \faCircleO & \faCircleO & \faCircleO &   &   &   & \faCircleO & \faCircleO & \faCircleO\tabularnewline
Programmering och datastrukturer & \faCircleO & \faCircleO & \faCircleO & \faCircleO & \faCircleO & \faCircleO & \faCircleO &   & \faCircle & \faCircleO &   & \faCircleO &  \tabularnewline
Linjär algebra & \faCircleO & \faCircleO & \faCircleO & \faCircleO & \faCircleO & \faCircleO & \faCircleO &   &   &   & \faCircleO & \faCircleO & \faCircleO\tabularnewline
Introducerande projekt & \faCircleO & \faCircleO & \faCircleO & \faCircleO & \faCircleO & \faCircleO & \faCircleO & \faCircleO & \faCircle & \faCircleO & \faCircleO & \faCircleO & \faCircle\tabularnewline
Envariabelanalys 1 & \faCircleO & \faCircleO & \faCircleO & \faCircleO & \faCircleO & \faCircleO & \faCircleO &   &   &   & \faCircleO & \faCircleO & \faCircleO\tabularnewline
Databaser och datamodellering & \faCircleO & \faCircleO & \faCircleO & \faCircleO & \faCircleO & \faCircleO & \faCircleO & \faCircleO &   & \faCircleO & \faCircleO & \faCircleO & \faCircleO\tabularnewline
Objektorienterad programmering & \faCircleO & \faCircleO & \faCircleO & \faCircleO & \faCircleO & \faCircleO & \faCircleO & \faCircleO &   & \faCircleO & \faCircleO & \faCircleO & \faCircleO\tabularnewline
Tillämpad sannolikhetslära och statistik & \faCircleO & \faCircleO & \faCircleO & \faCircleO & \faCircleO & \faCircleO & \faCircleO &   &   &   & \faCircleO & \faCircleO & \faCircleO\tabularnewline
Mekanik & \faCircleO & \faCircleO & \faCircleO & \faCircleO & \faCircleO & \faCircleO & \faCircleO &   &   &   & \faCircleO & \faCircleO & \faCircleO\tabularnewline
Jämnlöpande program & \faCircleO & \faCircleO & \faCircleO & \faCircleO & \faCircleO & \faCircleO & \faCircleO & \faCircleO &   &   &   & \faCircleO &  \tabularnewline
Ellära och magnetism & \faCircleO & \faCircleO & \faCircleO & \faCircleO & \faCircleO & \faCircleO & \faCircleO &   &   &   & \faCircleO & \faCircleO & \faCircleO\tabularnewline
Teknisk kommunikation &   &   & \faCircleO & \faCircleO & \faCircleO & \faCircleO & \faCircleO &   &   & \faCircleO & \faCircleO & \faCircleO & \faCircleO\tabularnewline
Algoritmer & \faCircleO & \faCircleO & \faCircleO & \faCircleO & \faCircleO & \faCircleO & \faCircleO & \faCircleO &   & \faCircleO & \faCircleO & \faCircleO & \faCircleO\tabularnewline
Mjukvaruutvecklingsprojekt & \faCircleO & \faCircleO & \faCircleO & \faCircleO & \faCircleO & \faCircleO & \faCircleO & \faCircle & \faCircle & \faCircleO & \faCircleO & \faCircleO & \faCircle\tabularnewline
Hållbar utveckling &   & \faCircleO & \faCircleO & \faCircleO & \faCircleO & \faCircleO & \faCircleO & \faCircleO &   &   & \faCircleO & \faCircleO &  \tabularnewline
Envariabelanalys 2 & \faCircleO & \faCircleO & \faCircleO & \faCircleO & \faCircleO & \faCircleO & \faCircleO &   &   &   & \faCircleO & \faCircleO & \faCircleO\tabularnewline
Flervariabelanalys & \faCircleO & \faCircleO & \faCircleO & \faCircleO & \faCircleO & \faCircleO & \faCircleO &   &   &   & \faCircleO & \faCircleO & \faCircleO\tabularnewline
Datorns uppbyggnad & \faCircleO & \faCircleO & \faCircleO & \faCircleO & \faCircleO & \faCircleO & \faCircleO &   &   &   &   & \faCircleO &  \tabularnewline
Numeriska metoder & \faCircleO & \faCircleO & \faCircleO & \faCircleO & \faCircleO & \faCircleO & \faCircleO &   &   &   & \faCircleO & \faCircleO & \faCircleO\tabularnewline
Mjukvaruarkitektur & \faCircleO & \faCircleO & \faCircleO & \faCircleO & \faCircleO & \faCircleO & \faCircleO & \faCircleO & \faCircle & \faCircleO & \faCircleO & \faCircleO & \faCircleO\tabularnewline
Inbyggda system & \faCircleO & \faCircleO & \faCircleO & \faCircleO & \faCircleO & \faCircleO & \faCircleO & \faCircleO &   &   &   & \faCircleO &  \tabularnewline
Reglerteknik & \faCircleO & \faCircleO & \faCircleO & \faCircleO & \faCircleO & \faCircleO & \faCircleO &   &   &   &   & \faCircleO &  \tabularnewline
Datorgrafik & \faCircleO & \faCircleO & \faCircleO & \faCircleO & \faCircleO & \faCircleO & \faCircleO & \faCircleO &   &   &   & \faCircleO &  \tabularnewline
Datornät & \faCircleO & \faCircleO & \faCircleO & \faCircleO & \faCircleO & \faCircleO & \faCircleO & \faCircleO &   &   & \faCircleO & \faCircleO & \faCircleO\tabularnewline
Industriell ekonomi &   &   &   &   &   &   &   & \faCircleO &   &   & \faCircleO & \faCircleO &  \tabularnewline
Vetenskapliga metoder & \faCircleO & \faCircle & \faCircleO & \faCircleO & \faCircleO & \faCircleO & \faCircleO & \faCircleO &   & \faCircleO & \faCircle & \faCircleO & \faCircle\tabularnewline
Datorsäkerhet & \faCircleO & \faCircleO & \faCircleO & \faCircleO & \faCircleO & \faCircleO & \faCircleO & \faCircle &   & \faCircleO & \faCircleO & \faCircleO & \faCircle\tabularnewline
Självständigt arbete (G2E) & \faCircleO & \faCircleO & \faCircleO &   & \faCircleO & \faCircleO & \faCircleO & \faCircleO &   & \faCircleO & \faCircleO & \faCircleO &  \tabularnewline
\bottomrule
\end{tabular}
}
\end{sidewaystable}

\begin{sidewaystable}[H]
\centering
\caption{Koppling mellan examensmål och kurser i årskurs 1--3. För varje kurs och examensmål anges i vilken grad (procent) kursen uppfyller målet.\label{tab:appkmr1}}
\resizebox{.9\linewidth}{!}{~\% 
\begin{tabular}{p{10cm}lllllllllllll}
\toprule
\textsf{\textbf{Kurs}} & \textsf{\textbf{A.1}} & \textsf{\textbf{A.2}} & \textsf{\textbf{B.1}} & \textsf{\textbf{B.2}} & \textsf{\textbf{B.3}} & \textsf{\textbf{B.4}} & \textsf{\textbf{B.5}} & \textsf{\textbf{B.6}} & \textsf{\textbf{B.7}} & \textsf{\textbf{B.8}} & \textsf{\textbf{C.1}} & \textsf{\textbf{C.2}} & \textsf{\textbf{C.3}} \tabularnewline
\midrule
Diskret matematik&25~\%&17~\%&33~\%&25~\%&22~\%&13~\%&23~\%&0~\%&0~\%&0~\%&20~\%&14~\%&50~\% \tabularnewline
Programmering och datastrukturer&50~\%&17~\%&17~\%&13~\%&11~\%&20~\%&23~\%&0~\%&100~\%&33~\%&0~\%&14~\%&0~\% \tabularnewline
Linjär algebra&50~\%&33~\%&33~\%&38~\%&33~\%&20~\%&31~\%&0~\%&0~\%&0~\%&20~\%&29~\%&50~\% \tabularnewline
Introducerande projekt&25~\%&33~\%&67~\%&63~\%&56~\%&47~\%&46~\%&67~\%&100~\%&33~\%&60~\%&57~\%&100~\% \tabularnewline
Envariabelanalys 1&25~\%&17~\%&33~\%&25~\%&22~\%&13~\%&23~\%&0~\%&0~\%&0~\%&20~\%&14~\%&50~\% \tabularnewline
Databaser och datamodellering&25~\%&50~\%&83~\%&75~\%&67~\%&47~\%&54~\%&17~\%&0~\%&33~\%&20~\%&29~\%&50~\% \tabularnewline
Objektorienterad programmering&25~\%&33~\%&67~\%&75~\%&56~\%&47~\%&54~\%&33~\%&0~\%&33~\%&20~\%&29~\%&50~\% \tabularnewline
Tillämpad sannolikhetslära och statistik&25~\%&33~\%&50~\%&38~\%&33~\%&20~\%&31~\%&0~\%&0~\%&0~\%&20~\%&14~\%&50~\% \tabularnewline
Mekanik&25~\%&33~\%&50~\%&38~\%&33~\%&20~\%&31~\%&0~\%&0~\%&0~\%&20~\%&14~\%&50~\% \tabularnewline
Jämnlöpande program&25~\%&33~\%&33~\%&38~\%&22~\%&27~\%&31~\%&33~\%&0~\%&0~\%&0~\%&14~\%&0~\% \tabularnewline
Ellära och magnetism&50~\%&33~\%&50~\%&38~\%&33~\%&27~\%&38~\%&0~\%&0~\%&0~\%&20~\%&29~\%&50~\% \tabularnewline
Teknisk kommunikation&0~\%&0~\%&17~\%&13~\%&11~\%&7~\%&8~\%&0~\%&0~\%&67~\%&20~\%&14~\%&50~\% \tabularnewline
Algoritmer&50~\%&50~\%&67~\%&75~\%&56~\%&47~\%&62~\%&50~\%&0~\%&33~\%&40~\%&43~\%&50~\% \tabularnewline
Mjukvaruutvecklingsprojekt&25~\%&33~\%&67~\%&75~\%&56~\%&53~\%&54~\%&100~\%&100~\%&33~\%&60~\%&71~\%&100~\% \tabularnewline
Hållbar utveckling&0~\%&17~\%&17~\%&13~\%&11~\%&7~\%&8~\%&17~\%&0~\%&0~\%&20~\%&14~\%&0~\% \tabularnewline
Envariabelanalys2&25~\%&17~\%&33~\%&25~\%&22~\%&13~\%&23~\%&0~\%&0~\%&0~\%&20~\%&14~\%&50~\% \tabularnewline
Flervariabelanalys&25~\%&17~\%&33~\%&25~\%&22~\%&13~\%&23~\%&0~\%&0~\%&0~\%&20~\%&14~\%&50~\% \tabularnewline
Datorns uppbyggnad&50~\%&33~\%&33~\%&25~\%&22~\%&20~\%&31~\%&0~\%&0~\%&0~\%&0~\%&14~\%&0~\% \tabularnewline
Numeriska metoder&75~\%&33~\%&33~\%&38~\%&33~\%&27~\%&38~\%&0~\%&0~\%&0~\%&20~\%&43~\%&50~\% \tabularnewline
Mjukvaruarkitektur&25~\%&33~\%&67~\%&63~\%&44~\%&47~\%&46~\%&33~\%&100~\%&33~\%&20~\%&29~\%&50~\% \tabularnewline
Inbyggda system&50~\%&50~\%&67~\%&75~\%&56~\%&47~\%&62~\%&33~\%&0~\%&0~\%&0~\%&14~\%&0~\% \tabularnewline
Reglerteknik&75~\%&50~\%&33~\%&38~\%&33~\%&27~\%&38~\%&0~\%&0~\%&0~\%&0~\%&29~\%&0~\% \tabularnewline
Datorgrafik&75~\%&50~\%&33~\%&50~\%&33~\%&33~\%&46~\%&17~\%&0~\%&0~\%&0~\%&29~\%&0~\% \tabularnewline
Datornät&75~\%&50~\%&67~\%&88~\%&67~\%&53~\%&69~\%&50~\%&0~\%&0~\%&40~\%&57~\%&50~\% \tabularnewline
Industriell ekonomi&0~\%&0~\%&0~\%&0~\%&0~\%&0~\%&0~\%&33~\%&0~\%&0~\%&20~\%&14~\%&0~\% \tabularnewline
Vetenskapliga metoder&50~\%&100~\%&83~\%&63~\%&78~\%&60~\%&62~\%&50~\%&0~\%&33~\%&100~\%&71~\%&100~\% \tabularnewline
Datorsäkerhet&50~\%&50~\%&83~\%&88~\%&67~\%&53~\%&69~\%&100~\%&0~\%&33~\%&60~\%&71~\%&100~\% \tabularnewline
Självständigt arbete (G2E)&25~\%&33~\%&17~\%&0~\%&22~\%&27~\%&23~\%&17~\%&0~\%&33~\%&60~\%&29~\%&0~\% \tabularnewline
\tabularnewline
\bottomrule
\end{tabular}
}
\end{sidewaystable}

\begin{sidewaystable}[H]
\centering
\caption{Koppling mellan examensmål och kurser i årskurs 4 och 5. För varje kurs och examensmål anges om målet uppfylls helt (\faCircle) eller delvis (\faCircleO).\label{tab:appkmd2}}
\resizebox{.9\linewidth}{!}{% 
\begin{tabular}{p{10cm}ccccccccccccc}
\toprule
\textsf{\textbf{Kurs}} & \textsf{\textbf{A.1}} & \textsf{\textbf{A.2}} & \textsf{\textbf{B.1}} & \textsf{\textbf{B.2}} & \textsf{\textbf{B.3}} & \textsf{\textbf{B.4}} & \textsf{\textbf{B.5}} & \textsf{\textbf{B.6}} & \textsf{\textbf{B.7}} & \textsf{\textbf{B.8}} & \textsf{\textbf{C.1}} & \textsf{\textbf{C.2}} & \textsf{\textbf{C.3}} \tabularnewline
\midrule
Modellering och simulering av system & \faCircleO & \faCircleO & \faCircleO & \faCircleO & \faCircleO & \faCircleO & \faCircleO & \faCircleO &   &   & \faCircleO & \faCircleO &  \tabularnewline
Kompilatorkonstruktion & \faCircleO & \faCircleO &   & \faCircleO & \faCircleO & \faCircleO & \faCircleO &   &   &   & \faCircleO & \faCircleO &  \tabularnewline
Formella metoder & \faCircle & \faCircleO & \faCircleO & \faCircleO & \faCircleO & \faCircleO & \faCircleO & \faCircleO &   &   & \faCircleO & \faCircleO &  \tabularnewline
Optimering & \faCircleO & \faCircleO & \faCircleO & \faCircleO & \faCircleO & \faCircleO & \faCircleO &   &   &   & \faCircleO & \faCircleO & \faCircleO\tabularnewline
Projekt i modellbaserad utveckling & \faCircle & \faCircleO & \faCircleO & \faCircleO & \faCircleO & \faCircleO & \faCircleO & \faCircle & \faCircle & \faCircleO & \faCircleO & \faCircleO & \faCircle\tabularnewline
Maskininlärning & \faCircle & \faCircleO & \faCircleO & \faCircleO & \faCircleO & \faCircleO & \faCircleO & \faCircleO &   & \faCircleO & \faCircleO & \faCircleO &  \tabularnewline
Parallelldatorprogrammering & \faCircleO & \faCircleO & \faCircleO & \faCircleO & \faCircleO & \faCircleO & \faCircleO & \faCircleO &   & \faCircleO & \faCircleO & \faCircleO &  \tabularnewline
Djup maskininlärning & \faCircleO & \faCircleO & \faCircleO & \faCircleO & \faCircleO & \faCircleO & \faCircleO & \faCircleO &   & \faCircleO & \faCircleO & \faCircleO &  \tabularnewline
Lean startup &   &   & \faCircleO & \faCircleO & \faCircleO & \faCircleO & \faCircleO & \faCircleO &   & \faCircleO & \faCircleO & \faCircleO & \faCircleO\tabularnewline
Projekt i dataintensiva system & \faCircleO & \faCircleO & \faCircleO & \faCircleO & \faCircleO & \faCircleO & \faCircleO & \faCircle & \faCircle &   & \faCircle & \faCircle & \faCircle\tabularnewline
Informationsvisualisering & \faCircleO & \faCircleO &   & \faCircleO & \faCircleO & \faCircleO & \faCircleO & \faCircleO &   &   & \faCircleO & \faCircleO & \faCircleO\tabularnewline
Datautvinning & \faCircle & \faCircleO &   & \faCircleO & \faCircleO & \faCircleO & \faCircleO & \faCircleO &   &   & \faCircleO & \faCircleO & \faCircleO\tabularnewline
Avancerad informationsvisualisering och tillämpningar & \faCircleO & \faCircleO &   & \faCircleO & \faCircleO & \faCircleO & \faCircleO & \faCircleO &   &   & \faCircleO & \faCircleO & \faCircleO\tabularnewline
Vetenskapliga metoder inom datavetenskap & \faCircleO & \faCircle & \faCircleO & \faCircleO & \faCircleO & \faCircleO & \faCircleO & \faCircleO &   & \faCircle & \faCircle & \faCircleO & \faCircle\tabularnewline
Projekt i visualisering och dataanalys & \faCircleO & \faCircleO & \faCircleO & \faCircleO & \faCircleO & \faCircleO & \faCircleO & \faCircle &   &   & \faCircleO & \faCircleO & \faCircle\tabularnewline
Självständigt arbete (A2E) & \faCircleO & \faCircleO & \faCircleO &   & \faCircleO & \faCircleO & \faCircleO & \faCircleO &   & \faCircleO & \faCircleO & \faCircleO &  \tabularnewline
\bottomrule
\end{tabular}
}
\end{sidewaystable}

\begin{sidewaystable}[H]
\centering
\caption{Koppling mellan examensmål och kurser i årskurs 4 och 5. För varje kurs och examensmål anges i vilken grad (procent) kursen uppfyller målet.\label{tab:appkmr2}}
\resizebox{.9\linewidth}{!}{% 
\begin{tabular}{p{10cm}ccccccccccccc}
\toprule
\textsf{\textbf{Kurs}} & \textsf{\textbf{A.1}} & \textsf{\textbf{A.2}} & \textsf{\textbf{B.1}} & \textsf{\textbf{B.2}} & \textsf{\textbf{B.3}} & \textsf{\textbf{B.4}} & \textsf{\textbf{B.5}} & \textsf{\textbf{B.6}} & \textsf{\textbf{B.7}} & \textsf{\textbf{B.8}} & \textsf{\textbf{C.1}} & \textsf{\textbf{C.2}} & \textsf{\textbf{C.3}} \tabularnewline
\midrule
Modellering och simulering av system&75~\%&67~\%&67~\%&88~\%&67~\%&60~\%&62~\%&33~\%&0~\%&0~\%&20~\%&29~\%&0~\% \tabularnewline
Kompilatorkonstruktion&75~\%&17~\%&0~\%&13~\%&11~\%&20~\%&15~\%&0~\%&0~\%&0~\%&20~\%&29~\%&0~\% \tabularnewline
Formella metoder&100~\%&17~\%&17~\%&50~\%&33~\%&40~\%&46~\%&33~\%&0~\%&0~\%&20~\%&29~\%&0~\% \tabularnewline
Optimering&50~\%&17~\%&33~\%&25~\%&22~\%&20~\%&23~\%&0~\%&0~\%&0~\%&40~\%&14~\%&50~\% \tabularnewline
Projekt i modellbaserad utveckling&100~\%&83~\%&83~\%&88~\%&89~\%&80~\%&85~\%&100~\%&100~\%&67~\%&80~\%&86~\%&100~\% \tabularnewline
Maskininlärning&100~\%&67~\%&67~\%&75~\%&56~\%&53~\%&62~\%&33~\%&0~\%&33~\%&20~\%&29~\%&0~\% \tabularnewline
Parallelldatorprogrammering&75~\%&67~\%&67~\%&88~\%&67~\%&60~\%&62~\%&50~\%&0~\%&33~\%&20~\%&43~\%&0~\% \tabularnewline
Djup maskininlärning&75~\%&67~\%&50~\%&63~\%&56~\%&47~\%&46~\%&33~\%&0~\%&67~\%&20~\%&43~\%&0~\% \tabularnewline
Lean startup&0~\%&0~\%&17~\%&13~\%&11~\%&7~\%&8~\%&50~\%&0~\%&67~\%&40~\%&29~\%&50~\% \tabularnewline
Projekt i dataintensiva system&75~\%&83~\%&83~\%&88~\%&89~\%&80~\%&77~\%&100~\%&100~\%&0~\%&100~\%&100~\%&100~\% \tabularnewline
Informationsvisualisering&75~\%&17~\%&0~\%&13~\%&11~\%&20~\%&15~\%&33~\%&0~\%&0~\%&60~\%&57~\%&50~\% \tabularnewline
Datautvinning&100~\%&17~\%&0~\%&13~\%&11~\%&20~\%&23~\%&33~\%&0~\%&0~\%&80~\%&71~\%&50~\% \tabularnewline
Avancerad informationsvisualisering och tillämpningar&75~\%&17~\%&0~\%&13~\%&11~\%&20~\%&15~\%&33~\%&0~\%&0~\%&60~\%&57~\%&50~\% \tabularnewline
Vetenskapliga metoder inom datavetenskap&50~\%&100~\%&83~\%&63~\%&78~\%&67~\%&62~\%&33~\%&0~\%&100~\%&100~\%&71~\%&100~\% \tabularnewline
Projekt i visualisering och dataanalys&75~\%&83~\%&83~\%&88~\%&89~\%&73~\%&77~\%&100~\%&0~\%&0~\%&80~\%&86~\%&100~\% \tabularnewline
Självständigt arbete (A2E)&25~\%&33~\%&17~\%&0~\%&22~\%&33~\%&23~\%&17~\%&0~\%&33~\%&60~\%&29~\%&0~\% \tabularnewline
\bottomrule
\end{tabular}
}
\end{sidewaystable}
